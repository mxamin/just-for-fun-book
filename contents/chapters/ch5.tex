\chapter{مقالات}
\section{دارایی معنوی}
بحث دارایی معنوی این روزها آن قدر داغ است که بعید است به اتاقی داخل
شوم و پوستری یا نوشته‌ای در حمایت از یکی از طرفین بحث در آن‌جا نصب نشده
باشد. بعضی‌ها فکر می‌کنند که امتیازنامه‌ها یا دیگر قوانین حمایت از
مالکیت‌های معنوی باعث تباهی و فساد جهان خواهند شد. به اعتقاد آن‌ها مشکل
این قوانین آن نیست که از اعتدلال خارج شده‌اند بلکه به باور آن‌ها قوانین
مرتبط با مالکیت‌های معنوی اصولا نوعی شر است که باید هر چه زودتر از دست
آن خلاص شد. طرف مقابل فکر می‌کند که کل اقتصاد جهان بر مبنای حقوق دارایی
معنوی بنا شده است و تمام تلاش شان این است که قونین مربوط به
امتیازنامه‌ها را مستحکم‌تر کنند.

نتیجه این است که دیوارنوشته‌های له و علیه این موضوع، پر رنگ‌تر و
برجسته‌تر شوند.

مشخص است که منظورم از اتاق، بیشتر اتاق‌های مجازی درون اینترنت است و نه
کافه‌های شبانه سن جوز\RFootnote{لینوکس در پاورقی می‌نویسد \dbquote{البته
    هر کسی که در سن جوز زندگی کرده باشد به شما خواهد گفت که این شهر
    زندگی شبانه ندارد و کسانی که بخواهند شب‌ها تفریح کنند باید با ماشین
    به سن ماتئو بروند.}}. گوشه و کنار اینترنت پر است از مشاجرات مربوط
به قوانین دارایی معنوی و مردمی که درباره همه چیز صحبت می‌کنند، از
اصلاحیه اول\RFootnote{اصلی ترین قانون تضمین کننده آزادی بیان در
  آمریکا} گرفته تا این موضوع که دارایی معنوی ممکن است در آینده توسعه
نرم‌افزارهای بازمتن را غیر ممکن کند.

وقتی که من می‌خواهم به این مساله فکر کنم به جایی می‌رسم که رسما می‌توان
آن را دیوانگی خواند.

مساله این نیست که من نظری ندارم: به نظر من دارایی معنوی چیز ارزشمندی
است ولی این دقیقا چیزی است که هر دو طرف دعوا آن را قبول دارند. به شما
گفتم که، جریان خیلی گیج کننده است. معمولا در آخر بحث من می‌بینم که به
نفع هر دو طرف استدلال کرده‌ام. جریان هم این است که دو طرف بحث، از دو
چیز مستقل دفاع می‌کنند و این وسط دارایی معنوی فقط یک اسم است که هر دوی
آن‌ها به کارش می‌برند.

برای خیلی‌ها - از جمله من - دارایی معنوی چیزی است مربوط به اختراعات
انسانی، یعنی دقیقا همان چیزی که انسان را از حیوان جدا می‌کند (و خب از
انگشت و این جور چیزها). در این فضا خود اسم \dbquote{دارایی معنوی} یک
توهین است چون اختراع یک کالا نیست که بشود مالک آن بود یا آن را خرید و
فروش کرد. اختراع، همان آفرینش است و ارزشمندترین کاری که موجودات انسانی
می‌توانند بکنند. ابداع، هنر است؛ آن‌هم یک هنر پررنگ و درخشان. مونالیزا
یک ابداع است درست همان طور که برنامه حاصل از یک شب پر کار پشت کامپیوتر
که برنامه‌نویس به آن افتخار می‌کند، هنر است. چنین چیزی را نمی‌شود ارزش
گذاری کرد یا حتی فروخت چون بخشی از هویت فردی است که آن را به وجود
آورده.

این نوع از تولید - خواه به شکل نقاشی باشد یا موسیقی یا مجسمه‌سازی یا نوشتن یا برنامه‌نویسی - باید مقدس باشد. تولید کننده و محصول، رابطه‌ای با هم دارند که غیرقابل تقلیل است.q درست مثل رابطه مادر و فرزند یا رابطه بین غذای چینی و ام.اس.جی.\LFootnote{MSG}. اما در عین حال محصول باید در اختیار هر کس دیگری که می‌خواهد از آن استفاده کند یا آن را تغییر دهد هم باشد چون این محصول، همان انسانیت است.

و در همین حال، در سمت دیگر صنعتی هست با ارزش تقریبی هفت گازیلیون
میلیارد دلار آمریکا در سال که دارایی معنوی نامیده می‌شود. این روزها
خلاقیت انسانی اتیکت قیمت خورده و اتفاقا قیمتش هم خیلی بالا است. خلاقیت
کمیاب است و در نتیجه نه فقط گران، که بسیار هم پر منفعت. این زاویه دید
باعث یک بحث جدید می‌شود که کاملا با قبلی فرق دارد و آدم‌های متفاوتی را
هم درگیر می‌کند. آدم‌هایی در اینجا بحث می‌کنند که حاصل خلاقیت بشری را
\dbquote{دارایی} می‌نامند. این آدم‌ها الزاما وکیل نیستند.

یک بار دیگر به عنوان این فصل نگاه کنید تا متوجه شوید که آدم‌های معتقد
به \dbquote{دارایی} تا الان که برنده بوده‌اند. حداقل اسم آن‌ها که پیروز
شده، پس مشکل چیست؟

مشهورترین مفهوم در دنیای دارایی معنوی، کپی‌رایت است. کپی‌رایت در واقع
شرایطی است که سازنده طی آن اعلام می‌کند که در مورد این محصول، چه حقوقی
را به دیگران تفویض می‌کند. \dbquote{صاحب} محصول حق دارد اعلام کند که
دیگران با پذیرش چه قواعدی باید از محصول او استفاده کنند.

کپی‌رایت دار کردن یک محصول هم کار ساده‌ای است. نیازی به ثبت آن نیست: شما
به شکل خودکار صاحب کپی‌رایت محصولی هستید که ساخته‌اید. این وضع، تفاوت
عمده‌ای دارد با دیگر انواع قوانین مربوط به دارایی‌های معنوی. بر خلاف
قوانین پیچیده حقوق تجاری و لوگوها، داشتن کپی‌رایت حق هر شهروند است و نه
در اختیار شرکت‌های بزرگ. شما با کشیدن، نوشتن یا کلا ساختن یک چیز منحصر
به فرد، می‌توانید صاحب کپی‌رایت آن شوید. گفته می‌شود که داشتن کپی‌رایت یک
محصول به سادگی نوشتن \dbquote{\lr{©} کپی رایت ۲۰۰۰، جای اسم شما} است ولی
صادقانه به شما بگویم که حتی نیازی به این کار هم نیست. چه بنویسید و چه
ننویسید، کپی رایت محصول شما متعلق به شما است. آن نوشته فقط به دیگران
کمک می‌کند تا در صورت علاقه به استفاده از محصول شما، راحت‌تر شما را پیدا
کنند.

شکی نیست که مالکیت کپی رایت چیزی کمک چندانی به کسی نمی‌کند. واقعیت این
است که مالکیت چیزی که ساخته‌اید، به این معناست که می‌توانید شیوه استفاده
از آن را شخصا تعیین کنید. مثلا حق دارید محصول هنری خود را به کسی
بفروشید و هیچ کسی به جز اداره مالیات حق ندارد در این باره جلوی شما را
بگیرد. ولی جریان چیزی بیشتر از پول صرف است و این دقیقا همان چیزی است
که باعث گیجی بیشتر مردم می‌شود.

برای مثال شما به عنوان صاحب کپی‌رایت یک کالا، این قدرت را دارید که با
کالای تان، کارهای جالب‌تری از فروش آن بکنید. مثلا اینکه برای آن یک مجوز
تعیین کنید. این کار حتی از فروختن‌ یک چیز هم بهتر است. با استفاده از یک
مجوز درست، می‌توانید به جای فروش خود یک اثر هنری، مجوز آن را به دیگران
بفروشید و بر اساس آن مجوز دیگران حق داشته باشند در حالی که اثر هنوز
متعلق به شما است، از آن استفاده‌ بکنند. انتخاب یک مجوز درست به اصطلاح
باعث خواهد شد تا هم بتوانید کیک‌تان را بخورید و هم آن را نگه‌ دارید. این
دقیقا همان روشی که باعث شد مایکروسافت رشد کند: مجوز یک کالا را دائما
به دیگران بفروشد در حالی که کالا هنوز متعلق به خودش است. بیخود نیست که
مردم عاشق داشتن این جور مجوزها هستند.

ببینم، تا این جای کار متوجه مشکل شده‌اید یا نه؟ اگر هنوز متوجه نشده‌اید
باید بگویم که خیلی مواظب کلاهبردارهای حرفه‌ای باشید!

مشکل اصلی دارایی معنوی این شده که صاحب آن می‌تواند تا بینهایت بار آن را
بفروشد بدون اینکه چیزی را از دست بدهد. شما هیچ ریسکی نمی‌کنید و در واقع
حتی این را در مجوز خود می‌گنجانید که اگر کالا عیبی پیدا کرد یا باعث
عیبی شد، شما هیچ مسوولیتی نخواهید داشت. به نظر نامعقول است؟ دقیقا!

مشکل این است که هیچ کس از مصرف کننده حمایت نمی‌کند.

اوضاع از این هم بدتر است. صاحب کپی‌رایت نه تنها حق فروش محصولش را تا
ابد خواهد داشت، که می‌تواند علیه هر کسی هم که محصولی شبیه او تولید کند،
شکایت کند. واضح است که صاحب کپی‌رایت، صاحب حقوق کارهای مشتق از کار اصلی
هم هست.

واضح است؟ نه چندان. خط مرز بین الهام و تقلید کجاست؟ اگر دو نفر به شکل
مستقل به یک ایده برسند چه؟ آیا بحث سر این است که کدام یکی زودتر قطار
فروش دوباره و سه باره و هزار باره همان محصول را راه خواهند انداخت و
دیگری حق نخواهد داشت حتی کارش را به کسی عرضه کند؟ مساله این نیست که
فقط از مصرف کنندگان حمایت نمی‌شود، جریان این است که \dbquote{دارایی
  معنوی}، از دیگر افراد خلاق هم حمایت نمی‌کند و جلوی بروز خلاقیت آنان
را می‌گیرد.

چیزی که بحث را زشت‌تر هم می‌کند این است که بسیاری از مدافعین قوانین
دارایی‌های معنوی قوی‌تر، استدلال‌های خود را بر مفاهیمی مثل \dbquote{دفاع}
از حقوق هنرمندان و مختراع بنا می‌کنند. چیزی که به نظر می‌رسد در این میان
مغفول می‌ماند این واقعیت است که نتیجه دادن قدرت بیشتر به یک عده از
مردم، گرفتن قدرت از دیگران است.

و در این شرایط غیرمنتظره نخواهد بود اگر بشنوید که بیشترین شرکت‌هایی که
استدلال‌های شان به سمت تقویت قوانین کپی‌رایت است، دقیقا همان‌هایی هستند
که بیشترین نفع را از این قوانین می‌برند. این قوانین توسط هنرمندان یا
مخترعین تقویت نمی‌شود، بلکه مدافع این قوانین شرکت‌هایی هستند که زندگی‌شان
به خلاقیت افراد دیگر وابسته است. و البته وکلا را هم نباید فراموش
کرد. نتیجه؟ قوانینی مشابه قانون نه چندان مشهور کپی‌رایت هزاره
دیجیتال\RFootnote{\lr{Digital Millennium Copyright Act (DMCA)} - که
  البته این روزها بسیار مشهور بوده و بسیاری ویدئوهای یوتیوب با استناد
  به آن حذف می‌شوند.} که آخرین بقایای حقوق مصرف کننده کالاهای کپی‌رایت
دار را از ایشان گرفت.

اگر حالا به این نتیجه رسیده‌اید که کپی‌رایت از نظر من چیز نامناسبی‌ است،
باید بگویم که در اشتباهید. من گاهی واقعا عاشق کپی‌رایت هستم و تنها
مشکلم هم این است که نباید روی حقوق نویسنده بیش از حد تاکید شود. قرار
نیست ترتیب مصرف کننده را بدهیم. من این را نه به عنوان یک مصرف کننده که
به عنوان تولید کننده یک کالای کپی‌رایت‌دار می‌گویم. چه در مورد این کتاب و
چه در مورد لینوکس.

من به عنوان فرد دارای کپی‌رایت، حقوق خودم را محفوظ می‌دانم ولی با هر
حقی، التزاماتی هم همراه می‌شود. من ملزم می‌شوم که از حقوقم استفاده
منصفانه بکنم نه اینکه از آن به عنوان اسلحه‌ای علیه کسانی که این حقوق را
ندارند، بهره ببرم. همان طور که یک آمریکایی بزرگ یک جایی گفته
\dbquote{نپرسید که کپی‌رایت چه کاری برای من کرده بلکه بپرسید که شما
  چکاری برای کپی‌رایت می‌توانید بکنید} یا یک همچین چیزی.

و در نهایت باید بگویم که حتی با وجود مواردی مثل کپی رایت هزاره
دیجیتال، کپی‌رایت هنوز شکلی نسبتا معتدل از دارایی معنوی است. مفهوم
\textbf{استفاده منصفانه}\RFootnote{\lr{Fair Use} - به این معنا که
  افراد باید حق داشته باشند از محصولات دارای کپی رایت با نیت‌های درست
  به شکلی آزاد استفاده کنند.} هنوز معتبر است و داشتن کپی‌رایت چیزی به
معنی مالکیت تمامی حقوق مربوط به اثر توسط مولف نیست.

اما در مورد حق اختراعها، علایم تجاری و اسرار تجاری دیگر نمی‌شود همین
نظر را داشت. در دنیای دارایی‌های معنوی، این‌ها مواد مخدر سنگین
هستند. بحث درباره حق اختراع نرم‌افزار آن قدر در شرکت‌ها باعث دعوا شده که
حرف زدن در این مورد در حلقه‌های فنی رسما حرکتی تحریک آمیز و دور از ادب
به حساب می‌آید،‌ درست مثل حرف زدن در مورد مالکیت اسلحه، سقط جنین، مصرف
حشیش و بهتر بودن مزه پپسی از کوکاکولا. دلیلش هم این است که حق اختراع
ها در بسیاری از جهات کنترل کامل بر ابداعات را به مالکان حق اختراع
می‌دهند. بدون اینکه جنبه‌های مثبت کپی‌رایت را حفظ کنند.

یکی از بدترین جنبه‌های حق اختراع در مقابل کپی‌رایت این است که شما با
ابداع یک چیز جدید، به خودی خود صاحب حق اختراع آن نمی‌شوید بلکه باید
تمامی مسیر دردناک و پیچیده و دشوار برای درخواست یک حق اختراع را در دفتر
مخصوص به این کار طی کنید. درخواست برای یک حق اختراع درست مثل ایستادن
در صف دریافت گواهینامه رانندگی است، با این اختلاف که باید به همراه
حداقل دوازده وکیل متخصص حق اختراع در صف بایستید و صف هم در حدود دو سال
تمام طول می‌کشد! خلاصه‌اش این است که دریافت یک حق اختراع کاری نیست که
عصر چهارشنبه که بچه‌ها خوابند، بروید و انجامش بدهید.

برای درک عمق فاجعه این را هم بگویم که گاهی اداره ثبت حق اختراع، ممکن
است منابع لازم برای بررسی اینکه آیا اختراع شما واقعا یک اختراع است را
هم نداشته باشد. آن‌ها انیشتین را استخدام نکرده‌اند\RFootnote{لینوس در
  پاورقی می‌نویسد \dbquote{البته واقعیت این است که انیشتین در دوره ای
    که مشغول کار روی نظریه نسبیت خاص بود، در اداره ثبت اختراع هم کار
    می کرد ولی این یک استثناء بود و بیشتر کارمندان آن اداره هم به این
    امر واقفند.}} تا اختراعات را بررسی کند و در نتیجه بررسی اختراعات
جدید، معمولا به درستی انجام نمی‌شود. منظورم این است که در بسیاری از
مواقع اختراعات مشکل‌دار و ناتمام هم ثبت می‌شوند. این اداره فرق زیادی با
یک پست‌خانه که کارمندانش همگی دکترا داشته باشند ندارد.

نتیجه چیست؟ به دلایلی کاملا مشخص، افراد خیلی کمی صاحب حق اختراع چیزی
هستند. این شرکت‌ها هستند که هزاران اختراع را تصاحب می‌کنند و وقتی شرکتی
آن‌ها را تهدید می‌کند که به خاطر نقض یکی از موارد ثبت اختراع متعلق به آن
شرکت در فلان کالا، شکایت خواهد کرد، با تهدیدی مشابه، جوابش را
می‌دهند. دنیای ثبت اختراع، این روزها تفاوت چندانی با جنگ سرد سابق ندارد
با این اختلاف که این بار سلاح اتمی، جایش را به دارایی معنوی داده است و
این موضوع، چیزی از ترسناکی جنگ کم نمی‌کند. مردمی که این بار باید در
پناهگاه‌های زیرزمینی مخفی شوند، مخترعین مستقلی هستند که از یک طرف با یک
سیستم دیوانه طرفند و از طرف مقابل پول لازم برای استخدام ۱۲۰۰۰ وکیل
برای دفاع از خود را ندارند.

اگر فکر می‌کنید همه چیز را دیده‌اید، وقت آن شده تا با مواد مخدر قوی‌تری
از دنیای دارایی‌های معنوی آشنا شوید: اسرار تجاری. مزیت \dbquote{اسرار
  تجاری} نسبت به انواع قبلی در این است که دیگر نه چیزی به نام دفتر
اسرار تجاری وجود دارد و نه هیچ برگه‌ای که لزومی به پر کردن آن
باشد. برای اضافه کردن یک پروژه به \dbquote{اسرار تجاری} کافی است یک
برچسب \dbquote{سرّی} به آن بزنید و به بقیه هم همین را بگویید. البته
می‌توانید درباره چیستی پروژه به هرکسی که دوست دارید توضیح بدهید ولی
باید ذکر کنید که این حرف‌ها سری هستند.

این کاری است که مردم همیشه کرده‌اند و احتمالا دلیل پیدایش حقوق مربوط به
ثبت اختراع هم همین بوده است. قوانین ثبت اختراع به وجود آمده‌اند تا
افراد و شرکت‌ها را به افشای اسرار تجاری ترغیب کنند و به آن‌ها این تضمین
را بدهند که حتی در صورت افشای اسرار موفقیت تجاری، بازار از دست آن‌ها
خارج نخواهد شد. یک جور این-درمقابل-آن؛ شما اعلام می‌کنید که راز موفقیت
تان چه بوده است و قوانین تضمین می‌کنند تا فلان سال، بازار در انحصار شما
باشد.

پیش از دوران ثبت اختراع، مردم و شرکت‌ها از اسرار تجاری‌شان با چنگ و
دندان دفاع می‌کردند و گاهی حتی این مخفی‌کاری به جایی می‌رسید که آن‌ها را
با خود به گور می‌بردند. شکی نیست افشا نشدن تکنولوژی‌های پیشرفته برای
همگان، شدیدا به ضرر روند تکاملی تکنولوژی خواهد بود. وعده حقوق انحصاری،
ثبت اختراع را به مشوقی قدرتمند تبدیل کرده تا آدم‌ها بدانند که در صورت
افشای اسرار تجاری، توان رقابتی خود را در مقابل رقبا از دست نخواهند داد
و در نتیجه با اطمینان خاطر بیشتری پیشرفت‌های تکنولوژیک خود را علنی
کنند.

به هرحال این ماجرا مربوط به آن روزها است و حالا دورانی گذشته و ما در
این روزها زندگی می‌کنیم. این روزها بنا به دلایلی ژرف، حتی اسرار تجاری
هم توسط قانون حمایت می‌شوند. هر عقل سالمی درک می‌کند که وقتی رازی علنی
شد، دیگر یک راز نیست. تنها در راهروهای طویل و پیچ در پیچ قوانین مربوط
به دارایی‌های معنوی است که یک راز می‌تواند حتی بعد از اینکه همگان از آن
مطلع شدند، کماکان یک راز باقی بماند. در این راهروها اگر برای کارفرمای
ناجوری کار کنید، حتی دانشی که در مغز شما است می‌تواند موجبی شود برای
شکایت از شما. بعضی از قوانین مربوط به دارایی‌های معنوی، واقعا ترسناک
شده‌اند.

در نگاهی وسیع تر، بازمتن جنبش صلح است. جنبش صلحی برای پایان دادن به
جنگ طولانی دارایی‌های معنوی. در حالی که بسیاری از مردم دیدگاه‌های خود را
در مورد بازمتن و کاری که قرار است انجام دهند دارند، در اکثر آن‌ها
می‌توان این نقطه اشتراک را دید که بازمتن، جنبشی با تکنولوژی بالا است
برای برقراری آرامش و خنثی کردن سلاح کپی‌رایت در جنگ دارایی‌های معنوی.

بازمتن می‌خواهد از سلاح کپی‌رایت استفاده جدیدی بکند. قرار است این بار
کپی‌رایت که تا دیروز سلاحی بود علیه مردم، تبدیل شود به کارت دعوتی از
مردم برای پیوستن به تفریح دیگران. همان مانترای\RFootnote{\lr{Mantra} -
  مانترا به دعاهای ادیان شرقی گفته می‌شود. عبارت‌هایی که با تکرار بسیار
  باعث ایجاد تغییراتی می‌شوند.} قدیمی: عشق بورزید و جنگ نکنید (البته در
سطحی انتزاعی و احتمالا با توجه به بعضی گیک‌هایی که من می‌شناسم، بسیار
انتزاعی).

البته مثل هر عقیده فلسفی دیگری، دیدگاه مقابلی هم موجود است و این همان
جایی که است من یکبار دیگر می‌توانم از پزشگ گواهی رسمی بگیرم که شیزوفرنی
دارم.

تا الان سعی کردم توضیح بدهم که چرا بسیاری از مردم معتقدند که دارایی
معنوی و بخصوص قدرت‌مندتر شدن قوانین مربوط به آن، بد است. خیلی از افراد
جامعه بازمتن (و صادقانه بگویم که حتی افراد بیرون از آن) هستند که
معتقدند به این دلایل باید کل سلاح‌های اتمی را نابود کرد و با برانداختن
قوانین انحصاری، به کل جنگ سرد خاتمه داد. بقیه مخالفند.

دیدگاه مقابل این است که بله، ممکن است دارایی معنوی ناعادلانه باشد و
بله، قوانین مربوط به آن هم به نفع شرکت‌های بزرگ هستند و منافع مصرف
کنندگان را نادیده می‌گیرند، ولی هر چه باشد این شیوه تا امروز که نافع
بوده! این قوانین قدرت را در دستان قدرتمندان متمرکز می‌کنند و دقیقا به
همین دلیل که سلاح قوی‌ای هستند، باعث پیشرفت بازار می‌شوند. مشخصا همان
روابطی که باعث می‌شدند سلاح‌های هسته‌ای قدرت نهایی در جنگ سرد باشند،
اینجا هم باعث جذابیت قوانین دارایی‌ معنوی در جنگ تکنولوژی شده‌اند. و در
تکنولوژی پول هست.

 و حلقه پسخوردی هم که به وجود می‌آید، بسیار قوی است. از آنجایی که
 دارایی معنوی چیز خوبی برای پول درآوردن است، پول زیادی هم صرف تولید
 دارایی معنوی بیشتر خواهد شد. این واقعیت بسیار مهم است و تقریبا همان
 چیزی است که در تاریخ هم باعث شده جنگ‌ها منشاء اختراع و جهش‌های مهندسی
 باشند (خود کامپیوتر هم در ابتدا با مقاصد صرفا نظامی به وجود آمد). جنگ
 مجازی حقوق دارایی‌ معنوی، باعث شده آن قدر منابع صرف توسعه تکنولوژی شود
 که پیش از این هیچ گاه سابقه نداشته. این چیز خوبی است.

معلوم است که من منطقا معتقدم که اختصاص منابع به یک موضوع باعث پیشرفت
آن نخواهد شد. برای مثال به صنعت موسیقی نگاه کنید. سالی کاجیلیون دلار
صرف این می‌شود که استعدادهای آینده را کشف کنند و هیچ کس متقاعد نشده که
اسپایس گرلز (که به دلیل هنرشان به میزان کافی تحسین شده‌اند) قابل مقایسه
با ولفگانگ آمادئوس موزارت (که در فقر مرد) است. پس شکی نیست که پول ریختن
به پای یک مساله باعث ظهور نوابغ نخواهد شد.

اما این نظریه که پول نابغه نمی‌سازد در مدل‌های بلند مدت صنعتی کارایی
چندانی ندارد. نبوغ آن قدر غیرقابل پیش‌بینی تقسیم شده و یافتنش آن قدر
مشکل است که برنامه‌ریزی بلند مدتی که منحصرا مبتنی بر کشف و جذب نوابغ
باشد، به جایی نخواهد رسید. توسعه تکنولوژیک (و متاسفانه موسیقی)، این
روزها نه مبتنی بر انیشتین‌ها (و موزارت‌ها) که وابسته به لشکر عظیمی از
مهندسین زحمتکش (و در مورد موسیقی، دختران جوان) است که حداکثر ممکن است
گاه گاه جرقه‌ای از خلاقیت بروز بدهند. منابع بیشتر، باعث بروز هنر والا
نخواهد شد، اما رشد آرام و مستمر را تضمین خواهد کرد. در نهایت هم این
بهتر است.

شاید مفهوم لشکر مهندسین زحمتکش، بار رمانتیک و کشش بسیار کمتری نسبت به
یک استعداد خارق‌العاده داشته باشد. فقط کافی است تعداد فیلم‌هایی که در
مورد \dbquote{دانشمند دیوانه} دیده‌اید را با آن‌هایی که در مورد
\dbquote{لشکر مهندسین زحمتکش} ساخته‌ شده‌اند، مقایسه کنید. وقتی صحبت از
کسب و کار است، احتمالا همه به دنبال جرقه‌های خلاقیت هستند اما چیزی که
بیشتر مورد توجه است، پیشرفت‌های کوچک اما مستمر در طول زمان است.

اینجاست که نور دارایی معنوی، می‌درخشد: دارایی معنوی بالیده و تا به
امروز موفق بوده است تا مانند جام مقدس تکنولوژی مدرن، به این ماشین بزرگ
سوخت برساند. به لطف دارایی معنوی، ماشین بزرگ تکنولوژی تا امروز بدون
اختلال به رشد آرام خود ادامه داده است. تکنولوژی دیگر شاهد جهش‌های عظیم
نیست، اما رشد آن کاملا قابل اتکا است.

پس من هر دو طرف را می‌بینم. البته باید اعتراف کنم که در اکثر مواقع
ترجیح می‌دهم فقط طرف مفرح و خلاقانه دنیای تکنولوژی را نظاره‌گر باشم؛
دنیایی که در آن عوامل اقتصادی همیشه تعیین کننده نیستند. من رویایی
دارم؛ رویای من روزی است که قوانین دارایی معنوی بر مبنای اخلاقیات نوشته
شوند و نه در این مورد که چه کسی قرار است سهم بزرگتری از کیک را تصاحب
کند.

به من اعتماد کنید. من از اقتصاد سر در می‌آورم ولی در عین حال نمی‌توانم
آرزو نکنم که اقتصاد چنین تاثیر منفی‌ای بر قوانین دارایی‌های معنوی مرتبط
با تکنولوژی‌های نوین نداشته باشد. مشوق‌های اقتصادی که در پی تقویت قوانین
دارایی‌های معنوی می‌آیند و ناتوانی ما از استفاده از عباراتی مثل
\dbquote{استفاده منصفانه} یا \dbquote{خلاق} در متون رسمی باعث شده که
این دو دیدگاه مرتبط با دارایی معنوی، این قدر جدا از هم رشد کنند. درست
مثل دعوای دو همسایه، اینجا هم هیچ یک از طرفین حاضر نیستند قبول کنند که
جواب صحیح احتمالا جایی در وسط این دو حد نهایی، قرار دارد.

همان طور که تصویب متاسف کننده قانون کپی‌رایت هزاره دیجیتال نیز نشان
داد، مشوق‌های اقتصادی به خوبی کار می‌کنند. سوال این است که چه نوعی از
قانون دارایی معنوی می‌تواند رشد تکنولوژی را تضمین کند بدون اینکه آن را
به طور کامل زیر نظر منافع خام مادی درآورد.

مساله وقتی جدی‌تر خواهد شد که بدانیم تکنولوژی مدرن (و بخصوص اینترنت) در
حال تضعیف اشکال قدیمی حفاظت از دارایی‌های معنوی هستند و این روند آن قدر
سریع در حال رخ دادن است که ما از آن عقب مانده‌ایم و کسی هم توان پیش‌بینی
آن را نداشته است. چه کسی تصور می‌کرد که مادربزرگ‌های میانه غربی آمریکا،
روزی دستورات سوزن‌دوزی را به شکل غیرقانونی از طریق اینترنت به اشتراک
بگذارند؟ کپی آثار هنری - و خود تکنولوژی - در مقایس بالا آن قدر همه‌گیر
و آسان شده است که این روزها شرکت‌هایی که منافع شان در دارایی‌های معنوی
است، هراسان در جستجوی هر راهی هستند که بتوان از طریق آن جلوی این کار
را گرفت. آن‌ها همه تلاش شان را می‌کنند تا کپی آثار را ممنوع کنند یا حتی
در صورت امکان، تکنولوژی‌ای که این کپی را ممکن می کند هم غیرقانونی اعلام
کنند.

این تصویر چه مشکلی دارد؟ مشکل اینجاست که وقتی همه تلاش معطوف به این
می‌شود که جلوی استفاده غیرقانونی از یک محصول گرفته شود، استفاده قانونی
از آن‌ هم سخت‌تر می‌شود. نمونه مشهور این امر در دنیای لینوکس، دعوای
قانونی مشهور به \lr{DeCSS} است.

در مورد \lr{DeCSS} شرکت‌های سرگرمی از افرادی که به دنبال باز کردن کد
دی.وی.دی.ها به منظور به اشتراک گذاشتن این کد روی اینترنت بودند، شکایت
کردند. برای قاضی مهم نبود که هدف نهایی این افراد قانونی است. او رای
داد که چون محصول پروژه قابلیت استفاده غیرقانونی را دارد، حتی اشاره به
اینکه از کجا می‌توان کدها را پیدا کرد هم در آمریکا غیرقانونی است (نام
\lr{DeCSS} از ترکیب پیشوند \lr{De} به معنی \dbquote{برعکس} یا
\dbquote{بازکردن} و \lr{CSS} ساخته شده بود که مخفف سیستم مخفی‌سازی
محتوا\LFootnote{Content Scrambling System} است).

این نمونه عالی‌ای است از کاربرد قوانین دارایی معنوی نه برای رشد خلاقیت
که به منظور کنترل بازار و محدود کردن آنچه مصرف کننده حق دارد یا حق
ندارد انجام دهد. نمونه‌ای از حرکت اشتباه قوانین دارایی معنوی.

به هرحال این استفاده اشتباه از قدرت دارایی معنوی محدود به موارد
تکنولوژیک هم نیست. یک مثال کلاسیک دیگر مربوط است به استفاده از قوانین
اسرار تجاری برای تعقیب و متوقف کردن کسانی که سعی کردند عموم مردم را
نسبت به کلیسای ساینتولوژی\RFootnote{\lr{Scientology} - دینی که مدعی
  می‌شود که بر اساس منطق و علم بنا شده ولی در عمل چون یارای دفاع علمی
  از عقایدش را ندارد به سرکوب مخالفان از طرق قانونی و گاهی غیرقانونی
  رو آورده.} آگاه کنند. کلیسای علم‌شناسی با موفقیت کتاب مقدس خود
(\dbquote{تکنولوژی پیشرفته}) را به عنوان یک سِرّ تجاری ثبت کرده بود و با
استفاده از قوانین دارایی معنوی، جلوی عمومی شدن این کتاب را گرفت.

اما شق دیگر چیست؟ به این فکر کنید که یک قانون دارایی معنوی بیاید که
حقوق دیگران را هم ذکر کند. به این فکر کنید که قوانین دارایی معنوی،
ممکن است باز بودن یا به اشتراک گذاشتن را تشویق کنند. مثلا می‌توانید
قانونی را در نظر بگیرید که بگوید شما می‌توانید اسرار خود را داشته باشید
- چه فنی و چه دینی - ولی این قانون، تضمینی حقوقی برای مخفی ماندن آن سر
نباشد.

بعله، می‌دانم. گاهی غیر واقع‌بین می‌شوم.

\section{پایانی بر کنترل}
راه ماندن و رشد کردن، این است که بهترین محصولی که می‌توانید را
بسازید. اگر با این کار نماندید و رشد نکردید، بدانید که دلیلی برای
ماندتان وجود نداشته. اگر نتوانید خودروی خوبی بسازید، مثل صنایع
اتومبیل‌سازی آمریکا در دهه ۱۹۷۰، سقوط خواهید کرد. موفقیت نتیجه کیفیت
است و اینکه به آدم‌ها چیزی را بدهید که می‌خواهند.

موفقیت نتیجه تلاش برای کنترل مردم نیست.

مشکل اینجاست که در بسیاری از مواقع، انگیزه اصلی افراد و شرکت‌ها حرص و
طمع است. این موضوع در طولانی مدت سبب شکست خواهد شد. طمع باعث تصمیمات
دیوانه‌وار و تلاش برای کنترل کامل بر دیگران خواهد شد. اینها بد هستند و
اتفاقا همان چیزهایی هستند که در طولانی مدت سبب فاجعه یا فاجعه‌های کوچک
خواهند شد. مثالی که در ذهن همه هست، موفقیت سریع تکنولوژی تلفن‌های همراه
در اروپا، در مقابل رقیبان آمریکایی شان است. شرکت‌های آمریکایی هر یک به
تنهایی تلاش می‌کردند تا با کنترل بازار، محصولات خاص خود را به فروش
برسانند، در حالی که اروپایی‌ها روی یک استاندارد – جی.اس.ام. - توافق
کردند و رقابت را تبدیل کردند به اینکه چه کسی می‌تواند محصول استاندارد
بهتری بسازد و خدمات بهتری ارائه دهد. شرکت‌های آمریکایی که دچار رقابت
بیمارگونه‌شان بودند، خیلی زود در این مسابقه عقب ماندند. در مقابل،
شرکت‌های اروپایی که استانداردهای یکدیگر را پذیرفته بودند، همگی از کشش
بازار استفاده کردند و به سود مورد نظر رسیدند. به همین دلیل است که
بچه‌های پراگ\RFootnote{پایتخت جمهوری چک} سال‌ها قبل از آن که بچه‌های
پئوریا\RFootnote{شهری در بخش مرکزی ایلینویز آمریکا} خبردار شوند که
می‌شود با موبایل سر جلسه امتحان تقلب کرد، داشتند با پیامک برای هم جک
فوروارد می‌کردند.

اگر سعی کنید با کنترل کردن منابع، پول در بیاورید باید بدانید که به
زودی از بازار عقب خواهید ماند. این امری جبری است و تاریخ هم پر است از
نمونه‌های آن. سال‌های ۱۸۰۰ آمریکا را در نظر بگیرید. در غرب هستید و منابع
آب کشاورزان محلی را کنترل می‌کنید. خسیس هستید و برای آب پول زیادی طلب
می‌کنید. کار به جایی خواهد رسید که یک نفر نفع مادی‌اش را در این خواهد
دید که راهی اختراع کند تا آب را از جایی دورتر ولی ارزان تر از شما به
کشاورزان برساند. شما ورشکست خواهید شد. یا سیستم لوله‌کشی مدرن اختراع
می‌شود و آب را می‌توان از هر جایی به هرجایی رساند. در هر صورت انحصار شما
خواهد شکست و هیچ چیز مفیدی در دستان شما باقی نخواهد ماند. این جریان
همیشه اتفاق افتاده و واقعا عجیب است که بعضی‌ها هنوز آن را نمی‌بینند.

یکی دو قرنی جلوتر بیایید و به سال‌های پایانی قرن بیستم و صنعت موسیقی
نگاه کنید. منبعی که این بار تحت کنترل درآمده، تفریح است. یک شرکت
مالکیت حقوق مربوط به کار یک هنرمند را در اختیار دارد. این هنرمند چند
آهنگ خوب و موفق تولید می‌کند، اما شرکتی که مالکیت آثار را در اختیار
گرفته روی هر سی دی فقط یک یا دو آهنگ خوب را قرار می‌دهد. با اینکار شرکت
می‌تواند به جای یک سی دی از منتخب بهترین آهنگ‌ها که همه به دنبال آن
هستند، چندین و چند سی دی بفروشد. حالا وقت آن است که کسی تکنولوژی‌ای مثل
\lr{MP3} را ابداع کند. به ناگهان همه می‌توانند از اینترنت موسیقی دانلود
کنند. حالا ام.پی.۳ دارد چیزی را به مردم می‌دهد که به دنبالش بوده‌اند: حق
انتخاب.

اگر قیمت یک سی دی ۱۰ دلار باشد و فقط حاوی یک یا دو آهنگ خوب باشد،
هرکسی به این فکر می‌افتد که از طریق اینترنت هر آهنگ خوب را 1.5 دلار
بخرد و یک مجموعه عالی برای خودش جمع کند. حالا مردم دیگر اسیر شرکت‌های
سرگرمی‌ای نیستند که با خست تنها حاضر بودند، قطعات کوچکی از موسیقی خوب
را روی هر سی دی عرضه کنند. حالا مردم حق انتخاب دارند. حالا این سوال که
چرا شرکت‌های موسیقی از تکنولوژی ام.پی.۳. و تکنولوژی‌های خواهرش مثل تورنت
و نپستر تا حد مرگ می‌ترسند، به سادگی قابل پاسخ دادن است. وضعیت امروز ما
درست مثل سال‌های ۱۸۰۰ است که قیمت آب آن قدر بالا رفته بود که یک نفر به
فکر اختراع ابزاری افتاد که از طریق آن بتوان آب را از نقاط دور به هر
جایی منتقل کرد.

اما حریف ما صنعتی است با تاریخی از تلاش برای کنترل مصرف کنندگان؛ آن هم
نه فقط کنترل آن‌ها از طریق انتشار آهنگ‌های خاص، که از طریق کپی‌رایت و
تکنولوژی. این همان صنعتی است که در دهه ۱۹۶۰ برای چندین سال تلاش کرد
حتی بعد از معرفی نوارهای کاست، جلوی کپی کردن موسیقی روی آن توسط مصرف
کنندگان را بگیرد. این صنعت فکر می‌کرد که نوار کاست رسانه مناسبی برای
کپی‌برداری غیرقانونی از موسیقی است و به همین دلیل به روش‌های مختلفی سعی
کرد تا از کپی‌رایت خودش محافظت کند. این بهانه بدی بود. دستاویز قرار
دادن دارایی معنوی و صحبت از قواعد اخلاقی فقط و فقط بستری بودند برای
حفظ کنترل شرکت‌های موسیقی بر صنعت و سود حاصل از آن. واقعیت این است که
نوار کاست هیچگاه به صنعت موسیقی صدمه‌ای نزد. معلوم است که مردم موسیقی
را برای استفاده شخصی از صفحه‌های گرامافون روی نوارهای کاست جدید کپی
می‌کردند ولی این فقط به معنای خرید صفحات بیشتر بود به منظور کپی کردن
آن‌ها. چند دهه بعد، سی‌دی به بازار آمد و پخش‌ کننده‌های آن جوری تنظیم شدند
که نتوان به راحتی از آن روی نوار کاست کپی گرفت. دوباره وحشت همه جا را
فرا گرفت. بعد کاست‌های دیجیتال آمدند که از نمونه‌گیری متفاوتی استفاده
می‌کردند - ۴۸ کیلوهرتز به جای ۴۴.۱ کیلوهرتز - تا جلوی کاربرانی که
می‌خواستند سی‌دی‌هایشان را روی کاست‌های دیجیتال کپی کنند گرفته شود. دوباره
شرکت‌ها داشتند سعی می‌کردند با بستن دست و پای کاربران، کنترل خود را بر
صنعت حفظ کنند.

صنعت موسیقی از طریق تلاش برای کنترل هر تکنولوژی موفق جدید، تنها به
افراد انگیزه داده است که شیوه‌های جدیدی برای گذر از این محدودیت‌ها
بیابند. واقعا آن‌ها نمی‌خواهند این را ببینند؟

این بحث به ناچار ما را به دی.وی.دی‌ها رهنمون می‌شود. این بار صنعت سرگرمی
ابزاری اختراع کرده بود که صدا و تصویر بسیار بهتری از وی.اچ.اس. داشت و
از آن کوچکتر و قابل استفاده‌تر هم بود. اما آن‌ها برای جلوگیری از کپی
شدن، روی آن رمز گذاشتند و بعد برای خراب‌تر کردن اوضاع، کدی مربوط به
موقعیت جغرافیایی هم به آن اضافه کردند. اگر در فرودگاه سانفرانسیسکو یک
دی.وی.دی. بخرید، احتمالا در اروپا پخش نخواهد شد. این برای شرکت‌های
سودجو خیلی جذاب بود: هی! ما می‌توانیم دی.وی.دی.ها را در اروپا گران تر
از آمریکا بفروشیم! پس باید مطمئن شویم که اروپایی ها نمی‌توانند از
آمریکا دی.وی.دی. بخرند.

آیا واقعا صنعت سرگرمی نتوانسته بود نتیجه واضح این جریان را پیش‌بینی
کند؟ قیمت آب آن قدر گران شده بود که می‌صرفید یک نفر روشی برای انتقال آب
از سرزمین‌های دور به هرکجا که لازم باشد اختراع کند.

بله، در حینی که صنعت سرگرمی طمّاعانه تلاش می‌کرد تا مردم را از طریق
تکنولوژی کنترل کند، رمز دی.وی.دی. شکسته شد - البته نه توسط کسانی که
می‌خواستند آن را کپی کنند بلکه توسط کسانی که می‌خواستند آن را روی
لینوکس‌های شان ببینند. اتفاقا اینها دقیقا افرادی بودند که می‌خواستند
دی.وی.دی. بخرند اما امکانش را نداشتند چون این دیسک‌ها روی دستگاه آن‌ها
جواب نمی‌داد. فکر می‌کنید عکس العمل شرکت‌ها چه بود؟ جلوگیری از گسترش
بازار فروش دی.وی.دی. و شکایت از کسانی که با شکستن قفل دی.وی.دی.ها باعث
به وجود امکان پخش آن‌ها روی لینوکس شده بودند.

یک بار دیگر ثابت شد که استراتژی‌های کوتاه مدت به ضرر منافع بلند مدت عمل
می‌کنند.

صنعت سرگرمی فقط یک مثال است. مساله مشابهی سال‌هاست که در صنعت نرم‌افزار
هم در جریان است. به همین دلیل است که استراتژی بسته بندی نرم‌افزار
مایکروسافت و مجبور کردن کاربر به استفاده از یک مجموعه بسته بندی شده،
محکوم به شکست است. اما این موضوع در لینوکس اتفاق نمی‌افتد چون اگر یک
نفر شما را مجبور کند که حتما لینوکسش را با فلان نرم‌افزار همراه آن
استفاده کنید، یک نفر دیگر نرم‌افزار دوست نداشتنی را از آن حذف خواهد کرد
و دوباره به شما اجازه انتخاب خواهد داد. با این کار شما خواهید توانست
دقیقا از چیزی استفاده کنید که به آن نیاز دارید و نه از چیزی که فروشنده
شما را مجبور به خرید آن کرده.

واقعا بیهوده است که تلاش کنیم مردم را با استفاده از تکنولوژی کنترل
کنیم. این کار نه فقط به شرکت اجبار کننده صدمه خواهد زد، که حتی روند
استفاده از آن تکنولوژی را هم کُند خواهد کرد. نمونه اخیر، جاوا است که به
شدت جذابیت روزهای آغازین خود را از دست داده است. سان میکروسیستمز با
تلاش برای کنترل محیط جاوا، باعث این شکست شده. جاوا هنوز خوب پیش می‌رود
ولی شکی نیست آن قدر که باید، سریع رشد نکرده.

سان سعی نکرد تا از خود جاوا پول در بیاورد اما این شرکت از این زبان به عنوان ابزاری برای متمایز کردن کامپیوترهایش استفاده کرد و در عین حال سعی کرد جاوا را راهی معرفی کند برای خارج کردن ما از چنگال مایکروسافت و البته فروختن سخت‌افزارهای بیشتر. سان با اینکه سعی نمی‌کرد جاوا را به منبع درآمد تبدیل کند، اما می‌خواست آن را تحت کنترل خود نگه دارد و به همین دلیل تمام مجوزهایی که برای استفاده از جاوا ارائه می‌کرد به شدت محدود کننده بود. 

جاوا محصول خوبی است اما مشکل اینجاست که سان بیش از حد سعی می‌کند با
مایکروسافت مقابله کند. انگیزه آن‌ها ترس، بیزاری و نفرت است، آن هم از
نوعی که در دهه ۱۹۹۰ در صنعت شایع شده بود. به دلیل همین ترس از
مایکروسافت و همین نفرت نسبت به آن، آن‌ها انتخاب‌های صحیحی برای مجوزها
نکردند. استفاده از این محصول برای همه و حتی برای همکاران سان هم مشکل
بود. به همین دلیل است که شرکت‌هایی مثل هیولیت پاکارد\RFootnote{همان
  شرکت \lr{HP}} و آی.بی.ام. در نهایت دست به کار توسعه جاواهای خود
شدند. آن‌ها هم فقط به دنبال این بودند که با سان مقابله کنند.

سان دو بار تلاش کرد تا از طریق دو سازمان استانداردسازی مختلف، جاوا را
استاندارد کند. ولی هر دوبار به دلیل مسایلی که به از دست دادن کنترلش
مربوط می‌شد،‌ پا پس کشید. سان از یک طرف می‌خواست این زبان را استاندارد
کند، ولی در عین حال نمی‌خواست کنترلش را بر آن از دست بدهد. حرف
سازمان‌های استانداردساز هم این بود که \dbquote{اگر می‌خواهی استانداردش
  کنیم، نمی‌توانی همه چیز را خودت دیکته کنی} و همین شد که سان جریان را
متوقف کرد. این مثالی از شرکتی است که سعی می‌کند تکنولوژی را به شکلی
کنترل کند که از نظر مصرف کننده بی‌معنا است. چنین شرکتی همیشه شکست خواهد
خورد. این جریان باعث می‌شود تکنولوژی هم شکست بخورد - یا پذیرش آن زمان
بیشتری ببرد.

این را در مقابل استراتژی \dbquote{اگر چیزی را دوست داری آزادش کن}
شرکت‌هایی مثل پالم کامپیوتینگ\RFootnote{\lr{Palm Computing} - یکی از
  اولین سازندگان کامپیوترهای دستی قابل حمل؛ چیزهایی شبیه به تلفن‌های
  همراه امروزی} قرار دهید. دوستان پالم، محیط توسعه نرم‌افزار مخصوص
کامپیوترهای دستی خود را آزاد کردند و آن‌هم نه فقط برای شرکت‌های همکار که
حتی برای افراد مستقلی که علاقمند بودند برای این کامپیوترها برنامه
بنویسند. باز اعلام کردن \lr{API}ها باعث شد که افراد و شرکت‌های علاقمند
بتوانند به راحتی به ابزار برنامه نویسی روی پالم دسترسی پیدا
کنند. نتیجه این کار تشکیل یک حلقه برنامه نویسی به دور کامپیوترهای پالم
بود. این کار باعث شد پالم به مفهومی فراتر از یک شرکت که در بازاری جدید
به دنبال کسب سهم است تبدیل شود. حالا شرکت‌هایی بودند که به شکل اختصاصی
برای سخت‌افزار پالم بازی می‌فروختند یا برنامه تقویمی مفصل‌تر از آن چیزی
که خود پالم نوشته بود به مشتری عرضه می‌کردند. مشتریان حالا حق داشتند در
مورد چیزی که دوست دارند استفاده کنند، تصمیم بگیرند و همه راضی
بودند. از جمله پالم که به خاطر باز اعلام کردن محیط برنامه‌نویسی‌اش به
بازار بزرگی دستی یافته بود.

هندسپرینگ\LFootnote{Handspring}هم تجربه مشابهی در مورد دستگاهش به نام
ویزور\LFootnote{Visor} داشت. این دستگاه رقیب پالم بود و از سیستم‌عامل
پالم استفاده می‌کرد و حالا شرکت تصمیم گرفته بود در باز کردن دستگاهش یک
قدم هم جلوتر برود و به شرکت‌های سخت‌افزاری هم اجازه بدهد که برای ویزور،
لوازم جانبی مانند جی.پی.اس. یا گیرنده‌های موبایل بسازند. مانند تجربه
پالم، در اطراف ویزور هم حلقه‌ای ایجاد شد از برنامه‌نویسان و سازندگان
سخت‌افزارهای جدید که هر روز سعی می‌کردند امکانات جدیدی به این دستگاه
اضافه کنند و همه راضی بودند.

کاری که سان می‌توانست بکند این بود که به همه اجازه بدهد جاوای خودشان را
داشته باشند - بدون اینکه مجبور باشند زیر چتر سان بروند - و در عوض خودش
سعی کند که بهترین کار را ارائه دهد. این نشانه شرکتی می‌بود که از رقابت
نمی‌ترسد و به خاطر طمع، کور نشده است. این نشانه شرکتی می‌بود که به خودش
اعتماد دارد و وقتش را برای دشمنی با این و‌ آن هدر نمی‌دهد.

\section{راه جذاب پیش رو}
آیا واقعا در جهان چیزی منزجر کننده تر از کسی هست که سعی می‌کند صنعت را
پیشگویی کند؟ منظورم افراد خود بزرگ بینی هستند که سعی می‌کنند در این
باره که قطار تفریحی تکنولوژی قرار است ما را به کجا ببرد، پُرچانگی
می‌کنند. البته بر این باور هستم که این افراد مشغول کار مهمی هستند. آن‌ها
در جلسات بحث شرکت می‌کنند و سخنرانی‌های افتتاحیه کنفرانس‌ها را برگزار
می‌کنند. منظورم جلسات و کنفرانس‌هایی هستند که مثل قارچ در حال رشد
هستند. افرادی که می‌خواهند پول‌های کلانی صرف تکنولوژی کنند، هزاران دلار
خرج می‌کنند تا بتوانند با حضور در جلسات و کنفرانس‌ها، نظرات این افراد را
بشنوند. این جریان باعث می‌شود کار هتل‌ها و آشپزها و پیش‌خدمت‌ها رونق پیدا
کند و به همین دلیل احساس من این است که این افراد مشغول کار مفیدی
هستند.

و حالا دیوید اصرار دارد که من هم به نوبه خودم یکی از آن فصل‌های
\dbquote{صنعت به کجا می‌رود} را بنویسم. فکر کردن به این جریان ذهنم را
مغشوش می‌کند ولی به هرحال دیوید کسی است که یک بار حین بوگی‌سواری من را
از غرق شدن نجات داده و همچنین او فکر می‌کند که خوانندگان احتمالا علاقه
بیشتری به خواندن \dbquote{صنعت به کجا می‌روند} دارند تا \dbquote{معنای
  زندگی.} حالا که این طور است دهنم را می‌بندم و می‌نویسم.

به هرحال.

برای شروع باید بگویم که تا جایی که یادم هست، هیچ وقت نتوانسته‌ام هیچ
چیز را به شکل قابل قبولی پیش‌بینی کنم. آیا توانسته بودم پیش‌بینی کنم که
سیستم‌عامل کوچکی که برای استفاده خودم نوشته بودم یک روز جهانگیر شود؟
نه. این گسترش من را هم غافلگیر کرد. البته این دفاع را دارم که بگویم
هیچ کس دیگری هم با هیچ روش پیشگویی‌ای، از این امر خبر نداده بود. موفقیت
لینوکس همان قدر که باعث شگفتی من شد، بقیه را هم مبهوت کرد. پس شاید کار
من از بقیه بهتر بوده و کسی چه می‌داند؟ شاید روزگاری به خاطر این فصل، به
نوسترآداموس صنعت شهره شوم.

شاید هم نه. خب به هرحال باید بنویسم.

مطمئنا ما می‌توانیم به تجربه گذشته‌مان نگاه کنیم. می‌توانیم نشانه‌های تلخی
را ببینیم که باعث شد شرکت شکست‌ناپذیری مانند
ای.تی.اند.تی\LFootnote{AT\&T} به زانو دربیاید و در نتیجه می‌توانیم
پیش‌بینی کنیم که در صورت گذشت زمان کافی، علف‌های هرز روزی جای ساختمان‌های
سبز و زیبای ردموند\RFootnote{\lr{Redmond} - مقر اصلی شرکت مایکروسافت}
را خواهند گرفت. درست همان طور که ستاره سینمای امروز را فردا با صورتی
پر از چین و چروک خواهیم دید، قهرمان تکنولوژی امروز هم فردا با یک مدل
جدیدتر جایگزین خواهد شد. تلاش شرکت‌ها برای اختراع مجدد ساخته قبلی‌شان یا
هر چیزی که در آینده آن را بنامند، فایده‌ای نخواهد داشت و چین و چروک را
بر صورت امثال ای.تی.اند.تی. شاهد خواهیم بود.

بهتر است اسمش را تکامل بگذاریم. بحث پیچیده‌ای نیست؛ فقط این جریان ساده
است که هیچ تجارتی تا ابد دوام نمی‌آورد.

اما موتور این تکامل چیست؟‌ آیا مثل چیزی که بعضی‌ها فکر می‌کنند، عامل این
تکامل چیزی در درون خود تکنولوژی است که روزی باعث خواهد شد تا
کامپیوترها آن قدر پیشرفت کننده که بر انسان چیره شوند و از نژاد ما چیزی
جز کمی زباله به جای نماند؟ یا شاید هم دلیل این تکامل، اختراعاتی باشند
که یکی بعد از دیگری مشغول پیش‌برد صنعت شده‌اند؟

به نظر من هیچ‌کدام. 

تکنولوژی دقیقا همان چیزی است که ما داریم از آن می‌سازیم و نه تکنولوژی و
نه مسایل مادی قادر به تغییر خواسته‌ها و کشش‌های طبیعی و بنیادین انسان
نیستند. تکامل باعث خواهد شد تا تکنولوژی هم مثل هر چیز دیگر از بقای صرف
جامعه مبتنی بر ارتباط به کاربرد مفرح بودن سوق پیدا کند (تذکر: بله این
نظریه را در اوایل کتاب هم خوانده‌اید و اگر خواندن را ادامه بدهید یک بار
دیگر هم آن را خواهید دید).

انسان‌ها، خواهی نخواهی حیوانات گروهی هستند و تکنولوژی هم به آن کمک
خواهد کرد.

پس کل چیزهایی که در مورد توانایی‌های ده سال آینده تکنولوژی شنیده‌اید را
رها کنید. مطمئنم که حرف‌های خیلی مربوطی هم نیستند. ما سی سال قبل موفق
شدیم انسان را در ماه پیاده کنیم ولی هیچ وقت این کار را تکرار
نکردیم. به نظر من دلیل این موضوع این بود که کشف کردیم ماه جای حوصله
سربری است که هیچ زندگی شبانه‌ای در آن جریان ندارد؛ جایی شبیه
سن‌جوز. نتیجه این است که مردم دوست ندارند به ماه برگردند و کل
تکنولوژی‌ای که برای اینکار مصرف شد، حرام شده است. ماه هنوز خالی است.

وقتی می‌خواهید در این باره صحبت کنید که آینده تکنولوژی دارد به کجا
می‌رود، مهمترین بخش صحبت تان باید در این مورد باشد که مردم چه چیزی دوست
دارند. همین که این را فهمیدید، تنها موضوعی که باقی می‌ماند این است که
چطور می‌توانید خیلی سریع کالای مورد علاقه مردم را در مقیاس بالا تولید
کنید و قیمتش را آن قدر پایین بیاورید که توده مردم بتوانند بدون فدا
کردن چیز دیگری که دوستش دارند، آن را به دست بیاورند. بقیه مسایل واقعا
ارزشی ندارد.

مواظب باشید چون قرار است کمی از موضوع پرت شویم. چیزی که واقعا می‌فروشد،
برداشت است نه واقعیت. مثلا وقتی یک سفر دریایی می‌خرید، شرکت‌ها برداشت
شما از آزادی، دریاهای شور و غذای خوب و کشتی عشاق را به شما می‌فروشند و
نه واقعا یک کابین شلوغ و کوچک را. چیزی که مهم است این است که شما در
کشتی مثل یک پرنده احساس آزادی می‌کنید.

همه این ها به چه معناست؟ این ها توضیح می‌دهند که برای مثال چرا مردم این
همه در مورد پلی‌استیشن ۲ سونی یعنی پیشرفته‌ترین ابزار تکنولوژی امسال
هیجان دارند، آنهم حتی پیش از اینکه به بازار بیاید (البته چند روز بعد
از اینکه این محصول به بازار عرضه شد، مشغول نوشتن این خطوط هستم). به
نظر من جامعه تفریح-محور دارد واقعیت پیدا می‌کند!

باید به مشکلی که در برداشت از کامپیوتر شخصی وجود دارد هم اشاره
کرد. واضح است که صنعت کامپیوترهای شخصی نگران کنسول‌های بازی است و دلیل
اصلی این امر هم آن است که سال‌ها نگران آن‌ها نبوده، چون فکر می‌کرده
کنسول‌های بازی ابزارهایی ارزان و کم ارزش هستند که هیچ تهدیدی برای
کامپیوترهای شخصی شیک و گران به حساب نمی‌آیند.

من شخصا متقاعد شده‌ام که اگر پانزده‌سال دیگر هنوز هم مشغول بحث در مورد
سیستم‌عامل باشیم و سیستم‌عامل هنوز چیز مهمی در کامپیوتر به حساب بیاید،
به این معنا است که یک جای کار شدیدا مشکل دارد. این حرف شاید از طرف کسی
که شهرتش را مدیون نوشتن یک سیستم‌عامل است عجیب به نظر برسد، ولی واقعیت
این است که اگر آماری صحبت کنیم، هیچ کس سیستم‌عامل را نمی‌خواهد.

در واقع کسی کامپیوتر را هم نمی‌خواهد. چیزی که اکثر ما می‌خواهیم یک
اسباب‌بازی جادویی است که بتوانیم با آن در وب گشت بزنیم، مشق‌های ترم را
بنویسیم، با آن بازی کنیم، حساب‌های مان را نگه‌داریم و این جور
کارها. تقریبا همه افراد ترجیح می‌دهند که بدون اطلاع از حضور کامپیوتر یا
سیستم‌عامل، بتوانند به این کارها برسند.

به همین دلیل است که بسیاری از تحلیل‌گران از دستگاه‌هایی مثل پلی‌استیشن ۲
سونی خوش شان می‌آید. آن‌ها می‌بینند که این دستگاه‌های ساده به سرعت در حال
فتح برخی از قلمروهای کامپیوترهای شخصی هستند، بدون اینکه کسی از مهاجرت
به آن‌ها بترسد یا نگران باشد که نکند نتواند با آن‌ها کار کند. این جریان
در حوزه‌های دیگر هم در حال اتفاق افتادن است و ما هر روز دستگاه‌های
بیشتری به خانه‌مان می‌آوریم که بدون اینکه متوجه باشیم، برای کارشان از
برنامه‌های پیچیده کامپیوتری استفاده می‌کنند.

پس کاندیدای من برای مایکروسافت آینده،‌ سونی است. البته به شرطی که
بتواند مثل چند وقت اخیر، مرتب و منظم پیشرفت کند. حالا ادعا نمی‌کنم که
این پیش‌بینی یک نوع بلاگفکر نوسترآداموسی است (بله. می‌دانم که چنین
واژه‌ای وجود ندارد ولی باید وجود داشته باشد). احتمالا افراد دیگری هستند
که با پیش‌بینی من موافقند اما من لازم می‌بینم کمی مفصل‌تر بحث کنم و بگویم
که چرا این جریان در حال رخ دادن است.

من، آن گونه که خیلی‌ها قبلا به اشتباه کرده‌اند، مرگ کامپیوترهای شخصی را
پیش‌بینی نمی‌کنم. دلیل پایه‌ای قدرت کامپیوترهای شخصی همیشه با آن‌ها خواهد
بود. این کامپیوترهای همه کاره در دنیای کامپیوتر معادل چاقوی سوییسی
هستند. این دستگاه‌ها پیچیده هستند و کسانی که از تکنولوژی خوش شان نمی‌آید
را می‌ترسانند. اتفاقا قدرت آن‌ها هم در همین است چون یک ابزار تک منظوره
نیستند که فقط برای یک کار طراحی شده باشند. قابلیت تطبیق و انعطاف‌پذیری
دقیقا همان‌ چیزی است که آن‌ها را جذاب می‌کند.

حالا می‌رسیم به نکته‌ مشترکی که کلیه این ابزارها را به هم متصل می‌کند:
ارتباطات. همه جا. شما نمی‌توانید دو ساعت را بدون چک کردن ایمیل سپری
کنید؟ مشکلی نیست معتاد عزیز. احساس گناه می‌کنید از اینکه یک روز را به
دور از کامپیوتر و در ساحل سپری کنید؟ اشکالی ندارد. حالا این امکان
فراهم شده در ساحل باشید و کماکان به اینترنت متصل بمانید. یادتان باشد:
چیزی که خوب فروش می‌رود، واقعیت نیست بلکه احساس است. احتمال اینکه شما
واقعا به کنار دریا بروید خیلی کم است اما احساس آزادی رفتن به آنجا و
قطع نشدن ارتباطات، چیزی است که به خوبی به فروش می‌رود. این جا بالاخره
اندازه مهم شده. اندازه کوچک به شما نشان می‌دهد که تکنولوژی جدید اولا
ترسناک نیست و ثانیا جزو حقوق بدیهی است.

ولی در این وسط جای لینوکس و بازمتن کجاست؟ کسی نخواهد دانست. لینوکس و
بازمتن در قلب ماشین‌های سونی خواهد بود. شما آن را نخواهید دید، متوجه
حضورش نخواهید بود، اما به هرحال در آنجا خواهد بود و همه چیز را به حرکت
در خواهد آورد. لینوکس در تلفن موبایل شما خواهد بود و این تلفن نه یک
وسیله مکالمه ساده، که مرکز ارتباطی کل وسایلی خواهد بود که در حینی که
شما از اینترنت بی‌سیم خود به دور هستید، به شبکه نیازمندند.

می‌بینید؟ فقط بحث زمان است و البته پول. 

\section{چرا بازمتن مهم است}
آی.بی.ام. شرکتی است با تاریخچه‌ای پر از دردسر برای آدم‌ها. درآمد اولیه
این شرکت نتیجه گرفتار کردن چند مشتری و کسب اطمینان از این امر بود که
هیچ شرکت دیگری نخواهد توانست جای پایش را در بازار محکم کند. در واقع
این روش اصلی پولدار شدن تقریبا تمام شرکت‌های کامپیوتری است. هنوز هم
اکثر شرکت‌ها به همین شیوه کار می‌کنند. بعد که شرکت آی.بی.ام. کامپیوتر
شخصی را ابداع کرد، تکنولوژی آن را در اختیار دیگران هم گذاشت تا از روی
آن کپی کنند. این تصمیم بیش از هر تصمیم دیگری به وقوع انقلاب
کامپیوترهای شخصی کمک کرد، که این انقلاب هم به نوبه خود باعث انقلاب
اطلاعات، انقلاب اینترنت، اقتصاد نوین - یا هر مفهومی دیگری شد که
تغییرات عظیم جهان امروز را به آن نام می‌خوانند.

این بهترین بازنمایی از منافع بی‌حد و مرز ناشی از فلسفه بازمتن است. با
اینکه کامپیوتر شخصی با پیروی از مدل بازمتن طراحی نشده بود اما مثال
بسیار خوبی است از تکنولوژی‌ای که باز اعلام شد و دیگر افراد یا شرکت‌ها
توانستند آن را کپی کنند، بهبود دهند یا بفروشند. در خالص‌ترین حالت،
بازمتن به هرکسی اجازه می‌دهد تا در توسعه یا استثمار مالی یک پروژه شریک
شود. لینوکس بدون شک موفق‌ترین مثال است. چیزی که در اتاق خواب شلوغ و
پلوغ من شروع شد، آن قدر رشد کرد که توانست به بزرگترین پروژه جمعی تاریخ
بشر تبدیل شود. این پروژه با ایدئولوژی‌ای آغاز شد که بین توسعه دهندگان
نرم‌افزار رواج داشت. بر اساس این تفکر، کد منبع نرم‌افزارهای کامپیوتری
باید برای همگان در دسترس باشد و جی.پی.ال. - مجوز ضد کپی‌رایت - ابزار
قدرتمند این جنبش است. امروزه این اندیشه رشد کرده و تبدیل به روشی برای
توسعه و بازتولید بهترین تکنولوژی ممکن شده است. موفقیت از این هم فراتر
رفته و در بازار تجاری جهان هم کاملا پذیرفته شده. شاهد این امر، پذیرش
روزافزون لینوکس به عنوان یک وب‌سرور امن و قیمت بالای سهام آن در بورس
است.

چیزی که توسط ایدئولوژی خلق شده بود، ثابت کرد که در بازار هم می‌تواند به
یک تکنولوژی موفق تبدیل شود. حالا هم که بازمتن ثابت کرده می‌تواند چیزی
بیش از یک تکنولوژی در حوزه فنی و تجارت باشد. در دانشگاه حقوق هاروارد،
پروفسور لری لسیگ\LFootnote{Larry Lessig} (که حالا در دانشگاه استانفورد
است) و چارلز نسون\LFootnote{Charles Nesson} از مدل بازمتن برای
بررسی‌های قانونی استفاده کرده‌اند. آن‌ها پروژه قانون باز را شروع کردند که
طی آن حقوق‌دانان داوطلب، متخصصین و دانشجویان می‌توانستند از طریق وب‌سایت
پروژه در بحث‌های مرتبط با قانون کپی‌رایت آمریکا مشارکت کنند و در مورد آن
پیشنهاد دهند و نظرات یکدیگر را نقد کنند. ایده این پروژه این است که با
مشارکت تعداد زیادی ذهن حقوقی، بهترین نظریات رشد خواهند کرد و طی
ارسال‌ها و جواب‌ها، کوهی از اطلاعات حقوقی جمع‌آوری خواهد شد. این سایت در
جمع‌بندی زیبایی برای تفاوت شیوه کار خود با روش‌های سنتی می‌نویسد:
\dbquote{چیزی که به خاطر محرمانه نبودن مباحثات از دست می‌رود با تعدد
  منابع و چندوجهی بودن مباحث، جبران می‌شود.} (در فضای کامپیوتری باید
بگوییم: با حضور چند میلیون چشم، تمام باگ‌های نرم‌افزاری خنثی می‌شوند.)

در مقایسه با آن چیزی که سالیان سال تحقیق دانشگاهی نام داشته، این یک
نگاه جدید است که افق‌های تازه‌ای را در برابر دیدگان ما قرار می‌دهد. مثلا
به این فکر کنید که این شیوه تحقیق تا چه حد می‌تواند به کشف درمان
بیماری‌ها سرعت ببخشد. یا مشکلات دیپلماسی بین‌المللی با حضور بهترین
ذهن‌های جهان چه قدر تقویت خواهد شد. همان طور که جهان کوچکتر می‌شود، سرعت
زندگی و صنعت سریعتر می‌شود و تکنولوژی و اطلاعات در دسترس تعداد بیشتری
از مردم قرار می‌گیرد، بیشتر و بیشتر کشف می‌کنیم که نگه داشتن همه چیز در
مشت خودمان، یک اشتباه روشی است.

نظریه پشت بازمتن، ساده است. در مورد یک سیستم‌عامل نظریه این است که کد
منبع - دستورات برنامه‌نویسی که زیربنای کل سیستم هستند - باید آزاد
باشند. هر کسی می‌تواند آن را بهتر کند، تغییر دهد یا از آن به نفع خود
استفاده کند به شرطی که تمام این تغییرات، بهتر کردن‌ها یا استفاده‌ها باید
به آزادی در اختیار دیگران نیز گذاشته شوند. به ذن فکر کنید. پروژه به
هیچ کس و همه کس تعلق دارد. وقتی پروژه‌ای باز می‌شود، شاهد رشد سریع و
مستمر آن هستیم. گروه‌های مشارکت کننده‌ای که به شکل هم زمان روی پروژه کار
می‌کنند سبب می‌شوند که پروژه سریعتر از هر حالتی که پشت درهای بسته ممکن
بود، رشد کند و به پیش برود.

این همان چیزی است که در لینوکس شاهدش هستیم.  تصور کنید: به جای یک تیم
توسعه که مانند راهبان در یک مکان مخفی مشغول برنامه‌نویسی باشند، یک غول
برنامه‌نویس در حال برنامه‌نویسی برای شماست. عملا میلیون‌ها نفر از
روشن‌ترین مغزهای جهان مشغول مشارکت در پروژه هستند و افرادی مشغول بررسی
خروجی‌ها هستند که اصولا شغل اصلی‌شان آزمایش نرم‌افزار نیست بلکه کاربران
واقعی سیستم هستند.

اولین باری که آدم‌ها ایده بازمتن را می‌شنوند، مساله به نظرشان خنده‌‌دار
می‌آید. شاید اینکه گسترش این ایده چندین سال به طول کشیده به همین دلیل
باشد. گسترش بازمتن مدیون ایدئولوژی‌اش نبود. مطرح شدن بازمتن وقتی سرعت
گرفت که توانست ثابت کند که بهترین شیوه برای توسعه و بهینه‌سازی بالاترین
تکنولوژی موجود است. این روزها هم بازمتن برنده بازار است و این پیروزی
حاصل موفقیت‌های قبلی بازمتن در به ثمر رساندن پروژه‌ها است. شرکت‌ها
توانسته‌اند حول سرویس‌های مرتبط با پروژه‌های بازمتن شکل بگیرند و بعضی‌ها
هم از بازمتن استفاده کرده‌اند تا تکنولوژی‌شان را رواج دهند. وقتی پول به
یک جا سرازیر می‌شود، مردم متقاعد می‌شوند که طرح موفق بوده.

یکی از چیزهایی که در دنیای بازمتن کمتر درک شده، این است که چگونه
بهترین برنامه‌نویسان جهان حاضر می‌شوند بدون پول برای یک پروژه وقت
بگذارند و کار کنند. برای جواب به این مساله باید انگیزه‌های این افراد
بپردازیم. در جامعه‌ای که ادامه حیات کمابیش تضمین شده است، پول دیگر
اصلی‌ترین انگیزه افراد نیست. ثابت شده که افراد وقتی بهترین محصول را
تولید می‌کنند که انگیزه‌ آن‌ها شور و اشتیاق به کار باشد؛ یعنی وقتی که کار
برای شان تفریح باشد. این مساله همان قدر در مورد نمایشنامه‌نویسان،
مجسمه‌سازان و مخترعان صحت دارد که در مورد برنامه‌نویسان. مدل بازمتن به
افراد فرصت می‌دهد تا بر اساس شور و اشتیاق شان کار کنند. کار به همراه
بهترین برنامه نویسان جهان - و نه کسانی که از سر اتفاق کارفرمای مشترکی
پیدا کرده‌اند - برای آن‌ها مفرح است. برنامه‌نویسان بازمتن در تلاش برای
کسب اعتبار در بین دیگر برنامه‌نویسان مطرح جهان هستند و این انگیزه
فوق‌العاده‌ای است برای ارائه بهترین کارها.

به نظر می‌رسد که بیل گیتس این را نمی‌فهمد. شاید هم الان متوجه این نکته
شده و از سوالی که در سال ۱۹۷۶ پرسیده، شرمنده باشد. او در ۱۹۷۶ طی
نامه‌ای به برنامه‌نویسان بازمتن نوشت: \dbquote{کاری که شما می‌کنید این
  است که جلوی نوشته شدن نرم‌افزار خوب را می‌گیرید. چه کسی حاضر است کار
  تخصصی را در قبال هیچ چیز انجام دهد؟}

در واقع برای درک مفهوم بازمتن، یک روش این است که به شیوه‌ای فکر کنیم که
دین چندین قرن قبل (و البته توسط بعضی موجودات در همین روزها) سعی می‌کرد
علم را فهم کند. علم یک چیز خطرناک، خرابکار و ضد نظام تبلیغ می‌شد و این
همان چیزی است که این روزها بعضی از شرکت‌های نرم‌افزاری، در مورد بازمتن
فکر می‌کنند. دقیقا همان طور که علم برای تخریب دین اختراع نشده بود،
بازمتن هم به وجود نیامده تا نظام نرم‌افزار را از هم بپاشد. بازمتن به
وجود آمده تا بهترین تکنولوژی‌ها را خلق کند و ببیند که تکنولوژی به کدام
سمت در حرکت است.

علم به خودی خود مولد پول نیست. انباشت این همه ثروت، کارکرد جانبی علم
بوده است. وضع بازمتن هم همین است. بازمتن به شرکت‌های جدیدی اجازه فعالیت
داده که در حال خارج کردن بخشی از بازار از دست شرکت‌های قدیمی هستند؛
بازهم همان طور که محصولات جانبی علم باعث تضعیف سلطه کلیسا شدند. این جا
شاهد شرکت‌های کوچکی مثل وی.ای.لینوکس هستیم که با استفاده از مزایای
بازمتن، به ناگهان به عنوان رقیبی برای شرکت‌های عظیم و جاافتاده مطرح
می‌شوند. به گفته سر ایزاک نیوتن، آن‌ها بر دوش غول‌های دیگر ایستاده‌اند.

بعله. همان طور که بازمتن پیشرفته می‌کند و سهمی از اقتصاد جهانی به دست
می‌آورد،‌ برنامه‌نویسان آن نیز کم کم شناخته می‌شوند و بیشتر و بیشتر به
عنوان افراد قابل استخدام به آن‌ها نگاه می‌شود. شرکت‌ها شروع کرده‌اند به
بررسی فهرست همکاران برنامه‌ها که از قدیم روشی بوده‌اند برای ذکر اسامی
افرادی که در پروژه‌ها همکاری داشته‌اند. این شرکت‌ها کسانی که مشارکت‌های
خوبی در پروژه‌های مختلف داشته‌اند را کشف می‌کنند و مسوولین منابع انسانی
خودشان را با یک گونی پول و پیشنهادهایی در مورد سهام شرکت پیش آن‌ها
می‌فرستند. در پاراگراف قبل گفتم که پول اصلی‌ترین انگیزه افراد نیست و
هنوز هم نظرم را درباره آن جمله عوض نکرده‌ام. ولی لازم است اضافه کنم که
به نظر من پول در قبال کار سخت، اصلا پاداش بدی نیست. وقتی قرار است باک
بی.ام.و را پر از بنزین کنید، اتفاقا پول بسیار هم به درد بخور می‌شود.

کارکردهای جانبی بازمتن هم - مثل علم - بی‌نهایت است. این روش چیزهایی
می‌سازد که تا امروزه غیرقابل ساخت به نظر می‌رسیدند و با این کار بازارهای
جدید را به وجود می‌آورد. با وجود لینوکس و دیگر نرم‌افزار بازمتن، افراد و
شرکت‌ها می‌توانند محصولات جدید و اختصاصی‌ای برای خود بسازند که تا دیروز
ساخت آن‌ها غیرممکن به نظر می‌رسیده. هیجان انگیز است وقتی می‌بینیم که
بسیاری از چیزهایی که این روزها با استفاده از لینوکس ساخته شده، در
اوایل اصولا به ذهن ما هم خطور نکرده بودند. لینوکس در چین هم دارد مشهور
می‌شود. پیش از این بخش عمده‌ای از صنعت نرم‌افزار آسیا، به ترجمه
نرم‌افزارهای آمریکایی و اروپایی اکتفا می‌کرد اما حالا دوستان ما در آن
بخش دنیا، مشغول ساختن نرم‌افزارهای جدید با استفاده از لینوکس هستند. من
واقعا به مردی افتخار می‌کنم که در نمایشگاه کامدکس جلو آمد و پمپ بنزینی
که بر مبنای لینوکس ساخته بود را به من نشان داد. آن نمونه یک پمپ بنزین
لینوکسی بود که می‌توانست فرد را طی سه دقیقه‌ای که باک مشغول پر شدن است
به سی.ان.ان. متصل کند و تیتر اخبار را به او نشان دهد. این مرد هم روی
شانه غول‌های دیگر ایستاده بود.

لذت بخش است وقتی می‌بینیم که افراد از محصولی مانند لینوکس برای ساخت یک
پمپ‌بنزین بهتر استفاده کرده‌اند. این ابداعی است که هیچ وقت در یک شرکت
بسته انجام نمی‌شود، چون اگر یک شرکت بخواهد لینوکس را به شکل تجاری به
بازار عرضه کند، شکی نیست که باید به دنبال محل اثبات شده آن یعنی سرورها
یا میزکارهای سطح بالا برود. اما لینوکس به شرکت‌ها این اجازه را داده که
خودشان در این باره که دوست دارند با آن چه کار کنند، تصمیم‌ بگیرند. پس
لینوکس را بیش از هر چیز دیگر در دستگاه‌های درون‌ساز\RFootnote{\lr{Embed}
  - کامپیوترهایی که درون ابزارهای دیگر قرار می‌گیرند تا کار کنترل آن‌ها
  را ساده‌تر کنند. مانند \lr{ECU} خودرو}
می‌بینید. تیوو\RFootnote{\lr{Tivo} - سیستم ضبط خودکار کانال‌های
  تلویزیون} لینوکس دارد، وب اسلیت\LFootnote{Web Slate} ترنسمتا هم
لینوکس دارد و در تلفنی\LFootnote{Telephony} هم شاهد حضور لینوکس
هستیم. همین جا است که میلیاردها دلار ثروت از قِبَل بازمتن تولید می‌شود.

من دوست دارم به جهان اجازه دهم که خودش مراقب خودش باشد. با کنترل نکردن
تکنولوژی، باعث می‌شوید که کاربران آزاد باشند. شما محصول را در اختیار
دیگران قرار می‌دهید و آن‌ها تصمیمات مربوط به خود را اتخاذ می‌کنند - مثلا
در این مورد که از محصول شما به عنوان بخشی از محصول یا سرویس خود بهره
بگیرند. این کار تلاشی برای گسترش لینوکس نیست بلکه منظور نهایی این است
که لینوکس را آزاد بگذاریم تا خود به خود گسترش یابد. این جریان فقط
منحصر به لینوکس نیست بلکه موضوع بحث ما هر چیزی است که بازمتن باشد.

بازمتن مهم است.

مردم در مورد لزوم آزادی بیان شک نمی‌کنند. این آزادی‌ای است که آن را با
فدا کردن جان به دست آورده‌ایم. از آزادی همیشه باید با جان دفاع کرد اما
در عین حال انتخاب این دفاع، در ابتدای کار، سخت است. این موضوع در مورد
باز بودن هم صادق است. لازم است تصمیم بگیرید که باز باشید. شاید در
ابتدا بسیار سخت به نظر برسد ولی در نهایت رضایت خاطر را به ارمغان خواهد
آورد.

سیاست را در نظر بگیرید. اگر منطقی که در مورد بازمتن گفتم در سیاست هم
صادق باشد، همیشه حکومت تک حزبی خواهیم داشت. شکی نیست که حکومت تک حزبی
ساده‌تر و قابل درک‌تر از سیستم پیچیده انتخاباتی است؛ روشی باز در سیاست
که اکثر دنیا بر مبنای آن اداره می‌شود. در سیستم تک حزبی نیازی به دردسر
کشیدن برای پرسیدن نظر دیگران و جلب توافق کسی نیست. استدلال به نفع آن
هم ساده است: حکومت آن قدر کار مهم و پیچیده‌ای است که لزومی نیست دائما
وقت آن صرف پرسیدن نظر افراد شود. عجیب است که مردم به راحتی مشکل این
استدلال را در مورد حکومت می‌بینند اما متوجه اشکال مشابه در دنیای صنعت و
تجارت نمی‌شوند و در برخورد با آن، احساس ناامنی می‌کنند

استدلال‌هایی که یک شرکت می‌کنند تا نرم‌افزارهای خود را باز نکند، متقاعد
کننده است. مدیران می‌گویند که تا به حال کارها به این شیوه انجام
نشده. این ترسناک است. مردم از تغییر می‌ترسند و بخشی از این ترس حاصل
ناتوانی در پیش‌بینی نتایج است. اگر شرکتی به وضع موجود بچسبد و حاضر به
تغییر نشود، احتمالا راحت‌تر در این باره که به کدام سو در حال حرکت است
قضاوت می‌کند و گاهی این پیش‌بینی مهمتر از موفقیت بزرگی است که لازمه‌اش
تغییر است. خیلی از شرکت‌ها ترجیح می‌دهند به شیوه‌ای قابل پیش‌بینی موفق
باشند تا به شیوه‌ای غیرقابل پیش‌بینی خیلی خیلی خیلی موفق.

برای شرکت‌ها راحت نیست که یک محصول حاضر و آماده را، بازمتن اعلام
کنند. این کار دردسرهایی دارد. یکی اینکه در طول سالیانی که این نرم‌افزار
تولید شده، شرکت حجم زیادی اطلاعات داخلی تولید کرده‌ است و این اطلاعات
درون شرکتی، نقطه تمایز این شرکت و دیگر شرکت‌ها شده. شرکت‌ها از فکر شریک
شدن این اطلاعات با دیگران می‌هراسند و احساس می‌کنند که با افشای این
اطلاعات، دلیل وجودی آن‌ها به خطر خواهد افتاد. آن‌ها به این دقت نمی‌کنند
که وجود این اطلاعات داخلی، مثل یک سد عمل می‌کند و دیگران را نیز از
مشارکت در پروژه‌ها باز می‌دارد.

البته من شرکت‌هایی را دیده‌ام که از بسته بودن به بازبودن گراییده‌اند. یکی
از این داستان‌ها مربوط به وپیت\LFootnote{Wapit} است. شرکتی فنلاندی که
ارائه دهنده خدمات و پشتیبانی زیرساخت به بسیاری از دستگاه‌های تعاملی
است. این پروژه شامل یک وب سرور هم می‌شود که به شکل یک تلفن دیواری ارائه
شده. برای آن‌ها تصمیم باز کردن نرم‌افزارهای شان منطقی به نظر
می‌رسید. آن‌ها می‌خواستند سرویس‌های شان را ارائه کنند ولی پیش از این باید
زیرساخت خود را توسعه می‌دادند. این کار مستلزم نوشتن حجم زیادی نرم‌افزار
است. یک دردسر واقعی. پس به جای دیدن این جریان به این شکل که
\dbquote{قرار است دارایی‌های معنوی‌مان را با دیگران شریک شویم} گفتند
\dbquote{نوشتن نرم‌افزار یک کار زمان‌گیر مهندسی‌است اما با نگه‌داشتن آن در
  شرکت، ارزش اضافه‌ای تولید نخواهد شد.}

چند عامل هم بود که به نفع وپیت عمل می‌کرد. اول اینکه این پروژه، ابعاد
وسیعی نداشت. دوم اینکه تصمیم باز کردن برنامه، در اولین روزهای تاسیس
شرکت گرفته شد. مدیریت اعلام کرد که نیروهای داخلی شرکت توان انجام پروژه
را دارند اما شرکت به دنبال تولید چیزی است فراتر از آن چیزی که
توانایی‌اش را دارد. همچنین تصمیم نهایی بر این شد که بازمتن کردن پروژه
روش خوبی است برای پیشبرد وپ\RFootnote{\lr{WAP} - پروتکلی برای دریافت صفحات
  وب بر روی گوشی‌ها که امروزه با پیشرفت مرورگرهای تلفن‌ها و تبلت‌ها دیگر
  کاربرد چندانی ندارد.} به عنوان استانداردی که برای دیگران نیز قابل
پذیرش باشد.

در همان شروع بازی، آن‌ها نظر من را خواستند و من به آن‌ها گفتم که باید با
این احساس که همه تصمیمات باید درون شرکت گرفته شوند، مقابله کنند. به
آن‌ها گفتم که اگر قرار است در جلساتی، تصمیماتی در مورد پروژه گرفته شود،
درهای این جلسات نباید به روی افراد خارج از شرکت، بسته باشند. با تبدیل
تصمیم‌گیری‌ها به یک امر درون شرکتی، فعالان بیرون از شرکت دچار از
خودبیگانگی می‌شوند و آن‌ها احساس می‌کنند که به شبکه اصلی تصمیم‌گیری دسترسی
ندارند. فکر کنم این اصلی‌ترین و حساس‌ترین مشکل برای ایجاد و مدیریت یک
پروژه بازمتن در یک شرکت تجاری است. شاید حرف زدن از باز کردن پروژه‌ها
آسان باشد اما پروژه‌های بازمتن شرکتی به راحتی ممکن است تبدیل به یک
رابطه دو قطبی \dbquote{ما-آن‌ها} شوند. تصمیمات ممکن است به شیوه راحت
گرفته شوند: با نشستن پشت میزهای یک کافه تریا و بحث کردن و نتیجه‌گیری و
غفلت از شرکت دادن دیگران در این روند. اگر این وضع پیش بیاید، خارجی‌ها
هم با دیدن اینکه تصمیمات از قبل در کافه‌تریای شرکت گرفته شده‌اند، خود را
از روند توسعه نرم‌افزار کنار می‌کشند.

این همان مشکلی است که گریبان نت‌اسکیپ را هم گرفت. در بهار ۱۹۹۸ و چند
ماه بعد از اینکه نت‌اسکیپ کد منبع مرورگر نسل بعدی‌اش (مشهور به موزیلا)
را بازمتن اعلام کرد، هنوز بازمتن بودن پروژه چندان احساس نمی‌شد. مدت
زیادی طول کشید تا این پروژه واقعا به یک پروژه بازمتن تبدیل شد. در
اوایل یک گروه درون‌شرکتی از برنامه‌نویسان موزیلا وجود داشتند که به سختی
پچ‌های خارجی‌ها را قبول می‌کردند. همه داخلی‌ها همدیگر را می‌شناختند و حتی
اگر دور یک میز ننشسته بودند، یک کافی‌شاپ مجازی داشتند که در آن‌ قهوه
می‌خوردند و تصمیم‌ می‌گرفتند. آن‌ها عملا یک گروه بسته بودند. نت‌اسکیپ در
طول آن‌ ماه‌ها به جای تبدیل شدن به نمونه‌ای خوب از کاربرد بازمتن در توسعه
نرم‌افزار، تبدیل شده بود به یک سیبل رسانه‌ای علیه جنبش بازمتن. وقتی
زمزمه‌های عدم فعالیت پروژه به بیرون درز کرد، وضع موزیلا خراب‌تر شد اما
در عوض نت‌اسکیپ قانع شد که پروژه را واقعا باز کند. این روزها می‌بینیم که
پروژه واقعا فعال‌تر است.

وقتی دوستان می‌شوند که امکان باز اعلام کردن یک پروژه تجاری هم وجود
دارد، معمولا شروع می‌کنند به پرسیدن یک سری سوالات تکراری. یکی از سوالات
در مورد عکس‌العمل داخلی‌ها است به این امکان که فردی خارج از سازمان
بتواند نتیجه‌ای بهتر از آن‌ها تولید کند و همه دنیا هم این را ببینند. به
نظر من برنامه‌نویسان باید از این لحاظ بسیار خوشحال باشند، چون ثابت
کرده‌اند این قدر ارزشمندند که بدون انجام دادن اکثر کار،‌ دارند حقوق
می‌گیرند. از این نظر بازمتن، یا باز بودن هر چیزی، فوق‌العاده است. باز
بودن به همه نشان می‌دهد که کار را واقعا چه کسی دارد انجام می‌دهد و دیگر
کسی نمی‌تواند پشت مدیرش مخفی شود.

بازمتن بهترین اهرم است برای بکارگیری نیروهای خارج از سازمان ولی هنوز
برای پاسخ به نیازهای خود سازمان باید افرادی در سازمان شاغل باشند. این
آدم شاید رهبر پروژه نباشد. در واقع برای سازمان عالی است اگر کسی از
بیرون پروژه را هدایت کند و جلو ببرد و پول هم نگیرد. خیلی خوب است اگر
کسی از بیرون کار بهتری انجام دهد. مشکل وقتی بروز می‌کند که این رهبر
بیرونی پروژه را در راستای اهداف یا نیازهای شرکت به پیش نبرد. در این
صورت شرکت باید شخصا به نیازهایش جواب دهد. باز کردن یک پروژه شاید به
معنای کاهش منابع تخصیصی از درون سازمان باشد ولی به هیچ وجه به معنای
تعطیل کردن کار درون سازمان نیست. حتی احتمال دارد پروژه آن قدر بزرگ شود
که هیچ شرکتی قادر به کنترل آن نباشد. منابع بیرونی ارزانتر هستند، نواقص
کمتری دارند و دیدگاه همه جانبه‌تری به پروژه می‌دهند ولی در مقابل کفه
دیگری هم داریم: سیستم ممکن است دیگر فقط به نیازهای شرکت توجه نکند. حالا
دیگر محصول برای پاسخگویی به نیازهای مشتریان بهینه می‌شود.

شاید آزاردهنده‌ترین مرحله، پذیرفتن این واقعیت باشد که ممکن است بیرونی‌ها
بهتر از داخلی‌ها بفهمند. مشکل دیگر پیدا کردن یک رهبر فنی قوی در داخل
شرکت است. این آدم باید کسی باشد که در دو سطح مورد اعتماد همگان باشد -
هم در سطح فنی و هم در سطح سیاسی. کسی که بتواند درک کند که ممکن است
پروژه از ابتدا مشکل‌دار طراحی شده باشد. به جای مخفی کردن این گونه
مشکلات، رهبر باید بتواند افراد را متقاعد کند که همه چیز باید از صفر
شروع شود، به عبارت دیگر باید به بقیه توضیح دهد که راه را اشتباه
آمده‌اند. این چیزی نیست که آدم‌ها از شنیدنش لذت ببرند و به همین دلیل هم
برای قبول کردن این حرف، گوینده آن باید کسی باشد که دیگران به حرفش
احترام بگذارند.

با توجه به سیاست‌های درون‌شرکتی و شیوه‌ای که سازمان‌ها معمولا کار می‌کنند،
رهبر فنی باید انسانی باشد با شخصیت قوی. او باید کسی باشد که به راحتی
با ایمیل کار کند و در مجادله‌ها هم طرف کسی را نگیرد. از عبارت
\dbquote{طرفین} استفاده نکردم چون نمی‌خواهم به نظر بیاید که در مباحث،
دو جبهه متخاصم - یعنی گروه داخلی و گروه خارجی - داریم. این رهبر فنی از
شرکت پول می‌گیرد تا کار بازمتن بکند. همه باید بدانند که این آدم پول
نمی‌گیرد که از شرکت یا خارجی‌ها حمایت کند بلکه پول می‌گیرد تا پروژه را به
پیش ببرد. داشتن رهبری که بیش از حد به شرکت نزدیک است خطرناک خواهد
بود. ممکن است افراد به قابلیت‌های فنی او اعتماد کنند ولی در مورد
تصمیم‌های غیرفنی‌اش اعتراض خواهند داشت.

باید بتوانید یک آدم شریف پیدا کنید. 

به همین دلیل است که این همه سال تلاش کرده‌ام تا با هیچ کدام از شرکت‌های
توزیع کننده لینوکس کار نکنم. به خصوص این روزها که اسکناس‌ها در حال
نمایان شدن هستند، این مساله مهمتر هم شده است. با این همه دلار که این
طرف و آن طرف جابه جا می‌شود،‌ مردم شروع می‌کنند به پرسش از انگیزه‌ها. برای
من مهم است که همیشه به بی‌طرفی مشهور بوده‌ام. نمی‌دانید چقدر برایم مهم
است که این بی‌طرفی حفظ شود. گاهی فکر می‌کنم این تلاش ممکن است من را
دیوانه کند.

درست است. حق با شما است. باید موعظه را متوقف کنم. بازمتن برای هر شخصی
و هر پروژه‌ای و هر شرکتی نیست اما هرچه قدر که مردم پول بیشتری از سهام
شرکت‌های لینوکسی در بیاورند بیشتر و بیشتر مطمئن خواهند شد که بازمتن،‌
همهمه ناشی از حرف‌های درگوشی بچه دبیرستانی‌های ایده‌آلیست نیست.

تمام چیزها را باز کنید و شانس به شما رو خواهد کرد. من تمام سال‌هایی که
خبرنگارها از من در مورد بازمتن پرسیده‌اند را در این مورد صحبت
کرده‌ام. الان پنج سالی می‌شود. آن روزها فکر می‌کردم که باید دائما توضیح
بدهم که چه چیزی در مورد بازمتن این قدر خوب و مهم است. و صادقانه بگویم
که این کار برایم مثل یک سفر بی‌پایان بوده است؛ مثل راه پیمودن در گل و
لای.

حالا مردم مساله را فهمیده‌اند. 

\section{شهرت و ثروت}
\dbquote{مسئولیت ناشی از شهرت چیست؟} این چیزی است که برخی مردم از من
خواهند پرسید و بگذارید صادقانه بگویم که \dbquote{مسئولیت} اصولا
\dbquote{مسئولیت} نیست. مشهور بودن مفرح است و مشاهیری که خلاف این را
می‌گویند یا می‌خواهند مودب باشند یا می‌خواهند به افراد غیر مشهور بگویند
که وضع آن‌ها بهتر است. از آدم‌های مشهور انتظار داریم که متواضع باشند و
بگویند که شهرت زندگی ساده آن‌ها را خراب کرده است.

به هرحال واقعیت این است که همه به شهرت و ثروت فکر می‌کنند. من که
می‌کردم. نوجوان که بودم دوست داشتم روزی دانشمند مشهوری شوم؛ شبیه آلبرت
انیشتین ولی بهتر. چه کسی است که چنین آرزویی نداشته باشد؟ شاید بعضی‌ها
نخواهند دانشمند شوند ولی دوست خواهند داشت که قهرمان اتومبیل‌سواری باشند
یا خواننده راک یا مادر ترزا یا شاید هم رییس‌جمهور ایالات متحده.

رسیدن به آن آرزو، عملا کار راحتی بوده. شکی نیست که من آلبرت انیشتین
نشده‌ام اما از اینکه باعث تغییری شده‌ام و از اینکه کار معناداری کرده‌ام،
احساس رضایت می‌کنم. اینکه به خاطر این کار نزد دیگران شناخته‌ شده‌ام هم به
رضایتم اضافه می‌کند.  دفعه بعد اگر کسی را دیدید که از شهرت یا ثروت
اظهار نارضایتی می‌کرد، توجه زیادی به حرفش نکنید. دلیل اظهار نارضایتی از
شهرت یا ثروت این است که جامعه انتظار دارد افراد مشهور و ثروت‌مند این
گونه حرف بزنند.

اما آیا همه بخش‌های ثروت و شهرت خوب هستند؟ بدون شک نه. مطمئنا مشهور
بودن نکات منفی‌ای هم دارد. آدم‌های داخل خیابان من را نمی‌شناسند (یا لااقل
معمولا کسی در خیابان من را نمی‌شناسد) ولی حجم زیاد ایمیلی که من دریافت
می‌کنم باعث می‌شود ایمیل‌هایی که جواب دادن شان ارزشمند است،‌ لا به لای
آن‌ها گم شوند. گاهی هم ایمیل‌هایی دارم که از یک طرف جواب‌دادن شان سخت است
و از طرف دیگر نمی‌شود به آن‌ها جواب نداد. جواب کسی که از شما می‌خواهد
برای پدر تازه درگذشته‌اش یک متن تحسین آمیز بفرستید را چه می‌دهید؟ من هیچ
وقت به آن ایمیل جواب ندادم و هنوز هم احساس گناه می‌کنم. جواب دادن آن
نامه احتمالا برای آن فرد بسیار مهم بود و برای من هم به همچنین، اما تا
به امروز هم آن ایمیل جواب داده نشده و احتمالا مسکوت ماندن آن، دو نفر
را تا به امروز ناراحت کرده است.

یا با کسی که از شما درخواست کرده در کنفرانسی سخنرانی افتتاحیه را ایراد
کنید چه می‌کنید وقتی که نه به کنفرانس علاقه‌ای دارید و نه برای این کار
انگیزه‌ای؟ چگونه می‌خواهید به مردم بفهمانید که با وجود اینکه مدت‌ها است
به پیام‌های تلفنی ضبط شده‌تان گوش نمی‌کنید، یک آدم بی‌ملاحظه و حرامزاده
نیستید. بالاخره هم شما را یک آدم بی‌ملاحظه و حرامزاده خواهند دید. در
نهایت به این نتیجه می‌رسیم که من به خاطر علاقه‌ام به یک موضوع، باید به
همه این موضوعات فکر کنم و آن موضوع چیزی نیست به جز لینوکس.

چیزی که رد خور ندارد این است که در نهایت باید \dbquote{نه} گفتن را یاد
گرفت یا حتی توجه نکردن به درخواست‌ها را. یکی از دلایلی که من از ایمیل
خوشم می‌آید این است که به راحتی می‌توان آن را نادیده گرفت - یک ایمیل کمتر
و بیشتر در بین صدها ایمیلی که من روزانه دریافت می‌کنم فرق زیادی
نمی‌کند. این رسانه آن قدر غیر شخصی شده که به ندرت ایمیلی آن قدر شخصی
می‌بینم که پاسخ ندادن به آن برایم منجر به احساس گناه شود. این گاهی
اتفاق می‌افتد (به کمی عقب‌تر رجوع کنید) ولی بسیار به ندرت. وقتی هم بحث
گذشتن از کنار درخواست مطرح نیست، گفتن \dbquote{نه} در ایمیل بسیار
ساده‌تر از گفتن آن در مکالمه رودررو یا حتی پشت تلفن است.

مبنای این مشکلات، تصور و انتظاراتی است که مردم نسبت به آدم‌های مشهور
پیدا می‌کنند و در نظر گرفتن این واقعیت که بدون شک نمی‌توان مطابق با
انتظارات همگان زندگی کرد. این مشکل را حین نوشتن این کتاب هم داشتم؛ در
حالی که یک کتاب شخصی می‌نوشتم، لازم بود به انتظارات دیگران نیز پاسخ دهم
تا خوانندگان احساس نکنند که این کتاب آن چیزی نبوده که انتظارش را
داشته‌اند.

البته بعضی از انتظارات هم واقعا احمقانه‌اند. پیش می‌آید که بعضی‌ها از من
انتظار دارند تا مثل یک راهب مدرن زندگی کنم؛ در فقر و انزوا و فقط هم به
این خاطر که تصمیم گرفته‌ام لینوکس را به آزادی در اختیار دیگران بگذارم و
دیدگاه سنتی و اقتصادی نسبت به نرم‌افزار را کنار بگذارم. اینجاست که
احساس می‌کنم باید از خودم دفاع کنم و توضیح دهم که از پول خرج کردن لذت
می‌برم و خوشحالم که پونتیاک گرند ای.ام. قدیمی‌ام را به یک ماشین جدیدتر و
مفرح‌تر ارتقاء داده‌ام.\RFootnote{لینوس در زیرنویس می‌گوید
  \dbquote{پونتیاک گرند ای.ام. هیچ مشکلی ندارد و ماشین خوبی است. فکر
    می‌کنم یکی از مرسوم‌تری ماشین‌های آمریکا هم باشد و چندباری هم
    خبرنگارها به من گفته‌اند که با دیدن خودروی معمولی من شگفت زده
    شده‌اند. این خودرو حتی ژاپنی‌ هم نیست! این روزها بعضی مردم احساس
    می‌کنند که دیگر برای‌شان محترم نیستم چون ساعت‌ها در مورد رنگ ماشین
    جدیدم که خیلی کمتر از قبلی کاربردی است، فکر کرده‌ام - یک
    بی.ام.و. زد ۳ که - یادتان که نرفته؟ - فقط برای تفریح. این خودرو
    عملا به هیچ دردی جز \xquote{تفریح} نمی‌خورد و دقیقا به همین خاطر
    است که دوستش دارم.}}

حالا و بعد از پرسش مربوط به \dbquote{مسوولیت شهرت} باید به سراغ پرسش
دوم برویم. \dbquote{آیا موفقیت لینوس (و/یا لینوکس) را تباه خواهد کرد؟}
آیا من به بچه ننری تبدیل خواهم شد که در مورد خودش کتاب می‌نویسد چون
دوست دارد اسمش را در قفسه کتابفروشی‌ها ببیند و از این راه پول خرید
ماشین جدیدش را تامین کند؟

جواب مشخص است؛ بله.

به هر حال فردی را در نظر بگیرید که فلسفه تمام زندگی‌اش مبتنی بر تفریح
بوده و شهرت و ثروت کافی را به آن اضافه کنید و ببینید که از او چه
انتظاری می‌توانید داشته باشید. یک آدم نیکخواه؟ من که این طور فکر
نمی‌کنم. اهدای پول به خیریه‌ها پیش از اینکه در حین نوشتن این کتاب توسط
دیوید مطرح شود، اصولا به فکر من هم نرسیده بود. وقتی دیوید این سوال را
پرسید با تعجب به او نگاه کردم و اولین چیزی که به فکرم رسید طنزی در
مورد یکی از خیریه‌ها بود. من هیچ وقت در طول زندگی‌ام مسوولیت مالی را
تجربه نکرده‌ام.

آیا موفقیت شیوه نگاه به چیزها را تغییر می‌دهد؟ بله. لینوکس وقتی که فقط
بیست و پنج گیک فنی کاربرش بودند موجودی کاملا متفاوت نسبت به زمانی بود
که بیست و پنج میلیون کاربر معمولی حداقل گاه گاه از آن استفاده می‌کردند
(یا هر جمعیتی که این روزها از آن استفاده می‌کنند). این مقایسه در مورد
زمانی که کاربران لینوکس فقط به خاطر تفریح از آن استفاده می‌کردند و حالا
که موفقیت تجاری بزرگی پشت کاربردهای آن نفهته است هم صدق می‌کند.

در مورد شخص لینوس هم همینطور است. چیزها تغییر کرده‌اند و نفی این
واقعیت، چیزی را به عقب بر نمی‌گرداند. لینوکس همان جنبشی که پنج سال پیش
بود نیست و لینوس هم لینوس آن روزها نیست. چیزی هم که از همه بیشتر مرا
به لینوکس جذب کرده همین واقعیت است که لینوکس هیچ وقت چیزی تکراری نبود
و هر روز شاهد چیزهای جدیدی در آن بودیم. این چیزهای جدید هم فقط در حوزه
فنی رخ نمی‌دادند، بلکه مفهوم خود لینوکس هم در روند موفقیت، تغییر
می‌کرد. اگر این طور نبود، زندگی بیش از حد خسته کننده می‌شد.

پس به جای استفاده از لغت \dbquote{تباه} ترجیح می‌دهم بگویم که موفقیت
تجاری هم من و هم لینوکس را \dbquote{تغییر} داده است. نسبت به استفاده
از لغت \dbquote{رشد کردن} هم مردد هستم - به نظرم اگر بحث رشد باشد،
داشتن سه دختر بیشتر باعث رشدم شده است تا لینوکس - پس همان بهتر است که
از \dbquote{تغییر} استفاده کنم. این تغییر در بسیاری موارد رو به جلو
بوده ولی از خلوص کاسته است. لینوکس در ابتدا فقط در خدمت افراد فنی بود
و بهشتی برای گیک‌ها. سنگر محکمی بود برای دنیای خالص که در آن فقط
تکنولوژی مهم بود و نه هیچ چیز دیگر.

این روزها آن بحث دیگر چندان موضوعیت ندارد. لینوکس هنوز پس‌زمینه فنی قوی
دارد، اما داشتن میلیون‌ها نفر کاربر باعث شده که هر تصمیم کوچک برای
تغییر، با مسوولیت بزرگی مواجه شود. حالا دیگر سازگار بودن با نسخه‌های
قبلی به ناگهان به یک موضوع حیاتی تبدیل شده و روزی- بیست سال بعد،‌ کسی
خواهد آمد و خواهد گفت که دیگر بس است! و سیستم‌عامل جدید خودش، یعنی
\dbquote{فردیکس} را خواهد نوشت.\RFootnote{لینوس در پاورقی توضیح می دهد
  \dbquote{شاید هم دایانیکس، چون امیدواریم که بیست سال بعد دنیای
    نرم‌افزار از سلطه مردان خارج شده باشد.}} آن سیستم‌عامل جدید دیگر
نیازمند سازگار بودن با کوله‌باری از تکنولوژی‌های تاریخی نخواهد بود و این
دقیقا رمز موفقیتش خواهد بود.

اما چیزی که مرا بیش از حد مفتخر می‌کند این است که حتی در دوره
\dbquote{فردیکس} هم، چیزها به وضعیت پیش از لینوکس باز نخواهند گشت. اگر
لینوکس هیچ کار نکرده باشد، به مردم نشان داده است که شیوه جدیدی برای
انجام کارها وجود دارد و ثابت کرده است که ما می‌توانیم کارهای خودمان را
در امتداد کارهای دیگران توسعه دهیم. مدت‌های زیادی است که بازمتن به وجود
آمده اما لینوکس باعث شد این مفهوم بین عموم مردم جا بیفتد. حالا دیگر
وقتی فردیکس بیاید، نیازی ندارد که کارها را از صفر شروع کند.

به خاطر تمام این چیزهایی که گفتم، دنیای به جایی کمی بهتر تبدیل شده.

\begin{journal}
تقریبا یکسال بعد از اینکه کار روی این کتاب را شروع کرده بودیم، من و
لینوس یک عصر جمعه را اختصاص دادیم به رفتن به پیست ماشین‌رانی‌ای که چندین
ماه قبل، در آن‌جا با هم مسابقه داده بودیم. این بار لینوس برنده شد. هم
در رانندگی سریعتر و هم در زدن اهداف. بعدا که سراغ غذاهای ترکی رفته
بودیم، شکستم را به گردن خستگی یک روز پر مشغله انداختم.

لینوس نگاهی به من انداخت و گفت: \dbquote{در عوض فقط سه ماه دیگر باید
  این شغل را تحمل کنی.}

\dbquote{چطور؟}

\dbquote{مگر سه ماه دیگر موعد دریافت سود سهام نیست؟}

دلیلی که این بخش را نوشتم، این بود که یادآوری کنم که چند ماه قبل که در
همین محل با لینوس مسابقه داده بودم، بعد از مسابقه گفته بود که حافظه
ضعیفی دارد و در بسیاری از مواقع،‌ تاو باید شماره‌ تلفن‌هایش را به او
یادآوری کند. حالا ناگهان نشان داده بود که تاریخ سرمایه‌گذاری در بورس یک
نفر دیگر را به یاد دارد و می‌تواند زمان سررسید توزیع سود را هم حساب
کند. او حتی می‌توانست دقیقا بگوید که وقتی جریان خرید بورس را به او
گفتم،‌ کجا ایستاده بودیم. به نظرم یک سال قبل از بازی کردن نقش پروفسور
کم‌حافظه‌ای لذت می‌برد که به جز نظریه ابرریسمان و میزان حافظه اولین
کامپیوترش، چیزی را به خاطر نمی‌آورد. حالا به نظر می‌رسید که حافظه این
پروفسور، حسابی تنظیم شده است.

ژانویه بود و در وان داغ کهنه من نشسته بودیم و با لینوس در این مورد
شوخی می‌کردیم که احتمالا موزه تاریخ دریایی از من خواهد خواست تا وانم را
به آن‌ها اهدا کنم. در آگوست بود که لینوس پرسید \dbquote{راستی کی قرار
  است وان را به موزه اهدا کنی؟} برای دانستن تاریخ آمدن آووتون هم نیازی
نبود به هیچ نوع دستگاه الکترونیکی رجوع کند. این روزها لینوس خیلی بیشتر
از پارسال در جریان وقایع زندگی شخصی دوستان و همکارانش است. در واقع حتی
درباره زندگی دوستان و همکاران من هم اطلاعاتی دارد. آدمی که اولین
جملاتش را با \dbquote{در واقع چیز زیادی از بچگی‌ام یادم نمی‌آید} شروع
کرده بود، حالا تبدیل شده به کسی که ناگهان می‌گوید \dbquote{برایت گفته‌ام
  که وقتی مادرم به من گفت ۱۰۰ مارکی که برای خرید اولین ساعتم کم داشتم
  را از پدر بزرگم بگیرم، چقدر خجالت کشیدم؟}

به یاد آوردن چیزها، تنها یکی از تغییراتی بود که در طول این یک سال مهم
از زندگی لینوس، شاهدش بودم. چیزهای دیگری هم عوض شدند. در نوامبر، به یک
سفر خانوادگی به لس آنجلس رفتیم که به صحنه نوشتن مقدمه \dbquote{معنای
  زندگی} تبدیل شد. سفیر فنلاند، خانه ویلایی‌اش را در اختیار ما گذاشته
بود و قبل از سفر،‌ لینوس داشت سعی می‌کرد از قفسه مشروبات یک فروشگاه
بزرگ، شراب مناسبی برای هدیه انتخاب کند. به قفسه چشم دوخته بود و بدون
اینکه حرکتی بکند گفت: \dbquote{برای انتخاب کمکم کن. هیچ سر رشته‌ای از
  شراب ندارم.} ده ما بعد، لینوس می‌توانست بین دو مارک مشابه، آن را که
برای نوشیدن حین دیدن یک فیلم حادثه‌ای در خانه مناسب است، انتخاب
کند. تازه دیدم که مشروب را پیش از خوردن در گیلاس می‌چرخاند تا کیفیتش را
بسنجد.

بحث ورزش هم هست. در اولین دیدارم با لینوس، دیدم که دیدگاهش به وضعیت
بدنش، مشابه گیک‌های دیگر است. فلسفه آن‌ها این است که بدن فقط حاملی است
برای مغزشان و در نتیجه در مواردی لینوس حتی به ورزش نکردن افتخار هم
می‌کرد. مشخص است که وضع تاو فرق داشت. کتاب‌های کاراته‌اش یک قفسه کامل را
اشغال کرده بودند و نوارهای ورزشش روی تلویزیون انباشه بودند.  لینوکس آن
روزها گفته بود \dbquote{شاید پنج سال دیگر دکترها به من بگویند که باید
  وزنم را کاهش دهم و ورزش کنم.}

من خودم ورزش را دوست دارم و سعی کرده‌ام ورزش را به یکی از برنامه‌های بین
خودم و توروالدز تبدیل کنم. می‌خواستم او را به موج سواری ببرم ولی به
نظرم بوگی‌سواری منطقی‌تر آمد. ما یک روز عصر به \dbquote{هاف مون بی}
رفتیم و لباس و تخته اجاره کردیم. لینوس شدیدا در مقابل ایده رفتن به
داخل آب سرد اقیانوس مقاومت می‌کرد، حتی با وجود لباس مخصوص این کار. اما
چند دقیقه نگذشته بود که به شکلی معجزه آسا درحال لذت بردن از موج‌های
اقیانوس بود و مثل یک بچه پنج ساله، فریاد می‌زد که \dbquote{فوق‌العاده
  است} و دستش را روی دست من می‌کوبید. بدون شک پانزده دقیقه نگذشته بود
که پایش گرفت - می‌گفت به خاطر عدم تناسب اندامش است - و مجبور شد
بوگی‌سواری را متوقف کند (وقتی پایش گرفت همان طور در آب نشست و اجازه داد
که موج‌ها او را به این طرف و آن طرف ببرند. اولین چیزی که به ذهنم رسید
این بود که \dbquote{لعنت! اگر این آدم این جا خفه شود، میلیون‌ها نرد
  علیه من به پا خواهند خواست.})

ما در طول مرحله آماده کردن کتاب، کلی ورزش کردیم: تنیس بازی کردیم، با
هم مسابقه شنا دادیم، در گریت آمریکا\RFootnote{\lr{Great America} - نام یک
  پارک تفریحی بزرگ در حوالی سن جوز} کارهای پرهیجان کردیم و توپ‌های گلف
را این طرف و آن طرف انداختیم. در اواخر کتاب، کار به جایی رسیده بود که
لینوس دیگر دوست نداشت یک جا بنشینیم و جلوی ضبط‌صوت من صحبت کنیم. او
دوست داشت درگیر هر فعالیتی بشود که من برنامه‌ریزی کرده بودم؛ از حمام گل
گرفته تا دوچرخه سواری کوهستان و بیلیارد و هر چیز دیگر. بعد از یک بازی
تنیس در نزدیک خانه من و در حالی که خیس عرق بود، گفت \dbquote{می‌توانم
  تمام عمرم را تنیس بازی کنم.} آن بار هم راکت و هم کفش ورزشی را قرض
گرفته بود. از آن روز به بعد، همیشه یک کفش ورزشی برای مواقع لزوم در
صندوق عقب ماشینش می‌گذاشت.
\end{journal}

\section{معنای زندگی 2}
هیچ وقت برای تان پیش آمده که یک شب گرم تابستانی به ستاره‌ها نگاه کنید و
از خودتان سوال کنید که چرا این جایید؟ جایگاه شما در جهان چیست و قرار
است با زندگی‌تان چه کار بکنید؟

خب بعله، من هم برایم پیش نیامده.

اما به هرحال این روزها من به یک نظریه در مورد زندگی، جهان و همه چیز
رسیده‌ام - یا لااقل در مورد زیرمجموعه‌ای از آن به نام \dbquote{زندگی}
شما در مقدمه این کتاب با نظریه من آشنا شده‌اید و چون تا این جای کتاب با
من بودید، می‌خواهم کمی بیشتر در مورد آن توضیح بدهم.

نظریه من وقتی که به ستاره‌ها چشم دوخته بودم و در وسعت بیکران آن‌ها غرق
شده بودم، به ذهنم خطور نکرده. این نظریه را وقتی کشف کردم که داشتم برای
یک سخنرانی آماده می‌شدم. وقتی شما در یک حوزه خاص مشهور می‌شوید، مردم از
شما انتظار دارند که درباره دیگر شاخه‌های علوم که برای میلیون‌ها سال
بشریت را حیران کرده‌اند نیز جواب‌هایی داشته باشید. از همه بدتر اینکه از
شما انتظار می‌رود این دیدگاه‌های ناب نسبت به مسایل حل نشده را در مقابل
جمعی از افراد ناشناس هم به زبان بیاورید.

این جریان واقعا نامربوط است. من سراغ لینوکس رفتم چون یک گیک تکنولوژی
بودم. دلیل شهرت من به خاطر توانایی سخنرانی‌ام نیست چه برسد به توانایی‌ام
در فلسفه‌بافی. البته چیزهای مربوط در زندگی خیلی کم هستند و به همین خاطر
شکایتی از این یکی هم ندارم.

برگردیم به موضوع.

این بار به یک مراسم در برکلی دعوت شده بودم به نام
\textbf{وب‌راش}\LFootnote{Webrush} من معمولا این جور دعوت‌ها را رد می‌کنم
اما این بار دعوت از طرف سفیر فنلاند در ایالات متحده انجام شده بود و
چون من یک آدم میهن پرست هستم (یا حداقل نسبت به اینکه کشورم را به خاطر
برف دوست ندارم و آن را ترک کرده‌ام، احساس گناه می‌کنم) مثل احمق‌ها گفتم
\dbquote{اوکی. جاگ گور دت.}\RFootnote{\lr{Jag gör det} - به فنلاندی
  یعنی \dbquote{بله مطمئنا می‌پذیرم}}

هیچ کس انتظار نداشت من در مورد معنای زندگی حرف بزنم و خودم هم اصلا به
این فکر نبودم. اما این برنامه درباره زندگی در جامعه شبکه‌ای بود و من
مسوولیت بخش اینترنت را داشتم و در عین حال نماینده فنلاند هم بودم. نام
فنلاند به خاطر نوکیا (که بنا به شهادت همه فنلاندی‌ها، بزرگترین، بهترین
و زیباترین شرکت جهان است) مشهور است و این یعنی مخابرات در سطح بسیار
بالا و جلسه هم که درباره شبکه بود. قبلا با شما در این مورد حرف زده‌ام
که در فنلاند تعداد موبایل‌ها بیشتر از آدم‌ها است و تحقیقاتی در جریان است
که در بدو تولد بتوانند این چیزها را به بدن نوزاد پیوند بزنند.

حالا من نشسته‌ام و دارم فکر می‌کنم که در مورد مخابرات و شبکه چه چیزی
باید بگویم. آه. یادم رفت بگویم که تقریبا همه سخنرانان دیگر، فلاسفه‌ای
بودند که درباره تکنولوژی نظر می‌دادند. به هرحال در برکلی بودیم دیگر. دو
چیزی که در برکلی بسیار مهم به شمار می‌آید، سیاست‌های برکلی‌ است و فلاسفه
برکلی.

خب حالا چه کار باید بکنم؟ اگر قرار است آن‌ها فلاسفه‌ای باشند که در مورد
تکنولوژی حرف می‌زنند، چرا من تکنولوژیستی نباشم که در مورد فلسفه حرف
بزنم؟ هیچ کس حق ندارد من را ترسو بخواند. می‌توانند بگویند که بی‌نهایت
ابله هستم ولی ترسو نه.

این گیک ترسو نیست.

حالا آنجا ایستاده‌ام. هیجان زده و در تلاش برای پیدا کردن موضوعی برای
سخنرانی فردا (من هیچ وقت پیش از اینکه به اندازه کافی دیر شده باشد،
شروع به آماده کردن سخنرانی نمی‌کنم. همین است که تقریبا تمام شب‌های قبل
از سخنرانی می‌توانید من را در حال فکر کردن به سخنرانی فردا پیدا
کنید). من پشت میز نشسته‌ام و به این فکر می‌کنم که این \dbquote{جامعه
  شبکه‌ای} چه جور چیزی است و نوکیا و بقیه شرکت‌های ارتباطاتی در آینده به
کدام سمت خواهند رفت.

بهترین فکری که به ذهنم می‌رسد، توضیح دادن معنای زندگی است. البته بحث
\dbquote{معنا} چندان مطرح نیست. مد نظرم بیشتر قانون زندگی است که از
این به بعد \dbquote{قانون لینوس} نام خواهد گرفت. این قانون چیزی شبیه
به قانون دوم ترمودینامیک است، با این تفاوت که به جای توضیح در مورد
انحطاط تدریجی جهان، تکامل تدریجی زندگی را تشریح می‌کند.

منظورم از \dbquote{تکامل}، مفهوم داروینی آن نیست. بحث داروین جریانی
مستقل است و من برای کنفرانس وب‌راش، بیشتر علاقمند بودم در مورد تکامل
جوامع توضیح بدهم. اینکه چگونه از جامعه صنعتی به جامعه ارتباطاتی
رسیده‌ایم و در آینده به کدام سمت خواهیم رفت و چرا. می‌خواستم بحث‌ام به
نظر جالب بیاید و آن قدر متقاعد کننده باشد که مخاطبین در طول پنل، جذب
شوند. هر کسی برای سخنرانی برنامه خودش را داشت و برنامه این بود که از
یک میزگرد مشترک با دو فیلسوف مهم، زنده بیرون بیایم.



دلیل رشد جوامع چیست؟ نیروی پیشرانه این رشد از کجا تامین می‌شود؟‌ آیا
واقعا آن طور که خیلی‌ها تصور می‌کنند، این تکنولوژی است که جوامع را به
جلو می‌برد؟ آیا کشف موتور بخار بود که باعث شد اروپا به یک جامعه صنعتی
تبدیل شود و بعد نوکیا و تلفن‌های موبایل بودند که ما را به سمت جامعه
ارتباطاتی راندند؟ به نظر می‌رسد این دیدگاه غالب فلاسفه باشد و آن‌ها دوست
دارند در این باره که چگونه تکنولوژی باعث رشد جوامع شده حرف بزنند.

من به عنوان یک تکنولوژیست، می‌دانم که تکنولوژی چیزی را به پیش
نمی‌راند. این جامعه است که تکنولوژی را جلو می‌برد، نه برعکس. تکنولوژی
فقط تعیین کننده محدوده کارهایی است که ما می‌توانیم انجام دهیم و در عین
حال هزینه هر کار را هم برای ما مشخص می‌کند.

تکنولوژی، همانند ابزارهایی که می‌سازد، بسیار خرفت است. تنها جنبه جذاب
تکنولوژی، کارهایی است که ما با استفاده از آن می‌توانیم بکنیم. نیروی
حرکت تکنولوژی، خواسته‌ها و نیازهای انسان است. ما این روزها به خاطر حضور
موبایل نیست که بیشتر با هم حرف می‌زنیم بلکه چون این روزها به آدمی
پرچانه تبدیل شده‌ایم و می‌خواهیم که بیشتر حرف بزنیم و موبایل نداشته‌ایم،
آن را اختراع کرده‌ایم. نوکیا را هم همین طور.

پس بحث من به این جا رسید که اگر قرار است تکامل یک جامعه را بررسی کنیم،
باید عاملی که انگیزه نیازهای انسان است را بشناسیم. آیا این انگیزه پول
است؟ موفقیت است؟ رابطه جنسی است؟ واقعا چه چیزی مبنای چیزهایی است که
انسان‌ها می‌خواهند؟

یکی از انگیزه‌هایی که احتمالا هیچ کس در آن شک نخواهد کرد و اگر هم بکند،
پاسخ آن ساده است، انگیزه بقا است. به هرحال این همان چیزی است که زندگی
را تعریف می‌کند: زنده بودن. این انگیزه پیروی کور از قانون دوم
ترمودینامیک نیست، بلکه در عوض محصول تلاش برای زنده ماندن است در دنیایی
شدیدا خصمانه که به راحتی می‌تواند به زندگی ما خاتمه دهد. پس بقا، عامل
اول تعیین کننده نیازهای ما است.

برای رتبه‌بندی عوامل بعدی،‌ باید نسبت آن‌ها با عامل اولی که در
پاراگراف‌های قبلی کشف کرده‌ام را بررسی کنم. سوال این نیست که
\dbquote{آیا به خاطر پول حاضرید آدم بکشید؟} بلکه باید بپرسیم که
\dbquote{آیا به خاطر پول حاضرید بمیرید؟} مشخص است که جواب \dbquote{نه}
است و به همین دلیل به راحتی می‌توانیم پول را از زمره انگیزه‌های پایه‌،
خارج کنیم.

اما بدون شک چیزهایی هم هستند که مردم حاضرند به خاطر آن‌ها
بمیرند. داستان‌ها و افسانه‌های زیادی داریم از قهرمانان و حتی حیوانات
قهرمانی که حاضر شده‌اند به خاطر هدفی بزرگ، خود را به کشتن دهند. پس بقای
صرف هم نمی‌تواند عامل پیش‌برنده جامعه باشد.

عواملی بعدی که به منظور مطرح کردن در برکلی به آن‌ها رسیدم، ساده بودند و
در پنل باعث بحث خاصی نمی‌شدند. احتمالا چند نفری با من موافقت می‌کردند
(یا از دیدگاه فرهنگی یک فنلاندی، آن‌ها مودب می‌ماندند). چیزهای خیلی
زیادی نیست که مردم حاضر باشند به خاطر آن جان شان را فدا کنند ولی روابط
اجتماعی بدون شک می‌تواند یکی از آن‌ها باشد.

نمونه‌هایی از روابط اجتماعی که می‌توانند باعث کم رنگ شدن انگیزه بقا
شوند، بسیارند. از رومئو و ژولیت در ادبیات بگیرید (که به خاطر چیز زمختی
مثل سکس نمردند بلکه به خاطر قطع رابطه اجتماعی، مرگ را انتخاب کردند) تا
سربازی وطن‌پرست که به خاطر کشور و خانواده‌اش (جامعه‌اش) مرگ را
می‌پذیرد. پس \dbquote{روابط اجتماعی} را به عنوان انگیزه شماره دو،
یادداشت کنید.

سومین انگیزه یا همان انگیزه نهایی، \dbquote{سرگرمی} است. شاید این
انگیزه به نظر مبتذل برسد، ولی شکی ندارم که نیروی فوق‌العاده‌ای
دارد. مردم هر روزه به خاطر کارهایی که تنها به خاطر تفریح انجام می‌دهند،
می‌میرند. مثلا پریدن از یک هواپیمای کاملا سالم فقط به خاطر احساس هیجان
ناشی از سقوط آزاد.

در عین حال سرگرمی، لازم نیست حتما پیش‌پا افتاده باشد. بازی شطرنج نوعی
سرگرمی است. همین طور است تلاش روشنفکرانه برای کشف اینکه جهان واقعا به
چه شیوه‌ای کار می‌کند. سرگرمی ممکن است تلاش افراد باشد برای کشف دنیاهای
تازه. به نظر من نیرویی که بتواند باعث شود یک نفر مشتاقانه در صندلی یک
سفینه بنشیند و با کلی مواد منفجره خیلی قوی به فضا پرتاب شود تا بتواند
زمین را از فضا نگاه کند، یک \dbquote{انگیزه حسابی} است.

حالا فهرست ما کامل شده: بقا،‌ وضعیت فرد در نظام اجتماع و سرگرمی. این ها
سه عاملی هستند که باعث می‌شوند ما کارهایی را بکنیم که داریم
می‌کنیم. بقیه عوامل چیزهایی هستند که احتمالا جامعه‌شناس‌ها به آن
\dbquote{عوامل معلول} می‌گویند: آن نوع عواملی که ناشی از عوامل پایه‌ای‌تر
هستند.

البته بحث ما چیزی است بیشتر \dbquote{این‌ها عواملی هستند که به آدم‌ها
  انگیزه‌ می‌دهند.} اگر بحث فقط همین بود که دیگر نمی‌شد آن را نظریه زندگی
نامید. مساله مهم این است که این سه عامل، ترتیبی طبیعی دارند و هر چیزی
که به نام زندگی شناخته‌ایم را شکل می‌دهند. صحبت این نیست که این سه عامل
به ما انگیزه می‌دهند بلکه باید توجه کنیم که این عوامل تعیین کننده
انگیزه‌های انواع دیگر موجودات هم هستند و به هر نوع رفتار زندگی‌گونه که
ما می‌شناسم قابل تعمیم هستند.

زنده بمان. اجتماعی بشو. تفریح کن. این یعنی پیشرفت و به هین دلیل هم هست
که \dbquote{فقط برای تفریح} را به عنوان اسم کتاب انتخاب کردیم. هر کاری
که ما می‌کنیم در نهایت به ابزاری برای تفریح بدل خواهد شد - حداقل اگر آن
رفتار فرصت کند تا به اندازه کافی تکامل پیدا کند.

باور نمی‌کنید؟

به این نگاه کنید که ما چگونه حیوانات را به \dbquote{بالاتر} و
\dbquote{پایین‌تر} طبقه‌بندی کرده‌ایم. همه حیوانات بقای خود را حفظ
کرده‌اند اما هر چه قدر که رده‌بندی \dbquote{بالا} بروید، شاهد موجودات
اجتماعی‌تر خواهید بود یعنی موجوداتی که روابط اجتماعی پیچیده‌تری با هم
دارند. در نهایت هم می‌رسیم به حیواناتی که می‌توانند تفریح کنند. مورچه با
اینکه روابط اجتماعی بسیار منظمی دارد، یک موجود رده بالا حساب نمی‌شود،
چون مثل گربه با غذایش بازی نمی‌کند و از رابطه جنسی هم لذت نمی‌برد.

چیزی ابتدایی (و لذت بخش) مثل رابطه جنسی را در نظر بگیرید. به نظر من
این رابطه به خودی خود یک انگیزه پایه‌ای نیست، بلکه نمونه‌ای عالی است از
اینکه یک رفتار ابتدایی چگونه در طول تکامل حیات، تحول پیدا کرده
است. هیچ شکی نیست که این رابطه در ابتدا رفتاری ساده برای حفظ بقای نسل
بوده است. به هرحال گیاهان هم به نوعی رابطه جنسی برقرار می‌کنند و احتمالا
میلیاردها سال قبل، این رابطه بین موجودات تک سلولی‌ای که بعدها تبدیل به
گیک‌ها و دیگر انواع انسان‌ها شده‌اند هم وجود داشته است. در این هم شکی
نیست که مدت‌ها قبل، رابطه جنسی از یک رابطه صرفا مربوط به بقا، به
رابطه‌ای اجتماعی تبدیل شده است. فقط انسان‌ها نیستند که برای ازدواج کلی
مراسم و قواعد اجتماعی دارند. به رقص پیچیده لک‌لک‌های
سندهیل\LFootnote{Sandhill} نگاه کنید و ببینید که چگونه در پی این
رقص‌ها، شریک دائمی زندگی‌شان را پیدا می‌کنند. در واقع این روزها مراسم
اجتماعی‌ای که با جفت‌یابی همراه شده آن قدر پیچیده و انرژی‌بر است که باید
برای آن کارکردی بسیار بیشتر از جفت‌یابی صرف قایل شد.

تفریح و سرگرمی؟ به شما اطمینان می‌دهم که این هم به بخشی از رابطه جنسی
تبدیل شده. نه فقط بین انسان‌ها که احتمالا در میان بسیاری از گونه‌های
\dbquote{برتر} هم رابطه جنسی با تفریح و سرگرمی‌ای همراه شده که آن را از
روند صرفا بقای نسلی خود خارج کرده است.

شما در همه جا می‌توانید شاهد تکامل انگیزه‌ها از بقا به سمت روابط اجتماعی
و سپس تفریح باشید. این بار جنگ را در نظر بگیرید: در ابتدا و در روزگاری
که تنها راه رسیدن به آب‌ آشامیدنی،‌ کشتن افرادی بوده که این منبع آب را
متعلق به خود می‌دانستند، هدف جنگ و کشتار، بقای نسل بود. مدت‌ها گذشت تا
جنگ تبدیل شد به روشی برای حفظ نظم اجتماعی و با ظهور سی.ان.ان.، جنگ به
یک سرگرمی تبدیل شد. چه دوستش داشته باشید و چه از آن متنفر باشید، این
روند در حال وقوع است.

کل تمدن هم در مقیاسی بزرگ‌تر از همین الگو پیروی می‌کند. تمدن در ابتدا
روشی بود برای زنده ماندن و قدرت‌گرفتن با استفاده از همکاری تعداد زیادی
از افراد. این امر هم مختص انسان‌ها نیست. خیلی از حیوانات و حتی گیاهان
هم برای دسترسی به شرایط زندگی بهتر، با یکدیگر زندگی و همکاری
می‌کنند. نکته جالب حرکت خود به خود جامعه از محوریت بقا به محوریت روابط
اجتماعی است. منظورم روندی است که طی آن جوامع انسانی برای کمک به
یکدیگر، راه‌های بزرگتر و بهتر و کانال‌های ارتباطی کاربردی‌تری می‌سازند تا
بتوانند روابط خود را گسترش دهند و از این طریق روند اجتماعی شدن خود را
تسریع می‌کنند.

در انتها جوامع به سمت سرگرمی محور بودن، به حرکت در می‌آیند. به
امپراتوری روم نگاه کنید که شهرت آن نه فقط به خاطر جاده‌های خوب و روابط
اجتماعی پیچیده است، که همچنین - به خصوص این روزها - به خاطر
خوش‌گذرانی‌ها و تفریحات شان نیز مشهورند.

یا اصلا آمریکای امروز را در نظر بگیرید. آیا کسی هست که بتواند ادعا کند
این صنعت عظیم فیلمسازی یا ساخت بازی‌های کامپیوتری چیزی به جز طلیعه‌داری
یک جامعه مبتنی بر سرگرمی است؟ این شرکت‌ها در چندین سال قبل فقط بازار
کوچکی پیش روی خود داشتند. اما حالا به عظیم‌ترین شرکت‌های ثروتمندترین
کشور جهان تبدیل شده‌اند.

چیزی که برای من به عنوان یک تکنولوژیست جالب است، پیگیری تکرار این الگو
در تکنولوژی‌ای است که خودمان می‌سازیم. ما اوایل دوران تکنولوژی مدرن را
عصر صنعتی می‌نامیم در حالی که در اصل باید عصر بقا از طریق تکنولوژی
نامیده شود. تکنولوژی تا دوران نه چندان دور فقط به منظور بقای راحت‌تر
استفاده می‌شد - توانایی پوشیدن لباس پشمی و انتقال سریع‌تر کالا. این یکی
از انگیزه‌های اصلی تکنولوژی بود.

ما عصر فعلی را عصر اطلاعات نامگذاری کرده‌ایم. این یک تغییر بزرگ است. ما
داریم از تکنولوژی برای ارتباطات و انتقال اطلاعات استفاده می‌کنیم که بر
خلاف انگیزه‌های مبتنی بر بقای سابق، یک فعالیت اجتماعی است. اینترنت و
این واقعیت که حجم زیادی از تکنولوژی این روزهای ما به سمت آن جریان پیدا
کرده، می‌تواند بهترین نشانه از این عصر باشد. معنای این نشانه این است که
لااقل بخش عمده‌ای از کشورهای صنعتی، خاصیت ادامه بقای تکنولوژی را
غیرقابل خدشه دیده‌اند و حالا به سراغ جنبه ارتباطی تکنولوژی رفته‌اند:
استفاده از تکنولوژی نه به منظور زندگی بهتر که به منظور بخشی جدا نشدنی
از زندگی اجتماعی.

هدف نهایی از همین حالا مشخص است؛ گذشتن از جامعه اطلاعاتی و رسیدن به
جامعه سرگرمی. جایی که اینترنت و ارتباط بیست و چهارساعته بی‌سیم غیرقابل
خدشه شود و دیگر اخبار مربوط به پیشرفت اینترنت، \dbquote{خبر} تلقی
نشود. در آن روز سیسکو یک بازار قدیمی است و شرکت دیزنی، صاحب جهان خواهد
بود. این دوره، احتمالا زیاد دور نیست.

\vspace*{20pt}

خب حالا همه این‌ها یعنی چه؟ احتمالا معنای زیادی ندارد. به هرحال نظریه
من درباب زندگی، قرار نیست شما را در مورد کارهایی که می‌کنید راهنمایی
کند. این نظریه حداکثر به این جا خواهد رسید که \dbquote{شاید بخواهی در
  برابرش مقاومت کنی ولی هدف نهایی زندگی، به هرحال تفریح خواهد بود.}

این نظریه تا حدی توضیح می‌دهد که چرا آدم‌ها با علاقه و اشتیاق روی
اینترنت به پروژه‌هایی مثل لینوکس می‌پردازند. برای من، مثل خیلی از آدم‌های
دیگر، لینوکس روشی بوده برای تلفیق دو مورد از انگیزه‌های پایه‌ای. حالا که
بقای ما تضمین شده، لینوکس دارد تفریح حاصل از تلاش فکری را با انگیزه
اجتماعی عضوی از یک حرکت دسته جمعی بودن تامین می‌کند. شاید خیلی از ما
یکدیگر را چهره به چهره ندیده باشیم، اما ایمیل برای ما چیزی بیشتر از یک
ابزار صرفا منتقل کننده اطلاعات است. پیوندهای دوستی و دیگر انواع ارتباط
اجتماعی می‌تواند با ایمیل هم شکل بگیرد.

این احتمالا به این معناست که اگر روزی با موجود هوشمند دیگری در کهکشان
روبرو شویم، احتمالا نخواهد گفت \dbquote{مرا به پیش رهبرتان راهنمایی
  کنید.} اولین جمله یک موجود هوشمندتر به ما، احتمالا این خواهد بود:
\dbquote{بیایید جشنی راه بیاندازیم رفقا.}

مطمئنا ممکن است من اشتباه کرده باشم.
