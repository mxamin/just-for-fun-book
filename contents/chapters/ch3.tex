\chapter[تولد یک سیستم‌عامل]{تولد یک سیستم‌عامل *}
\vfill
\begin{alertbox}
\textbf{* اخطار:}
\emph{در تمام فصل \textbf{تولد یک سیستم‌عامل} با زبان
  گیکی در سطح متوسط مواجه خواهید بود.}
\end{alertbox}

\section{بخش یکم}
بعضی از آدم‌ها، تاریخ را با ماشین‌هایی که داشته‌اند یا شغل‌های شان یا مکان
زندگی یا حتی عشق‌هایی که داشته‌اند به یاد می‌آورند. سال‌های زندگی من، با
کامپیوترها مشخص می‌شوند.

سال‌های نوجوانی را با سه کامپیوتر سپری کردم. اولی همان کمودور
\lr{VIC-20} فوق‌الذکر که از پدربزرگم به ارث برده بودم. این یکی از اولین
کامپیوترهای \dbquote{خانگی} بود. یکی از اجداد \lr{PC}های
امروزی. کامپیوتر کمودور ۶۴ به نوعی برادر بزرگتر همین کامپیوتر بود و هم
خانواده بعدی هم آمیگاها بودند که در اروپا محبوبیت بیشتری پیدا
کردند. البته هیچکدام از این کامپیوتر موفق نشدند به مقبولیتی که بعدها
کامپیوترهای خانگی دیگر مثل \lr{PC}ها یا حتی اپل \lr{II} - که هم دوره
\lr{VIC} من بود - دست پیدا کردند، برسند.

آن روزها که هنوز \lr{PC}ها اینقدر زیاد نشده بودند، تقریبا همه برنامه
نویسی‌های مربوط به کامپیوترهای خانگی با زبان اسمبلی انجام می‌شد (باور
نمی‌کنم که شروع کرده‌ام به گذاشتن \dbquote{آن روزها...} در اول
جمله‌هایم). کامپیوترها،‌ سیستم‌عامل‌های خانگی خود را داشتند که چیزی مشابه
سیستم‌عامل داس\RFootnote{\lr{DOS} یا سیستم عامل دیسک یکی از اولین سیستم
  عامل‌های متنی کامپیوترهای پی سی بود} برای \lr{PC}ها بودند. این
سیستم‌عامل‌ها یک بارگزار ساده برای برنامه‌ها و یک محیط برنامه‌نویسی
ابتدایی داشتند. در آن دوره، استانداردهای چندانی وجود نداشت و هر شرکت
سعی می‌کرد بازار را در اختیار خودش بگیرد. کمودور هم به این تلاش، شهره
بود.

وقتی هر کاری که می‌توانستم با \lr{VIC-20} بکنم را کردم، شروع کردم به پس‌انداز
کردن پول برای خرید کامپیوتر بعدی. این در زندگی من مساله مهمی بود. همان
طور که قبلا هم گفته‌ام، یادم نیست که کدام فامیل در کدام دوره در کجا
زندگی می‌کرده است و خیلی چیزهای دیگر را هم فراموش کرده‌ام ولی مسیری که
برای تصاحب کامپیوتر دوم‌م رفته‌ام را به این راحتی‌ها فراموش نخواهم کرد.

من از پول‌هایی که به عنوان هدیه کریسمس و هدیه تولد به من داده می‌شد، کمی
پس‌انداز داشتم (چون در ۲۸ دسامبر به دنیا آمده بودم، عملا هدایای کریسمس
و تولد با هم ادغام می‌شدند). مقداری پول هم از طریق کار تابستانی به
عنوان تمیزکننده پارک‌های هلسینکی به آن اضافه می‌شد. پارک‌های هلسینکی
فضاهای سبز مرتبی نیستند که به عنوان یک چشم انداز جذاب درست شده باشند
بلکه جنگل‌هایی هستند که هر طور خواسته‌اند، رشد کرده‌اند. کاری که ما باید
می‌کردیم این بود که شاخه‌های بیش از حد رشد کرده را ببریم یا چوب‌های خشک
را از روی زمین جمع کنیم. این کار جالبی بود - من همیشه طرفدار فعالیت‌های
درون فضای باز بوده‌ام. در یک دوره هم روزنامه‌پخش‌ می‌کردم. البته روزنامه‌
که نه، نامه‌های تبلیغاتی! حالا که به جریان فکر می‌کنم می‌بینم که هیچ وقت
خیلی اهل کار تابستانی نبوده‌ام ولی گاهی شغل‌های موقت داشته‌ام. احتمالا
بیشترین درآمد من از مستمری‌های مدرسه بوده است.

در فنلاند مرسوم است که مردم به مدارس کمک‌های مالی می‌کنند. حتی به مدارس
ابتدایی دولتی. از سال چهارم، این پول بر اساس چیزی که اهدا کننده در ذهن
داشته، به بچه‌ها می‌رسد. یادم هست که یک بار اهدا کننده خواسته بود که
پولش به محبوب‌ترین فرد کلاس برسد. این مساله در کلاس ششم بود و ما برای
انتخاب محبوب‌ترین فرد رای گیری کردیم. شاید لازم باشد اضافه کنم که برنده
من نبودم. آن پول، حدود ۲۰۰ مارک فنلاندی بود که حدود ۴۰ دلار می‌شود. این
در حقیقت پول زیادی نبود ولی برای یک کلاس ششمی که آن را به خاطر محبوب
بودن دریافت می‌کرد، زیاد به حساب می‌آمد.

این پول معمولا به کسی داده می‌شد که در یک درس یا رشته ورزشی بهتر از
بقیه باشد و بسیاری از جایزه‌ها هم مختص مدارس خاصی هستند یا از طریق دولت
در اختیار مدارس قرار داده می‌شوند. گاهی در طول زمان، این جوایز کمتر و
کمتر می‌شدند. فراموش نمی‌کنم که جایزه‌ای بود که ارزش مالی‌اش از یک پنی
بیشتر نبود. در اینجور مواقع،‌ مدرسه چیزی به این جایزه اضافه می‌کرد تا
مقدار آن با معنی‌تر شود اما به هرحال ارزش نهایی خیلی کم بود. این جوایز
بیشتر به این خاطر داده می‌شدند که سنت کمک مالی به دانش‌آموزان حفظ
شود. یکی از خوبی‌های فنلاند این است که سنت‌های آموزشی‌اش را جدی می‌گیرد.

من هرسال به خاطر \dbquote{مرد ریاضی} بودن، این مستمری‌ها و جوایز را
می‌گرفتم. در دبیرستان، جوایز بزرگتر بود. بزرگترین آن‌ها حدود ۵۰۰ دلار
بود و اکثر پول کامپیوتر دوم من هم از همین جا می‌آمد وگرنه پول تو جیبی‌
هفتگی من به کامپیوتر خریدن نمی‌رسد. راستی کمی پول هم از پدرم قرض کردم.

سال ۱۹۸۶ یا ۱۹۸۷ بود. من شانزده یا هفده ساله بودم. سال‌های بسکتبال را
پشت سر گذاشته بودم. وقت بسیار زیادی را صرف تصمیم‌گیری در این باره کردم
که چه کامپیوتری بخرم. قدیم‌‌ترها \lr{PC}ها کامپیوترهای چندان جذابی نبودند و
در نتیجه وقتی که درباره کامپیوتر آینده‌ام خیال‌پردازی می‌کردم، مطمئن بودم
که سراغ یک \lr{PC} نخواهم رفت.

من در نهایت سینکلر کیو.ال.\LFootnote{Sinclair QL} را انتخاب کردم که
احتمالا اکثر شما برای به یاد آوردن‌اش خیلی جوان هستید. جریان از این
قرار است که سینکلر یکی از اولین کامپیوترهای ۳۲ بیتی برای استفاده
کاربران خانگی بود. بنیانگذار شرکت، سر کلیو سینکلر\LFootnote{Sir Clive
  Sinclair}،‌ در واقع استیو وزنیاک\RFootnote{\lr{Steve Wosniak} - از
  بنیانگذاران و مغز فنی شرکت اپل} انگلستان بود. او سه کیت کامپیوتری
ساخت که در آمریکا با عنوان تیمکس\LFootnote{Timex} فروخته
می‌شدند. بعله!‌ همان شرکتی که ساعت‌های تیمکس را می‌سازد، کامپیوترهای
سینکلر را هم وارد آمریکا کرده و با نام تجاری خود به فروش رساند. اولین
سری به شکل کیت‌‌های آماده ساخت فروخته شده و سری‌های بعدی به شکل
کامپیوترهای آماده به کار.

سیستم‌عامل کامپیوترهای سینکلر کیو.داس خوانده می‌شد. آن دوران همه چیزش را
از حفظ بودم. این سیستم‌عامل برای یک کامپیوتر خاص نوشته شده بود و برای
آن روزها، از زبان بیسیک نسبتا پیشرفته‌ای پشتیبانی می‌کرد و گرافیک خوبی
هم داشت. چیزی من را بسیار به این سیستم‌عامل علاقمند کرده بود،
چندوظیفه‌گی\LFootnote{Multitask} آن بود: می‌توانستید چندین برنامه را به
شکل همزمان اجرا کنید. البته بخش بیسیک، قابلیت چند وظیفه‌گی را نداشت و
در هر لحظه فقط می‌شد یک برنامه بیسیک را اجرا کرد. اما اگر برنامه‌های خود
را به زبان اسمبلی می‌نوشتید، سیستم‌عامل می‌توانست از طریق تسهیم‌زمانی، آن
را به شکل همزمان با دیگر برنامه‌ها اجرا کند.

کامپیوتر یک تراشه ۶۸۰۸۰ هشت مگاهرتزی داشت که نسخه دوم و ارزان‌تر سری
۶۸۰۰۰ بود که تراشه‌ای ۳۲ بیتی با یک رابط ۱۶ بیتی به دنیای خارج بود؛
یعنی عملیات درون تراشه با ۳۲ بیت و تمام ارتباطات آن با دنیای بیرون
(مانند حافظه، دیسک‌ و ...) با ۱۶ بیت انجام می‌شد. از آن‌جایی که تراشه
تنها می‌توانست اطلاعات ۱۶ بیتی را از حافظه بخواند،‌ اینگونه عملیات
سریع‌تر از عملیات ۳۲ بیتی انجام می‌شدند. این طراحی بسیار مرسوم بود و حتی
این روزها هم در بسیاری از سیستم‌های جاسازی شده\RFootnote{\lr{Embedded
    System} به معنی سیستم هایی که داخل سیستم های دیگر جاسازی می شوند و
  آن‌ها را کنترل می کنند. مانند کامپیوتر مرکزی خودرو} و اتوموبیل‌ها از
همین معماری استفاده می‌شود.

تراشه ۶۸۰۸۰ که در کامپیوتر من استفاده شده بود، به جای ارتباطات ۱۶ بیتی
با دنیای خارج از پردازشگر مرکزی، از ارتباطات ۸ بیتی برای این منظور
استفاده می‌کرد. با وجود ارتباط ۸ بیتی با دنیای خارج، عملیات درونی
پردازشگر هنوز ۳۲ بیتی بود. این باعث می‌شد برنامه‌نویسی با آن لذت‌بخش‌تر
باشد.

من ۱۲۸ کیلوبایت - نه مگابایت - حافظه داشتم که برای زمان خودش حافظه
زیادی بود. کامپیوتر \lr{VIC-20}ی که این کامپیوتر جایگزین‌اش شده بود،
فقط سه و نیم کیلوبایت حافظه داشت. در عین حال از آنجایی که پرادزشگر
مرکزی ۳۲ بیتی بود، این کامپیوتر می‌توانست بدون هیچ مشکلی به کل حافظه
موجود دسترسی داشته باشد؛ مساله ای که پیش از این تصورش هم نمی‌رفت. این
دلیل اصلیی بود که من این کامپیوتر را انتخاب کردم. تکنولوژی جذاب بود و
من عاشق پردازشگر مرکزی‌اش بودم.

امیدوار بودم با خرید کامپیوتر از مغازه‌ای که آشنای یکی از دوستانم بود،‌
بتوانم تخفیف بگیرم اما متوجه شدم که کامپیوتر دلخواه من را ندارند و
برای دریافت آن باید کلی منتظر بمانم. توان انتظار نداشتم پس سری به
بزرگترین کتاب فروشی هلسینکی یعنی آکادمیسکا
بوکهاندلن\LFootnote{Akademiska Bokhandeln} زدم که بخشی را هم به فروش
کامپیوتر اختصاص داده بود. کامپیوترم را مستقیما از همانجا خریدم.

قیمت کامپیوتر نزدیک به ۲۰۰۰ دلار بود. این قانون مدت‌ها دوام داشت که
جدیدترین کامپیوترها حدود ۲۰۰۰ دلار قیمت داشتند. همین یکی دو ساله است
که این قانون از اعتبار افتاده. حالا می‌شود یک کامپیوتر شخصی جدید را با
۵۰۰ دلار خرید. مثل اتومبیل‌. کسی اتومبیلی زیر ۱۰۰۰۰ دلار نمی‌سازد. گاهی
اصلا ارزشش را ندارد. مطمئنا شرکت‌ها می‌توانند اتومبیل‌های ۷۰۰۰ دلاری
بسازند ولی استدلال آن‌ها این است که اگر کسی بتواند اتومبیل ۷۰۰۰ دلاری
بخرد، احتمالا ترجیح می‌دهد با دادن ۳۰۰۰ دلار بیشتر، اتومبیلی با
قابلیت‌های بهتر یا امکانات بیشتر مثل کیسه هوا دریافت کند. اگر
اتومبیل‌های جدید امروزی را با اتومبیل‌های پانزده سال پیش مقایسه کنید،
قیمت‌ها تقریبا برابر هستند. در حقیقت با در نظر گرفتن تورم، اتومبیل‌ها
کمی هم ارزان‌تر شده‌اند. ولی کیفیت بسیار بهتر شده است.

در مورد کامپیوترها هم مساله همین بود. وقتی کامپیوتر چیزی نبود که هر
کسی بخرد، قیمت‌اش از ۲۰۰۰ دلار پایین‌تر نمی‌آمد. اگر کامپیوترها گران‌تر
می‌شدند، شرکت‌ها دیگر نمی‌توانستند تعداد زیادی از آن‌ها را بفروشند. قیمت
دقیقا به اندازه‌ای بود که فروش برود ولی ارزان‌تر شدنش به زیاد شدن فروش
کمک خاصی نکند. مردم حاضر بودند ۲۰۰ دلار بیشتر را بدهند و کامپیوتر
بهتری بگیرند.

در دو سال اخیر، ساختن کامپیوتر بسیار ارزان‌تر شده است و قابلیت‌هایش نیز
بسیار پیشرفت کرده‌اند. شرکت‌ها افراد زیادی که حاضر بودند ۲۰۰ دلار بدهند
تا کامپیوتر کمی بهتر بخرند را از دست داده‌اند و چون دیگر فقط به خاطر
قابلیت‌های کمی بهتر نمی‌شود فروش را بالا برد، مجبور شده‌اند سر قیمت رقابت
کنند.

قبول می‌کنم: در ۱۹۸۷ یکی از دلایلی که باعث می‌شد سینکلر خوب فروش برود،
ظاهر باحال آن بود.

رنگ آن سیاه مات یک دست بود با یک صفحه کلید سیاه و زاویه‌های نود
درجه. شبیه کامپیوترهای پر زرق و برق پر از انحنا نبود. تلاش می‌کرد نهایت
کامپیوتر باشد. کیبرد تقریبا دو سه سانتی‌متر ارتفاع داشت چون بخشی از خود
کامپیوتر بود. اکثر کامپیوتر‌های آن دوره همین طور طراحی می‌شدند. در سمت
راست صفحه‌کلید، جایی که انتظار داریم صفحه‌کلید عددی باشد، یک حلقه
نامتناهی از نوار کاست قرار داشت. چیزی که فقط کامپیوترهای سینکلر آن را
استفاده کردند. کاربرد این وسیله شبیه دیسک گردان بود البته با این تفاوت
که به دلیل طراحی نواریش، برای رسیدن به اطلاعاتی که دنبال آن‌ها بودید،
باید نوار را تا سر اطلاعات مورد نظر می‌چرخاندید. بعدها مشخص شد که این
وسیله ایده خوبی نیست، چون اطمینان و راحتی دیسک‌ها را ندارد.

پس من نزدیک به ۲۰۰۰ دلار خرج کامپیوتر سینکلرم کردم. بیشترین کاری که با
آن می‌کردم، تمام کردن یک پروژه و رفتن سراغ پروژه بعدی بود. همیشه دنبال
یک کار جالب برای انجام دادن بودم. یک مفسر و کمپایلر زبان
فورت\LFootnote{Forth} داشتم تا با آن ور بروم. فورت زبان عجیبی بود که
دیگر کسی با آن کار نمی‌کند. یک زبان خاص و جالب که در دهه ۱۹۸۰ برای
کارهای متنوعی استفاده می‌شد. ولی به دلیل پیچیدگی‌های برنامه‌نویسی با آن
هیچ وقت تبدیل به زبانی مرسوم نشد و افراد غیرفنی از آن استفاده
نکردند. این زبان در عمل بی‌مصرف بود.

من برای خودم چند ابزار برنامه‌نویسی نوشتم. یکی از اولین چیزهایی هم که
برای دستگاهم خریدم، یک کارت توسعه دارای حافظه \lr{EEPROM} (حافظه فقط
خواندنی قابل پاک‌کردن و برنامه‌نویسی مجدد با برق) بود. این حافظه‌ای است
که با استفاده از یک دستگاه خاص می‌توانید چیزهایی را روی آن بنویسید و
حتی وقتی برق کامپیوتر را قطع می‌کنید، اطلاعات روی آن باقی می‌مانند. با
این دستگاه می‌توانستم ابزارهایی که خودم نوشته بودم را بدون اینکه مجبور
باشم هربار آن‌ها را در \lr{RAM} (حافظه با دسترسی اتفاقی) بارگزاری کنم، همیشه
دم دست داشته باشم. در عین حال با استفاده از این ابزار، حافظه ارزشمند
\lr{RAM} برای بقیه کارهای کامپیوتر باقی می‌ماند.

چیزی که من را به سیستم‌عامل‌ها علاقمند کرد: یک کنترل کننده فلاپی خریدم
تا مجبور نباشم از میکرودرایو خود سینکلر استفاده کنم اما درایوری که روی
این کنترل کننده بود چنگی به دل نمی‌زند و در نهایت خودم نشستم و کنترل
کننده آن را نوشتم. طی نوشتن این کنترل کننده به چند باگ\RFootnote{اشکال
  در سیست های کامپیوتری. برای اولین بار اشکالات کامپیوترها به خاطر گیر
  کردن حشرات در آن‌ها مشاهده می‌شدند و به همین دلیل از کلمه
  \dbquote{باگ} به معنی حشره برای اشاره به این مفهوم استفاده شده.} در
سیستم‌عامل هم پی بردم - یا لااقل به چند ناهماهنگی بین چیزی که راهنماها
ادعا می‌کردند سیستم‌عامل انجام می‌دهد و آنچه که واقعا انجام می‌داد. این‌ها
را کشف کردم چون برنامه‌ای که نوشته بودم درست کار نمی‌کرد.

کدهای من همیشه، اوم...، بدون نقص هستند. پس مطمئن هستم مشکل باید از جای
دیگری باشد. پس بررسی را ادامه دادم و سیستم‌عامل را
دیس‌اسمبل\RFootnote{\lr{Disassemble} - تبدیل کد اجرایی به کد
  اسبملی. نوعی مهندسی معکوس برای رسیدن به کد قابل تغییر از یک برنامه
  اجرایی} کردم.

می‌توانید کتاب‌هایی را بخرید که حاوی بخش‌هایی از کدهای سیستم‌عامل
باشند. این کمک می‌کند. همچنین نیازمند یک دیس‌اسمبلر هستید؛‌ ابزاری که
برنامه به زبان ماشین را می‌گیرد و آن را به زبان اسمبلی ترجمه می‌کند. این
برنامه هم کمک بزرگی است چون وقتی با زبان ماشین روبرو هستید، دنبال کردن
دستورات بسیار مشکل است. مثلا جهش‌ها فقط به آدرس‌های عددی اشاره می‌کنند و
پیگیری‌ آن‌ها دردسر زیادی دارد. یک دیس‌اسمبلر خوب، این آدرس‌های عددی را به
اسم‌های معنادارتری ترجمه می‌کند یا حتی به خود شما اجازه می‌دهد تا اسم‌های
مورد نظرتان را وارد کنید. در عین حال کمک می‌کند تا مجموعه‌ای از
دستورالعمل‌ها که کار خاصی انجام می‌دهند را شناسایی کنید. من هم دیس‌اسمبلر
خودم را داشتم که ترجمه‌های نسبتا خوبی انجام می‌داد و لیست‌های قابل فهمی
تولید می‌کرد. اگر برنامه کار نمی‌کرد می‌توانستم به آن بگویم که در طول
لیست جلو برود و از جای خاصی، اجرای برنامه را پیگیری کند و با اینکار
دقیقا می‌دیدم که سیستم عامل مشغول انجام چه کاری است. گاهی هم نه به خاطر
کشف باگ‌ها، که به خاطر درک بهتر اینکه چه چیزی در جریان است، از
دیس‌اسمبلر استفاده می‌کردم.

یکی از چیزهایی که در مورد تنفرم بودد، وضعیت \dbquote{فقط خواندنی}
سیستم‌عامل بود. نمی‌توانستید آن را تغییر بدهید. می‌شد کدهایی را به
بخش‌هایی از آن اضافه کرد ولی فقط به همان بخش‌هایی که از قبل این قابلیت
در آن‌ها تعبیه شده بود. بسیار بهتر می‌بود اگر می‌شد کلا سیستم‌عامل را با
یک سیستم‌عامل جدید جایگزین کرد. پیاده‌سازی سیستم‌عامل در حافظه
رام\RFootnote{\lr{ROM} - حافظه فقط خواندنی که یکبار روی آن می نویسید و
  بعد فقط از آن می خوانید. این حافظه معمولا برای بارگزاری برنامه ها یا
  سیستم عامل در کامپیوتر استفاده می‌شود.} (فقط خواندنی) ایده بدی است.

با وجودی چیزهایی که درباره شیفتگی تکنولوژیک فنلاندی‌ها گفتم، سینکلر
کیو.ال. نتوانست جای پای محکمی در بین هفتمین ملت بزرگ اروپا، پیدا
کند. به خاطر بازار کوچک سینکلر در فنلاند، هربار که می‌خواستید برای
ماشین فوق‌العاده و لبه‌تکنولوژی‌تان تجهیزات جانبی بخرید، مجبور بودید این
کار را با پست و از طریق انگلستان انجام دهید. اول باید سراغ کاتالوگ‌ها
می‌رفتید و به امید یافتن کسی که قطعه مورد نظر شما را بفروشد، آن‌ها را
زیر و رو می‌کردید. بعد باید چک‌های تضمینی به اسم فروشنده تهیه می‌کردید و
چند هفته‌ای برای دریافت جنس منتظر می‌ماندید (می‌بینید که هنوز دوره آمازون
و کارت‌های اعتباری شروع نشده بود). این دقیقا همان‌کاری بود که وقتی
می‌خواستم حافظه دستگاه‌ام را از ۱۲۸ کیلوبایت به ۶۴۰ کیلوبایت ارتقاء بدهم
انجام دادم. همین ماجرا وقتی که می‌خواستم یک اسمبلر بخرم تا کدهای
اسمبلی‌ام را به کدهای ماشین (صفر و یک) ترجمه کند و وقتی که یک ادیتور
خریدم تا از آن به عنوان ویرایشگر متن استفاده کنم،‌ تکرار شد.

اسمبلر و ادیتور به خوبی کار می‌کردند ولی هر دو روی میکرودرایو بودند و
نمی‌شد آن‌ها را به \lr{EEPROM} منتقل کرد. برای حل این مشکل، ادیتور و اسمبلر
خودم را نوشتم و از آن‌ها برای تمام کارهای برنامه‌نویسی استفاده کردم. هر
دو را با اسمبلی نوشتم که طبق استانداردهای امروزی، کار احمقانه‌ای
بوده. نوشتن به اسمبلی بسیار کندتر و پیچیده‌تر است و فکر کنم حل یک مساله
با اسمبلی صد برابر بیشتر از حل همان مساله با زبانی مثل سی طول بکشد که
آن روزها هم موجود بود.

من چند دستور به مفسر سینکلر اضافه کردم و در نتیجه اگر مثلا می‌خواستم
چیزی را ویرایش کنم فقط دستورش را صادر می‌کردم و یک لحظه بعد ویرایشگر
حاضر و آماده، زیر دستم بود. ادیتوری هم که خودم نوشته بودم، سریع‌تر از
ادیتوری بود که با ماشین به من داده شده بود. یک جورهایی مفتخر بودم که
برنامه‌ من می‌تواند با سرعت بیشتری کارکترها را روی صفحه بریزد. معمولا با
ماشینی مثل آن، زمان قابل توجهی طول می‌کشد تا صفحه پر از کاراکتر شود و
برای خالی کردن خط‌های جدید، شروع به حرکت به سمت بالا کند. من به این
افتخار می‌کردم که در ادیتور من حروف آن قدر سریع تایپ می‌شدند که حرکت
سریع صفحه به سمت بالا، به نمایشگر فرصت عملکرد صحیح نمی‌داد و کاراکترها
حین حرکت محو به نظر می‌رسیدند. این برای من مهم بود. این مساله باعث می‌شد
ماشین چابک‌تر به نظر برسید و من می‌دانستم که برای به دست آوردن این سرعت،
کلی کار کرده‌ام.

در این دوره آدم‌های زیادی نبودند که من بشناسم‌شان و به اندازه‌ من درگیر
کامپیوترها باشند. در مدرسه یک باشگاه کامپیوتر داشتیم ولی من وقت چندانی
در آن نمی‌گذراندم. آن‌جا بیشتر به درد بچه‌هایی می‌خورد که می‌خواستند با
کامپیوتر آشنا شوند. در کل دبیرستان من حدود ۲۵۰ دانش‌آموز وجود داشت و
بعید می‌دانم جز من کسی در آن مدرسه بوده باشد که پیش از ده سالگی با
کامپیوتری کار کرده باشد.

یکی از کارهایی که دوست داشتم با سینکلر کیو.ال. بکنم، نوشتن کپی بازی‌های
مشهور بود. من مشابه‌هایی برای بازی‌هایی که روی \lr{VIC-20} داشتم می‌نوشتم و
گاهی قابلیت‌های جدید هم به آن اضافه می‌کردم. البته معمولا بازی‌ها بهتر
نمی‌شدند: کامپیوتر بهتر شده بود ولی مفهوم‌ همان مفهوم قدیمی بود.

بازی مورد علاقه من، آسترویدز\RFootnote{\lr{Asteroids} - بازی‌ای که باید
  در آن با چرخاندن و گاز دادن و شلیک کردن، سفینه فضایی خود را در
  لابلای سنگ‌های فضایی و بشقاب پرنده‌ها زنده نگهدارید.} بود و هیچ وقت هم
نتوانستم کپی خوبی از آن را بنویسم. مشکل اینجا بود که در آن دوره همه
نمونه‌های خوب آستروئید،‌ با گرافیک برداری نوشته می‌شدند. این بازی‌ها به
جای اینکه از گرافیک مبتنی بر نقاط استفاده کنند، از شیوه‌ای استفاده
می‌کردند که لامپ‌های کاتدی بر اساس آن ساخته شده بودند؛ یعنی شلیک
الکترون‌ها از یک تفنگ الکترونی در پشت صفحه نمایش و منحرف کردن این
الکترون‌ها با استفاده از میدان‌های مغناطیسی. با این شیوه می‌شد به دقت و
وضوح بسیار بهتری رسید اما پیاده‌سازی آن ساده نبود. می‌توانستید یک مشابه
برای بازی‌های رده آستروئید بنویسید ولی روی کامپیوتری که قابلیت‌های
گرافیکی لازم را نداشت، امکان نداشت بشود کیفیت نمونه اصلی را به دست
آورد.

یادم هست که با اسمبلی، یک کپی از بازی پک‌من\RFootnote{\lr{PacMan} -
  بازی ای که در آن یک موجود زردرنگ با فرار از روح‌ها باید همه سکه های
  موجود در یک ماریچ را بخورد} نوشتم. اولین قدم این بود که ببینم
شخصیت‌های حاضر در پک‌من قرار است چه شکلی باشند. بعد سعی می‌کردید که آن
شخصیت را در یک ماتریس شانزده در شانزده جا دهید و رنگ آن‌را مشخص
کنید. اگر آدم هنرمندی بودید، نتیجه خوب از آب در می‌آمد. اما اگر - مثل
من - آقای بی‌هنر بودید، نتیجه چیزی شبیه به پسرعمومی بیمار پک‌من بود.

قبول! کپی من چندان جذاب نبود اما خودم به آن افتخار می‌کردم. برنامه حاصل
قابل بازی کردن بود و من آن را برای مجله‌ای که کدهای کامپیوتری را چاپ
می‌کرد ارسال کردم. من برنامه‌های دیگری را هم به مجلات فروخته‌ام و احساسم
این بوده که این کار طبیعی است.

نه. 

یکی از مشکلات این بود که کدها به اسمبلی نوشته شده بودند. به عبارت دیگر
اگر کوچکترین اشتباهی در تایپ کد از مجله می‌کردید، برنامه کار نمی‌کرد.

چند بازی هم از خودم نوشتم ولی نوشتن بازی مغز خاصی می‌خواهد. بازی
نیازمند بهره‌وری خیلی بالا از توانایی‌های کامپیوتر است و باید خیلی خیلی
در سخت‌افزار پایین بروید. این‌کار را می‌توانستم بکنم ولی ذهنیت بازی‌کن‌ها
را نداشتم. چیزی که یک بازی را دوست داشتنی می‌کند، معمولا سرعت یا گرافیک
بالای آن نیست. باید چیزی باشد که باعث شود شما به بازی بچسبید - چیزی که
باعث شود شما بازی را ادامه دهید. مثل فیلم‌های سینمایی. جلوه‌های ویژه یک
چیز هستند و چیزی که پشت بازی قرار دارد یک چیز دیگر. و هیچ وقت پشت
بازی‌هایی که من نوشتم، چیز خاصی نبود. بازی باید یک پیشرفت داشته باشد،
یک ایده. معمولا پیشرفت بازی این است که دائما سریع‌تر می‌شود. همان کاری
که پک‌من می‌کند. بعضی وقت‌ها هم مسیرها پیچیده‌تر می‌شوند یا هیولاها دقیق‌تر
شما را تعقیب می‌کنند.

یکی از نکاتی که من را جذب پک‌من می‌کرد، سر و کله زدن با این مساله بود که
چگونه باید شخصیت‌ها را حرکت دهیم، بدون اینکه حرکت آن‌ها چشمک‌زن یا دارای
پرش به نظر برسد. این چشمک‌زدن، مشکل اصلی بازی‌های کامپیوتری در قدیم بود
چون بدون سخت‌افزارهای خاص، حرکت کاراکترها به سادگی باعث چشمک‌زدن آن‌ها
می‌شود. در حالت طبیعی برای حرکت دادن یک کاراکتر، کاراکتر را از مکان اول
حذف می‌کنیم و بعد آن را در مکان دوم ترسیم می‌کنیم. اگر نتوانید زمان‌بندی
دقیقی برای اینکار داشته باشید، کاربران لحظه‌ای که هیچ شخصیتی روی صفحه
نیست را خواهند دید و این باعث چشمک‌زدن شخصیت شما خواهد شد. برای رفع این
مشکل چندین راه حل وجود دارد. می‌توانید اول شخصیت را در جای جدید بکشید و
بعد شخصیت قبلی را پاک کنید؛ اما باید توجه داشته باشید که آن قسمت‌هایی
از شخصیت قبلی که توسط شخصیت جدید پوشانده شده‌اند، پاک نشوند. با استفاده
از این روش، چشمک‌زدن لعنتی حذف می‌شود و در عین حال گاهی حرکت با دیدن
سایه‌ای از شخصیت در حال حرکت، بسیار هم مطبوع‌تر می‌شود. مغز تفسیر خوبی از
این جریان ارائه می‌دهد. حالا به جای چشمک‌زدن، احساس واقعی حرکت را
دارید. مشکل این روش این است که هزینه‌بر است و کلی از منابع سیستم را صرف
خودش می‌کند.

اینکه بازی‌ها دقیقا در لبه تکنولوژی قرار گرفته‌اند و اولین برنامه‌هایی
هستند که برنامه‌نویسان می‌نویسند، دلیلی دارد. بخشی از دلیل مربوط به این
واقعیت است که در بسیاری از مواقع، باهوش‌ترین برنامه‌نویس‌ها بچه‌های
پانزده، شانزده‌ ساله‌ای هستند که در اتاق‌خوابشان مشغول ور رفتن با
کامپیوتر هستند (این چیزی است که شانزده سال پیش به آن اعتقاد داشتم و
هنوز هم فکر می‌کنم درست است). اما دلیل دیگری هم برای پیشرو بودن بازی‌ها
هست: بازی‌ها سخت‌افزارها را به جلو هل می‌دهند.

اگر به کامپیوترهای امروزی نگاه کنید می‌بینید که برای کارهای معمول به
اندازه کافی سریع هستند. اما درست وقتی به محدودیت‌های کامپیوترها
پی‌می‌برید که سراغ بازی‌های هیجان‌انگیز جدید و بخصوص این بازی‌های سه بعدی
که این روزها مشهور شده‌اند بروید. اصولا بازی‌ها یکی از آن حوزه‌هایی هستند
که در آن به راحتی می‌توانید بگویید آیا کارها در حال انجام شدن در زمان
واقعی هستند یا خیر. در کار با ویرایش‌های متن، یکی دو ثانیه مکث در اینجا
و آنجا به کسی برنمی‌خورد اما در هنگام بازی اگر چیزی کمی بیشتر از یکدهم
ثانیه طول بکشید، آزار دهنده خواهد بود. بازی‌های آن دوره، بسیار ساده
بودند. این روزها برنامه‌نویسی بخش کوچکی از ساخت هر بازیی است. اگر ساخت
بازی را با فیلم‌سازی مقایسه کنید، برنامه‌نویسی در حد فیلم‌برداری است.

خب. من سینکلر کیو.ال. را برای سه سال داشتم. از دبیرستان تا دانشگاه
هلسینکی و سپس ارتش فنلاند. کامپیوتر خوبی بود ولی به هرحال دیگر آماده
بودم تا از آن جدا شوم. در سال آخر یا کمی زودتر، کمبودهایش را کشف کرده
بودم. پرازنده ۶۸۰۰۸ پردازشگر خوبی بود ولی من داشتم درباره نسل بعدی آن
یعنی ۶۸۰۲۰ چیز می‌خوانم و جذب مباحثی مثل مدیریت حافظه و صفحه‌بندی
می‌شدم. این کامپیوترها می‌توانستند کارهایی بکنند که وقتی در سطح نزدیک به
سخت‌افزار مشغول کار هستید، بسیار ارزشمندند.

یکی از چیزهایی که در سینکلر کیو.ال. بسیار ناراحتم می‌کرد این بود که با
وجود توانایی چندکارگی در سیستم عامل، به خاطر نبود حفاظت از حافظه‌‌های
مشترک، هر لحظه امکان فروریزی سیستم وجود داشت. هر وظیفه‌ای که کار
اشتباهی انجام می‌داد، کل سیستم دچار فروریزی می‌شد.

سینکلر کیو.ال. آخرین تلاش سر کلیو سینکلر برای طراحی و ساخت کامپیوتر
بود. دلیل اصلی هم عدم موفقیت شرکت از نظر اقتصادی بود. فنّآوری آن جذاب
بود ولی شرکت مشکلات تولید داشت و مشکلات کیفی به معنی شکست تبلیغاتی
است. در عین حال بازار هم در حال رقابتی‌تر شدن بود.

در دهه ۱۹۸۰ کم کم می‌شد تصور کرد که یک راننده تراموای معمولی هم بتواند
فقط برای کارهای ویرایش متنش، صاحب کامپیوتر شود و همه نشانه‌ها هم به نفع
کامپیوترهای شخصی یا همان \lr{PC} ها بودند. بعله!‌ کامپیوترهای اصیل
\lr{IBM PC} شروع به سرازیر شدن به قفسه‌های مغازه‌ها کردند و با وجود
نواقص فنی‌ای که داشتند، بسیار موفق ظاهر شدند. این کامپیوترهای همه جا
حاضر بژ، مهر تایید آی.بی.ام را داشتند و این چیز کمی نبود. یک جاذبه
دیگر: لوازم جانبی استاندارد و به راحتی قابل تهیه بودند.

من دائما درباره پردازشگرهای جدیدی که می‌توانستند پاسخگوی نیازهای من
باشند چیز می‌خواندم. برایم واضح بود که ۶۸۰۲۰ با وجود جذاب به نظر رسیدن،
به جایی نخواهد رسید. این امکان هم وجود داشت که یک پردازشگر جدید برای
سینکلرم بخریم و آن را ارتقاء دهم. آن روزها اینکار به معنی بازسازی
ماشین بود. سیستم‌عامل هم چیزی درباره مدیریت حافظه نمی‌دانست و در این
صورت مجبور بودم خودم آن را بنویسم. پس ماجرا این بود:
هووومممم.... اینکار قدم بزرگی است و خریدن یک پردازشگر جدید هم خرج
زیادی دارد.

و این جریان خریدهای دائمی برای کامپیوتر هم دردسر بزرگی بود. آن روزها
کاتالوگ درست و حسابیی وجود نداشت که کلیه تجهیزات سینکلر در آن باشند و
من بتوانم تلفن را بردارم و کمی حافظه بیشتر سفارش دهم. کار من شده بود
سفارش پستی از انگلیس (البته در مورد نرم‌افزار مشکلی نبود چون
نرم‌افزارهای مورد نیازم را خودم می‌نوشتم).

این دردسر یک جنبه مثبت هم داشت. وقتی قرار شد از شر ماشین خلاص شوم،
تصمیم‌ گرفتم قطعات اضافی آن را هم بفروشم - یعنی دیسک سختی که خریده بودم
چون حتی یک لحظه دیگر نمی‌توانستم میکرودرایو را تحمل کنم و رم اضافه‌ای که
داشتم. مردم در خیابان صف نکشیده بودند تا وسایل دست دوم سینکلر من را
بخرند و تنها روش این بود که در یک مجله کامپیوتری آگهی بدهم و دعا
کنم. همین‌جا بود که دوست خوبم جوکو ویروماکی\LFootnote{Jouko Vierumaki}
را دیدم که بعدها معلوم شد به جز من احتمالا تنها کسی در فنلاند است که
یک سینکلر کیو.ال. دارد. او به آگهی من جواب داد و با یک قطار از
لاهتی\LFootnote{Lahti} به شهر ما آمد تا قطعات سینکلر من را بخرد. و
همین‌جا بود که من را با اسنوکر آشنا کرد.

\section{بخش دوم}
در سال اول دانشکده، سینکلر روی میز کنار تخت بود. درست کنار پنجره رو به
پیترسگاتان\LFootnote{Petersgatan}، ولی چندان برنامه‌ای با
آن ننوشتم. بخشی از این مساله بر می‌گشت به علاقه‌ام به اینکه روی درس‌ها
تمرکز کنم ولی شاید دلیل اصلی این بود که دچار کمبود پروژه شده
بودم. کمبود پروژه که داشته باشید، دچار کمبود انگیزه هم می‌شوید. در این
حالت سعی می‌کنید سراغ چیزی بروید که به شما انگیزه بدهد.

به نظرم آن دوره بهترین وقت بود که در ارتش ثبت نام کنم چون به هرحال
باید روزی اینکار را می‌کردم. نوزده ساله بودم و ناراضی از ضعف‌های
کامپیوترم و پروژه خاصی هم برای انجام نداشتم. سوار قطاری شدم و به
لاپلند رفتم.

قبلا برایتان گفته‌ام که چقدر در این مورد که در ارتش چه می‌گذرد و به طور
خاص در مورد نیازهای فیزیکی آن، بی‌اطلاع‌ بودم. بعد از یازده ماه در ارتش
بودن و تقلا کردن با تجهیزات نظامی، احساس می‌کنم کاملا حق دارم تا بقیه
عمرم را در آرامش و بی‌حرکتی‌ای بگذرانم که تنها ورزشش وارد کد از طریق
صفحه کلید و ضرب گرفتن روی لیوان‌های آبجو باشد (در واقع اولین فعالیت‌های
مشابه ورزش، تقریبا ده سال بعد از زمان خلاصی از خدمت نظامی بود که دیوید
مرا قانع کرد با او در امواج خروشان هاف‌مون بی\LFootnote{Half Moon Bay}
به بوگی‌سواری\RFootnote{شکل ساده ای از موج سواری که با قرار دادن یک
  تخته کوچک در زیر سینه انجام می‌شود.} بروم. من تقریبا غرق شدم و پاهایم
برای چند روزی درد می‌کرد.)

نظام وظیفه در ۷ می ۱۹۹۰ تمام شد. هرچند که تاو برای تان خواهد گفت که من
نمی‌توانم تاریخ ازدواج‌مان را به یاد بیاورم، اما تاریخ پایان سربازی، هیچ
وقت از یادم نخواهد رفت.

اولین کاری که می‌خواستم بکنم، آوردن یک گربه بود.

دوستی داشتم که گربه‌اش چند هفته قبل فارغ شده بود و من یکی از بچه
گربه‌های باقیمانده را برداشتم. یک بچه گربه سفید، مذکر، زیبا و به خاطر
گذراندن اولین هفته‌های زندگی در خارج از خانه، قادر به زندگی درون و
بیرون آپارتمان مادرم. اسمش را رندی\LFootnote{Mithrandir} گذاشتم که
مخفف میتراندیر76، جادوگر خوب رمان ارباب حلقه‌ها است. حالا او یازده‌ ساله
است و مثل صاحب‌اش کاملا به سبک زندگی در کالیفرنیا عادت کرده.

نه، فکر نکنم کل آن تابستان کار مفیدی کرده باشم. کلاس‌های تابستان تا
پاییز شروع نمی‌شدند. کامپیوترم چنگی به دل نمی‌زد پس اکثر اوقات با
کت‌حوله‌ای کهنه‌ام در خانه می‌گشتم یا با رندی بازی می‌کردم یا در موارد
معدودی با دوستان بیرون می‌رفتم تا آن‌ها بتوانند به تلاش‌های من در بولینگ
و اسنوکر نخودی بخندند. قبول! گاهی هم در مورد کامپیوتر آینده‌ام
خیال‌پردازی می‌کردم.

من با مساله بغرنج یک گیک روبرو بودم. مثل هر منزه‌طلب کامپیوتری که با
۶۸۰۰۸ بزرگ شده باشد، \lr{PC} را تحقیر می‌کردم اما وقتی در ۱۹۸۶ تراشه‌های
۳۸۶ بیرون آمدند، \lr{PC}ها کم کم شروع کردند به جذاب شدن. آن‌ها
می‌توانستند هر کاری که ۶۸۰۲۰ قادر بود بکند را انجام دهند و در ۱۹۹۰ هم
تولید انبوه و معرفی نمونه‌های ارزان و سازگار با این کامپیوترها، باعث شد
قیمت‌آن‌ها شدیدا افت کند. من شدیدا حواسم به مسایل پولی بود چون هیچ پولی
نداشتم. پس این کامپیوتر کامپیوتری بود که من می‌خواستم. هم‌چنین به خاطر
پر شدن بازار از \lr{PC}، قطعات جانبی آن‌ها هم به آسانی یافت
می‌شد. علی‌الخصوص وقتی صحبت از سخت‌افزار بود، من چیزی می‌خواستم که
استاندارد باشد.

تصمیم گرفتم این پرش را انجام دهم و به سراغ سخت‌افزار جدید بروم. یادگیری
و کار با یک پردازشگر جدید، مفرح بود. این موقعی بود که شروع به فروش
قطعات سینکلرم کردم.

هر کسی کتابی دارد که زندگی‌اش را تغییر داده. انجیل. سرمایه. سه‌شنبه‌ها با
ماری. هر چیزی که لازم است بدانم را در مهدکودک یاد گرفته‌ام. انواع و
اقسام کتا‌ب‌ها (مخلصانه آرزو دارم با خواندن مقدمه این کتاب و نظریه من در
مورد معنای زندگی، شما تصمیم بگیرید تا این کتاب را به عنوان کتاب متحول
کننده خود نام ببرید). کتابی که من را به مرحله جدیدی پرتاب کرد،
سیستم‌های عامل: طراحی و اجرا نوشته آندرو
س. تاننباوم\LFootnote{Operating Systems: Design and Implementation, by
  Andrew S. Tanenbaum} بود.

من برای کلاس‌های پاییز ثبت‌نام کرده بودم و چیزی که بیشتر از همه انتظارش
را می‌کشیدم کلاس برنامه‌نویسی زبان سی و سیستم عامل یونیکس بود. برای
آماده شدن برای این کلاس‌، کتاب ذکر شده را در تابستان خریدم با این امید
که با پیش‌‌مطالعه به سر کلاس درس بروم. در این کتاب، آندرو تاننباوم،
استاد دانشگاه آمستردام درباره مینیکس\LFootnote{Minix} که یک ابزار کمک
آموزشی یونیکس است که خودش نوشته، صحبت می‌کند. مینیکس همچنین یک مشابه
جمع و جور برای یونیکس است. درست بعد از خواندن مقدمه و درک فلسفه پشت
یونیکس و اینکه این سیستم‌عامل چقدر قدرتمند، تمیز، زیبا و توانا برای
انجام کارهای مختلف است، تصمیم گرفتم روی ماشین آینده‌ام یونیکس نصب
کنم. البته من باید مینیکس نصب می‌کردم چون تنها نسخه‌ای بود که دیدم واقعا
کاربردی است.

با شروع به فهم یونیکس، در خودم تلنگری حس کردم و صراحتا بگویم که اثر
این تلنگر هیچ وقت فروکش نکرد (امیدوارم شما هم بتوانید همین حرف‌ را
درباره چیزی بگویید).

\section{بخش سوم}
سال تحصیلی‌ای که در پاییز ۱۹۹۰ شروع شد، اولین سالی بود که در آن،
دانشگاه هلسینکی از یونیکس، سیستم‌عامل قدرتمندی که در اواخر دهه ۱۹۶۰ در
آزمایشگاه‌های بل ساخته شده اما در جای دیگری توسعه یافته بود، استفاده
می‌کرد. در اولین سال تحصیل من، ما یک \lr{VAX} داشتیم که دارای سیستم‌عامل
\lr{VMS} بود. این سیستم وحشتناک بود و محال بود کسی با خودش بگوید
\dbquote{وای!  کاش یکی از این‌ها را در خانه داشتم.} در عوض همه
می‌پرسیدند \dbquote{اوه! حالا این کار را باید چطور انجام دهم؟.} استفاده
از آن مشکل بود. ابزارهای زیادی نداشت. با آن نمی‌شد به این راحتی‌ها به
اینترنت که روی یونیکس پیاده سازی شده بود، وصل شد. حتی به سادگی
نمی‌توانستید متوجه شوید که حجم یک فایل چقدر است. می‌پذیرم که \lr{VMS}
برای یکسری از مقاصد، مثلا بانک‌های اطلاعات بسیار خوب بود اما از آن نوع
سیستم‌عامل‌هایی نبود که برای شان هیجان داشته باشید.

دانشگاه فهمیده بود که باید به سراغ چیزهای جدید برود. دنیای دانشگاهی
شیفته یونیکس بود و دانشگاه هم یک \lr{Micro VAX} خرید که
اولتریکس\LFootnote{Ultrix} یا نسخه یونیکس شرکت دیجیتال اکوییپمنت را
اجرا می‌کرد. این راهی بود برای غوطه ور شدن در محیط یونیکس.

با خواندن کتاب آندرو تاننباوم، و یاد گرفتن چیزهایی که اگر تراشه ۳۸۶
داشتم می‌توانستم به سراغ‌شان بروم، بیشتر و بیشتر مشتاق تجربه دنیای
یونیکس می‌شدم. هیچ راهی نبود که ۱۸۰۰۰ مارک فنلاندی جور کنم و یکی
بخرم. می‌دانستم که اگر ترم پاییزی شروع شود، می‌توانم تا وقتی که کامپیوتر
شخصی خودم که بتواند یونیکس اجرا کند را نخریده‌ام، می‌توانم از طریق
سینکلرم به کامپیوتر یونیکس دانشگاه متصل شوم.

پس آن تابستان دو کار اصلی کردم: هیچ چیز و خواندن ۷۱۹ صفحه کتاب
سیستم‌های عامل: طراحی و اجرا. این کتاب جلد قرمز یک جورهایی روی تخت من
زندگی می‌کرد.

دانشگاه هلسینکی، نسخه شانزده کاربره \lr{Micro VAX} را خریده بود و این
یعنی فقط ۳۲ نفر حق داشتند در کلاس \dbquote{سی و یونیکس} ثبت نام کنند؛
احتمالا اینگونه محاسبه کرده بودند که ۱۶ نفر صبح با سیستم کار کنند و ۱۶
نفر شب. معلم هم مثل بقیه ما در یونیکس تازه‌کار بود. از قبل این را اعلام
کرد و در نتیجه مشکلی نداشتیم. او کتاب درسی را یک فصل جلوتر از
دانشجویان می‌خواند در حالی که بعضی از دانشجویان سه فصل از کلاس جلو
بودند. این که بچه‌ها سوال‌هایی بکنند که به دو سه فصل آینده مربوط شود تا
ببینند آیا معلم تا آن‌جا را خوانده است یا نه، به یک بازی تبدیل شده بود.

همه ما بچه‌هایی در جنگل‌های یونیکس بودیم و کلاس هم هر جلسه پیش‌
می‌رفت. چیزی که از این کلاس‌ها مشخص بود این بود که پشت یونیکس، یک فلسفه
نهفته است. این نکته را از همان اولین ساعت کلاس می‌شد فهمید. بقیه کلاس،
تنها به ارائه جزییات می‌پرداخت.

چیزی که یونیکس را ممتاز می‌کند، ایده‌های بنیادینی است که این سیستم‌عامل
به دنبال‌شان است. این سیستم‌عامل تمیز و زیبا است. از حالت‌های خاص اجتناب
می‌کند. یونیکس مفهوم پروسس را عمده می‌کند - پروسس هر چیزی است که کاری
انجام می‌دهد. بگذارید یک مثال ساده بزنم. در یونیکس پوسته فرمان، چیزی که
برای دستور دادن به سیستم‌عامل در آن دستوراتی را وارد می‌کنید، بخشی از
خود سیستم‌عامل نیست (در حالی که مثلا در داس\RFootnote{سیستم عامل
  دیسک. یکی از اولین سیستم عامل های متنی که برای کامپیوترهای پی سی
  ساخته شد.} این طور است). این پوسته فقط یک وظیفه\LFootnote{Task}
است. مثل هر وظیفه دیگر. فقط مساله این است که این وظیفه دستورات را از
صفحه‌کلید می‌خواند و خروجی را روی نمایشگر نشان می‌دهد. هر چیزی که در
یونیکس کاری می‌کند، یک پروسس است. علاوه بر این، فایل‌ها را هم دارید.

این طراحی ساده همان چیزی بود که من و خیلی‌های دیگر را (حداقل بین ما
گیک‌ها) فریفته یونیکس کرد. تقریبا هر کاری که در یونیکس می‌کنید تنها از
شش عمل ساده ساخته شده (که \textbf{فراخوانی‌های سیستمی}\LFootnote{System
  Call} نامیده می‌شوند چون فراخوانی‌هایی هستند که از سیستم‌عامل درخواست
انجام کارها را می‌کنند). شما می‌توانید با استفاده از این شش فراخوانی
سیستمی، تقریبا هر چیزی بنویسید و هر کاری بکنید.

مفهوم \textbf{فورک}\LFootnote{Fork} یکی از عملیات های پایه‌ای یونیکس
است. وقتی پروسه فورک می‌کند، یک کپی کاملا مشابه از خودش را می‌سازد. با
این‌کار دو کپی کاملا مشابه از یک چیز دارید. کپی فرزند، معمولا یک پروسه
دیگر را اجرا می‌کند - خودش را با یک برنامه‌دیگر جایگزین می‌کند. و این
دومین عمل اصلی است. چهار عمل اصلی دیگر عبارت هستند از: باز کردن، بستن،
خواندن و نوشتن و همه آن‌ها روی فایل عمل می‌کنند. این شش عمل اصلی، عناصر
تشکیل دهنده سیستم‌عامل یونیکس هستند.

بدون شک هزاران فراخوانی سیستم دیگر وجود دارند تا همه جزییات را پوشش
دهند، ولی وقتی که شش فراخوانی اصلی را درک کردید، یونیکس را
فهمیده‌اید. یکی از زیبایی‌های یونیکس همین است که بفهمید برای انجام
کارهای پیچیده، نیازی به رابط‌های پیچیده ندارید. با ترکیب متقابل اجزای
ساده، می‌توان به هر مقدار پیچیدگی‌ رسید. کاری که باید کرد ایجاد کانال‌های
ارتباطی (که در زبان یونیکسی، پایپ\RFootnote{\lr{Pipe} که با علامت
  \code{|} نمایش داده می‌شود.} نامیده می‌شوند) بین پروسه‌های ساده به
منظور حل مسایل پیچیده است.

یک سیستم زشت، سیستمی است که برای حل هر مساله، یک رابطه پیچیده داشته
باشد. یونیکس درست برعکس است و به شما آجرهایی را می‌دهد که با آن‌ها
می‌توانید هر چیزی بسازید. این دقیقا معنای طراحی یک سیستم تمیز است.

همین مساله در مورد زبان‌ها هم صادق است. انگلیسی بیست و شش حرف دارد که
می‌توانید با استفاده از آن‌ها هر چیزی بنویسید. در مقابل چینی را داریم که
برای هر چیزی که بخواهید بگویید، یک علامت مجزا دارد. در چینی با پیچیدگی
شروع می‌کنید و امکان ترکیب چیزهای پیچیده با هم، بسیار اندک است. جریان
شبیه رویکرد \lr{VMS} است که برای هر کار، عملیات پیچیده و جذابی دارد اما
امکان استفاده از این اجزا به شکلی جز طراحی اولیه، وجود ندارد. ویندوز
هم به همین شیوه طراحی شده.

در مقابل یونیکس بنا به فلسفه \dbquote{کوچک زیبا است} پایه‌ریزی
شده. اینجا آجرهای کوچک و ساده‌ای دارید که با کنار هم چیدن آن‌ها می‌توانید
به نهایت پیچیدگی برای بیان دقیق آن چیزی که نیاز دارید، برسید.

به هرحال این همان روشی که فیزیک هم بر اساس آن کار می‌کند. در فیزیک هم
به دنبال قوانین پایه‌ای می‌گردیم که منطقا باید ساده و مفید
باشند. پیچیدگی جهان محصول ترکیب‌های شگفت‌انگیز این قوانین ساده با یکدیگر
است و نه محصول پیچیدگی خود قوانین حاکم بر جهان.

سادگی یوینکس خود به خود به وجود نیامده. یونیکس و مفهوم عملیات پایه‌ای
ساده‌اش با زحمت و دردسر توسط دنیس ریچی\LFootnote{Dennis Rickie} و کن
تامپسون\LFootnote{Ken Thompson} در آزمایشگاه‌های بل طراحی و نوشته
شد. به هیچ وجه نباید سادگی را با آسان بودن اشتباه گرفت. برای رسیدن به
سادگی نیازمند طراحی و سلیقه خوب هستیم.

برگردیم به مثال زبان‌: نوشتار تصویری و زبان‌های مبتنی بر اشکال، مثل
چینی، زودتر به وجود می‌آیند و \dbquote{ساده‌تر} هستند چون استفاده از
حروف به عنوان پایه‌های نوشتار، نیازمند تفکری انتزاعی‌تر است. به همین
ترتیب نباید سادگی یونیکس را حاصل پیشرفته‌ نبودن آن دانست - اتفاقا مساله
برعکس است.

توجه کنید که نمی‌گویم دلیلی اصلی به وجود آمدن یونیکس یک چیز خیلی
پیشرفته بود. مثل خیلی چیزهای دیگر در دنیای کامپیوتر، این یکی هم به
عنوان یک بازی شروع شد. یک نفر بود که می‌خواست روی
پی.دی.پی-۱۱\RFootnote{\lr{PDP-11} یکی از مینی‌کامپیوترهای اولیه ۱۶ بیتی
  که در طول دهه هفتاد بسیار محبوب بود.} بازی کند. یونیکس این طوری شروع
شد: پروژه شخصی دنیس و کن تا بتوانند جنگ کهکشان‌ها بازی کنند. از آن‌جایی
که این سیستم‌عامل چیز جدی‌ای حساب نمی‌شد، \lr{AT\&T} هم به آن نگاه تجاری
نداشت. در واقع \lr{AT\&T} یک شرکت انحصاری تنظیم شده بود و چیزی که اصلا
نمی‌توانست به سراغش برود، فروش کامپیوتر بود. به همین خاطر کسانی که
یونیکس را نوشتند، برنامه و کدهای منبع‌اش را به رایگان در اختیار دیگران
و بخصوص دانشگاه‌ها قرار دادند. این یک قدم بزرگ بود.

این‌ها باعث شدند تا یونیکس به عنوان یک پروژه بزرگ دانشگاهی مطرح شود. در
سال ۱۹۸۴، بالاخره \lr{AT\&T} اجازه پیدا کرد تا در تجارت کامپیوتر هم
سهیم شود، ولی تا آن موقع سال‌ها بود که دانشمندان زیادی - بخصوص در
دانشگاه کالیفرنیا-برکلی - تحت راهنمایی کسانی مثل بیل
جوی\LFootnote{Bill Joy} و مارشال کرک مکوییسک\LFootnote{Marchall Kirk
  McKuisk} روی این سیستم‌عامل کار کرده و آن را توسعه داده بودند. آدم‌ها
الزاما برای مستند کردن کارهایی که می‌کنند وقت چندانی صرف نمی‌کنند.

اما در اوایل دهه ۱۹۹۰، یونیکس به سیستم‌عامل شماره یک همه سوپرکامپیوترها
و سرورها تبدیل شده بود. این بازار بزرگی بود. یکی از مشکلات این بود که
حالا دیگر نسخه‌های مختلف این سیستم‌عامل مشغول رقابت با یکدیگر
بودند. بعضی از نسخه‌ها با وفاداری خاصی به کدهای پایه \lr{AT\&T} رشد
کرده بودند (و خانواده \lr{System V} خوانده می‌شدند). بعضی‌ها از نسخه
\lr{BSD}\RFootnote{\lr{Berkeley Software Distribution} که یکی از اولین
  سیستم عامل های مشابه یونیکس بود و هنوز هم انتخابی عالی برای سرورها
  به حساب می‌آید.} که در دانشگاه کالیفرنیا-برکلی توسعه داده شده بود
مشتق شده بودند و گروهی هم بودند که ترکیبی از این دو به حساب می‌آمدند.

یکی از مشتقات \lr{BSD} به طور خاص قابل ذکر است؛ پروژه \lr{386BSD} که
توسط بیل جولیتز\LFootnote{Bill Jolitz} بر اساس کد پایه \lr{BSD} نوشته
شده و به رایگان در اینترنت توزیع شده بود. این پروژه بعدها چند شاخه شد
و نسخه‌های آزاد \lr{BSD} (از جمله \lr{NetBSD}، \lr{FreeBSD} و
\lr{OpenBSD}) از آن به وجود آمد و توجه زیادی را در دنیای یونیکس به
خودش جلب کرد.

به هین دلیل بود که \lr{AT\&T} تازه از خواب بیدار شد و علیه دانشگاه
کالیفرنیا-برکلی شکایت کرد. کد اصلی متعلق به \lr{AT\&T} بود ولی بعدا
کارهای بسیاری در دانشگاه برکلی روی آن انجام شده بود. هیات مدیره
دانشگاه مدعی بود که دانشگاه حق توزیع یا فروش ارزان نسخه یونیکس خودش را
دارد. آن‌ها نشان دادند که آن قدر کد را تغییر داده‌اند که عملا دیگر
ردپاهای بسیار اندکی از کد اولیه در آن باقی مانده است و بسیاری از
قسمت‌ها بازنویسی شده‌اند. این پرونده حقوقی بعد از اینکه شرکت
ناول\LFootnote{Novell} سیستم‌عامل یونیکس را از \lr{AT\&T} خرید، با
توافق بسته شده. البته به این شرط که بخش‌هایی از کد که \lr{AT\&T} به شکل
عمومی منتشر کرده بود، از برنامه فعلی حذف شود.

در این حین، این دعواهای حقوقی دستاویزی شد برای یک بچه جدید تا با
استفاده از فرصت رشد کند و پراکنده شود. عملا، این مهلتی بود برای لینوکس
تا بازار را تصاحب کند. البته دارم از خودم جلو می‌افتم.

حالا که از موضوع پرت شدیم، می‌خواهم یک نکته دیگر را هم توضیح
دهم. یونیکس مشهور است به اینکه آدم‌های عجیب و غریب و حاشیه‌ای دنیای
کامپیوتر را به خودش جذب می‌کند. علیه این شهرت نمی‌شود حرف زد چون درست
است.

صادقانه باید بگویم که کلی آدم دیوانه در دنیای یونیکس هستند. نه
دیوانه‌های زنجیری. نه از آن دیوانه‌ها که سگ همسایه‌شان را مسموم
می‌کنند. منظورم آدم‌هایی با شیوه زیست بسیار متفاوت است.

به یاد بیاورید که فعال‌ترین سال‌های یونیکس، دهه‌های ۱۹۶۰ و اوایل ۱۹۷۰
بود. دوره‌ای که من در سبد لباس‌های چرک آپارتمان مادربزرگم خواب
بودم. این‌ها آدم‌هایی بودند که از گل نیرو می‌گرفتند - البته آدم‌های
فنی. بخش عمده فلسفه یونیکس-باید-آزاد-باشد، مربوط به شرایط آن دوره است
و نه نظریات سیستم‌عامل. آن دوره دوره ایده‌آل گرایی عمومی
بود. انقلاب. آزادی از قدرت. عشق آزاد (که من از دست دادمش هرچند که اگر
هم درباره‌اش می‌دانستم، نمی‌دانستم باید با آن چکار کنم). و باز بودن نسبی
یونیکس، هرچقدر هم که به نبود انگیزه‌های مالی مربوط شود، باعث شده بود تا
این سیستم‌عامل برای آن مردم جذاب شود.

اولین باری که من با این جنبه از یونیکس آشنا شدم، احتمالا حوالی ۱۹۹۱ و
به همت لارس ویرزنیوس\LFootnote{Lars Wirzenius} بود که من را با خودش به
یک همایش در دانشگاه پلی‌تکنیک هلسینکی برد (که همان طور که همه می‌دانند،
نه در هلسینکی که در کنار مرز اسپو\RFootnote{\lr{Espoo} - شهری در
  فنلاند که از مراکز تکنولوژی این کشور است و شرکت نوکیا هم در آن قرار
  داشت} قرار دارد- آن‌ها می‌خواستند ولو اگر شده فقط با اسم، به هلسینکی
مشهور متصل باشند). سخنران ریچارد استالمن\RFootnote{\lr{\lr{Richard
      Stallman}} - از بنیانگذاران جنبش آزادی نرم افزار و از مبلغان
  فعال این فلسفه که لقب پیامبر این جنبش را بر دوش می کشد.} بود.

ریچارد استالمن خدای نرم‌افزار آزاد\LFootnote{Free Software} است. او در
۱۹۸۴ شروع به کار روی یک جایگزین و مشابه یونیکس کرد و آن را سیستم گنو
(\lr{GNU}) نامید. گنو خلاصه \textbf{گنو یونیکس نیست}\LFootnote{Gnu is
  Not Unix} و یکی از چندین خلاصه بازگشتی‌ای است که یکی از حرف‌هایش، به
خودش باز می‌گردد. این نوعی شوخی کامپیوتری است که غیرکامپیوتری‌ها از آن
سر در نمی‌آورند. بودن با ما گیک‌ها خیلی مفرح است.

از این مهمتر، آر.ام.اس. (او ترجیح می‌دهد این طور صدایش بزنند)، نویسنده
\textbf{بیانیه نرم‌افزار آزاد}\LFootnote{Free Software Mnaifesto} و
همچنین لیسانس کپی‌رایت نرم‌افزار آزاد یا همان \lr{General Public
  License} است که به طور خلاصه \lr{GPL} خوانده می‌شود. عملا او پیشرو
مفهوم نرم‌افزار آزاد به عنوان یک امر تصمیم‌گیری شده و نه اتفاقی است. قبل
از او، یونیکس اصلی بنا به یکسری حوادث به شکل آزاد در اختیار دیگران
قرار گرفته بود.

باید اعتراف کنم که من چندان نسبت به موضوعات پیچیده‌ای که برای
آر.ام.اس. آن قدر اهمیت داشتند و دارند آگاه نیستم. حتی باید بگویم که
درباره بنیاد نرم‌افزار آزاد که او تاسیس کرده‌بود و از آن دفاع می‌کرد هم
اطلاعات کمی داشتم. با در نظر گرفتن این واقعیت که چیز چندانی از سخنرانی
سال ۱۹۹۱ به یاد نمی‌آورم، باید بگویم که احتمالا در آن دوره تاثیر زیادی
در زندگی من نداشته است. من به فنّآوری علاقه‌داشتم، نه به سیاست - در
بچگی به اندازه کافی سیاست دیده بودم. اما لارس یک ایده‌آل‌گرا بود و تا
آخر جلسه ماند و گوش کرد.

برای اولین بار در ریچارد تیپ ایده‌آل یک هکر ریشو با موی بلند را
دیدم. این تیپ از هکرها در هلسینکی زیاد پیدا نمی‌شود.

درست که من نور هدایت را در آن سخنرانی ندیدم، ولی یک چیزی باید درونم
تکان خورده باشد چون بعدها برای لینوکس از لیسانس جی.پی.ال. استفاده
کردم. بازهم دارم از خودم جلو می‌افتم.

\section{بخش چهارم}
دوم ژانویه ۱۹۹۱. اولین روزی که مغازه‌ها بعد از کریسمس و تولد بیست و یک
سالگی من باز هستند و این دو برای من پردآمدترین اتفاقات طول سال هستند.

با پول کریسمس و تولد، تصمیم اقتصادی بزرگم مبنی بر خرید کامپیوتری به
قیمت ۱۸۰۰۰ مارک فنلاند را گرفتم که حدود ۳۵۰۰ دلار می‌شد. البته اینقدر
پول نداشتم و برنامه این بود که یک سوم قیمت را پرداخت کنم، کامپیوتر را
به خانه ببرم و بعد بقیه قیمت را قسطی بپردازم. کامپیوتری که انتخاب کرده
بودم، ۱۵۰۰۰ مارک قیمت داشت ولی چون من در طول سه سال و قسطی می‌پرداختم،
باید ۱۸۰۰۰ می‌دادم.

من به یک مغازه کوچک رفتم. سازنده برایم مهم نبود و به همین خاطر یک
کامپیوتر سفید بدون اسم را انتخاب کردم. برای خرید،‌ فروشنده فهرستی از
قیمت‌ها و میزان رم و پردازنده و اندازه دیسک سخت به شما نشان می‌داد و
انتخاب می‌کردید. من دنبال قدرت بودم. می‌خواستم به جای ۲ مگابایت، ۴
مگابایت رم داشته باشم. سرعت مورد نظرم هم ۳۳ مگاهرتز بود. البته
می‌توانستم سراغ ۱۶ مگاهرتز هم بروم اما نه! من بهترین چیز را می‌خواستم.

شما به فروشنده می‌گفتید چه چیزی می‌خواهید و او کامپیوتر را برایتان سر هم
می‌کرد. در عصر اینترنت و تحویل در محل، این مساله کمی عجیب است. باید سه
روز بعد بر می‌گشتید و کامپیوتر را تحویل می‌گرفتید، اما این سه روز مثل یک
هفته گذشت. روز ۵ ژانویه، از پدرم خواستم برای رفتن به مغازه و به خانه
آوردن کامپیوتر به من کمک کند.

نه فقط هیچ اسمی نداشت، که هیچ توضیحی هم همراه کامپیوتر نبود. یک جعبه
خاکستری ساده. من این کامپیوتر را به خاطر باحال بودن‌ ظاهرش نخریده
بودم. ظاهر این ماشین با مونیتور ۱۴ اینچش که ارزان‌ترین چیزی بود که من
می‌توانستم بخرم، خیلی حوصله‌ سر بر بود. ولی به هر حال این کامپیوتر چیز
قرص و محکمی بود. منظورم از قرص و محکم کامپیوتر قدرتمندی است که کمتر
کسی توان داشتن‌اش را داشت. نمی‌خواهم بگویم که آن کامپیوتر خیلی کاربردی
ولی غیرجذاب (چیزی مثل استیشن‌های ولوو) بود. واقعیت این بود: من دنبال
کامپیوتر قابل اتکایی بود که به راحتی بتوانم برایش قطعات جانبی بخرم؛
چیزی که بدون شک به زودی لازم می‌شد.

کامپیوتر با یک نسخه محدود شده داس فروخته شده بود. من می‌خواستم مینیکس
اجرا کنم پس یک نسخه‌ از آن سفارش دادم و حدود یک ماهی طول کشید تا این
سیستم‌عامل به فنلاند برسد. کتاب مینیکس را می‌توانستید از مغازه‌های
کامپیوتری بخرید. ولی به دلیل کم بودن تقاضا برای خود سیستم‌عامل، باید آن
را به یک کتابفروشی سفارش می‌دادید. قیمت آن هم ۱۶۹ دلار بود به اضافه
هزینه پست، به اضافه مالیات، به اضافه هزینه تبدیل پول و به اضافه یکسری
چیز دیگر. آن موقع به نظرم این مساله خیلی ظالمانه بود. صادقانه بگویم که
هنوز هم همین نظر را دارم. ماهی که حرام شد،‌ به نظرم مثل شش سال طول
کشید. حتی از چند ماهی که منتظر خریدن کامپیوتر بودم هم بدتر بود.

زمستان طولانی و سرد بود. هربار که قدم از خانه بیرون می‌گذاشتید این خطر
وجود داشته که با تنه پیرزنی که انتظار می‌رفت به جای تلوتلوخوردن در
خیابان، در خانه و جلوی تلویزیون مشغول تماشای مسابقه هاکی و بافتن ژاکت
یا پختن سوپ برای خانواده‌اش باشد، روی برف‌ها ولو شوید. عملا تمام آن ماه
را با کامپیوترم \textbf{شاهزاده ایرانی}\RFootnote{\lr{Prince Of
    Persia} - یکی از بازی‌های مشهور کامپیوترهای پی سی روی سیستم عامل
  داس} بازی کردم. موقعی هم که بازی نمی‌کردم، مشغول خواندن کتاب‌هایی
بودند که به من نشان می‌دادند کامپیوتر جدیدم چگونه کار می‌کند.

مینیکس بالاخره در یک بعد از ظهر جمعه رسید و همان شب هم نصب‌اش کردم. نصب
برنامه مستلزم این بود که شانزده عدد فلاپی را یکی یکی در کامپیوتر
بگذاریم. تمام آخر هفته به این گذشت که به فضای جدید کامپیوترم عادت
کنم. چیزهایی که درباره سیستم‌عامل جدید دوست داشتم و از آن مهمتر چیزهایی
که دوست‌شان نداشتم را یاد گرفتم. سعی کردم برای حل مشکلاتی که دوست‌شان
نداشتم، برنامه‌هایی که به آن‌ها عادت داشتم را از کامپیوتر دانشگاه دریافت
کنم. در کل حدود یک ماه یا حتی کمی بیشتر طول کشید تا این کامپیوتر را
واقعا کامپیوتر خودم بکنم.

اندرو تاننباوم، پروفسور دانشگاه آمستردام که مینیکس را نوشته بود،
می‌خواست این برنامه را یک ابزار آموزشی نگاه دارد. به همین دلیل، مینیکس
قدرت چندانی نداشت. البته وصله‌هایی برای مینیکس وجود داشت - که آن را بهتر
می‌کرد - از جمله وصله مشهور یک هکر استرالیایی به نام بروس
اوانز\LFootnote{Bruce Evans} که خدای مینیکس ۳۸۶ به حساب می‌آمد. اصلاحات
او، مینیکس را روی ۳۸۶ بسیار قابل‌استفاده‌تر کرده بود. من حتی قبل از گرفتن
کامپیوتر هم خبرنامه‌های آنلاین مینیکس را دنبال می‌کردم و در نتیجه از
همان اول می‌دانستم که می‌خواهم این نسخه بهبود یافته مینیکس را اجرا
کنم. اما به خاطر قوانین مربوط به مجوز، باید اول نسخه اصلی مینیکس را
می‌خریدید و سپس با کلی تلاش، کاری می‌کردید که اصلاحات و وصله‌های اوانز با
آن همراه شوند. این کار بزرگی بود.

چیزهایی در مینیکس بود که باعث نارضایتی من می‌شد. بدترین آن‌ها، شبیه‌ساز
ترمینال بود و چون برنامه‌ای بود که از طریق آن به کامپیوتر دانشگاه متصل
می‌شدم، اهمیت زیادی هم داشت. هربار که می‌خواستم از طریق خط تلفن برای
استفاده از یونیکس قدرت‌مند یا آنلاین شدن، به کامپیوتر دانشگاه متصل شوم،
باید از این برنامه استفاده می‌کردم.

پس پروژه‌ای برای ایجاد شبیه‌ساز ترمینال خودم شروع کردم. هدف من نوشتن
شبیه‌ساز زیر مینیکس نبود بلکه می‌خواستم در پایین‌ترین‌ لایه سخت‌افزاری،
برنامه‌ام را اجرا کنم. این پروژه همچنین راهی بود برای درک بسیار بهتر از
اینکه سخت‌افزار ۳۸۶ چگونه کار می‌کند. همان طور که اشاره کردم، در هلسینکی
زمستان بود. من یک کامپیوتر حسابی داشتم و مهمترین بخش پروژه این بود که
ببینم این ماشین چگونه کار می‌کند و تفریح کنم.

از آنجایی که می‌خواستم در سطح خود فلز کامپیوتر برنامه بنویسم، باید از
بایوس\RFootnote{\lr{BIOS} - سیستم راه انداز کامپیوترهای پی سی که
  مسوولیت دادن کنترل به سیستم عامل را بر عهده دارد} شروع می‌کردم و
اولین کد، رام\LFootnote{ROM} است که کامپیوتر بعد از روشن شدن، اجرا
می‌کند. بایوس کدهای بعدی را از روی دیسک یا فلاپی می‌خواند که در مورد
برنامه من، انتخاب فلاپی بود. بایوس اولین سکتور فلاپی را می‌خواند و شروع
به اجرای آن می‌کند. این اولین پی.سی. من بود و باید یاد می‌گرفتم که همه
این کارها چطور انجام می‌شود. همه این‌ها در حالتی که به آن \textbf{حالت
  واقعی}\LFootnote{Real Mode} می‌گویند، اجرا می‌شود اما برای اینکه
بتوانیم از کل توان پردازنده مرکزی استفاده کنیم و آن را در وضعیت ۳۲
بیتی بکار بگیریم، باید به حالتی برویم که به آن \textbf{حالت حفاظت
  شده}\LFootnote{Protected Mode} می‌گویند. برای اینکار باید کلی کار
پیچیده صورت بگیرد.

پس برای نوشتن یک شبیه‌ساز ترمینال به این روش، لازم است دقیقا بدانید که
پردازنده مرکزی چطور کار می‌کند. در حقیقت دلیل اینکه برنامه را به زبان
اسمبلی نوشتم، این بود که درباره سی.پی.یو. چیزهای بیشتری یاد بگیرم. چیز
دیگری که باید بدانید، این است که چطور روی صفحه بنویسید، چطور از صفحه
کلید بخوانید و چگونه روی مودم بخوانید و بنویسید. (امیدوارم خوانندگانی
غیرگیکی که با جرات از پریدن به \hyperref[ch4]{فصل فرش قرمز} سر باز زده
اند را از دست ندهم.)

من می‌خواستم دو ترد\RFootnote{\lr{Thread} - منظور پروسه ای است که به
  شکل مستقل در حال اجرا است} مستقل داشته باشم. یک ترید از مودم
می‌خوانَد و روی صفحه نمایش می‌دهد و آن یکی از صفحه کلید می‌خوانَد و روی
مودم می‌نویسد. دو پایپ\LFootnote{Pipe} هم در هر دو جهت وجود دارند. به
این کار سوییچ وظایف\LFootnote{Task Switching} می‌گویند و ۳۸۶ سخت‌افزار
بخصوصی برای مدیریت آن دارد. به نظرم این ایده خیلی باحال بود.

اولین برنامه‌های من به این شکل بودند که یک ترید دائما حرف \code{A} را
روی صفحه می‌نوشت و ترید دیگر حرف \code{B} را (می‌دانم که خیلی جذاب نیست.)
برنامه‌ را طوری نوشته بودم که هر یک از این تریدها چندین بار در ثانیه
اجرا شوند. با استفاده از وقفه زمان‌سنج، اول صفحه پر از \code{AAAAAAA}
می‌شد و بعد ناگهان \code{BBBBBBBB} ها شروع به نوشته شدن می‌کردند. از نظر
کاربردی، این برنامه واقعا به درد نخور، اما شیوه خوبی است برای نشان
دادن و فهمیدن اینکه برنامه مبتنی بر سوییچ وظایفی که نوشته بودم، به
خوبی کار می‌کرد. نوشتن این برنامه شاید یک ماه طول کشید چون همه چیز را
باید قدم به قدم یاد می‌گرفتم.

در نهایت موفق شدم تا دو ترید قبلی که یکی \code{AAAAAA} می‌نوشت و یکی
\code{BBBBBB} را به شکلی تغییر دهم که یکی از مودم بخواند و روی صفحه
بنویسد و یکی هم از صفحه‌کلید بخواند و اطلاعات را روی مودم منتقل
کند. حالا من برنامه شبیه‌ساز ترمینال خودم را داشتم.

هر وقت که می‌خواستم اخبار را بخوانم، فلاپی حاوی برنامه را در دیسک‌گردان
می‌گذاشتم و ماشین را بوت می‌کردم و با برنامه خودم، مشغول خواندن اخبار از
روی کامپیوتر دانشگاه می‌شدم. اگر لازم می‌شد بخشی از برنامه را بهتر کنم
یا آن را تغییر دهم،‌ باید کامپیوتر را در مینیکس بوت می‌کردم و در آنجا
برنامه نویسی را ادامه می‌دادم.

نسبتا به این ماجرا افتخار می‌کردم. 

خواهرم سارا هم از موفقیت‌ بزرگ من مطلع بود. نسخه‌های اولیه را به او نشان
داده بود و او با خیره شدن به \code{AAAAAAA} ها و \code{BBBBBB} ها برای
حدود پنج ثانیه، بدون هیچ هیجانی گفته بود \dbquote{خوبه} و رفته
بود. آن‌جا بود که فهمیدم خروجی خیلی هم جذاب نیست. غیرممکن است بشود به
کسی که جریان را درک نمی‌کند توضیح داد که علی‌رغم اینکه چیز فوق‌العاده‌ای
نمی‌بیند، جریانات هیجان‌انگیزی در پشت زمینه جریان دارد. هیجان آن برنامه
دقیقا به همان اندازه بود که به یک نفر، یک جاده آسفالت شده را نشان
بدهیم. شاید تنها کس دیگری که برنامه را دید، لارس\LFootnote{Larsw} بود؛
تنها دانشجوی سوئدی زبان گرایش کامپیوتر به جز من، که با من هم‌ورودی بود.

ماه مارس بود، شاید هم آوریل و اگر هم برف‌ها در پیترزگارتان شروع به
آب‌شدن کرده بودند، من خبر نداشتم یا برایم مهم نبود. من بیشتر وقتم را در
کت‌حوله‌ای و پشت کامپیوتر نه چندان خوشگلم می‌گذراندم در حالی که پنجره‌ها
با مقواهای سیاه پوشانده شده بودند تا من را از نور آفتاب و دنیای بیرون،
جدا نگه‌ دارند. هر ماه قسط‌های کامپیوتر شخصی جدیدم‌ را می‌دادم و قرار بود
این جریان تا سه سال ادامه داشته باشد. چیزی که نمی‌دانستم این بود که فقط
یک‌سال به این کار ادامه خواهم داد. تا آن موقع لینوکس نوشته می‌شد و افراد
بسیاری بیشتر از سارا و لارس آن را می‌دیدند. در آن هنگام، پیتر
آنوین\LFootnote{Peter Anvin} که حالا با من در ترنسمتا کار می‌کند، از
طریق اینترنت از افراد خواهد خواست تا قسط‌های کامپیوتر من را بپردازند.

همه می‌دانستند که من از لینوکس هیچ پولی درنمی‌آورم. به همین خاطر شروع
کردند به گفتن اینکه \dbquote{بذار روی اینترنت یک پولی جمع کنیم و قسط کامپیوتر
  لینوس رو بدیم.}

این فوق‌العاده بود. 

من هیچ پولی نداشتم. همیشه این حس را داشتم که خوب نیست کسی از آدم پول
بخواهد یا آدم از کسی گدایی کند اما حقیقت این بود که این بار مردم
خودشان پولشان را به من می‌دادند و این... بازهم دارم از خودم جلو می زنم.

لینوکس این گونه شروع شد. با تبدیل شدن برنامه آزمایشی من به یک بسته
شبیه‌ساز ترمینال.

\begin{journal}
مجله ردهارینگ\LFootnote{Red Herring} من را برای تهیه گزارش به
اولو\LFootnote{Oulu} فرستاد، که علی‌رغم موقعیت مزخرف‌اش و اینکه فقط چند
ساعت رانندگی با منطقه قطبی فاصله داشت، در حال تبدیل شدن به یک مرکز
تکنولوژیک بود. این فرصت خوبی بود برای ملاقات با پدر و مادر و خواهر
لینوس در هلسینکی.

پدرش نیلز (که با اسم نیک شناخته می‌شد) من را در لابی هتل سوکوس
واکون\LFootnote{Sokos Hotel Vakuna} در مرکز شهر هلسینکی ملاقات
کرد. مرتب، با یک عینک‌ کلفت و ریشی مشابه لنین. به تازگی یک دوره
چهارساله را برای رادیو فنلاند، در مسکو به پایان رسانده بود و مشغول
نوشتن یک کتاب درباره روسیه بود و می‌خواست تصمیم بگیرد که آیا سمتی را در
واشنگتن که به آن علاقه‌ای نداشت بپذیرد یا نه. چند ماه قبل یک جایزه ملی
معتبر را برده بود که به گفته همسر سابقش، باعث شده بود \dbquote{تا حد
  زیادی خوش‌اخلاق‌تر} شود.

در اوایل عصر، من را با ولووی وی.۴۰ خودش به توری از محل زندگی دوران
کودکی لینوس و مدرسه‌ای مکعبی برد که پدر و پسر در آن درس خوانده
بودند. همچنین از کنار خانه پدربزرگ گذشتیم که لینوس سه ماه اول زندگی‌اش
را در آن گذرانده بود و بعد به ساختمانی با منظره پارک رسیدیم که هفت سال
بعد خانواده در آن گذشته بود. نیک یکی از این هفت سال را در مسکو مشغول
تحصیل بود تا یک کمونیست شود؛ درست وقتی که لینوس پنج ساله بود. بعد
ساختمان زرد رنگی را نشان داد که لینوس و خواهرش بعد از طلاق در آن
گذرانده بودند. یک مغازه فیلم‌های بزرگسال، جایگزین فروشگاهی شده بود که
لینوس وسایل الکترونیک خود را از آن می‌خرید. در نهایت به بازدید
ساختمان‌هایی رفتیم که پدربزرگ مادری لینوس در آن زندگی می‌کرد و لینوکس در
آن متولد شده بود. آنا،‌ مادر لینوس هنوز در آن‌جا زندگی می‌کند.

نیک آدمی بامزه، باهوش و ناراضی از خود است که شباهت‌های رفتاری چندی با
لینوس دارد، از جمله مالاندن چانه با دست وقتی که مشغول حرف زدن
است. لبخند آن‌ها هم مشابه است. پدر بر خلاف پسر یک ورزشکار است که در تیم
بسکتبال بازی می‌کند، روزی پنج مایل می‌دود و صبح‌ها در رودخانه یخ زده شنا
می‌کند. در پنجاه و پنج سالگی، با اعتماد به نفس یک ورزشکار سی و پنج ساله
راه می‌رود. یک ناهماهنگی دیگر بین پدر و لینوس هم این است: پدر زندگی
عاشقانه پر ماجرایی دارد.

شام را در یک رستوران شلوغ مرکز هلسینکی می‌خوریم. جایی که نیک درباره
مشکلات لینوس جوان به عنوان فرزند یک کمونیست فعال که معمولا در خیابان
سخنرانی می‌کند و در یک دوره هم صاحب یک دفتر بوده است، صحبت می‌کند. او
می‌گوید که گاهی بچه‌های دیگر لینوس را به دلیل عقاید سیاسی رادیکال پدرش
دست می‌انداخته‌اند و حتی بعضی از پدر و مادرها بچه‌هایشان را از بازی کردن
با لینوس، منع می‌کردند. نیک شرح می‌دهد که به عقیده او، دوری کردن لینوس
از عقاید چپ، ریشه در همین مشکلات دوران کودکی دارد. می‌گوید \dbquote{اجازه
  نمی‌داد درباره این موضوعات صحبت کنم. اگر شروع می‌کردم به حرف زدن، اتاق
  را ترک می‌کرد.} ادامه می‌دهد که \dbquote{در بهترین حالت،‌ شروع می‌کرد
  ساز مخالف کوک کردن. می‌دانم که بچه‌ها به خاطر پدر عجیب و غریب، لینوس
  را در مدرسه دست می‌انداختند. پیام این ماجرا برای من این بود : بابا من
  رو قاطی این چیزها نکن.}

با ماشین به خانه نیک بر می‌گردیم. جایی که قول می‌دهد بنشینیم و آبجو
بخوریم. خانه در شمال خیابان اصلی است و مجموعه‌ای از بلوک‌های مسکونی است
که در دهه ۱۹۲۰ برای کارگران ساخته شده بودند. از پله‌ها بالا می‌رویم و
بعد از درآوردن کفش‌ها، وارد خانه می‌شویم. فضای خانه، با لامپ‌های پوشیده
در سبدهای بافتنی، دیوارآویزهای جهان سوم و گیاهان آپارتمانی، فضای
ضدفرهنگی دهه‌های ۱۹۶۰ را به ذهن می‌آورد. او پشت میز آشپزخانه می‌نشیند،
آبجو می‌ریزید و درباره وظیفه پدری صحبت می‌کند. \dbquote{پدر نباید فکر
  کند کسی است که بچه‌ها را به جایی که هستند، رسانده.} موبایل را
برمی‌دارد تا به زنی که با او زندگی می‌کند تلفن کند. می‌گوید که لینوس تازه
شروع کرده کتاب‌های تاریخی‌ای را بخواند که سال‌ها پدرش اصرار داشته بخواند
و اضافه می‌کند که او احتمالا تا به حال کتاب شعر پدربزرگش را نخوانده.

از نیک می‌پرسم که آیا هیچ وقت به برنامه‌نویسی احساس علاقه کرده و آیا تا
به حال از لینوس خواسته تا اصول برنامه‌نویسی را به او آموزش دهد. می‌گوید
که هرگز. دلیل می‌آورد که پدر و پسر دو موجود مستقل هستند و برایم توضیح
می‌دهد که تلاش برای ورود به دنیای پر شور و حال لینوس، از نظر او
\dbquote{تجاوز به روحش} تعبیر می‌شده. در نقش پدر یک آدم مشهور، راحت به
نظر می‌رسد. در مقاله‌ای از یک روزنامه که بعد از دریافت جایزه درباره نیک
نوشته شده، از طرف او نقل شده است که حتی زمانی که برای آوردن لینوس از
زمین بازی به پارک می‌رفت، بچه‌های دیگر با اشاره به او، به یکدیگر
می‌گفتند: \dbquote{نگاه کن! اون پدر لینوس است!}

\vspace*{30pt}

سارا توروالدز با قطار از خانه‌اش در شهری کوچک در غرب هلسینکی آمده
بود. جایی که تابلوهای خیابان، اول به سوئدی نوشته می‌شوند و بعد به
فنلاندی و جایی که می‌تواند به راحتی اجاره خانه‌ای با سونا و وان بزرگ را
بپردازد و جایی که می‌تواند از اینکه در خیابان‌ها سوئدی بیشتر از فنلاندی
به گوش می‌خورد، لذت ببرد. او توضیح می‌دهد که در یک اقلیت، اقلیت بوده
است: او در نوجوانی تصمیم گرفته کاتولیک شود. عملی که او را بخشی از
اقلیت ده درصدی غیر لوتری فنلاند می‌کند و باعث می‌شود پدر خداناباورش برای
چند هفته او را عاق کند.

او حالا به هلسینکی آمده تا در یک برنامه حمایت شده از طرف دولت، به
کودکان اصول کاتولیسم را آموزش دهد. او دختری فعال و دوست‌داشتنی است و در
بیست و نه سالگی، نمونه‌ای است از آدم معتقدی که پر از انرژی و مشغول
فعالیت است. پوست روشن و صورت گردش یادآور شباهتی مبهم بین او و برادر
بزرگ‌ترش است، ولی شکی نیست که خواهر بسیار اجتماعی‌تر از برادر است. تمام
مدت مصاحبه، مشغول تایپ روی گوشی تلفن‌همراه‌اش است و احتمالا دارد برای
کسانی که در ادامه روز خواهد دید، پیام می‌فرستد. هر چند وقت یکبار هم
نگاهی به گوشی می‌اندازد تا جواب‌ها را بخواند. او شغل موفقی به عنوان یک
مترجم دارد.

ظهر شده و با سارا برای خوردن ناهار به پیش مادرش می‌رویم. در حین راه در
بسیاری از مکان‌های مهم دوران کودکی توقف می‌کنیم: پارک گربه، دبستان و چند
جای دیگر. \dbquote{مادر و پدر من، کمونیست‌هایی بودند که کارت عضویت‌شان
  همیشه همراه‌شان بود. ما جوری بزرگ شدیم که فکر می‌کردیم اتحاد جماهیر
  شوروی جای خوبی است.} و ادامه می‌دهد که \dbquote{ما به مسکو هم
  رفتیم. چیزی که از همه بیشتر یادم است، مغازه اسباب‌بازی فروشی بسیار
  بزرگ آن‌جا بود. بزرگ‌تر از هر چیزی که در هلسینکی داریم.} پدر و مادرش
وقتی شش ساله بوده از هم جدا شده‌اند. \dbquote{وقتی به ما گفتند که پدرم
  به خوبی و خوشی از خانه می‌رود را یادم هست. من هم فکر کردم که تصمیم
  خوبی است. دعواها تمام می‌شد. در اصل او سفرهایی طولانی به مسکو می‌کرد و
  به همین دلیل ما به نبودنش عادت داشتیم.} سارا وقتی که ده سال داشت،
تصمیم گرفت تا به جای زندگی در کنار مادر و لینوس، پیش پدرش برود که در
آن دوران به شهر اسپو\RFootnote{\lr{Espoo} - شهری تکنولوژیک در فنلاند
  که مرکز نوکیا هم د رآن قرار داشت.} در مجاورت هلسینکی نقل مکان کرده
بود. می‌گوید \dbquote{مساله این نبود که نمی‌خواستم پیش مادرم باشم. در
  اصل نمی‌خواستم پیش لینوس باشم. با رفتن من به پیش پدرم، فقط آخر هفته‌ها
  با هم دعوا می‌کردیم. ما همیشه با هم دعوا داشتیم. البته هی که بزرگ‌تر
  شدیم، دعواها هم کمتر شدند.}

ما به خانه آنا توروالدز که در طبقه اول یک آپارتمان بود رفتیم و او به
پیشوازمان آمد. اسم مستعارش میکی بود. اجازه نداد بر اساس سنت قدیمی
فنلاندی، کفش‌هایم را پیش از ورود به خانه در بیاورم. گفت: \dbquote{احمق
  نباش!  اینجا همین حالا هم کثیف است. احتمالا نمی‌توانی از این بدترش
  کنی.}  قدکوتاه، مو مشکی و بسیار نکته‌بین بود. چند لحظه بعد از وارد
شدن ما، تلفن زنگ زد. معاملات املاکی بود که می‌خواست آپارتمان کناری را
به من نشان دهد تا من بتوانم در برگشت به آمریکا، وضعیت آن را برای لینوس
تعریف کنم و همچنین مدارک را هم با خودم به آمریکا ببرم تا در صورت پسند،
لینوس بتواند آن‌جا را بخرد تا پایگاهی در هلسینکی داشته باشد. وارد
آپارتمان وسیع شدیم؛ جایی که کارمند معاملات املاک که به طرز غریبی من را
به یاد هنرپیشه نقش آنت بنینگ\LFootnote{Anette Bening} در فیلم زیبای
آمریکایی می‌انداخت. از ما خواست تا قبل از دیدن خانه، روکش‌های آبی‌رنگی را
روی کفش‌هایمان بکشیم. چند لحظه بعد این کارمند با لبخندی زننده داشت
می‌گفت: \dbquote{خب حالا اینجا این اتاق است، یک اتاق عالی برای گذاشتن
  اجناس عتیقه و ارزشمندی که نمی‌خواهید نور آفتاب آن‌ها را خراب کند.}
میکی نگاه شیطنت‌باری به من انداخت و گفت: \dbquote{اوه چه روش خوبی برای
  گفتن اینکه این اتاق اصلا آفتاب‌گیر نیست.}

به آشپزخانه که برگشتیم، نیکی پشت یک میز مثلثی که با یک رومیزی رنگارنگ
تزیین شده بود نشست و در فنجان‌هایی واقعا بزرگ، برای همه قهوه
ریخت. آپارتمان او هم مانند آپارتمان شوهر سابقش، با هنرها و کتاب‌های
اقوام گوناگون تزیین شده بود. پرده‌های اتاق، پرده‌های
ماریمکوی\LFootnote{Marimekko} سیاه و سفید بودند. آپارتمان در اصل سه
اتاق و یک آشپزخانه داشت. بعد از اینکه بچه‌ها خانه را ترک کرده بودند،
میکی به اتاق‌خواب بزرگ که سابقا توسط سارا اشغال شده بود، نقل مکان کرده
بود. بعد دیوارهای اتاق‌خواب لینوس و اتاق‌خواب قبلی خودش را خراب کرده بود
تا یک پذیرایی/آشپزخانه بزرگ درست کند. به یک گوشه خالی اشاره کرد و گفت:
\dbquote{این آنجایی است که کامپیوتر لینوس قرار داشت. فکر کنم کم کم
  باید یک جور پلاک شناسایی به آن‌جا آویزان کند. نظر شما چیست؟.} خیلی
راحت گپ می‌زد. با دستورزبان و دایره‌لغاتی آن قدر قوی که در گفتن جمله‌ای
مثل \dbquote{یکی از آن خزهایی نبود که در خیابان پلاس اند} حتی یک لحظه
مکث هم نمی‌کرد. روی دیوار اتاق‌خوابش، یک پرچم بزرگ اتحاد جماهیر شوروی
بود. هدیه‌ای به لینوس از طرف جوکو ویرومکای\LFootnote{Jouko Vierumkai}
بود که طی مسابقات بین‌المللی اسکی آن را آورده بود. لینوس سال‌ها آن را در
یک قفسه گذاشته بود اما حالا میکی آن را بالای تخت‌اش آویزان می‌کرد.

میکی آلبومی بیرون آورد که معدود عکس‌های خاطرات خانواده در آن بود. لینوس
دو یا سه ساله، لخت در ساحل ایستاده بود. لینوس در همان سن و سال ولی در
قلعه‌ای مشهور نزدیک هلسینکی. لینوس در اوایل دوران بلوغ با ظاهری نه
چندان دوست داشتنی و کمی زمخت. میکی در تولد شصت سالگی پدرش؛ پروفسور
آمار. در این عکس برادر و خواهر بزرگ‌ترش را نشان می‌دهد: \dbquote{خواهرم
  در نیویورک روان‌پزشک است و برادرم فیزیکدان اتمی. اما من! من گوسفند
  سیاه گله بودم. درسته؟ اما در عوض من اولین نوه را به دنیا آوردم.} این
را می‌گوید و یک سیگار آتش می‌زند.

ناهار را در رستورانی که به افتخار ویلت چمبرلین\LFootnote{Wilt
  Chamberlain} نامگذاری شده، می‌خوریم. همان زمان که میکی قهوه سفارش
می‌دهد، سارا با موبایل‌اش ور می‌رود. نیکی برایم می‌گوید که چطور با پدر
لینوس درباره اینکه آیا لازم است پستانک را از لینوس بگیرند یا نه، بحث
می‌کردند: از طریق یادداشت گذاشتن برای هم روی میز آشپزخانه. همچنین از
حافظه ضعیف لینوس و ناتوانی‌اش در بیادآوری چهره‌ها صحبت
می‌کند. \dbquote{اگر با او مشغول دیدن فیلمی باشید و قهرمان فیلم لباس
  آبی‌اش را عوض کند و زرد بپوشد، لینوس خواهد پرسید که: این یارو کیست؟}
صحبتی هم درباره سفر خانواده با دوچرخه به سوئد می‌شود. از خوابیدن
شب‌هنگام کنار رودخانه سرد و دزدیده‌شدن دوچرخه سارا در همان روز نخست و
خرج شدن کل بودجه سفر برای خرید یک دوچرخه جدید. چادر زدن روی یک صخره و
تنها گذاشتن لینوس در تمام طول روز در چادر در حالی که مادر و دختر مشغول
شنا و ماهی‌گیری بوده‌اند و در نهایت از اینکه هنگام برگشت از ماهیگیری،
متوجه شده‌اند تنها چیزی که جلوی افتادن چادر به دریای بالیتک را گرفته،
لینوس بوده که بی‌توجه به تغییرات آب و هوایی، تمام مدت در چادر خواب بوده
است.

میکی به دورانی که لینوس در اتاقش مخفی می‌شده و مثل یک برده، به کامپیوتر
خدمت می‌کرده، می‌خندند. \dbquote{نیک همیشه به من می‌گفت که لینوس را بیرون
  بیاندازم و مجبورش کنم که شغلی پیدا کند، ولی لینوس مزاحم من نبود. چیز
  زیادی هم لازم نداشت و فقط با کامپیوترش مشغول بود. این تمام زندگی‌اش
  بود، تمام علاقه‌اش. حق هم داشت اینکار را بکند چون من از کاری که می‌کرد
  هیچ سر در نمی‌آوردم.}

این روزها مادر هم به اندازه همه از فعالیت‌های پسرش مطلع است. میکی و
بقیه اعضای خانواده در معرض سوالات دائمی رسانه‌ها قرار دارند. این سوالات
معمولا با لینوس هم مطرح می‌شوند، ولی او می‌گوید که بهتر است هر فرد
خانواده هر طور که صلاح می‌داند پاسخ بدهد. اما به هرحال هر وقت که آن‌ها
به سوالی جواب می‌دهند، آن را برای لینوس هم می‌فرستند تا او هم پاسخ را
تایید کند.

ماه‌ها قبل که من برای کسب اطلاع از دوران بچگی لینوس به میکی ایمیل زدم،
پاسخ خیلی کامل و تشریحی‌ای گرفتم. عنوان مقاله‌ مادرش این بود:
\dbquote{بزرگ‌کردن لینوس از زمانی که یک نِرد کوچک بود.} آن‌جا نوشته بود
که بچه‌ نوپایش همان نشانه‌های علاقمندی به علم را نشان می‌داد که برادر و
پدر آنا نیز نشان داده بودند:

\dbquote{وقتی می‌فهمید {کسی شیفته علم است} که وقتی مشکلی جلوی‌اش قرار
  می‌گیرد یا چیزی او را آزار می‌دهد چشمانش می‌درخشند. کسی که بعد از دیدن
  مشکل، دیگر صدای شما را نمی‌شود، کسی که دیگر جواب ساده‌ترین سوالات را
  هم نمی‌دهد، کسی که فعالیت ذهنی‌اش کل فعالیت‌های دیگرش را تحت‌الشعاع قرار
  می‌دهد، کسی که در حین کار برای حل یک مساله، غذا و خواب را هم فراموش
  می‌کند و در نهایت کسی که از تلاش باز نمی‌ایستد.  البته بدون شک این فرد
  توقف می‌کند و به زندگی روزمره هم می‌پردازد ولی بعد دوباره با اشتیاق
  قبلی به مساله بر‌می‌گردد و مشغول حل مشکل می‌شود. این آدم شیفته علم
  است.}

در آن مقاله درباره کشمکش‌های برادری و خواهری بین لینوس و سارا هم نوشته
بود. درباره بحث و جدل‌های آن‌ها درباره هر موضوع کوچک (سارا: \dbquote{من
  مزه قارچ/جگر/هرچیز دیگری رو دوست ندارم} لینوس: \dbquote{چرا دوست
  داری!}) و همین طور احترام همراه با دلخوری‌اش را نسبت به
خواهرش. \dbquote{لینوس یک بار در حالی که پنج یا شش ساله بود، حسادت
  همراه با احترام خود نسبت به خواهرش را این طور بیان کرده بود: من هیچ
  وقت فکر جدیدی ندارم. من فقط به چیزهایی فکر می‌کنم که بقیه قبلا به آن
  فکر کرده‌اند. من فقط دوباره به آن‌ها فکر می‌کنم. ولی سارا فکرهایی می‌کند
  که قبلا هیچ‌کس نکرده.}

این خاطرات شاید نشان دهنده‌ این باشند که به نظر من لینوس هیچ استعداد
\dbquote{خاصی} ندارد یعنی استعدادش به طور خاص در کامپیوتر نیست. اگر کامپیوتر
نشد، یک چیز دیگر. یک زمان دیگر و یک مکان دیگر ممکن است لینوس روی یک
چیز کاملا متفاوت تمرکز کند و به نظرم این کار را هم خواهد کرد. (منظورم
این است که امیدوارم لینوس تا آخر عمر به توسعه لینوکس نچسبد.) انگیزه او
کامپیوتر یا شهرت و پول نیست. او صادقانه به دنبال کنجاوی‌هایش و فتح
مشکلات پیش رو است و البته اگ بخواهم درست‌تر بگویم، \textbf{حل مشکلات به
شیوه‌ای صحیح} چون شیوه صحیح تنها روشی است که او را ارضا می‌کند.

فکر می‌کنم همین الان گفته باشم که لینوس به عنوان یک بچه چطور موجودی
بود. بله! بزرگ‌کردنش ساده بود. تنها چیزی که لازم داشت یک مشکل
بود. بقیه‌اش با خودش بود. وقتی هم که روی کامپیوتر متمرکز شد، بزرگ‌کردنش
ساده‌تر هم شد. همان طور که من و سارا می‌گفتیم؛ کافی بود به لینوس یک کمد
اضافی و یک کامپیوتر خوب بدهیم و گاه گداری هم از شکاف کمد برایش پاستای
خام بریزیم و او کاملا خوشحال خواهد بود.

به جز اینکه... و وقتی بحث به اینجا می‌رسید قلب من به دهنم می‌آمد. در
دنیای به این بزرگی چطور ممکن است لینوس با دختری دوست شود؟ این تنها
موردی بود که در تمام طول مادر بودن برایش واقعا دعا کرده‌ام. واقعا هم
کار کرد! تاو را وقتی در دانشگاه درس می‌داد، دید و وقتی دیدم که برای چند
روز گربه و کامپیوترش را فراموش کرده فهمیدم که طبیعت بالاخره پیروز شده
است.

فقط امیدوارم هیولای شهرت او را از هدفش زیاد منحرف نکند (به نظر نمی‌رسد
شهرت او را چندان عوض کرده باشد ولی به هرحال این روزها نرم‌خوتر شده است
و بیشتر با آدم هایی که نزدیکش می‌روند حرف می‌زند. حتی به نظر می‌رسد برای
\dbquote{نه} گفتن مشکل دارد. البته به نظر من این بیشتر به شوهر و پدر
شدنش ربط دارد تا به هیاهوی رسانه‌ها.)

و واضح است که مادر و دختر این هیاهوی رسانه‌ای را به خوبی دنبال
می‌کنند. اواخر ژانویه ۲۰۰۰ است و فردای روزی که ترنسمتا قرار بوده اعلام
کند که طی این مدت مشغول چه پروژه‌ای بوده است. اوایل ناهار است که نیکی
از سارا می‌پرسد: \dbquote{امروز توی روزنامه‌ها درباره اون آدمی که خودت
  می‌دونی و اون چیزی که خودت می‌دونی، چیزی بود؟}

آن‌شب، نیکی حین رفتن به سر کار از تاکسی می‌خواهد تا جلوی هتل من بایستد و
یک صندلی کودک از چوب صنوبر به من می‌دهد تا شخصا به پاتریشیا
برسانم. همین طور یک نقشه از آپارتمانی که برای لینوس موجود است.
\end{journal}

\newpage
\begin{wellbox}
درباره اولین باری که احساس کردم لینوس کار ارزشمندی کرده.

فکر کنم اوایل سال ۱۹۹۲ بود. بدون برنامه خاصی داشتم با دوچرخه‌ام به سمت
خانه کاملا درهم و برهم لینوس می‌رفتم تا ببینمش. همان طور که داشتیم
ام.تی.وی. نگاه می‌کردیم، از لینوس درباره سیستم‌عامل جدیدش پرسیدم. معمولا
جواب‌های بی‌ربطی می‌داد. اما این بار مرا پیش کامپیوترش برد (از‌‌ آشپزخانه
بهم ریخته،‌ به اتاق آشوب‌زده‌اش رفتیم)

لینوس نام کاربری و عبارت‌عبورش (همین باشه یا از رمزعبور استفاده کنیم؟)
را به کامپیوتر داد و یک خط فرمان ظاهر شد. او چند کاربرد ابتدایی خط
فرمان را نشان داد که چندان هم چیز چشمگیری نبود. بعد از چند لحظه یکی از
آن لبخندهای لینوسی را زد و گفت: \dbquote{شبیه داس است، نه؟}

من که تا حدی تحت تاثیر قرار گرفته بودم با سر تایید کردم. البته شوکه
نبودم چون چیزی که می‌دیدم شبیه داس بود و واقعا چیز جدیدی نداشت. باید
می‌دانستم که لینوس هیچ‌وقت بدون دلیل آنطور لبخند نمی‌زند. او به سمت
کامپیوتر برگشت و چند کلید ترکیبی زد و یک صفحه ورود دیگر ظاهر شد. یک
لاگین جدید و یک خط فرمان جدید. لینوس دو خط فرمان جدید هم باز کرد و گفت
در آینده افراد مختلف خواهند توانست از این طریق به شکل جداگانه به همین
سیستم وارد شوند.

آن‌موقع که باور کردم لینوس چیزی فوق‌العاده خلق کرده است. البته با این
جریان مشکلی نداشتم چون هنوز من بودم که در میز اسنوکر، فرمانروایی
می‌کردم.

\vspace*{10pt}
\hfill جوکو \dbquote{آووتون} ویروماکی
\end{wellbox}

\begin{wellbox}
برای من کل جریان به این معنا بود که تلفن همیشه اشغال بود و کسی
نمی‌توانست به ما زنگ بزند... از یک جایی به بعد، از چهارگوشه دنیا کارت
پستال به خانه سرازیر شد. فکر کنم آن موقع که فهمیدم مردم دنیا واقعا
دارند از چیزی که او درست کرده، استفاده می‌کنند.

\hfill سارا توروالدز
\end{wellbox}

\section{بخش پنجم: زیبایی برنامه‌نویسی}
درست نمی‌دانم چطور باید شیفتگی‌ام به برنامه‌‌نویسی را بیان کنم، ولی به
هرحال سعی‌ام را خواهم کرد. برای کسی که برنامه‌نویسی می‌کند، اینکار
جذاب‌ترین چیز در دنیا است. بازی‌ای بسیار درگیر کننده‌تر از شطرنج، بازی‌ای
که در آن شما قوانین را می‌سازید و بازی‌ای که نتایج چیزهایی هستند که شما
تعریف‌شان کرده‌اید.

البته هنوز هم برای افرادی که از بیرون به قضیه نگاه می‌کنند، برنامه‌نویسی
حوصله‌برترین فعالیت دنیا است.

بخشی از هیجان اولیه موجود در برنامه‌نویسی را می‌توان به راحتی توضیح داد:
این واقعیت که هر دستوری که به کامپیوتر بدهید، با دقت تمام آن را اجرا
خواهد کرد. بدون کوچک‌ترین اشتباهی. تا ابد. بدون هیچ شکایتی.

این ماجرا به خودی خود جذاب است. 

اما اطاعت بی‌چون و چرا هر چند جذاب است، مشخصه یک دوست خوب نیست. در
حقیقت همین مساله باعث می‌شود که کامپیوتر خیلی زود حوصله‌بر شود. چیزی که
باعث می‌شود مردم با این شدت جذب کامپیوتر شوند این است که برای حل یک
مشکل، علاوه بر دادن دستور به کامپیوتر، لازم است کشف کنید که چگونه باید
این دستور را بدهید.

من شخصا متقاعد شده‌ام که علوم کامپیوتر اشتراکات بسیاری با فیزیک
دارد. هر دوی آن‌ها در این مورد بحث می‌کنند که جهان در سطح بنیادینش چگونه
کار می‌کند. مطمئنا تفاوت هم در این است که در فیزیک بحث بر سر کشف چگونگی
کارکرد جهان است و در علوم کامپیوتر، بحث بر سر ساخت این جهان. در حوزه
کامپیوتر، شما خالق جهان هستید. شما باید هر چیزی که پیش می‌آید را کنترل
کنید. اگر اینکار را خوب انجام دهید، خدای کامپیوتر خواهید بود. البته در
مقیاسی کوچک.

و البته احتمالا با گفتن این حرف، نیمی از جمعیت جهان را ناراحت کرده‌ام.

اما این واقعیت دارد. شما باید دنیای خود را بسازید و تنها چیزی که در
این خلقت شما را محدود می‌کند، توانایی‌های ماشین و این روزها بیشتر و
بیشتر، توانایی‌های خودتان است.

\vspace*{20pt}

به یک خانه درختی فکر کنید. می‌توانید روی یک درخت خانه‌ای بسازید که کار
کند، یک ورودی داشته باشد و مستحکم هم باشد ولی هر کسی فرق یک خانه
مستحکم و یک خانه درختی زیبا که از شکل و خواص درخت استفاده‌ای خلاقانه
کرده است، را می‌داند. این کار،‌ ترکیب هنر و مهندسی‌ است. این همان دلیلی
است که برنامه‌نویسی، فریبنده و فوق‌العاده می‌شود. کارایی گاهی بعد از جذاب
بودن،‌ زیبا بودن یا شوکه کننده بودن قرار می‌گیرد.

برنامه‌نویسی، تمرین خلاقیت است. 

چیزی که اولین بار مرا به دنیای کامپیوتر کشاند، کشف این روند بود که
کامپیوترها چگونه کار می‌کنند. یکی از لذت‌های من وقتی بود که کشف کردم
کامپیوتر مثل ریاضی است: جهان را باید خودتان و با قوانین خودتان
بسازید. در فیزیک شما با قوانین موجود درگیر هستید ولی در ریاضیات - مثل
برنامه‌نویسی - تا وقتی که قوانینی که وضع کرده‌اید با یکدیگر سازگار باشند
می‌توانید راه را ادامه دهید. لزومی ندارد ساختارهای ریاضی با هیچ منطق
بیرونی سازگار باشند، ولی چیزی که هیچ گاه نمی‌توان از آن عدول کرد،
سازگاری درونی قوانین با یکدیگر است. همان طور که هر ریاضی‌دانی به شما
خواهد گفت، می‌توانید ساختارهای ریاضی‌ای بسازید که در آن، سه به علاوه سه
برابر دو بشود. می‌توانید هر کاری که دوست دارید بکنید. ولی باید توجه
کنید که در حین اضافه شدن پیچیدگی ساختار، اجزای ساختار کماکان با یکدیگر
و با جهانی که شما خلق کرده‌اید، سازگار باقی بمانند. اگر قرار است این
دنیا زیبا باشد، نباید هیچ کاستی‌ای در آن راه بیابد. این دقیقا شیوه کار
در جهان برنامه‌نویسی هم هست.

یکی از دلایلی که مردم را تا به این حد شیفته کامپیوتر می‌کند، این است که
آن‌ها می‌توانند با توسل به کامپیوتر دنیاهای جدیدی بسازند و در آن‌ها دست
به تجربه بزنند و بیاموزند که چه چیزهایی ممکن است. در ریاضی می‌شود به
سراغ تمرین‌های فکری رفت و درباره ممکن‌ها سخن گفت. مثلا وقتی صحبت از
هندسه می‌شود،‌ اکثر مردم به هندسه اقلیدسی فکر می‌کنند. کامپیوتر به مردم
کمک کرده است که بتوانند هندسه‌‌های مختلف را به نمایش بکشند؛ هندسه‌هایی که
به هیچ وجه اقلیدسی نیستند. با استفاده از کامپیوتر می‌توانیم این دنیاهای
جدید را مشاهده کنیم و ببینیم که چطور کار می‌کنند. مجموعه
مندلبرت\LFootnote{Mandelberot} را به یاد دارید؟ تصاویر فراکتالی که بر
اساس معادلات مندلبرت ایجاد شده بودند؟ اینها تصاویری بودند که بر اساس
دنیایی کاملا ریاضی ساخته شده بودند که پیش از کامپیوترها به هیچ عنوان
امکان ظهور نداشتند. مندلبرت این قواعد قراردادی را درباره جهان جدیدی
نوشته که پیش از این وجود نداشت و با واقعیت بیرونی هم ارتباطی
نداشت. کامپیوترها کمک کردند کشف کنیم که این قوانین، تصاویر زیبایی هم
خلق می‌کنند. با کامپیوتر و برنامه‌نویسی، می‌توان جهان‌های جدیدی ساخت و
گاهی این جهان‌ها و الگوها، واقعا زیبا هستند.

البته در بیشتر مواقع،‌ کار ما این نیست. ما معمولا فقط برنامه‌هایی
می‌نویسیم که قرار است مشکل خاصی را حل کنند. در این حالت، شما جهان جدیدی
نمی‌سازید بلکه مشکلی در درون جهان کامپیوتر را حل می‌کنید. مشکل از طریق
اندیشیدن به آن حل می‌شود. فقط هم عده معدودی هستند که می‌توانند ساعت‌ها
جلوی یک صفحه نورانی بنشینند و به یک مشکل فکر کنند. فقط خوره‌ها و
گیک‌هایی مثل من.

سیستم‌عامل پایه هر چیز دیگری است که در ماشین اتفاق می‌افتد و ساخت یک
سیستم‌عامل، چالشی واقعی است. وقتی سیستم‌عاملی می‌نویسید، در حال ساختن
جهانی هستید که تمام برنامه‌های دیگر در درون آن زندگی خواهند کرد. در
حقیقت شما دارید قوانینی را می‌نویسید که بنا بر آن‌ها برنامه‌ها خواهند
دانست درون این دنیا، چه چیزی مجاز و چه چیزی غیرمجاز است. البته هر
برنامه‌ای همین کار را می‌کند، ولی سیستم‌عامل پایه‌ای ترین سطح این برنامه‌ها
است. مثل نوشتن قانون اساسی سرزمینی جدید. تمام برنامه‌های دیگر، قوانین
معمولی این سرزمین خواهند بود.

بعضی وقت‌ها قوانین چیزهای با معنایی نیستند و ما به دنبال معنا هستیم. ما
دوست داریم بتوانیم به راه حل نگاه کنیم و احساس کنیم که جواب صحیح را با
شیوه صحیح به دست آورده‌ایم.

شاگرد مثبت کلاس را به یاد دارید که در مدرسه همیشه جواب‌های صحیح را
داشت؟ او زودتر از هر کس دیگری مساله را حل می‌کرد و نکته این بود که
تلاشی هم برای زودتر حل کردن به کار نمی‌برد. او هیچ‌وقت سعی نمی‌کرد یاد
بگیرد که هر مساله را چطور باید حل کرد بلکه فقط به شیوه صحیح به مشکل
نگاه می‌کرد. وقتی که راه حل او را می‌شنیدید، احساس می‌کردید راه حل او با
معناترین راه حل برای این مساله است.

در کامپیوترها هم مساله به همین صورت است. می‌توانید با آزمایش کلیه
جواب‌های ممکن، جواب را پیدا کنید ولی این راه احمقانه است. در عوض سعی
کنید مساله را آن قدر خرد کنید که اصولا دیگر مساله‌ای باقی نماند و
ناگهان احساس کنید که مساله خودبخود حل شده است. به شیوه دیگری به مساله
نگاه کنید و ناگهان به روشنی خواهید رسید: مشکل اصولا به این دلیل وجود
داشت که شما از زاویه اشتباهی به مساله می‌نگریستید.

احتمالا بهترین مثال در این مورد نه در دنیای کامپیوتر که از دنیای ریاضی
است. داستان مربوط می‌شود به ریاضی‌دان مشهور کارل فردریش گاوس و معلمی که
حوصله تدریس نداشت و از آن‌ها خواست تا اعداد ۱ تا ۱۰۰ را با یکدیگر جمع
کنند و جواب را بگویند. معلم انتظار داشت که اینکار حدود یک روز از
دانش‌آموزان وقت بگیرد. ولی ریاضی دان نابغه آینده، در عرض پنج دقیقه جواب
صحیح را اعلام کرد: ۵۰۵۰. شیوه حل مساله به هیچ وجه جمع کردن یکی یکی
اعداد از یک تا صد نبود. این روش حوصله‌سربر و احمقانه است. چیزی که گاوس
کشف کرد این بود که جمع ۱ و ۱۰۰ می‌شود ۱۰۱. همینطور جمع ۲ و ۹۹ می‌شود
۱۰۱. به همین ترتیب ۳ و ۹۸ هم ۱۰۱ است و الی آخر.  در نهایت ۵۰ و ۵۱ هم
حاصل ۱۰۱ را ایجاد خواهند کرد و در عرض چند لحظه می‌توان به این نتیجه
رسید که ما ۵۰ بار نتیجه ۱۰۱ را داریم و جواب نهایی، ۵۰۵۰ است.

احتمال دارد که این داستان ساختگی باشد ولی پیامش روشن است: ریاضی‌دان
واقعی،‌ مساله را از راه طولانی و حوصله‌سربر حل نمی‌کند، چون می‌تواند الگوی
پنهان شده در پشت مشکل را ببیند و از این الگو برای حل زیباتر سوال
استفاده می‌کند. مساله در علوم کامپیوتر هم مشابه است. مطمئنا می‌شود
برنامه‌ای نوشت که با جمع زدن ۱ تا ۱۰۰، مساله بالا را حل کند. این برنامه
برای کامپیوترهای امروزی یک چشم به هم زدن است اما برنامه‌نویس باهوش است
و جواب را از اول می‌داند. این آدم، برنامه‌ای می‌نویسد که به شیوه‌ای جدید و
زیبا به مساله حمله کند و در نهایت مشخص خواهد شد که راه حل او، راه حلی
بهتر است.

هنوز مشکل است توضیح بدهم که چه چیز جذابی در این وجود دارد که برای چند
روز کله خود را به دیوار بکوبید و سعی کنید کشف کنید که یک مساله چطور
باید به شیوه‌ای زیبا حل شود. ولی وقتی راه‌حل زیبای تان را پیدا کنید،
بهترین احساس جهان را خواهید داشت.

\section{بخش ششم}
شبیه‌ساز ترمینال من داشت دست و پا در‌می آورد و کامل‌تر می‌شد. از آن به طور
مرتب برای ورود به کامپیوتر دانشکده و خواندن ایمیل‌ها یا بحث در گروه
پستی مینیکس استفاده می‌کردم. مشکل این بود که لازم بود فایل‌هایی را آپلود
و دانلود کنم. به عبارت دیگر باید فایل‌ها را روی دیسک ذخیره می‌کردم. برای
اینکار باید برای شبیه‌ساز ترمینالم یک درایور دیسک هم می‌نوشتم. همچنین به
یک ساختار فایل نیاز داشتم تا بتوانم فایل‌ها را روی دیسک طبقه‌بندی کنم و
بتوانم فایل‌های مختلف را در بخش‌های مربوط به خودشان ذخیره کنم.

اینجا بود که احساس کردم ادامه پروژه کار زیادی می‌برد که ارزشش را ندارد
و تقریبا کل پروژه را متوقف کردم. ولی مساله این بود که کار دیگری برای
انجام نداشتم. بهار آن‌ سال به کلاس‌هایم می‌رفتم که چالشی خاصی در آن‌ها
نبود. تنها فعالیت هفتگی من، دیدارهای (یعنی پارتی‌های) چهارشنبه شب در
باشگاه اسپکتروم بود. با توجه به اینکه جزو دسته غیرحیوانات
اجتماعی\LFootnote{Social non-Animal} بودم، این تنها فعالیت من به جز
درس‌خواندن و برنامه‌نویسی بود. در بهار آن سال، بدون آن دیدارها (پارتی‌ها)
که کمک می‌کردند یک کم-گوشه‌-گیر باشم، به یک گوشه‌گیر کامل تبدیل
می‌شدم. اسپکتروم ساختار درونی‌ای داشت که زندگی اجتماعی را تسهیل می‌کرد و
فکر نمی‌کنم برنامه‌های چندانی را در آن‌جا از دست داده باشم. آن برنامه‌ها
برای من بسیار مهم بودند. در واقع، گاهی حتی به خاطر هیجان آن برنامه‌ها،
نمی‌توانستم بخوابم. من قبل از بعضی‌ از برنامه‌ها، شب را با این تلاش
می‌گذراندم که فردا با کمبود توانایی‌های اجتماعی‌ام چه کنم، چگونه دماغ
بزرگم را مخفی کنم و در مورد نداشتن دوست‌دختر چه توضیحی بدهم. فکر کنم
این یک مشکل همیشگی گیک‌ها است.

چیزی که سعی‌ می‌کنم بگویم این است که آن بهار، کار چندان جالبی برای انجام
نداشتم. در این میان فکر کردن به پروژه دیسک‌درایو/سیستم فایل پذیرفتی
بود. پس با خودم گفتم که به سراغش خواهم رفت. یک دیسک درایو نوشتم و چون
می‌خواستم از فایل‌هایم در سیستم‌عامل مینیکس هم استفاده کنم - و همچنین به
این دلیل که فایل سیستم مینیکس به خوبی مستند شده بود - سیستم‌ فایل خودم
را سازگار با سیستم‌فایل مینیکس طراحی کردم. با اینکار می‌توانستم فایل‌هایی
که در مینیکس داشتم را بخوانم و همچنین فایل‌هایی که توسط شبیه ساز
ترمینال خودم ساخته‌ام را هم بتوانم در مینیکس استفاده کنم.

این جریان کار خیلی زیادی برد. یک برنامه روزانه به شکل برنامه نویسی،
خواب، برنامه‌نویسی، خواب، برنامه‌نویسی، غذا (هله هوله)، برنامه‌نویسی،
خواب، برنامه‌نویسی، دوش (سریع)، برنامه‌نویسی. از همان موقع معلوم بود که
این پروژه دارد به سمت یک سیستم‌عامل پیش‌ می‌رود. پس من هم ذهنیتم از این
پروژه را از یک شبیه‌ساز ترمینال به سمت یک سیستم‌عامل تغییر دادم. فکر
می‌کنم این گذار در یکی از آن ماراتن‌های برنامه‌نویسی حاصل شد. شب یا روز؟
یادم نیست. یک لحظه در کت‌حوله‌ای سوراخ سوراخم پشت کامپیوتر نشسته بودم و
مشغول تغییر دادن شبیه‌ساز ترمینالم بودم تا قابلیت‌های جدیدی به آن اضافه
کنم. لحظه‌ای بعد متوجه شدم که آن قدر قابلیت‌های این شبیه‌ساز ترمینال زیاد
شده است که باعث شده به یک سیستم‌عامل جدید تغییر چهره بدهد.

من آن را \textbf{گنو-ایمکس شبیه سازهای ترمینال}\LFootnote{Gnu-emacs of
  terminal emulation programs} خودم نامیدم. گنو-ایمکس به عنوان یک
ادیتور شروع به کار کرد ولی کسانی که مشغول توسعه‌اش بودند، آن را به
میزبانی برای انواع و اقسام قابلیت‌ها تبدیل کردند. آن‌ها می‌خواستند
ادیتوری بنویسند که بتواند برنامه‌نویسی شود ولی جنبه قابل برنامه‌نویسی
بودن زیادی پیشرفت کرد و ادیتور به مخلوقی جهنمی تبدیل شد. این ادیتور
همه چیز دارد به جز یک ظرفشویی آشپزخانه و احتمالا به همین دلیل است که
گاهی شکلک این برنامه را، ظرفشویی آشپزخانه انتخاب می‌کنند. می‌گویند این
یک پروژه بسیار بزرگ برنامه‌نویسی است که قابلیت‌هایی بیش از هر چیزی دارد
که برای یک ادیتور لازم است. شبیه‌ساز ترمینال من هم مشغول طی کردن مسیری
مشابه بود. رشد شبیه ساز ترمینال من، داشت به چیز جدیدی منجر می‌شد.


\begin{emailbox}
\noindent\textbf{از:} \code{Torvalds@klaava.Helsinki.Fi} (لینوس بندیکت
توروالدز)

\noindent\textbf{به:} گروه خبری \code{comp.os.minix}

\noindent\textbf{موضوع:} \code{Gcc-1.40} و یک سوال مربوط به پوسیکس

\noindent\textbf{شناسه پیام:}
\code{<1191Jul13,10050.9886@klaava.Helsinki.Fi>}

\noindent\textbf{تاریخ:} ۳ جولای ۹۱ ساعت۱۰:۰۰:۵۰ جی.ام.تی.

\noindent سلام شبکه‌ای‌ها،

به خاطر پروژه‌ای که مشغول آن هستم (در مینیکس)، علاقمندم تعاریف
استاندارد پوسیکس را داشته باشم. ممکن است یک نفر من را به یک نسخه
(ترجیحا) قابل خواندن توسط ماشین از جدیدترین نسخه راهنمایی کند؟ سایت‌های
اف.تی.پی. خیلی خوب خواهند بود.
\end{emailbox}


بله. این اولین شاهد از گیکی در فنلاند است که دارد محدوده‌ توانایی‌های
کامپیوتریش را می‌آزماید. استانداردهای پوسیکس\LFootnote{POSIX} قوانین
دور و درازی هستند که صدها فراخوانی سیستمی یونیکس را تشریح می‌کنند. این
دستورها تمام فعالیت‌های کامپیوتر را کنترل می‌کنند و با فراخوانی‌های اساسی
مثل خواندن و نوشتن و باز کردن و بستن شروع می‌شوند. پوسیکس بدنه
استاندارد یونیکس است. سازمانی متشکل از تمام کسانی که می‌خواهند با هم در
مورد استانداردهای یونیکس توافق کنند. برای برنامه‌نویس‌ها، استانداردها
بسیار مهم هستند. چون از طریق آن‌ها می‌توانند برنامه‌هایی بنویسند که روی
بیش از یک کامپیوتر اجرا شوند. فراخوانی‌های سیستمی - بخصوص مهم‌هایش -
فهرستی از توابع مختلف را در اختیار من می‌گذاشت که زیرساخت‌های یک
سیستم‌عامل را تشکیل می‌دهند. من برنامه‌هایی می‌نوشتم که توابع مورد نظر را
به شیوه‌ای که خودم تصمیم گرفته بودم، اجرا کنند. در عین حال با پیگیری
پوسیکس، برنامه‌های من برای دیگران نیز قابل استفاده می‌شدند.

آن موقع نمی‌دانستم که این امکان هست که کپی سخت (نسخه کاغذی) پوسیکس را
از خود آن سازمان سفارش دهم. البته ارزشی هم نداشت. حتی اگر می‌توانستم
پول‌اش را بدهم، رسیدن نسخه‌ها به فنلاند از طریق پست زمان زیادی
می‌گرفت. این بود که می‌خواستم نسخه‌ نرمی پیدا کنم که قابل دریافت از طریق
سایت‌های اف.تی.پی. باشد.

هیچ کس جوابی حاوی پیوندی به فایل‌های پوسیکس نداد. پس مجبور شدم به سراغ
نقشه ب بروم. شروع بررسی استانداردها از راهنمای نسخه یونیکس دانشگاه که
از سرویس دهنده‌ سان میکروسیستمز\LFootnote{Sun Microsystems}
استفاده می‌کرد. این راهنماها حاوی نسخه‌ای ابتدایی از فراخوانی‌های اساسی
بودند و می‌شد از طریق آن‌ها کار را شروع کرد. می‌شد به راهنماها نگاه کرد و
دید که هر فراخوانی قرار است چه کاری را انجام دهد و بعد پشت کامپیوتر
نشست و تابعی برای انجام آن کار نوشت. صفحات راهنما نمی‌گفتند که چطور
باید وظایف را انجام داد و تنها به نتیجه نهایی اشاره می‌کردند. تازه بعضی
از فراخوانی‌ها را هم از کتاب آندرو تاننباوم و بعضی کتاب‌های دیگر
برداشتم. در نهایت یک نفر جلدهای کلفت حاوی استانداردهای پوسیکس را
فرستاد.

البته ایمیل من بدون جواب هم نماند. هر آدم مطلعی (و فقط هم آدم‌های مطلع
گروه خبری مینیکس را می‌خواندند) می‌توانست بگوید که پروژه من نوشتن یک
سیستم‌عامل است. مگر قوانین پوسیکس به چه درد دیگری می‌خوردند؟ پیام من
کنجکاوی آری لمکه\LFootnote{Ari Lemke}، استاد حل تمرین دانشگاه تکنولوژی
هلسینکی (که اگر اینقدر علاقمند به تئوری‌ها نبودم، آن‌جا درس می‌خواندم) را
برانگیخت. آری با فرستادن یک جواب دلگرم کننده، نوشت که بر روی
اف.تی.پی. دانشگاه، یک زیرشاخه برایم ساخته است تا هر وقت احساس کردم
سیستم‌عامل آماده شده، آن را در اختیار کسان دیگری بگذارم که ممکن است
علاقمند باشند آن را آزمایش کنند.

\section{بخش هفتم}
آری لمکه باید آدم خوشبینی بوده باشد. او مدت‌ها پیش از اینکه من چیزی
برای ارائه داشته باشم، مسیر اف.تی.پی. \code{ftp.funer.fi} را ساخت. من کلمه
رمز را داشتم و همه چیز تنظیم شده بود تا در موقع مناسب به سیستم وارد
شوم و فایل‌ها را در آن جا آپلود کنم. حدود چهارماه طول کشید تا احساس کنم
چیزی دارم که می‌توان آن را با جهانیان به اشتراک گذاشت یا حداقل با آری و
چند خوره سیستم‌عامل دیگری که گاه گداری با آن‌ها ایمیل رد و بدل می‌کردم.

هدف اصلی من ایجاد سیستم‌عاملی بود که در نهایت بتوانم از آن‌ به عنوان
جایگزین مینیکس استفاده کنم. قرار نبود کاری بیشتر از مینیکس بکنم،
برنامه اولیه این بود که چیزهایی که در مینیکس دوست دارم را تکرار کنم و
همینطور چند قابلیت دیگر را. برای مثال نه فقط شبیه‌ساز ترمینال مینیکس بد
بود، که کنترل وظیفه هم نداشت؛ یعنی نمی‌شد در حینی که نیازی به یک برنامه
نداشتیم، آن را به پشت زمینه منتقل کنیم. مدیریت حافظه مینیکس هم خیلی
ابتدایی بود و در سیستم عامل مک هنوز هم همین طور است.

روش نوشتن یک سیستم‌عامل این است که اول کشف کنید فراخوانی‌های سیستمی قرار
است چه کاری بکنند و بعد برنامه‌هایی بنویسید که این وظایف را به شیوه‌ای
که شما دوست دارید، عملیاتی کنند. به شکل عمومی، چیزی حدود چندصد
فراخوانی سیستمی وجود دارد که بعضی از آن‌ها نیازمند چند تابع گوناگون
هستند. البته بعضی‌ها هم ساده هستند. بعضی از فراخوانی‌های پایه‌ای بسیار
پیچیده هستند و پیاده سازی آن‌ها نیازمند کلی کار زیربنایی است. مثلا
فراخوانی‌های سیستمی \dbquote{خواندن} یا \dbquote{نوشتن} را در نظر
بگیرید. برای خواندن یا نوشتن از دیسک، نیازمند این هستید که قبلا یک
درایور دیسک نوشته باشید. حالا \dbquote{باز کردن} را در نظر
بگیرید. باید کل لایه فایل سیستم را بسازید تا یک تابع بتواند اسم فایلی
را بگیرد و آن را باز کند. نوشتن \dbquote{باز کردن} شاید چند ماه کار
برد ولی وقتی عملیاتی شد، از همان کد می‌شد در قسمت‌های دیگر هم استفاده
کرد.

این روش توسعه‌ اولیه بود. من از راهنماهای سان یا کتاب‌های دیگر،
استانداردها را می‌خواندم و یکی یکی فراخوانی‌های سیستمی را انتخاب می‌کردم
و سعی می‌کردم توابعی بنویسم که آن‌ها را عملیاتی کنند. کار سخت و
طاقت‌فرسایی بود.

دلیل: هیچ چیزی اتفاق نمی‌افتاد، هیچ پیشرفتی را عملا مشاهده
نمی‌کردید. می‌توانستید برنامه‌های کوچکی بنویسید که کد تازه نوشته شده را
آزمایش کنند ولی عملا چیز کاربردیی از این کدها بیرون نمی‌آمد. بعد از
مدتی، دیگر روند انتخاب تک تک فراخوانی‌ها از یک فهرست بلندبالا را کنار
می‌گذاشتید و احساس می‌کردید که فراخوانی‌ها آن قدر کامل شده‌اند که بتوانید
برنامه‌های واقعی را روی آن‌ها اجرا کنید. اولین برنامه‌ای که باید اجرا
کنید، پوسته\RFootnote{\lr{Shell} همان پوسته متنی است که در
  سیستم‌عامل‌هایی مانند لینوکس دستورات را داخل آن تایپ می کنیم.} است چون
اجرای دیگر برنامه‌ها بدون حضور پوسته، بسیار مشکل است. علاوه بر این،
پوسته شامل بسیاری از فراخوانی‌های سیستمی‌ای است که دیگر برنامه‌ها هم از
آن‌ها استفاده خواهند کرد. پوسته را اجرا کنید و فهرستی از فراخوانی‌هایی
را خواهید داشت که باید یک به یک بنویسیدشان.

در یونیکس، پوسته به نوعی مادر همه برنامه‌های دیگر است. پوسته آنجاست تا
برنامه‌های اجرایی دیگر را اجرا کند (برنامه اجرایی، فایلی است که به شکل
۰ و ۱ به ماشین می‌گوید که چکار کند. هربار که برنامه‌ای را به یک زبان
برنامه‌نویسی می‌نویسید، باید آن را از کد منبع به باینری ترجمه کنید.) در
عین حال این پوسته است که به شما اجازه می‌دهد وارد سیستم شوید. قبول! در
یونیکس اولین برنامه‌ای که به شکل سنتی اجرا می‌شود \code{init} نام دارد
ولی جرای \code{init} به حجم زیادی از زیرساخت نیاز دارد و کنترل کننده
کل اتفاقاتی است که روی می‌دهند. وقتی چیزی برای اجرا شدن نیست، داشتن
\code{init} هم لزومی ندارد.

پس به جای شروع به اجرای \code{init}، اولین کاری که
کرنل\LFootnote{Kernel} من می‌کرد، اجرای پوسته بود. من حدود بیست و پنج
فراخوانی سیستمی را نوشته بودم و همان طور که گفتم، این اولین برنامه‌
واقعی بود که می‌خواستم اجرا کنم. پوسته چیزی نبود که من نوشته‌ باشمش. من
یکی از پوسته‌های اصلی یونیکس که یکی از مشابه‌های پوسته‌ای به نام پوسته
بورن\LFootnote{Bourne Shell} را دانلود کرده و روی دیسک ریخته بودم. این
پوسته به عنوان یک نرم‌افزار روی اینترنت در دسترس همه بود و اسمش را از
یک شوخی ناجور گرفته بود. کسی که پوسته اصلی را نوشته بود \lr{Bourne}
نام داشت و در نتیجه این مشابه،\lr{Bourne-Again}\RFootnote{این شوخی ای
  است با مسیحیانی که بعد از مدت ها فکر می کنند به تازگی مسیحیت را کشف
  کرده اند و با تولدی دوباره، وظیفه دارند دیگران را نیز متوجه این کشف
  کنند} یا به اختصار \lr{bash} نام گرفته بود.

وقتی سعی‌ می‌کنید یک برنامه واقعی را از دیسک بارگزاری کنید، بدون شک با
یک باگ در درایور دیسک یا برنامه بارگذار مواجه خواهید شد. از آن‌جایی که
برنامه بارگزار نمی‌فهمد مشغول چه کاری است، همیشه فهرستی از کارهای در
حال اقدام را چاپ می‌کند. این بسیار مهم است چون با این روش دقیقا می‌فهمید
که اشکال در کجا بروز کرده است.

من در مرحله‌ای بودم که برنامه‌ام پوسته را از دیسک بارگزاری می‌کرد و هر
فراخوانی سیستمی که صدا زده می‌شد، ولی من هنوز آن را ننوشته بودم را چاپ
می‌کرد. من کامپیوتر را بوت کردم، پوسته را اجرا کردم و چیزی شبیه به این
ظاهر شد: \dbquote{فراخوانی سیستمی شماره ۵۱۲ نوشته نشده است.} من صبح و
شب به این نوشته‌ها نگاه می‌کردم و فراخوانی‌های جدید ر ا می‌نوشتم و قبلی‌ها
را اصلاح می‌کردم. این‌کار بسیار لذت بخش‌تر بود از اینکه فهرستی از
فراخوانی‌ها را جلویم بگذارم و یکی یکی آن‌ها را بنویسم. با این کار،
پیشرفت را می‌دیدم.

اواخر آگوست یا اوایل سپتامبر بود که توانستم پوسته را به طور کامل اجرا
کنم. از آن به بعد همه چیز آسان‌تر شد.

این مساله بزرگی بود. 

وقتی پوسته را راه‌انداختم، توانستم سریعا چند برنامه را کمپایل کنم. برای
مثال پوسته خیلی خیلی پیچیده‌تر از برنامه‌ای مثل \code{cp} (کپی) یا
\code{I} (برای گرفتن فهرست فایل‌ها) بود. هر چیزی که لازم داشتم، بخشی از
پوسته بود که قبلا نوشته شده بود و در نتیجه وقتی پوسته راه افتاد، از
کمی بالای صفر تا ۱۰۰، در مدت خیلی کمی پیموده شد. گاهی آن قدر همه چیز
آماده بود که من احساس کن فیکن می‌کردم. قبل از این، هیچ چیز کار نمی‌کرد.

بعله! احساس رضایت فوق‌العاده‌ای داشتم. فکر می‌کنم این مهم بود چون آن
تابستان به جز کامپیوتر، هیچ کار دیگری نکرده بودم. اغراق نمی‌کنم. از
آوریل تا آگوست، بهترین ایام سال در فنلاند است. مردم برای قایق‌سواری به
مجتمع‌الجزایر می‌روند و در سواحل آفتاب می‌گیرند و در سوناهای تابستانی‌شان
وقت می‌گذرانند. اما من به سختی می‌توانستم بگویم شب است یا روز و حتی چه
موقعی از سال است. آن پرده‌های سیاه و کلفت جلوی نور آفتاب تقریبا بیست و
چهار ساعته را می‌گرفتند، همین طور جلوی دنیای بیرون را. بعضی روزها - یا
شب‌ها؟ - از تخت بیرون می‌آمدم و مستقیما به سراغ کامپیوترم که کمتر از
نیم‌ متر با تخت فاصل داشت می‌رفتم. در نهایت پدرم شروع کرد تا در این باره
که چرا من یک شغل تابستانی نمی‌گیرم،‌ به مادرم غر بزند. مساله برای مادرم
مهم نبود: من او را اذیت نمی‌کردم. سارا از اینکه وقتی من آنلاین می‌شدم،
خطوط تلفن مدت‌ها مشغول می‌شد، کمی ناراضی بود. احتمالا خودش این جمله را
با مراعات کمتری می‌نوشت. اغراق نخواهد بود اگر بگویم که به جز از طریق
کامپیوتر، هیچ ارتباطی با دنیای خارج نداشتم. باشه!‌ شاید هفته‌ای یکبار
دوستی می‌آمد و به پنجره تقه‌ای می‌زد و اگر من مشغول بالا و پایین رفتن در
بخش مهمی از کد نبودم، به داخل دعوتش می‌کردم. (توجه کنید که مهمان همیشه
مرد بود. این قبل از دورانی بود که گیک‌ بودن باحال محسوب شود.) ما چای
می‌نوشیدیم و در آشپزخانه کوچک‌مان، یک ساعتی ام.تی.وی. نگاه می‌کردیم. حالا
که درست فکر می‌کنم، می‌بینم که گاهی اگر کسی مثل جوکو (که من او را
آوونتون صدا می‌زنم که به معنای \dbquote{قاتل اژدها} است و این خودش داستانی
دارد) به پنجره تقه می‌زد، ممکن بود با هم برویم بیرون و آبجویی بخوریم و
کمی اسنوکر بازی کنیم. اما صادقانه بگویم که در آن دوران هیچ چیز دیگری
در زندگی من وجود نداشت.

این را هم بگویم که طی آن دوره به هیچ وجه یک آدم بیچاره رنگ‌پریده
نبودم. پوسته کار می‌کرد و این به آن معنا بود که من پایه‌های یک سیستم‌عامل
را نوشته‌ام. این کار مفرح بود.

با عملیاتی شدن پوسته، شروع کردم به آزمایش برنامه‌های درون‌‌ساخت آن. بعد
آن قدر برنامه جدید کمپایل کردم که بتوانم واقعا کار مفیدی انجام
دهم. همه چیز را در مینیکس کمپایل می‌کردم و یک پارتیشن هارد را هم اختصاص
داده بودم به سیستم‌عامل جدید و پوسته را هم به آن انتقال داده بودم. پیش
خودم آن را لینوکس می‌خواندم.

صادقانه: هیچگاه نمی‌خواستم برنامه را با نام لینوکس منتشر کنم چون این
کار به نظرم خودخواهانه می‌آمد. پس اسمی که در نظر گرفته بودم چه بود؟‌
فریکس\LFootnote{Freax} (گرفتید؟ \lr{Freaks}\RFootnote{در انگلیسی به
  معنی چیزهای عجیب و غریب و آدم های ناهمگون} با پسوند مشهور
\lr{x}). در حقیقت بعضی از فایل‌های ساخت\RFootnote{\lr{Make File} -
  فایلی که به کمپایلر می گوید چگونه باید برنامه خاصی را کمپایل کند.}
اولیه - فایل‌هایی که مشخص می‌کنند فایل منبع چگونه باید کمپایل شود - برای
تقریبا نیم سال عبارت فریکس را در خود داشتند. البته این مساله ارزشی هم
نداشت چون در آن مرحله اصولا برنامه‌ای برای انتشار عمومی این نرم‌افزار
نداشتم.

\section{بخش هشتم}
\begin{emailbox}
\noindent\textbf{از:} \code{torvalds@klaava.helsinki.fi} (لینوس بندیکت
توروالدز)

\noindent\textbf{به:} گروه خبری مینیکس \code{comp.os.minix}

\noindent\textbf{موضوع:} دوست دارید در مینیکس چه چیزهایی ببینید؟

\noindent\textbf{خلاصه:} رای‌گیریی کوتاه درباره سیستم‌عامل جدید من

\noindent\textbf{شناسه پیام:}
\code{<1991Aug25.205708.9541@klaava.Helsinki.Fi>}

سلام به همه مینیکس کارها. من مشغول یک سیستم‌عامل (آزاد) هستم (فقط به
عنوان یک سرگرمی. مثل گنو بزرگ و حرفه‌ای نخواهد شد) برای کامپیوترهای
سازگار با ۳۸۶ (۴۸۶). این جریان از آوریل شروع شده و کم کم دارد آماده
می‌شود. دنبال هر جوابی از شما هستم که بگوید چه چیزهایی را در مینیکس
دوست دارید یا دوست ندارید چون سیستم‌عامل من هم تقریبا شبیه همان خواهد
بود (لایه فیزیکی فایل سیستم مشابه است (به خاطر مسایل عملی) و همین طور
چند چیز دیگر.)

تا الان، \code{Bash} (نسخه \code{1.08}) و \code{gcc} (نسخه
\code{1.40}) و چیزهای دیگری را روی آن اجرا کرده‌ام. همین می‌رساند که در
عرض چند ماه به یک چیز به دردبخور خواهم رسید و به همین دلیل دنبال
قابلیت‌هایی هستم که شما علاقمندید در آن باشد. هر پیشنهادی را استقبال
می‌کنم ولی این به آن معنا نیست که آن را به سیستم اضافه خواهم کرد \code{:-)}

\hfill لینوس (\code{torvalds@kruuna.helsinki.fi})

\noindent\textbf{پ.ن.} بله! هیچ کد مینیکسی در آن نیست و از فایل سیستم
مالتی‌ترید پشتیبانی می‌کند. قابل انتقال نیست (چون از سوییچ وظایف ۳۸۶ و
چند قابلیت خاص دیگر استفاده می‌کند) و به احتمال زیاد از هیچ چیزی جز
هارددیسک‌های \code{AT} پشتیبانی نخواهد کرد، چون آن‌ها تنها چیزی هستند که دارم
\code{:-(}
\end{emailbox}


خوره‌ترین‌های دنیای سیستم‌عامل احساس کردند که جرقه‌ای در حال تولد
است. پیشنهادهای چندانی در مورد مینیکس به دستم نرسید، ولی بعضی‌ها شروع
به پرس و جو کردند.

\begin{emailbox}
\noindent\textbf{\code{<}} بیشتر بگو! احتیاجی به \code{MMU} داره؟

\noindent\textbf{جواب:} بله

\noindent\textbf{\code{<}} چقدرش به سی است؟ مشکلات انتقال به دیگر
سیستم‌ها چیست؟ هیچ کس باور نمی‌کند که کلا غیرقابل انتقال باشد \code{):} من دوست
دارم به آمیگا منتقلش کنم.

\noindent\textbf{جواب:} اکثر به سی نوشته شده، ولی خب اکثر مردم چیزی که
من می‌نویسم را به عنوان سی قبول نخواهند کرد. چون به عنوان یک پروژه ۳۸۶
هم هست، از هر قابلیت اختصاصی آن که نام ببرید استفاده کرده. بعضی از
فایل‌های \dbquote{سی} من همان قدر که سی هستند، اسمبلی هم هستند.

\noindent همان طور که قبلا گفتم، از \code{MMU} استفاده می‌کند. هم برای
صفحه بندی (فعلا نه برای دیسک) و هم برای سگمنت‌بندی. همین سگمنت‌بندی است
که اینقدر به ۳۸۶ وابسته‌اش کرده (هر وظیفه ۶۴ مگ برای کد دارد و ۶۴ وظیفه
کلا به ۴ گیگ نیاز دارند).
\end{emailbox}


حتی چند نفری هم بودند که پیشنهاد کردند آزمایشگرهای بتا باشند. 

در نهایت چاره‌ای نبود جز فرستادن برنامه. این روشی بود که عادت کرده بودم
بر اساس آن برنامه‌هایم را مبادله کنم. تنها چیزی که باید واقعا در موردش
تصمیم می‌گرفتم، این بود که چه زمانی برای ارسال برنامه و سهیم شدن آن با
دیگران مناسب است. یا اگر بهتر بگویم: کی برنامه به اندازه کافی بهتر شده
تا از نشان دادن آن به دیگران خجالت نکشم؟

چیزی که نهایتا دنبالش بودم، این بود که کمپایلر و محیطی واقعی داشته
باشم که بتوان برای لینوکس در خود لینوکس برنامه نوشت و از مینیکس بی‌نیاز
بود. اما وقتی دیدم که پوسته گنو به خوبی روی لینوکس اجرا شد، آن قدر
احساس افتخار کردم که حس کردم آماده‌ام تا لینوکس را با دنیا شریک
شوم. همچنین علاقه داشتم کمی بازخورد هم بگیرم.

همان زمانی که پوسته با موفقیت کمپایل شد، چند کد باینری دیگر هم داشتم
که روی آن کار می‌کردند. عملا نمی‌شد کار خاصی در این سیستم‌عامل جدید کرد
ولی می‌دیدید که یک جورهایی یادآور یونیکس است. درحقیقت چیزی بود شبیه به
یک یونیکس مفلوج.

پس تصمیم گرفتم که در دسترس دیگران هم بگذارمش. البته به شکل عمومی جریان
را اعلام نمی‌کردم و در عوض با ایمیل‌های خصوصی به تعداد کمی از دوستان -
بین پنج تا ده نفر - اطلاع دادم که آن را روی سرویس دهنده
اف.تی.پی. گذاشته‌ام. بروس اوانس مشهور در دنیای مینیکس و آری لمکه هم جزو
این افراد بودند. کد منبع خود لینوکس و چند فایل اجرایی را آپلود کردم تا
افراد بتوانند کار را شروع کنند. همین طور در ایمیل به آن‌ها گفتم که برای
راه اندازی و تنظیمات اولیه آن باید چکار کنند. آن‌ها هنوز نیازمند مینیکس
- نسخه ۳۸۶ - بودند و باید از قبل \code{gcc} را به شکل نصب شده می‌داشتند. در
حقیقت باید دقیقا نسخه من از \code{gcc} را می‌داشتند و به همین خاطر آن را هم
عمومی کردم.

برای نسخه‌بندی، پروتکل خاصی هست. یک مساله روانی هم در آن دخیل است. وقتی
احساس می‌کنید که نسخه‌ای واقعا برای انتشار آماده است، آن را نسخه
\code{1.0} می‌نامید. قبل از این مرحله، شماره نسخه‌ها مشخص می‌کند که به
نظر شما چقدر از کار تا رسیدن به نسخه \code{1.0} باقی است. با در نظر
گرفتن این موضوع، من نسخه‌ای از سیستم‌عامل که در اف.تی.پی. گذاشتم را نسخه
\code{0.01} نامیدم. این عدد به همه گوشزد می‌کرد که این سیستم‌عامل به هیچ
وجه آماده انتشار نیست.

و بعله! تاریخ را دقیق یادم هست: ۱۷ سپتامبر ۱۹۹۱. 

بعید می‌دانم بیشتر از یکی دو نفر آن را تست کرده باشند. آن‌ها باید دردسر
نصب یک کمپایلر خاص، خالی کردن یک پارتیشن برای بوت کردن سیستم و در
نهایت کمپایل کردن کرنل جدید را تحمل می‌کردند تا تنها یک پوسته را اجرا
کنند. اجرای پوسته عملا تنها چیزی بود که سیستم‌عامل من قادر بود انجام
دهد. می‌توانستید کد منبع را چاپ کنید که فقط حدود ۱۰۰۰۰ خط بود. اگر با
فونت کوچک چاپ می‌کردید، چیزی کمتر از ۱۰۰ صفحه (این روزها این کد در
محدوده ۱۰ میلیون خط است).

یکی از دلایل اصلی‌ای که سیستم‌عامل را منتشر کردم این بود که نشان دهم این
مدت فقط جو نمی‌داده‌ام و واقعا کاری انجام شده است. در اینترنت حرف زدن و
ادعا کردن ارزشی ندارد. جدای از اینکه بحث سر چه چیزی است - چه سیستم‌عامل
و چه سکس - خیلی‌ها در اینترنت مشغول ادعاهای غیرواقعی هستند. پس خوب است
که بعد از صحبت با کلی آدم در این مورد که مشغول نوشتن یک سیستم‌عامل
هستید، بتوانید بگویید \dbquote{ببینید! واقعا یک کاری کرده‌ام. این همه
  وقت شما را فیلم نکرده بودم. نتیجه را ببینید...}

و آری لمکه که راه رسیدن برنامه به سرویس‌دهنده اف.تی.پی. را هموار کرده
بود از اسم فریکس خوشش نیامد. او اسم دیگری که در پروژه استفاده شده بود
یعنی لینوکس را بیشتر پسندید و ارسال من را به \code{pub/OS/Linux} تغییر
نام داد. قبول دارم که به این کار اعتراضی نکردم. اما به هرحال او بود که
این کار را کرد. پس من می‌توانم با صداقت بگویم که خودخواه نیستم. یا
حداقل می‌توانم با صداقت نسبی بگویم که خودخواه نیستم. نظر من این بود که
این اسم خوبی است و همیشه هم می‌توانم انتخابش را گردن کس دیگری
بیندازم. دقیقا همین کاری که الان دارم می‌کنم.

همان طور که گفتم، سیستم‌عامل من عملا چندان هم به درد نمی‌خورد. چون اگر
حافظه را بیش از حد پر می‌کردید، یا کار غیرطبیعی دیگری انجام می‌دادید، به
راحتی کرش\LFootnote{Crash} می‌کرد. حتی اگر کار غیرطبیعی‌ای هم
نمی‌کردید، با رها کردن سیستم‌عامل به حال خود برای مدت طولانی، می‌توانستید
باعث کرش کردن آن شوید. البته در آن دوره قرار هم نبود کسی این سیستم‌عامل
را طولانی‌مدت استفاده کند. قرار بود فقط دیده شود. باشه! قرار بود تحسین
هم بشود.

این سیستم‌عامل چیزی نبود به جز یک ابزار خاص برای چند نفری که به نوشتن و
بررسی سیستم‌عامل‌ها علاقمند بودند. آدم‌های بسیار فنی و در بین آدم‌های فنی
هم یک گروه خیلی خاص با علاقه‌ای مشترک.

باز‌خورد آن‌ها مثبت بود ولی مثبت به این معنی که \dbquote{خوب است این کار
  را هم بکند} یا \dbquote{به نظر جالب می‌رسد ولی روی دستگاه من که اجرا
  نشد.}

یک ایمیل را دقیق به خاطر دارم که نوشته بود بسیار از سیستم‌عامل من خوشش
آمده و یک پاراگراف را اختصاص داده بود به اینکه بگوید چقدر این برنامه
خوب است. بعد در این باره نوشته بود که کل هارددیسک کامپیوترش به خاطر
این آزمایش از بین رفته و درایور دیسک سخت باید یک مشکلی داشته باشد. او
تمام کاری که کرده بود را از دست داده بود، ولی هنوز کاملا مثبت برخورد
می‌کرد. خواندن اینجور ایمیل‌ها مفرح بود. یک گزارش در مورد باگی که کل
اطلاعات کسی را نابود کرده بود.

این دقیقا همان باز‌خوردی بود که من به دنبالش بودم. بعضی از باگ‌ها از
جمله آن باگی که باعث می‌شد پر شدن حافظه به کرش بیانجامد را کشف و رفع
کردم. قدم بزرگ انتقال \code{gcc} به لینوکس را هم برداشتم و نتیجه‌اش این
بود که حالا می‌شد برنامه‌های کوچک را در خود لینوکس نوشت. به عبارت دیگر
لازم نبود مردم قبل از نصب لینوکس، \code{gcc} من را نصب کنند.

\section{بخش نهم}
\noindent آیا شما هم اندوه روزهایی را می‌خورید که مردان، مرد بودند و
شخصا درایورهای شان را می‌نوشتند؟

\hfill -- اطلاعیه ارسال لینوکس نسخه \code{0.02}

\vspace*{10pt}

اوایل اکتبر، نسخه \code{0.02} ارائه شد که شامل اصلاح چند باگ و اضافه
شدن چند برنامه‌ جدید بود. ماه بعد نسخه \code{0.03} را منتشر کردم.

احتمالا در اواخر سال ۱۹۹۱ کار را متوقف می‌کردم. خیلی از کارهایی که به
نظرم جالب می‌آمد را تمام کرده بودم. همه چیز به شکل کامل کار نکرده بود
ولی کشف کرده بودم که در دنیای نرم‌افزار همین که احساس کردید مسایل
پایه‌ای را حل کرده‌اید، خیلی راحت انگیزه خود را برای حل جزییات از دست
می‌دهید. این همان چیزی بود که داشت برای من هم پیش می‌آمد. تلاش برای
باگ‌زدایی نرم‌افزار کار جذابی نیست. اما دو چیز اتفاق افتاد که باعث شد
راه را ادامه دهم. اول اینکه به شکل اتفاقی پارتیشن مینیکس کامپیوترم را
خراب و نابود کردم. دوم اینکه مردم هنوز برایم بازخورد می‌فرستادند.

تا آن روز با اینکه کامپیوتر را در لینوکس بوت می‌کردم، از مینیکس به
عنوان محل اصلی توسعه نرم‌افزار استفاده می‌کردم. بیشترین کاری که در
لینوکس می‌کردم، عبارت بود از خواندن خبرها از کامپیوتر دانشگاه توسط
برنامه شبیه‌ساز ترمینالی که نوشته بودم و از آن‌جایی که خط تلفن کامپیوتر
دانشگاه همیشه مشغول بود،‌ یک برنامه کوچک نوشته بودم که به شکل خودکار آن
قدر شماره می‌گرفت تا بالاخره خط آزاد شود. اما در دسامبر، اشتباها به جای
شماره گرفتن روی مودم، روی هارددیسک شماره گرفتم. در اصل قرار بود
\code{/dev/tty1} که درگاه سریال مودم بود را به شماره‌گیر بدهم، ولی
اشتباها \code{/dev/hda1} را به عنوان ابزار به شماره‌گیر خودکار دادم که
مشخص کننده دیسک‌سخت کامپیوترم بود. نتیجه کار این بود که اطلاعات
نامناسبی در حساس‌ترین نقاط سخت‌دیسک نوشته شد. درست جایی که مینیکس از آن
بوت می‌شد و دیگر نتوانستم مینیکس را بوت کنم.

این همان مرحله لحظه حساس بود: باید تصمیم می‌گرفتم که مینیکس را از اول
نصب کنم یا بپذیریم که لینوکس آن قدر کارا شده که برای کارهایم نیازی به
مینیکس ندارم. در حالت دوم باید برنامه‌های جدید برای لینوکس را در خود
لینوکس می‌نوشتم و هر وقت هم احساس می‌کردم که به خصوصیتی از مینیکس احتیاج
دارم که در لینوکس نیست، باید آن را به لینوکس اضافه می‌کردم. از نظر
مفهومی، ترک کردن محیط توسعه مادر و متکی کردن یک سیستم‌عامل به خودش قدمی
آن قدر بزرگ است که تصمیم گرفتم نسخه بعدی که در اواخر نوامبر منتشر شد
را \code{0.10} بنامم. چند هفته بعد، نسخه \code{0.11} هم درآمد.

از این‌جا بود که کم کم مردم واقعا شروع کردند به استفاده از لینوکس و
انجام کارهایی تحت آن. تا این موقع حداکثر چند باگ‌زدایی تک خطی برایم
ارسال می‌شد. ولی حالا دیگر مردم شروع کرده بودند به اضافه کردن قابلیت‌های
جدید به لینوکس و فرستادن آن‌ها برای من. یادم هست که رفتم و حافظه
کامپیوترم را از ۴ مگابایت به ۸ مگابایت ارتقاء دادم تا حافظه کافی برای
کارها داشته باشم. همچنین به بازار رفتم و یک کمک‌پردازنده ریاضی هم خریدم
چون دائما از من سوال می‌شد که آیا لینوکس از کمک‌پردازنده‌ها هم پشتیبانی
می‌کند یا نه. این سخت‌افزار جدید به کامپیوترم اجازه می‌داد تا محاسبات
اعداد اعشاری را بدون دردسر انجام دهد.

یادم هست که در دسامبر، آقایی از آلمان که فقط ۲ مگابایت رم داشت،
می‌خواست کرنل را کمپایل کند ولی نمی‌توانست \code{gcc} را اجرا کند چرا که
\code{gcc} به تنهایی بیشتر از یک مگابایت رم می‌خواست. او از من پرسید که
آیا می‌توان لینوکس را با کمپایلر کوچکتری که اینهمه حافظه نخواهد کمپایل
کرد. من هم تصمیم‌ گرفتم با وجود اینکه خودم به این موضوع نیازی نداشتم،
آن را فقط به خاطر او برآورده کنم. این خاصیت
حافظه-به-دیسک\LFootnote{Page to Disk} خوانده می‌شود و به این معنا است
که کسی که فقط دومگابایت حافظه دارد، می‌تواند برای جبران این نقیصه، از
دیسک به عنوان حافظه رم استفاده کند. تاریخ این ماجرا به حدود کریسمس
۱۹۹۱ برمی‌گردد. یادم هست که روز ۲۳ دسامبر داشتم تلاش می‌کردم حافظه به
دیسک را راه بیندازم. روز بیست و چهارم برنامه کار می‌کرد ولی گاه گداری
باعث کرش سیستم می‌شد. روز بیست و پنجم همه چیز به درستی کار می‌کرد. این
عملا اولین خصوصیتی بود که به خاطر یک نفر دیگر به لینوکس اضافه کرده
بودم.

و به این افتخار می‌کردم. 

تا به حال در این مورد به خانواده‌ام که گاه گداری برای خوردن یک وعده
گوشت و ماهی هارینگ در خانه مادربزرگ پدری (فارمار!) جمع می‌شدند،‌ چیزی
نگفته بودم. جامعه کاربران لینوکس به شکل روزانه در حال گسترش بود و حالا
دیگر هر روز از جاهایی که آرزوی دیدن‌شان را داشتم، ایمیل دریافت
می‌کردم. جاهایی مثل استرالیا و آمریکا. نپرسید چرا ولی هیچ وقت احساس
نکردم باید در این باره چیزی به مادر و پدرم، خواهرم یا بقیه فامیل
بگویم. آن‌ها از کامپیوتر سر در نمی‌آوردند. تصورم این بود که نخواهند
فهمید چه چیزی در جریان است.

تا آن‌جایی که به آن‌ها مربوط می‌شد،‌ کار من فقط اشغال کردن دائمی تلفن
بود. در هلسینکی پول تلفن در طول شب ثابت بود و به همین علت من هم سعی
می‌کردم بیشتر کارم را در دیروقت انجام دهم ولی خب گاهی هم تلفن در تمام
طول روز اشغال می‌ماند. حتی سعی کردم یک خط تلفن مجزا برای خودم بگیرم ولی
ساختمانی که خانه مادرم در آن قرار داشت آن قدر قدیمی بود که هیچ خط
اضافه‌ای نداشت و کسی هم علاقه‌ای به کشیدن خطوط جدید برای آن احساس
نمی‌کرد. سارا در آن دوره کاری نداشت جز اینکه با دوستانش تلفنی صحبت
کند. حداقل برداشت من که این بود. پس گاه‌گداری با هم دعوا داشتیم. البته
دعواهای مجازی. وقتی او مشغول حرف زدن بود من مودم را تنظیم می‌کردم تا
شماره بگیرد و حاصل اینکار صداهای بیب-بیب-بیببب در تلفن بود. اینکار
سارا را عصبانی می‌کرد ولی در عوض می‌فهمید که من واقعا به آزاد شدن خط
تلفن و خواندن ایمیل‌هایم احتیاج دارم. هیچ وقت ادعا نکرده‌ام که بهترین
برادر بزرگتر دنیا هستم.

حافظه به دیسک قدم بزرگی بود چون مینیکس هیچ وقت به سراغ آن نرفته
بود. این قابلیت در نسخه \code{0.12} اضافه شد که در اولین هفته از ژاویه
۱۹۹۲ توزیع شد. مردم سریعا شروع کردند به مقایسه لینوکس نه فقط با مینیکس
که با کوهیرنت\LFootnote{Coherent} که نسخه کوچکی از یونیکس بود و توسط
شرکت مارک ویلیامس\LFootnote{Mark Williams Company} گسترش یافته
بود. اضافه کردن حافظه به دیسک، باعث شده بود لینوکس از رقبای خود یک سر
و گردن جلوتر باشد.

خیز لینوکس از همان روز شروع شد. حالا کسانی را داشتیم که از مینیکس به
لینوکس سوییچ می‌کردند. در آن موقع لینوکس قادر نبود همه کارهایی که
مینیکس می‌کرد را انجام دهد، ولی از پس اکثر کارهایی که برای مردم ارزش
داشت، برمی‌آمد. البته لینوکس حالا یک قابلیت جدید هم داشت که همه به
دنبال آن بودند: حافظه-به-دیسکی که می‌توانست باعث شود افراد قادر باشند
برنامه‌هایی بزرگتر از حافظه کامپیوترشان را اجرا کنند. معنی این قابلیت
آن است که هر وقت حافظه کامپیوتر کم آمد،‌ بخشی از حافظه به دیسک منتقل
می‌شود و سیستم‌عامل به یاد می‌سپارد که آن را از کجا برداشته و در کجا
ذخیره کرده و در نهایت مقداری از حافظه که به این روش خالی شده است را به
برنامه‌های جدید اختصاص می‌دهد. این جریان برای هفته‌های اول سال ۱۹۹۲ چیز
مهمی به حساب می‌‌آمد.

ماه ژانویه بود که تعداد کاربران لینوکس از پنج، ده و بیست نفری که من
می‌توانستم با آن‌ها ایمیل داشته باشم و اسم‌های شان را به خاطر بسپارم
فراتر رفت و به صدها نفری رسید که دیگر قابل شناسایی نبودند. من همه
کاربران لینوکس را نمی‌شناختم و این مفرح بود.

درست در همان روزها یکی از این دروغ‌های اینترنتی هم در حال گردش در شبکه
بود. یک پسر فقیر به اسم کریگ\LFootnote{Craig} در حال مرگ از سرطان بود
و یک نامه زنجیره‌ای مشهور از شما می‌خواست که برایش کارت پستال
بفرستید. بعدا معلوم شد که این جریان شوخی‌ بیمارگونه یک آدم است. احتمالا
هیچ وقت کریگی وجود نداشته، چه برسد به اینکه از سرطان در حال مرگ
باشد. اما به هرحال این درخواست میلیون‌ها کارت پستال به آن آدرس جاری
کرد. من هم وقتی از مردم خواستم که در صورت استفاده از لینوکس به جای پول
برایم کارت پستال بفرستند، حرفی نیمه جدی و نیمه شوخی زده بودم. از نظر
من آن نامه یک جور جوک \dbquote{خدایا! یک ایمیل دیگه با درخواست ارسال
  کارت پستال} بود. آن روزها در دنیای کامپیوترهای شخصی، گرایش زیادی به
نرم‌افزارهای اشتراک‌افزار\LFootnote{Shareware} وجود داشت. برنامه را
دانلود می‌کردید و در صورت استفاده از شما انتظار می‌رفت که مبلغی در حد ده
دلار برای نویسنده بفرستید. مردم هم برای من ایمیل می‌زدند و می‌پرسیدند که
آیا علاقمندم پولی در حد سی دلار برایم بفرستند یا نه. باید جوابی به
آن‌ها می‌دادم.

الان که به گذشته نگاه می‌کنم به نظرم می‌رسد که درخواست پول ممکن بود مفید
باشد. چیزی حدود ۵۰۰۰ دلار وام دانشجویی داشتم و ماهی هم باید ۵۰ دلار
قسط کامپیوترم را می‌دادم. خرج‌های دیگرم عبارت بودند از پیتزا و
آبجو. البته لینوکس آن قدر من را مشغول خودش کرده بود که به ندرت بیرون
می‌رفتم؛ شاید حداکثر هفته‌ای یک بار. برای بیرون بردن دخترها هم که هیچ
پولی لازم نداشتم و هرچند امکان خرج پول برای ارتقاء سخت‌افزاری وجود
داشت، ضرورتی به این کار احساس نمی‌کردم. شاید یک پسر دیگر، برای
نرم‌افزاری که نوشته بود درخواست پول می‌کرد و آن را به عنوان بخشی از
اجاره خانه به مادر تک‌سرپرستش می‌داد. من هیچ وقت به این فکر نیافتادم. از
من شاکی باشید.

من بیشتر علاقمند بودم تا ببینم که مردم واقعا از لینوکس استفاده
می‌کنند. به جای پول، از آن‌ها کارت پستال خواستم و از همه جا کارت‌پستال
سرازیر شد. از نیوزلند گرفته تا ژاپن و از هلند تا ایالات متحده. معمولا
سارا بود که نامه‌ها را چک می‌کرد و به ناگهان متعجب شده بود که چطور این
برادر پردردسرش یکهو این‌همه دوست از سراسر دنیا پیدا کرده. این اولین
باری بود که احساس می‌کرد من در آن همه ساعتی که تلفن اشغال بود، مشغول
کار مفیدی بودم. تعداد کارت‌پستال‌ها به صدها عدد رسیده بود ولی هیچ ایده‌ای
ندارم که چه بلایی سر آن‌ها آمده است. احتمالا در یکی از اسباب‌کشی‌ها گم
شده‌اند. آووتون من را \dbquote{شخصی با حداقل نوستالژی ممکن} می‌خواند.

در حقیقت پول نخواستن من دلایل متعددی داشت. وقتی برای اولین بار لینوکس
را به اینترنت می‌فرستادم، احساس می‌کردم که قدم در مسیری گذاشته‌ام که
قرن‌ها دانشمندان و دانشگاه یان در آن حرکت کرده‌اند. به گفته سر ایزاک
نیوتن، احساس می‌کردم روی دوش غول‌ها ایستاده‌ام. حاصل کارم را به اشتراک
گذاشته بودم تا دیگران علاوه بر استفاده از آن، به من بازخورد هم دهند
(قبول! همچنین دنبال تمجید هم بودم). اینکه از کسانی که توانایی بهتر
کردن کار مرا داشتند، پول درخواست کنم چندان منطقی نبود. شاید اگر در
جایی به جز فنلاند که در آن بروز دادن کوچکترین نشانه‌ای از خست، با شک و
تردید نگریسته می‌شود بزرگ شده بودم، روش دیگری در پیش می‌گرفتم (البته این
قضیه بعد از موفقیت چشمگیر نوکیا و پیش آمدن این جریان که در جیب هر آدمی
در هر کجای جهان یک گوشی نوکیا است و حساب‌های بانکی تعدادی فنلاندی از
این راه هر روز پرتر و پرتر می‌شود، تا حدی تغییر کرده است). و بله! شاید
اگر تحت نظر یک پدربزرگ فدایی دانشگاه و یک پدر فدایی کمونیسم رشد نکرده
بودم هم، روند دیگری در پیش می‌گرفتم.

به هرحال به دنبال فروش لینوکس نبودم. البته نمی‌خواستم کنترلم بر آن‌ را
هم از دست بدهم. یعنی نمی‌خواستم کس دیگری توان فروش آن را داشته
باشد. این موضوع را به طور مشخص در یادداشت کپی‌رایتی که همراه نسخه‌
اولیه‌ای که در سپتامبر پخش کردم،، مشخص کرده‌ بودم. خوشبختانه بنا به
موافقتنامه برن که در قرن نوزدهم تصویب شده، شما مالک کپی‌رایت چیزی هستید
که تولید کرده‌اید مگر اینکه آن را به دیگری واگذار کنید. من به عنوان
صاحب کپی‌رایت، حق داشتم قوانین را مشخص کنم: حق دارید از سیستم‌عامل به
شکل رایگان استفاده کنید به شرطی که آن را به کسی نفروشید و اگر تغییری
در کدها دادید باید آن‌ها را به شکل کد منبع (و نه کدهای باینری که
غیرقابل دسترسی هستند) برای استفاده همگانی منتشر کنید. اگر شما با این
قوانین موافق نبودید، حق نداشتید کد اصلی را کپی کنید یا در آن تغییری
دهید.

خودتان قضاوت کنید. شش ماه از زندگی‌تان را روی چیزی می‌گذارید و می‌خواهید
آن را برای همه قابل دسترسی کنید، ولی نمی‌خواهید کس دیگری کنترل آن را در
دست بگیرد. من دوست داشتم مردم به این کد دسترسی داشته باشند و از آن
استفاده کنند و بنا به سلیقه خود آن را بهبود بخشند. اما در عین حال
می‌خواستم که بدانم مردم دارند با آن چکار می‌کنند. لازم بود من هم به کد
اصلی دسترسی داشته باشم تا اگر کسی تغییری مثبتی ایجاد کرد، خودم هم
بتوانم از آن بهره‌مند شوم. از نظر من بهترین روش برای کمک به توسعه
لینوکس این بود که آن را پاک نگه دارم. ورود پول به ماجرا، آب را گل‌آلود
می‌کرد. اگر پولی در بین نباشد، آدم‌های طمّاع هم وارد بازی نمی‌شوند.

با اینکه من علاقه‌ای به درخواست پول در مقابل لینوکس نداشتم، بعضی‌ها نسبت
به اینکه در مقابل دادن دیسک‌های حاوی سیستم‌عامل به دیگران درخواست کمی
پول داوطلبانه بکنند، شرمی نداشتند. در فوریه دیگر عجیب نبود اگر آدم‌هایی
را می‌دیدید که با دیسک‌های حاوی لینوکس در دست، به سراغ نشست‌های مرتبط با
یونیکس می‌روند. آن‌ها شروع کرده بودند به پرسیدن اینکه آیا اشکالی دارد
اگر در مقابل هر دیسک مبلغی در حد پنج دلار درخواست کنند که هزینه دیسک و
زمان مصرف شده را پوشش دهد. مشکل این بود که اینکار مخالف کپی‌رایت نوشته
شده توسط من بود.

دیگر وقت آن بود تا درباره سیاست \dbquote{لینوکس برای فروش نیست} تجدید
نظر کنم. از طرفی بحث‌های آنلاین در مورد لینوکس هم آن قدر زیاد شده بود
که دیگر نگران نبودم کسی لینوکس را برای خودش بردارد و فرار کند؛ چیزی که
بزرگترین کابوس من بود. حداقل انجام این کار بدون ایجاد کلی واکنش منفی،
امکان نداشت. اگر کسی به فکرش می‌زد تا لینوکس را بدزدد و آن را به یک
نرم‌افزار تجاری تبدیل کند بدون شک با واکنش‌های منفی زیادی روبرو
می‌شد. هکرهای زیادی در جامعه لینوکس بودند تا با دیدن این صحنه داد بکشند
که \dbquote{هی! این لینوکس است! تو حق نداری این کار را بکنی.} البته نه
به این مودبی که من گفتم.

چرخ لینوکس به حرکت درآمده بود. هر روز هکرهایی از سراسر دنیا تغییرات
پیشنهادی خود را برای من می‌فرستادند. ما به شکل دست جمعی در حال خلق
بهترین سیستم‌عامل این حوالی بودیم و به راحتی هم ممکن نبود از مسیر منحرف
شویم. به همین دلیل و از آن‌جایی که لینوکس دیگر شناخته شده بود، احساس
کردم اشکالی ندارد اگر مردم شروع به فروش آن کنند.

البته قبل از اینکه خودم را آقای نیکوکار جا بزنم، اجازه بدهید یک نکته
حیاتی دیگر در مورد این تصمیم را شرح دهم. واقعیت این است که برای
کاربردی کردن لینوکس، از ابزارهای زیادی استفاده کرده بودم که به شکل
آزاد روی اینترنت قرار داده شده بودند. من روی دوش غول‌ها بالا رفته
بودم. یکی از مهمترین این نرم‌افزارهای آزاد کمپایلر \code{gcc} بود. این
نرم‌افزار تحت کپی‌رایت پروانه جامع همگانی\LFootnote{General Public
  License} یا به شکلی که بیشتر در سطح جهان شناخته شده است \lr{GPL} (یا
کپی‌لفت) که فرزند معنوی ریچارد استالمن بود،‌ منتشر شده بود. در دیدگاه
\lr{GPL} پول جایگاهی ندارد. اگر کسی علاقمند به پرداخت باشد، می‌توانید
میلیون‌ها دلار از او درخواست کنید، اما باید کدهای منبع را هم در اختیار
بگذارید. در عین حال کسی که کدهای منبع را از شما می‌خرد یا می‌گیرد تمامی
حقوق شما را هم خواهد داشت. این یک ابزار فوق‌العاده است. البته من بر
خلاف طرفداران پر و پا قرص \lr{GPL} که معتقدند هر ابداع جدید نرم‌افزاری
باید بر اساس پروانه جامع همگانی برای تمام جهانیان قابل استفاده شود،
اعتقاد دارم که مبتکرین حق دارند در مورد شیوه استفاده از اختراع شان
شخصا تصمیم بگیرند.

من کپی‌رایت قدیمی را کنار گذاشتم و از \lr{GPL} استفاده کردم. یعنی از
کپی‌رایتی که استالمن آن را نوشته و گروهی از وکلا آن را بررسی کرده‌ان
(چون وکلا درگیر ماجرا هستند، این سند چندین صفحه را اشغال می‌کند).

کپی‌رایت جدید از نسخه \code{0.12} اعمال شد و یادم هست که شب اول از فکر
اینکه بخش تجاری با محصول من چه کار خواهد کرد، خوابم نبرد. حالا که به
گذشته نگاه می‌کنم این نگرانی به نظرم خنده‌دار می‌رسد، چون بخش تجاری توجه
نسبتا کمی به این جریان نشان داد. چیزی به من می‌گفت که باید مواظب
باشم. یکی از نگرانی‌هایم این بود - و هنوز هم هست - که کسی بیاید و
لینوکس را بدون توجه به کپی‌رایتش صاحب شود. آن موقع نگران این بود که
شکایت از کسی که در آمریکا این کپی‌رایت را نقض کند عملا غیر ممکن
است. هنوز هم این نگرانی را دارم. شکایت کردن و تعقیب قضایی افراد در این
گونه موارد مشکل نیست ولی من نگران افرادی هستم که تا وقتی قانونا متوقف
نشده‌اند، به این استفاده غیرقانونی ادامه می‌دهند.

و این ترس آزار دهنده هم هست که شرکت‌هایی در جاهایی مثل چین بدون توجه به
\lr{GPL} هرکار که بخواهند می‌کنند. عملا هیچ چیزی در قانون آن‌ها نیست که
جلوی نقض کپی‌رایت را بگیرد و در دنیای واقعی هم پیگیری قضایی این
قانون‌شکنان هیچ فایده‌ای نخواهد داشت. این همان کاری است که شرکت‌های
نرم‌افزاری بزرگ و صنایع موسیقی سعی کرده‌اند انجام دهند و تا امروز موفقیت
چندانی هم به دست نیاورده‌اند. نگرانی‌های من در برخورد با وقایع، تخفیف
پیدا کردند. شاید کسی برای مدتی کپی‌رایت را نقض کند، ولی در نهایت
آدم‌هایی که به قانون احترام می‌گذارند و آن‌هایی که تغییرات خود را برای
همه قابل دسترس می‌کنند، پیش می‌‌افتند. آن‌ها بخشی از روند پیشرفت کرنل
هستند. در مقابل آن‌هایی که تغییرات خود را در اختیار دیگران نمی‌گذراند
همان‌هایی هستند که از به روزرسانی‌ها هم بهره‌ای نمی‌برند و عقب می‌مانند و
مشتریان شان را از دست می‌دهند. این امید من است.

در کل من کپی‌رایت را از دو دیدگاه می‌بینم. فرض کنید کسی هست که روزی ۵۰
دلار درآمد دارد. آیا انتظار دارید این آدم ۲۵۰ دلار پول یک نرم‌افزار را
بدهد؟‌ به نظر من که اگر از نسخه غیرقانونی استفاده کند و آن ۵۰ دلار را
خرج غذا کند، کار غیراخلاقیی نکرده. این شکل از نقض کپی‌رایت، اخلاقا
مشکلی ندارد. به نظرم غیر اخلاقی- و احمقانه - است اگر کسی این
\dbquote{خلافکار} را تحت تعقیب قضایی قرار دهد. در مورد لینوکس هم مهم
نیست اگر یک نفر بدون توجه به \lr{GPL} از آن برای کاربردهای شخصی
استفاده کند. بحث بر سر کسی است که به دنبال پول‌دار شدن سریع است. اینکار
به نظر من غیراخلاقی است؛ چه در آفریقا باشد و چه در آمریکا. تازه همین‌جا
هم درجه‌بندی‌های مختلفی هست.

اما به هرحال طمع هیچ وقت خوب نیست.

\section{بخش دهم: مینیکس در مقابل لینوکس}
همه توجهات هم مثبت نبود. هرچند که هیچ وقت اهل جنگ و دعوا نبوده‌ام، اما
وقتی آندرو تاننباوم شروع به حمله به سیستم‌عاملی کرد که در حال جلو
افتادن از سیستم‌عامل خودش بود، باید از سیستم‌عامل و مردانگی‌ام دفاع
می‌کردم. از آن‌جایی که ماها نِرد هستیم، همه چیز با ایمیل پیش رفت.

البته چه کسی می‌تواند به او به خاطر عصبانیت‌اش ایراد بگیرد؟‌ قبل از اینکه
گروه خبری لینوکس تاسیس شود، من دائما از طریق گروه خبری مینیکس
اطلاعیه‌های لینوکس را پخش می‌کردم و از همان‌جا هم افراد علاقمند را پیدا
کردم. چرا اندرو باید از این جریان راضی باشد؟

برای تازه‌کارها بگویم که تاننباوم از این ناراضی بود که من از قواعد گروه
خبریش تخلف کرده بودم. در عین حال بدون شک از این هم ناراحت بود که
سیستم‌عاملش دارد زیر سایه یک سیستم‌عامل جدید قرار می‌گیرد که به تازگی از
جنگل‌های سرد فنلاند آمده و توسعه‌دهندگان زیادی مشغول جذب شدن به آن
هستند. در نهایت هم اینکه او نظر متفاوتی درباره شیوه صحیح نوشتن
سیستم‌عامل‌ها داشت. آن روزها آندرو جزو آن دسته از دانشمندان کامپیوتر بود
که می‌گفتند گرایش میکروکرنل بهترین شیوه طراحی سیستم‌عامل است. وی مینیکس
را هم به همین روش طراحی کرده بود. وضع آموئبا\LFootnote{Amoeba} که
سیستمی بود که آن روزها رویش کار می‌کرد، نیز به همین منوال بود.

این شیوه غالب اواخر دهه ۱۹۸۰ و اوایل ۱۹۹۰ بود، اما موفقیت لینوکس مشغول
تضعیف این ایده بود. به همین دلیل آندرو به فرستادن ایمیل‌های دوست
نداشتنی ادامه داد.

نظریه پشت میکروکرنل این است که سیستم‌عامل ذاتا چیز پیچیده‌ای است و در
نتیجه باید سعی کنیم با بخش بخش کردن آن، از پیچیدگی آن بکاهیم. پایه
ایده میکروکرنل این است که کرنل باید هسته هسته هسته باشد. به عبارت دیگر
کرنل باید حداقل کار ممکن را انجام بدهد. وظیفه اصلی کرنل برقراری ارتباط
است. هر چیزی که کامپیوتر بخواهد ارائه بدهد، سرویس‌هایی است که از طریق
کانال‌های ارتباطی می‌شود به آن‌ها دست پیدا کرد. در گرایش میکروکرنل، هر
مساله باید آن قدر کوچک شود تا دیگر هیچ بخش پیچیده‌ای در آن باقی نماند.

به نظر من اینکار احمقانه می‌آمد. درست است که هر بخش ساده،‌ کاملا ساده
است اما ارتباط این بخش‌های ساده، بسیار پیچیده‌تر از حالتی که می‌شود که
این سرویس‌ها به خود کرنل اضافه می‌شدند؛ مانند لینوکس. به مغز فکر
کنید. هر بخش مغز بسیار ساده است ولی روابط بسیار پیچیده این اجزاء،
می‌تواند یک سیستم بی‌نهایت پیچیده را ایجاد کند. این همان مشکل کلاسیک
\dbquote{بزرگ‌تر بودن کل از جزء} است. اگر چیزی را بردارید و نصف کنید و
بعد ادعا کنید که هر نیمه، پیچیدگی‌ای نصف پیچیدگی کل دارد،‌ پیچیدگی روابط
بین این دو نیمه را نادیده گرفته‌اید. ایده پشت میکروکرنل این است که کرنل
را به پنجاه قسمت مستقل تقسیم کنید و در نتیجه پیچیدگی هر قسمت بشود یک
پنجاهم پیچیدگی کرنل اولیه. اما چیزی که در نظر گرفته نمی‌شود، این واقعیت
است که پیچیدگی روابط بین این اجزا پیچیده‌تر از کل سیستم اولیه است و
تازه این در حالی است که اجزا هم آن قدرها ساده\RFootnote{لینوکس در
  اینجا از واژه \lr{Trivial} استفاده کرده که در دنیای برنامه نویسی به
  معنی برنامه‌ای است که آنقدر ساده شده که تلاش برای ساده‌تر کردن آن، وقت
  هدر دادن و حتی ایجاد کننده مشکلات خواهد بود.} نشده‌اند.

این اصلی‌ترین استدلال علیه میکروکرنل است. سادگیی که به دنبالش هستید، یک
توهم است.

لینوکس بسیار کوچک‌تر و بسیار ساده‌تر کار را شروع کرد و هیچ‌وقت هم ماژوله
شدن را اجبار نکرد. در نتیجه می‌توانستید هر کاری که می‌خواهید را بسیار
سرراست‌تر از مینیکس پیاده‌سازی کنید. یکی از مشکلات پایه‌ای من با مینیکس
این بود که اگر پنج برنامه مختلف را در آن اجرا می‌کردید و آن‌ها می‌خواستند
به پنج فایل مختلف دسترسی داشته باشند، کل کار به شکل سری انجام می‌شد. به
عبارت دیگر پنج پروسه داشتید که به سیستم فایل پیام می‌دادند:
\dbquote{ممکن است من از فلان فایل بخوانم؟} و بخشی از سیستم‌عامل که
مسوول پاسخ به این درخواست بود، یکی از آن‌ها را انتخاب می‌کرد و جوابش را
می‌داد و بعد سراغ درخواست بعدی می‌رفت.

تحت لینوکس که کرنلی‌است مونولیتیک، در این حالت پنج پروسه خواهید داشت که
هر کدام فراخوانی سیستمی خود را برای کرنل می‌فرستند. کرنل باید بسیار دقت
کند این پنج درخواست با هم قاطی نشوند، ولی در عوض می‌تواند به این پنج
پروسه و هرچند پروسه دیگر که نیاز داشته باشند، امکان دسترسی دائمی به
فایل‌ها را بدهد.

مشکل دیگر مینیکس این بود که علی‌رغم در اختیار داشتن متن آن، بنابر
توافقنامه‌اش نمی‌توانستید کار چندانی با آن متن بکنید. مثلا بروس اوانز را
در نظر بگیرید که تغییرات عمده‌ای در مینیکس داده بود و آن را بسیار
کاراتر کرده بود، ولی اجازه نداشت این تغییرات را به خود مینیکس اضافه
کند بلکه باید همه چیز را به صورت وصله‌های جانبی ارائه می‌کرد. از نقطه
نظر عملی، این یک فاجعه است. مثلا امکان ندارد بروس یک نسخه قابل اجرا از
تغییراتش ایجاد کند و به مردم اجازه بدهد تا به سادگی از نسخه‌ای بهتر
استفاده کنند. مردم به جای اینکار مجبور هستند برای رسیدن به سیستمی قابل
استفاده،‌ روندی چند مرحله‌ای را قدم به قدم طی کنند و اینکار برای بسیاری
از مردم، غیرعملی است.

اوایل ۱۹۹۲، تنها باری بود که کار به ارتباط مستقیم با آندرو تاننباوم
کشید. فرض کنید یک صبح یخبندان به سیستم لاگین کنید و با نسخه‌ای ویرایش
نشده از این پیام مواجه شوید:

\begin{emailbox}
\noindent\textbf{از:} \code{ast@cs.vu.nl} (اندی تاننباوم)

\noindent\textbf{به:} گروه خبری \code{comp.os.minix}

\noindent\textbf{موضوع:} دوره لینوکس گذشته است

\noindent\textbf{تاریخ:} ۲۹ ژانویه ۹۲ ساعت ۱۲:۱۲:۵۰ جی.ام.تی.

دو هفته‌ای در آمریکا بودم و در نتیجه وقت نکردم در مورد لینوکس نظرم را
بگویم (البته اگر بودم هم حرف چندانی برای گفتن نداشتم) ولی حالا به هر
دلیلی که باشد، حرف‌هایی دارم که باید بزنم.

همان طور که اکثر شما می‌دانید، مینیکس برای من یک سرگرمی شخصی است؛ چیزی
که بعد از ظهرها اگر از نوشتن کتاب خسته شده باشم و جنگ یا انقلاب یا بحث
مهمی در سنا هم نباشد که مستقیما از سی.ان.ان. پخش شود، به سراغش
می‌روم. شغل اصلی من، استادی دانشگاه و تحقیق در حوزه سیستم‌عامل‌ها است.

بنا به شغلم، حس می‌کنم تا حدی می‌دانم که سیستم‌عامل‌ها در یکی دو دهه آینده
به کدام سمت خواهند رفت. در دیدگاه من دو نکته مهم خودنمایی می‌کند:

\noindent ۱. سیستم‌های میکروکرنل در برابر مونولیتیک‌ها

بیشتر سیستم‌عامل‌های قدیمی مونولیتیک‌ هستند. یعنی کل سیستم‌عامل یک فایل
بزرگ \code{a.out} است که در \dbquote{حالت کرنل} اجرا می‌شود. این فایل
اجرایی حاوی مدیر پروسه‌ها، مدیر حافظه، سیستم فایل و تمام دیگر اجزای
مورد نیاز است. مثال‌هایی از این گونه، عبارت هستند از یونیکس، ام.اس-داس،
وی.ام.اس.، ام.وی.اس، او.اس.۳۶۰، مالتیکس و بسیاری دیگر.

در مقابل سیستم‌عامل‌های میکروکرنل را داریم که در آن‌ها اکثر عملیات سیستم
به شکل پروسه‌های مستقلی در خارج از کرنل پیاده سازی می‌شود. این پروسه‌ها
برای ارتباط از سیستم‌پیام رسان استفاده می‌کنند. وظیفه کرنل برقراری این
سیستم پیام‌رسانی، مدیریت وقفه‌ها، مدیریت سطح پایین پروسه‌ها و احتمالا
ورودی و خروجی است. نمونه‌هایی از این ایده عبارت‌ هستند از آر.سی.۴۰۰۰،
آموئبا، کروس، ماخ و ویندوز ان.تی. که هنوز منتشر نشده است.

هرچند می‌توانم درباره مزایا و معایب هریک داستان هزار و یک شب تعریف کنم
اما گفتن همین اکتفا می‌کنم که در بین کسانی که واقعا به طراحی سیستم‌عامل
اشتغال دارند، بحث تمام شده است. میکروکرنل برنده شده. مینیکس یک
سیستم‌عامل میکروکرنل است که در آن مدیریت حافظه و سیستم فایل دو پروسه
مجزا هستند که خارج از کرنل اجرا می‌شوند. درایورهای ورودی و خروجی‌ هم
پروسه‌های خاص خودشان را دارند. لینوکس یک سیستم به سبک مونولیتیک‌ است؛ یک
قدم بزرگ به عقب. قدمی به دهه ۱۹۷۰.

\noindent ۲. قابلیت انتقال

مینیکس طراحی شد تا قابلیت انتقال داشته باشد و تا به حال هم از سری
اینتل گرفته تا \code{680x0} (آتاری،‌ آمیگا، مکینتاش) و اسپارک و
\code{NS32016} آن را اجرا می‌کنند. لینوکس شدیدا به \code{80x86} وابسته
است و جای دیگری ندارد که برود.

البته اشتباه نشود. من به خاطر لینوکس خوشحالم چون تمام افرادی که سعی
می‌کنند مینیکس را به یک بی.اس.دی. یونیکس تبدیل کنند را از من دور
می‌کند. به هرحال با کمال صداقت بگویم که به تمام کسانی که به دنبال یک
سیستم‌عامل *مدرن* و *آزاد* هستند، پیشنهاد می‌کنم تا به دنبال یک سیستم
میکروکرنل و قابل انتقال باشند. مثلا گنو یا چیزی شبیه به آن.

\hfill اندی تاننباوم (\code{ast@cs.vu.nl})
\end{emailbox}

می‌دانستم که باید از شرافتم دفاع کنم پس نوشتم:

\begin{emailbox}
\noindent\textbf{از:} \code{torvalds@klaava.helsinki.fi} (لینوس بندیکت توروالدز)

\noindent\textbf{موضوع:} پاسخ به: دوره لینوکس گذشته است 

\noindent\textbf{تاریخ:} ۲۹ ژانویه ۹۲ ساعت ۲۳:۱۴:۲۶

\noindent\textbf{سازمان:} دانشگاه هلسینکی

خب با همچین موضوع بحثی، فکر کنم من هم باید چیزی بنویسم. پیشاپیش از
خوانندگان مینیکس از اینکه بازهم درباره لینوکس خواهند خواند، عذر
می‌خواهم. ترجیح می‌دادم در این تله نیافتم ولی فعلا که چاره‌ای نیست جز
جواب‌ دادن!

در مقاله \code{<12595@star.cs.vu.nl>} آقای \code{ast@cs.vu.nl} (اندی
تاننباوم) می‌نویسد:

دو هفته‌ای در آمریکا بودم و در نتیجه وقت نکردم در مورد لینوکس نظرم را
بگویم (البته اگر بودم هم حرف چندانی برای گفتن نداشتم) ولی حالا به هر
دلیلی که باشد، حرف‌هایی دارم که باید بزنم.

همان طور که اکثر شما می‌دانید، مینیکس برای من یک سرگرمی شخصی است؛ چیزی
که بعد از ظهرها اگر از نوشتن کتاب خسته شده باشم و جنگ یا انقلاب یا بحث
مهمی در سنا هم نباشد که مستقیما از سی.ان.ان. پخش شود، به سراغش
می‌روم. شغل اصلی من، استادی دانشگاه و تحقیق در حوزه سیستم‌عامل‌ها است.

واقعا به نظرتان این دلیل معقولی برای توضیح کمبودهای مینیکس است؟ متاسفم
ولی باختی: من بهانه‌های خیلی بیشتری دارم ولی لینوکس در همه زمینه‌های از
مینیکس سر است. تازه در این باره که بهترین بخش‌های مینیکس توسط بروس
اوانز نوشته شده، حرفی نمی‌زنم.

\noindent جواب اول: شما مینیکس را به عنوان یک سرگرمی شخصی مطرح
می‌کند. نگاه کنید ببینید چه کسی دارد از مینیکس پول در می‌آورد و چه کسی
لینوکس را مجانی پخش کرده. هنوز هم می‌گویید مینیکس یک سرگرمی شخصی است؟
مینیکس را به رایگان در اختیار مردم بگذارید و یکی از بزرگترین انتقاد‌های
من مرتفع می‌شود. در اصل این لینوکس است که سرگرمی من است (البته یک
سرگرمی بسیار جدی و ارزشمند): من هیچ پولی از لینوکس درنیاورده‌ام و حتی
بخشی از یک پروژه دانشگاهی هم نبوده است. آن را فقط و فقط در وقت آزاد
خودم و فقط و فقط روی ماشین خودم گسترش داده‌ام.

\noindent جواب دوم: شغل شما استادی دانشگاه و تحقیق است: این یکی دلیل
خوبی است برای صدمات مغزی‌ای که مینیکس از آن رنج می‌برد. فقط می‌توانم
امیدوارم باشم (و انتظار داشته باشم) که آموئبا به گندی مینیکس نباشد

\noindent ۱. سیستم‌های میکروکرنل در برابر مونولیتیک‌ها

درست است. لینوکس مونولیتیک است و می‌پذیرم که میکروکرنل‌ها زیباتر
هستند. اگر موضوع بحث اینقدر حساس نبود، شاید با بخش زیادی از نوشته‌های
شما موافقت می‌کردم. از دیدگاه نظریه و زیبایی‌شناسی، لینوکس بازنده میدان
است. اگر پروژه کرنل گنو بهار گذشته آماده شده بود، من اصولا زحمت شروع
این پروژه را هم به خودم نمی‌دادم: اما واقعیت این است که آماده نبود و
هنوز هم نیست. برگ برنده لینوکس، آماده بودن آن است.

مینیکس یک سیستم‌عامل میکروکرنل است {پاک شد، اما نکته را گرفته‌اید}
لینوکس یک سیستم به سبک مونولیتیک‌ است؛

اگر این تنها شرط برای \dbquote{خوب بودن} کرنل بود، حق با شما
بود. مساله‌ای که ذکر نکرده‌اید این است که مینیکس به خوبی از عهده وظایف
میکروکرنل برنیامده و با مالتی‌تسک واقعی (داخل کرنل) مشکل دارد. اگر من
سیستم‌عاملی نوشته بودم که با مالتی‌تسک مشکل داشت، به این راحتی بقیه را
محکوم نمی‌کردم؛ در واقع بیشترین تلاش من این بود که دیگران این شکست
مفتضح را نبینند.

\{ بله! می‌دانم که هک‌های مالتی‌تسک برای مینیکس وجود دارند ولی به هرحال
  آن‌ها هک هستند و برونس اوانز خواهد گفت که با تمام نسخه‌ها هم به خوبی
  سازگار نیستند \}

\noindent ۲. قابلیت انتقال

\dbquote{قابلیت انتقال مال آن‌هایی است که نمی‌توانند برنامه‌های جدید بنویسند}

\hfill - من، همین الان (نیمه شوخی نیمه جدی)

واقعیت این است که لینوکس بیشتر از مینیکس قابلیت انتقال دارد. می‌گویید
چطور؟ باید بگویم نه به آن معنایی که شما برداشت کرده‌اید. منظورم این است
که من لینوکس را تا جایی که می‌توانستم (بدون داشتن استانداردهای \lr{POSIX}
جلوی چشمم)، سازگار با استانداردها نوشته‌ام. انتقال نرم‌افزارها به لینوکس
معمولا بسیار ساده‌تر از انتقال آن‌ها به مینیکس است.

می‌پذیرم که قابلیت انتقال چیز خوبی است: ولی فقط وقتی که این کار با معنا
باشد. تلاش برای کاملا قابل انتقال کردن یک سیستم‌عامل ایده فوق‌العاده‌ای
نیست: پیروی از یک \lr{API} قابل انتقال کافی است. ایده زیربنایی
سیستم‌عامل استفاده از توانمندی‌های سخت‌افزار است در عین مخفی کردن آن‌ها
پشت لایه‌ای از فراخوانی‌های سطح بالا. این دقیقا همان کاری است که لینوکس
می‌کند: به کارگرفتن مجموعه‌ای وسیع‌تر از دستورات ۳۸۶ نسبت به آن‌چیزی که
دیگر کرنل‌ها استفاده می‌کنند. شکی نیست که این‌کار قابلیت انتقال کرنل را
پایین می‌آورد، اما در عوض طراحی را بسیار ساده‌ می‌کند. یک بده بستان ساده
و دلیل وجودی لینوکس.

این را هم قبول دارم که لینوکس تا نهایت غیرقابل انتقال بودن رفته است:
من ژانویه قبل ۳۸۶م را گرفتم و لینوکس تا حدی پروژه‌ای بود تا ریزه‌کاری‌های
آن را یاد بگیرم. اگر واقعا یک پروژه مستقل بود، احتمالا بخش‌هایی را قابل
انتقال‌تر می‌نوشتم. البته به هیچ وجه دنبال بهانه آوردن نیستم: وضع فعلی
نتیجه طراحی اولیه‌ام بوده و آوریل قبل که پروژه را شروع کردم، فکر
نمی‌کردم کسی روزی بخواهد از آن استفاده کند. خوشحالم بگویم که اشتباه
کرده بودم و از آن‌جایی که متن برنامه به شکل آزاد در دسترس همه قرار
دارد، هر کسی که بخواهد می‌تواند آن را به هر چیزی که بخواهد پورت کند؛
هرچند که کار ساده‌ای نخواهد بود.

\hfill لینوس

معذرت می‌خواهم اگر گاهی زیادی تند هستم: اگر هیچ چیز دیگری نداشته باشید،
مینیکس به اندازه کافی خوب است. اگر پنج یا ده تا ۳۸۶ اضافه داشته باشید،
که من ندارم، آموئبا هم ممکن است خوب باشد. من معمولا وارد دعواها نمی‌شوم
ولی وقتی بحث لینوکس است، کمی حساس می‌شوم.
\end{emailbox}

این داستان چند قسمت دیگر هم داشت و یکی از معدود دعواهای اینترنتی من
بود و شما متوجه نکته اصلی شده‌اید: حتی از همان اولین روزها هم صداهای
مخالفی وجود داشتند (شاید هم نکته اصلی این باشد که وقتی وارد یک فروم
الکترونیک می‌شوید مواظب باشید چون اشتباهات املایی و انشایی شما تا
ابدالدهر باقی خواهند ماند).

\begin{journal}
من و لینوس خانواده‌های مان را در کمپ گذاشتیم و یک روز عصر آخرهای جولای
را با هم در گوورهات‌ اسپرینگ\LFootnote{Gover Hot Springs}
گذراندیم. درست در جایی که به گفته لینوس توروالدز که لحظه‌ای برای نگاه
کردن به آن مکث کرده بود، گویی از وسط صفحات تبلیغی کداک در مجله نشنال
جغرافیک بیرون افتاده بود. آتشی در کنار یک جوی کوچک روشن کردیم و از
لینوس درخواست کردم تا برایم از زندگی‌اش تعریف کند، بخصوص در دوره‌ای که
درخواست برای لینوکس در حال افزایش بود و کاربران آن داشتند از محدوده
خوانندگان گروه خبری مینیکس فراتر می‌رفتند.

\dbquote{باید فوق‌العاده بوده باشد.} این نظر من بود و ادامه دادم که
\dbquote{سال‌ها در یک اتاق و پشت کامپیوترت بودی. با ارتباطی بسیار کم با
  دنیای خارج از سی‌.پی.یوی کامپیوترت. حالا یکهو از هر گوشه و کنار دنیا،
  مردم متوجه کار عظیم تو شده بودند و تو شده بودی مرکز این توجه. همه
  داشتند به تو...}

جواب این بود: \dbquote{تا جایی که یادم است، جریان برایم چندان مهم
  نبود. واقعا حس می‌کنم مهم نبود. در اصل این برخورد همان‌ چیزی بود که
  دور از انتظار هم نبود چون به هرحال مشکلی وجود داشت که باید حل
  می‌شد. از این نظر زیاد به جریان فکر می‌کردم ولی اهمیت عجیبی برایم
  نداشت. ماجرا بیشتر از نظر معنوی، برایم بزرگ بود.}

لینوس ادامه داد: \dbquote{مساله جذاب برایم این بود که آدم‌های زیادی به
  من انگیزه می‌دادند تا این پروژه را پیش ببرم. اوایل فکر می‌کردم پایان
  این پروژه برایم متصور است. پایانش جایی بودی که پروژه در آن تمام تمام
  می‌شد. اما این لحظه هیچ وقت نرسید چون آدم‌ها دائما به من انگیزه ادامه
  کار می‌دادند. آن‌ها خوراک فکری برایم فراهم می‌کردند و من ادامه
  می‌دادم. هیجان کار ادامه داشت و در غیر اینصورت من سراغ پروژه دیگری
  می‌رفتم. این شیوه کار من بود که تا وقتی کار مفرح بود، جلو می‌رفتم. به
  هرحال این مساله دغدغه فکری من نبود. به نظرم به دماغم یا اینجور چیزها
  بیشتر فکر می‌کردم تا به لینوکس.}

چند هفته بعد در مرکز خرید استانفورد بودیم. جایی که لینوس مشغول بررسی
کفش‌های دو و انتخاب یک کفش مناسب برای خود بود. فروشنده پرسید:
\dbquote{معمولا در هفته چند کیلومتر می‌دوید؟} لینوس لبخند زد. در طول ده
سال، در مجموع یک کیلومتر هم ندویده بود. ورزش در زندگی لینوس جایی
نداشت. اما وقتی سرحال‌تر بود، اعتراف کرد که بدش نمی‌آید چند کیلویی وزن
کم کند.

با دست که روی شکمش می‌زد گفت: \dbquote{احتمالا تاو به شما اصرار کرده که
  من را به ورزش ببرید تا این شکم را از دست بدهم!}

به شوخی جواب دادم: \dbquote{بعله! و به همسرت بگو که چک این ماه هنوز
  نقد نشده!}

مشغول دور زدن مجموعه استانفورد با ماشین بودیم تا جای مناسبی برای پارک
پیدا کنیم. شاید بعد از نیم ساعت کمی نرمش کردیم و از راه گلی‌ای که حاصل
خشک کردن دریاچه بود، شروع به دویدن به سمت هدف کردیم. یک آنتن بشقابی
بزرگ که پشت درخت‌ها پنهان بود. با بدجنسی سرعت نسبتا زیادی برای دویدن
انتخاب کردم ولی در کمال تعجب دیدم که لینوس حدود یک مایل درست پشت من
آمد. بعد نفسش برید و چند دقیقه بعد هر دو روی زمین چمنی که کنار دریاچه
بود، ولو شدیم.

پرسیدم: \dbquote{برخورد خانواده با اتفاقات مرتبط با لینوکس چطور بود؟ باید
  هیجان زده شده باشند!}

جواب داد: \dbquote{فکر کنم اصولا کسی متوجه جریان نشد. البته نه اینکه کسی توجهی
  نکند ولی خب من همه عمرم برنامه‌نویسی کرده‌ام و این ماجرا هم از نظر
  آن‌ها هیچ فرقی با بقیه زندگی‌ام نداشت.}

\dbquote{اما به هرحال باید در این باره با آن‌ها حرف زده باشی. مثلا یک
  بار که پدرت داشته با ماشین تو را به جایی می‌رسانده، ممکن است گفته
  باشی: اوه پدر! شاید باور نکنید ولی من یک کار جالب با کامپیوترم
  کرده‌ام که این روزها صدها نفر دارند از آن استفاده می‌کنند..}

جواب قاطع است: \dbquote{نه.} لینوس ادامه می‌دهد: \dbquote{اصلا حس نکردم
  که باید در این مورد با دوستان یا خانواده صحبت کنم. احساس می‌کردم که
  نباید در این مورد به کسی اصرار کنم. یادم هست که لارس ویرزنیوس در
  همان دوران تصمیم گرفته بود تا زنیکس\LFootnote{XENIX} که نسخه شرکت
  اسکو\LFootnote{SCO} از یونیکس بود را بخرد. یادم هست که سعی می‌کرد
  دلایلی مثل این بیاورد که \xquote{البته از اینکار من اشتباه برداشت
    اشتباه نکنی‌ها.} تا جایی که یادم هست من اصلا ناراحت نشده
  بودم. بعدها سوییچ کرد به لینوکس ولی این جریان برای من مهم نبود. برای
  من همین که مردم از آن استفاده می‌کردند جذاب بود و گرفتن پاسخ هم
  خوشحالم می‌کرد اما در عین حال اینها برایم چندان هم مهم نبود. من احساس
  نمی‌کردم که باید کلام مقدس را ترویج کنم. از اینکه مردم از کد نوشته
  شده توسط من استفاده کنند خوشحال می‌شدم اما هیچگاه این تصور را نداشتم
  که پخش کردن آن در دنیا، مهمترین کار روی کره زمین است. اینکه چند صد
  نفر از کد من استفاده کنند باعث نمی‌شد احساس کنم کار بسیار مهمی
  کرده‌ام. مساله بیشتر مفرح بود تا مهم. این روزها هم همین احساس را
  دارم.}

نمی‌توانستم ناباوری‌ام را پنهان کنم،‌ پرسیدم: \dbquote{پس احساس می‌کردی نیازی نیست به پدر و مادر و دوستانت در این مورد حرفی بزنی؟ در مورد چیزهایی که در حال اتفاق افتادن بود هیجان نداشتی؟}

پیش از جواب دادن، چند ثانیه‌ای مکث کرد. \dbquote{اصولا یادم نیست که آن روزها احساس داشتم، یا نه!}
\end{journal}

\begin{journal}

لینوس یک ماشین جدید خریده است. یک بی‌.ام.و زد ۳ با دو صندلی و سقف کنار
رونده. به قول خودش، این یک ماشین \dbquote{مفرح} است. رنگ ماشین آبی
متالیک است، بهترین رنگ برای ماشین‌های اسباب بازی پسربچه‌ها. دلیل انتخاب
این رنگ این بوده که بی.ام.و زد۳ رنگ زرد براق ندارد؛ وگرنه انتخاب اولش
زرد براق بود. می‌گوید که بی.ام.و زرد معمولی، \dbquote{مثل ادرار
  می‌ماند.} چندین سال‌ است که پونتیاکش را در نزدیک‌ترین فاصله به در ورودی
ترنسمتا پارک کرده ولی این ماشین را جایی دورتر پارک می‌کند تا در سایه
باشد و از پنجره هم دیده شود. حالا وقتی لینوس پشت کامپیوتر است، می‌تواند
از پنجره قربان صدقه ماشین جدیدش برود.

تقریبا یک سال قبل بود که با ماشین روباز موستانگی که من اجاره کرده
بودم، با هم از سانتاکروز بالا می‌رفتیم. یادم می‌افتد که آن روز لینوس از
من خواست تا بعد از بیرون آمدن از سونا کمی در پارکینگ بمانیم و ماشین‌های
اسپرت را نگاه کنیم. حالا داریم از همان کوه‌ها بالا می‌رویم اما این بار
در ماشین اسپرت لینوس. از جاده ۱۷ که دور می‌زند، لبخند دارد.

من می‌گویم: \dbquote{استحقاقش را داشتی} و کلی سی‌دی آهنگ از داشبورد
بیرون می‌آورم. می‌پرسم کدام آهنگ پینک فلوید را می‌خواهد و او می‌گوید:
\dbquote{با این آهنگ‌ها بزرگ شده‌ام. وقتی بچه‌ بودم هیچ‌وقت آهنگ نخریدم
  ولی جنیس جوپلین\LFootnote{Janis Joplin} همیشه در خانه بود. شاید
  مادرم می‌گذاشت. هرچند که می‌دانم طرفدار الویس کاستلو\LFootnote{Elvis
    Costello} بود.}

عصر جمعه است. یکی از آن عصرجمعه‌های درخشان کالیفرنیا که همه حس ها را
غرق لذت می‌کند: آسمان نیلگون برای چشم‌ها، آفتاب گرم برای پوست، رایحه
اکالیپتوس‌های کوهی، مزه شیرین هوا و موسیقی پینک‌فلوید از بلندگوهای تقویت
شده. احتمالا برای کسانی که سبقت می‌گیرند، ما جوان‌هایی قدیمی هستیم که در
ماشین آخرین مدلمان راک کلاسیک گوش می‌دهیم. البته ماشین‌های اندکی هستند
که از بی.ام.و زد ۳ لینوس سبقت بگیرند.

ماشین را کنار اتوبان و در ردیف ماشین‌هایی که اکثرا از ماشین لینوس
قدیمی‌تر هستند، کمی بالاتر از سانتا کروز، پارک می‌کنیم و پیاده، راه مان
را به سمت ساحلی که معمولا جمعیت چندانی در آن نیست ادامه می‌دهیم. در
آفتاب گرم، روی حوله‌ها پهن می‌شویم و قبل از درآوردن ضبط‌صوت از کوله پشتی،
چند دقیقه‌ای صبر می‌کنیم. دوباره از او می‌خواهم تا درباره لینوس روزهای
اول لینوکس، صحبت کند.

روی شن‌ها مربعی می‌کشد تا نمایانگر اتاقش باشد و بعد جای کامپیوتر و
تخت‌خواب را مشخص می‌کند. \dbquote{می‌توانستم از تخت‌خواب بیرون بخزم و
  ایمیل‌هایم را چک کنم.} و همین حرکت را با انگشتش نشان می‌دهد. ادامه
می‌دهد که: \dbquote{بعضی روزها اصلا از خانه خارج نمی‌شدم. ایمیل‌هایم را
  چک نمی‌کردم تا ببینم چه کسی به من ایمیل زده. بیشتر دنبال این بود که
  ببینم فلان مشکل حل شده یا نه. شبیه این بود که چک کنم ببینم چه چیز
  جذابی منتظر من است یا اگر مشکلی پیش آمده، چه کسی آن را حل کرده}
لینوس می‌گوید که زندگی اجتماعی‌اش در آن دوران رقت‌بار بوده و بعد که احساس
می‌کند منظور را نرسانده اضافه می‌کند \dbquote{از آن هم بالاتر}

می‌گوید که: \dbquote{البته صد در صد هم منزوی نبودم ولی خب حین رشد
  لینوکس هم، من کماکان یک آدم غیراجتماعی بودم. حتما متوجه شده‌ای که هیچ
  وقت تلفنی با کسی حرف نمی‌زنم. همیشه همینطور بوده. هیچ وقت به کسی زنگ
  نمی‌زنم. اکثر دوستانم از آن تیپ‌هایی هستند که راحت به افراد زنگ می‌زنند
  ولی من نه. می‌توانی حدس بزنی رابطه عاطفی چه خواهد شد اگر هیچ وقت به
  دختری زنگ نزنی. آن روزها فقط چند دوست داشتم که گاه گاهی به خانه
  می‌آمدند، در می‌زدند و در خواست می‌کردند برای یک فنجان چای داخل
  شوند. بعید می‌دانم کسی در آن دوران متوجه لینوکس می‌شد و با خود می‌گفت
  که این آدم دنیا را تکان خواهد داد. احتمالا هیچ‌کس چنین فکری نمی‌کرد.}

تنها فعالیت متناوب اجتماعی لینوس در آن دوران، گردهمایی‌های هفتگی انجمن
اسپکتروم بود که طی آن دانشجویان علوم، دور هم جمع می‌شدند. این دیدارها
هم معمولا محورهای تکنولوژیک داشتند.

\dbquote{نگران چه چیزهایی بودم؟ فقط زندگی اجتماعی. شاید نگرانی واژه
  مناسبی نباشد. بحث احساسی بود. گاهی به دخترها فکر می‌کردم. آن دوره
  لینوکس برایم چندان مهم نبود. هنوز هم تا حدی اهمیت چندانی ندارد. هنوز
  هم می‌توانم گاهی بیخیالش بشوم.}

\dbquote{در آن سال‌های اولیه ورود به دانشگاه، چیزهای اجتماعی خیلی مهم
  بودند. البته جریان این طور نبود که من مثلا قوز داشته باشم و نگران آن
  باشم که آدم‌ها به من بخندند. مساله این بود که من می‌خواستم دوست و این
  جور چیزها داشته باشم. یکی از دلایلی که اسپکتروم را دوست داشتم این
  بود که به من اجازه می‌داد بدون اینکه اجتماعی باشم، جزوی از یک ساختار
  اجتماعی باشم. روزهای جلسه یک آدم اجتماعی بودم و بقیه هفته پشت
  کامپیوتر. این جریان احساسی‌تر از هر چیز مرتبط با لینوکس بود. هیچ وقت
  به خاطر لینوکس ناراحتی نداشتم و هیچ شبی هم به خاطر آن بی‌خوابی
  نکشیدم.}

\dbquote{چیزی که من را واقعا ناراحت می‌کرد و هنوز هم باعث ناراحتی من
  است، خود تکنولوژی نیست بلکه تعامل‌های اجتماعی مرتبط با تکنولوژی
  است. مثلا ناراحتی من از نامه اندرو تاننباوم به خاطر مباحث تکنیکی
  مطرح شده در نامه و بحث‌های منتج از آن نبود. اگر آن نامه را هرکس دیگری
  فرستاده بود، از کنارش می‌گذشتم. مساله این بود که او این نامه را به
  فهرست پستی فرستاده بود و من را... من در مورد موقعیت اجتماعی‌ام در بین
  آدم‌هایی که آن گروه پستی را می‌خواندند حساس بودم و او داشت به این
  موقعیت حمله می‌کرد.}

\dbquote{یکی از چیزهایی که باعث خوبی و پیشرفت لینوکس شد، بازخوردهایی
  بود که می‌گرفتم. بازخوردها به این معنی بودند که لینوکس اهمیت داشت و
  من بخشی از یک گروه اجتماعی بودم. تازه من رهبر آن گروه اجتماعی
  بودم. شکی نیست که این مهم بود، مهم‌تر از آن که بخواهم درباره اش با
  پدر و مادرم صحبت کنم. من بیشتر دغدغه کسانی را داشتم که از لینوکس
  استفاده می‌کردند. من یک حلقه اجتماعی درست کرده بودم و مورد احترام
  افراد آن حلقه بودم. البته آن دوران این طور فکر نمی‌کردم و الان هم
  نظرم کاملا این نیست، ولی این باید مهمترین جنبه بوده باشد. به همین
  دلیل بود که آن قدر تند به اندرو تاننباوم جواب دادم.}

خورشید در حال غروب کردن در اقیانوس آرام است و وقت ترک ساحل. لینوس
اصرار دارد که من ماشینش را برانم تا حس کنم که چقدر خوب به فرامین جواب
می‌دهد. می‌گوید که از راه طولانی و پر پیچ خم شماره ۹ به سیلیکون‌ولی
برگردیم.

لینوس می‌گوید که جنگ ایمیلی با صاحب گروه مینیکس در نهایت به ایمیل‌های
خصوصی کشید، چون صحبت‌ها آن قدر ناجور بود که نمی‌شد آن‌ها را به شکل عمومی
ادامه داد. جنگ چند ماهی متوقف شده بود تا اینکه تاننباوم با ارسال
ایمیلی به لینوس، او را به تبلیغ پنج خطی یک نسخه تجاری از لینوکس در پشت
جلد مجله بایت ارجاع داده بود.

\dbquote{آخرین ایمیلی که از تاننباوم گرفتم این بود که از من می‌پرسید
  آیا واقعا این آن چیزی است که دنبالش هستم؟ آیا واقعا می‌خواهم افراد
  برنامه من را بفروشند. برایش یک جواب یک کلمه‌ای فرستادم: بله. و دیگر
  هیچ وقت از او ایمیلی نداشتم.}

تقریبا یکسال بعد که لینوس برای اولین سخنرانی عمومی‌اش به هلند رفته بود،
به دانشگاه محل تدریس تاننباوم رفت تا از او بخواهد که نسخه‌ای از
سیستم‌عامل‌ها: طراحی و اجرا ، کتابی که زندگی‌اش را شکل داده بود، برایش
امضا کند. او بیرون در منتظر ماند ولی تاننباوم پیدایش نشد. در آن تاریخ،
استاد جایی در بیرون از شهر بود و این دو هیچ گاه با هم ملاقات نکردند.
\end{journal}

\section{بخش یازدهم}
اتاق هتل به زور بالای صفر درجه بود. من در تخت دراز کشیده بودم،
می‌لرزیدم و به سخنرانی فردا فکر می‌کردم. در هلند آن‌طور که در فنلاند
اتاق‌ها را گرم می‌کنند، جایی را گرم نمی‌کنند و این اتاق با این پنجره‌های
بزرگ انگار فقط برای تابستان طراحی شده. اما سرما تنها چیزی نبود که من
را در ۴ نوامبر ۱۹۹۳ بیدار نگه‌داشته بود. من به شکل غیرقابل باوری، استرس
داشتم.

سخنرانی جلوی جمع همیشه نقطه ضعف من بوده است. در مدرسه از ما می‌خواستند
تا درباره موضوعی که درباره اش تحقیقی کرده‌ایم - موش و اینجور چیزها -
سخنرانی کنیم و این کار همیشه برای من غیر ممکن بود. من آن‌جا می‌ایستادم و
هیچ حرفی از زبانم خارج نمی‌شد. حتی وقتی که برای حل کردن مساله هم به پای
تخته می‌رفتم، مشکل داشتم.

حالا در هلند بودم. یعنی در اد\LFootnote{Ede} که تقریبا یک ساعت با قطار از
آمستردام فاصله داشت. اینجا بودم چون دعوت شده بودم که در جشن ده‌سالگی
گروه کاربران هلند سخنرانی کنم. می‌خواستم به خودم ثابت کنم که توان این
کار را دارم. سال قبل به مناسبت مشابهی از من خواسته بودند تا در اسپانیا
حرف بزنم و من به این دلیل که ترسم از صحبت برای جمع بیشتر از عشقم به
مسافرت به اسپانیا بود، نپذیرفته بودم. و آن موقع واقعا عاشق مسافرت
بودم. (هنوز هم سفر را دوست دارم ولی دیگر نه مثل بچه‌‌ای که به ندرت از
فنلاند بیرون رفته باشد. تنها جایی که رفته بودم، سوئد بود که گاهی برای
پیک‌نیک به آنجا می‌رفتیم و مسکو که وقتی شش ساله بودم، در آنجا سری به
پدرمان زده بودیم.) رد کردن دعوت به اسپانیا آن قدر برایم دردناک بود که
تصمیم گرفتم دعوت بعدی برای سخنرانی را حتما قبول کنم. اما حالا که در
تختواب خوابیده بودم و به این فکر می‌کردم که نخواهم توانست دهانم را باز
کنم یا از آن بدتر جلوی ۴۰۰ نفر به تته پته خواهم افتاد، احساس می‌کردم که
بهتر بود در تصمیم‌گیری برای آمدن عجله نمی‌کردم.

بله! واقعا اوضاع گند بود.

خودم را با حرف‌های همیشگی دلداری می‌دادم. مثلا اینکه جمعیت حاضر، خواهان
موفقیت من هستند و اصولا اگر من را دوست نداشته باشند، آن‌جا نخواهند
بود. تازه موضوع را هم دقیق می‌دانستم: دلایل فنی پشت تصمیم‌گیری‌های مرتبط
با هسته لینوکس و دلایل انتشار آزاد آن. ولی هنوز متقاعد نشده بودم که
سخنرانی با موفقیت همراه خواهد بود و مغزم مثل یک لوکوموتیو غیرقابل
نگه‌داشتن، دلایل شکست را بررسی می‌کرد. واقعا مشغول لرزیدن بودم و هوای
سرد بی‌اهمیت‌ترین دلیل بود.

سخنرانی چه شد؟ جمعیت با آدم وحشت زده‌ای که جلوی شان ایستاده بود، همراهی
کردند و کل حواس شان را دادند به تصاویر پاورپوینت (متشکرم مایکروسافت!)
و بعد هم به سوالات شان پاسخ دادم. در واقع پرسش و پاسخ بهترین قسمت
برنامه بود. بعد از سخنرانی من، مارشال کیرک مک کوسیک\LFootnote{Marshall
  Kirk McKusik} که جزو افراد اصلی یونیکس \lr{BSD} بود، جلو آمد و به من
گفت که سخنرانی‌ام به نظرش جذاب بوده. از این حرفش آن قدر خوشحال بودم که
می‌خواستم به زانو بیافتم و پاهایش را ببوسم. چند نفری هستند که در دنیای
کامپیوتر اخبارشان را دنبال می‌کنم و کیرک یکی از آن‌ها است. دلیلش هم این
است که در اولین سخنرانی من با من خیلی مهربانی کرد.

اولین سخنرانی خیلی سخت بود، ولی بعدی‌ها بهتر شد و اعتماد به نفس من هم
افزایش پیدا کرد. دیوید دائم از من می‌پرسد که بعد از گسترش لینوکس، وضعیت
من در دانشگاه چه تغییری کرد. اصلا یادم نمی‌آید که بعد از استادی ام کسی
به لینوکس اشاره کرده باشد یا دانشجویی مرا به انگشت نشان داده باشد. از
این خبرها نبود. اطرافیانم در مورد لینوکس می‌دانستند، اما اکثر هکرهایی
که روی آن کار می‌کردند، خارج از فنلاند بودند.

در پاییز ۱۹۹۲، به عنوان کمک استاد کلاس سوئدی دانشکده علوم کامپیوتر
مشغول به کار شدم (ماجرا این بود که آن‌‌ها برای کلاس‌های مقدماتی کامپیوتر،
به دنبال کمک‌استادهایی بودند که سوئدی صحبت کنند. در دانشکده علوم
کامپیوتر هم تنها دو دانشجوی ارشد سوئدی زبان بود: لارس و لینوس. آن‌ها
انتخاب چندانی نداشتند). اوایل حتی از اینکه پای تخته بروم و مساله‌ای را
حل کنم هم دچار استرس می‌شدم. ولی کم کم یاد گرفتم که به جای فکر کردن به
خجالت کشیدن، مشغول حل مساله شوم. سه سال بعد به \dbquote{کمک محقق}
ارتقاء مقام پیدا کردم. معنای این سمت آن بود که به جای حقوق گرفتن به
خاطر درس دادن، به خاطر نشستن در آزمایشگاه کامپیوتر و تحقیق، به خصوص در
مورد لینوکس، حقوق می‌گرفتم. این شروع روندی بود که طی آن کسی به من پول
می‌داد تا روی لینوکس کار کنم. این عملا همان چیزی است که در ترنسمتا هم
اتفاق می‌افتد.

دیوید: \dbquote{از کی مساله به یک موضوع جدی تبدیل شد؟}

من: \dbquote{هنوز هم مساله جدی نشده.}

باشه. باید اصلاحش کنم. مساله وقتی جدی‌تر شد که معلوم شد چند نفر در دنیا
هستند که به لینوکس به عنوان چیزی بیش از یک سیستم‌عامل اسباب‌بازی
وابسته‌اند. وقتی مردم شروع کردند به استفاده از لینوکس به عنوان یک
سیستم‌عامل اصلی، من متوجه شدم که اگر چیزی خراب شود، من مسوول هستم. یا
حداقل احساس کردم که من مسوول هستم (هنوز هم همین احساس را دارم). در طول
سال ۱۹۹۲ بود که لینوکس از یک سرگرمی به یک ابزار کامل در زندگی انسان‌ها،
در تجارت آن‌ها و در کسب درآمدشان تبدیل شد.

در ۱۹۹۲ و تقریبا یکسال بعد از اینکه من پروژه شبیه‌ساز ترمینال را شروع
کرده بودم، اولین سیستم پنجره \code{X} تحت لینوکس اجرا شد. معنی این حرف
آن است که به لطف پروژه \code{X} که در \lr{MIT} شروع شده بود، کاربران
لینوکس می‌توانستند از طریق محیط گرافیکی با کامپیوتر ارتباط براقرار کنند
و در پنجره‌های مختلف، برنامه‌های مختلفی را به شکل هم زمان اجرا کنند. این
یک تغییر بزرگ بود. یادم است که تقریبا از یکسال قبل، این موضوع مورد
شوخی من و لارس بود و به او می‌گفتم که روزی خواهد آمد که \code{X} را تحت
لینوکس اجرا کنیم. هیچ‌وقت فکر نمی‌کردم که این اتفاق به این سرعت
بیافتد. هکری به اسم اورست زبوروسکی\LFootnote{Orest Zborowski} توانست
\code{X} را به لینوکس پورت کند.

عملکرد پنجره‌ها، حاصل مدیریت سیستم سرویس‌دهنده \code{X} روی محیط گرافیکی
است. سرویس‌دهنده از طریق پیام‌هایی با کلاینت‌ها صحبت می‌کند. مثلا کلاینت‌ها
می‌گویند که \dbquote{من یک پنجره به این اندازه می‌خواهم.} این رابطه از
طریق لایه‌ای به نام سوکت‌ها یا به اصطلاح فنی‌تر یونیکس دامین سوکت‌ها ایجاد
می‌شود. این روش ارتباط داخلی یونیکس است. از همین سوکت‌ها در اینترنت هم
استفاده می‌کنیم. ارست اولین لایه سوکت را برای لینوکس نوشت تا \code{X}
را روی آن اجرا کند. رابط ارست کمی زمخت بود و با بقیه کدها به خوبی
هماهنگ نمی‌شد، ولی با این وجود من آن را پذیرفتم چون به آن احتیاج
داشتیم.

چند وقتی طول کشید تا به این واقعیت که ما یک رابط گرافیکی داریم عادت
کنم. فکر می‌کنم یکی دوسال اول زیاد از آن استفاده نمی‌کردم. ولی خب، این
روزها بدون آن نمی‌توانم زنده بمانم. وقتی من مشغول کارم، تعداد زیادی
پنجره باز هستند.

کار ارست نه تنها باعث شد ما پنجره داشته باشیم، که دروازه آینده را هم
برای مان باز کرد. دامین سوکت‌ها برای ارتباطات داخلی استفاده می‌شدند که
\code{X} به آن احتیاج داشت اما با استفاده از همین سوکت‌ها توانستیم جهش
بزرگی ایجاد کنیم و آن شبکه بیرونی بود - ارتباط کامپیوترها با
یکدیگر. بدون شبکه، لینوکس فقط به درد کسانی می‌خورد که از خانه‌شان
می‌خواستند با استفاده از مودم به جایی متصل شوند یا اصولا همه کارها را
روی یک کامپیوتر محلی انجام دهند. ما با خوشبینی شروع کردیم به توسعه
شبکه لینوکس بر پایه همان سوکت‌ها، هرچند که این سوکت‌ها اصولا برای کار
شبکه طراحی نشده بودند.

من آن قدر به نتیجه کار مطمئن بودم که تصمیم گرفتم شماره نسخه لینوکس را
با یک جهش بالا ببرم. برنامه اولیه من این بود که نسخه \code{0.13} را در
مارس ۱۹۹۲ منتشر کنم. اما با دیدن رابط گرافیکی که به خوبی کار می‌کرد،
احساس کردم که ۹۵ درصد راه برای ارائه یک سیستم‌عامل قابل اتکا برای
کارهای روزمره و دارای امکان ارتباطات شبکه‌ای فراهم شده است. پس نسخه
جدید را \code{0.95} نامیدم.

پسر! من چیزی سرم نمی‌شد. اگر نخواهم بگویم که کلا نفهم بودم. 

شبکه چیز ناجوری است و تقریبا دوسال طول کشید تا وضعیت شبکه لینوکس به
جایی رسید که قابل ارائه شد. وقتی شبکه را به یک سیستم اضافه می‌کنید، کلی
مساله جدید ایجاد می‌شود. بخصوص مسایل ایمنی. نمی‌دانید آن بیرون چه کسی
است و دارد چکار می‌کند. باید مواظب باشید که افراد با فرستادن پاکت‌های
بی‌ربط، باعث کرش کردن سیستم شما نشوند. دیگر نمی‌توانید کنترل کنید که چه
کسانی به کامپیوتر شما وصل شوند و افراد هم تنظیمات بسیار متنوعی
دارند. وقتی استاندارد مورد نظر \lr{TCP/IP} باشد، برنامه‌ریزی بازهم
سخت‌تر می‌شود. به نظر می‌رسید این مرحله تا ابد طول بکشد. در پایان سال
۱۹۹۳، قابلیت شبکه تقریبا تمام شده بود و قابل استفاده بود، هر چند که
خیلی‌ها هنوز با آن مشکلات جدی داشتند. شبکه‌های غیر ۸ بیتی در این سیستم
غیرقابل استفاده بودند.

به خاطر هیجان بی‌مورد در نامیدن نسخه قبلی با شماره \code{0.95}، حالا
حسابی در تله افتاده بودم. در طول دو سال باقی مانده تا عرضه نسخه ۱،
مجبور شدیم کلی بامبول سوار کنیم. اعداد چندانی بین \code{0.95} و ۱ وجود
ندارد ولی ما باید دائما به خاطر اصلاحات و باگ‌زدایی، نسخه می‌دادیم. وقتی
به \code{0.99} رسیدیم، شروع کردیم به اضافه کردن اعداد برای نمایش سطوح
پچ‌ها و بعد هم اضافه کردن حروف. مثلا در یک مرحله نسخه \code{0.99} سطح
پچ \code{15A} را داشتیم و بعد \code{0.99} سطح پچ \code{ 15B} را. این
ماجرا تا سطح پچ \code{15Z} ادامه داشت ولی بالاخره سطح پچ \code{16} را
که نسخه قابل استفاده بود، \code{1.0} نامیدیم. این نسخه با کلی هیاهو در
مارس ۱۹۹۴ و در دانشکده علوم کامپیوتر دانشگاه هلسینکی منتشر شد.

رسیدن به این مرحله واقعا پرآشوب بود ولی هیچ‌کدام از ماجراها نتوانست
جلوی عمومی شدن لینوکس را بگیرد. حالا گروه اینترنتی خودمان را داشتیم که
\code{comp.os.linux} نامیده می‌شد و حاصل خاکستر جنگ آتشین من و اندرو
تاننباوم بود. کلی آدم در این گروه عضو شده بود. آن روزها اینترنت
کابال\LFootnote{Internet Cabal} یعنی گروهی که کمابیش مسوول اداره
اینترنت بودند، آماری ماهانه از میزان جذب افراد به گروه‌های مختلف منتشر
می‌کردند. این آمارها دقیق نبودند ولی تنها معیاری بودند که میزان علاقه
مردم به سایت‌ها و موضوعات مختلف - و در این مورد لینوکس - را نشان
می‌دادند. بین همه گروه‌ها، \code{alt.sex} همیشه رتبه اول را داشت. (البته
مورد علاقه من نبودم ولی یکی دوبار آن را چک کردم تا ببینم این همه هیاهو
برای چیست. من یک نرد با نیروی جنسی پایین بودم که ترجیح می‌دادم به جای
خواندن در مورد پوزیشن‌های جدیدا کشف شده و اینجور چیزها در alt.sex، با
پردازنده ریاضی کامپیوترم ور بروم).

بر اساس آمار ماهانه کابال من به راحتی می‌توانستم میزان محبوبیت
\code{comp.os.linux} را بسنجم و واقعا هم این کار را می‌کردم (هرچند ممکن
است من الگو و قهرمان کسی باشم ولی هیچ وقت آن آدم از خودگذشته و
غیرخودخواه و فدای‌تکنولوژی‌ای که گاهی رسانه‌ها از من ترسیم می‌کنند
نیستم). در پاییز ۱۹۹۲، برآورد فهرست خبری ما، در حد چند ده هزار نفر
بود. این تعداد آدم خبرها را می‌خواندند تا ببینند چه خبر است، ولی همگی
کاربر لینوکس نبودند. در گزارش‌های ماهانه، فهرستی هم بود از چهل گروه که
بیشترین کاربر را داشتند و این فهرست به شکل پیش‌فرض در گروه خبری قرار
می‌گرفت. اگر گروه خبری شما یکی از این چهل گروه پرخواننده نبود،
می‌توانستید فهرست کامل را از یک گروه اختصاصی دیگر دریافت نمایید. من
معمولا باید می‌رفتم سراغ این فهرست کامل.

گروه خبری لینوکس در حال بالا آمدن از پلکان بود. یکبار که به یکی از چهل
گروه پرخواننده تبدیل شد، من بسیار خوشحال بودم. واقعا جالب بود. یادم
هست که نامه‌ای کنایه‌دار در \code{comp.os.linux} نوشتم و سیستم‌عامل‌های
مختلف از جمله مینیکس و رتبه گروه‌های شان را در آن آوردم و نوشتم
\dbquote{هی ببینید! ما محبوب‌تر از ویندوز هستیم} البته فراموش نکنید که
آن دوران طرفداران ویندوز در اینترنت نبودند. در ۱۹۹۳ بود که لینوکس یکی
از پنج گروه پرطرفدار شد. آن شب، مملو از رضایت به تخت رفتم چون لینوکس
به اندازه سکس محبوب شده بود.

البته در گوشه دنیایی که من زندگی می‌کردم خبر چندانی نبود. من واقعا
زندگی خاصی نداشتم. همان طور که قبلا هم گفتم، در این دوره با تلاش پیتر
آنوین، مردم ۳۰۰۰ دلار پول داده بودند تا من قسط کامپیوترم را بدهم و من
هم تا پایان ۱۹۹۳ این کار را کرده بودم. برای کریسمس هم دستگاهم را به یک
\code{486DX2} ارتقاء داده بودم که سال‌ها کامپیوتر اصلی من ماند. زندگی
من این بود: خوردن. خوابیدن. گاهی به دانشگاه رفتن. کد نویسی و کلی ایمیل
خواندن. خبر داشتم که بعضی از دوستان شیطنت‌هایی هم می‌کنند ولی من مشکلی
با زندگی‌ام نداشتم.

در حقیقت هم بیشتر دوستان دور و بر من، بازنده بودند.

\section{بخش دوازدهم}
سخنرانی در اد به من ثابت کرد که تحت هر شرایطی می‌توانم جان سالم به در
ببرم. حتی از وضعیت حادی که در آن لازم باشد جلوی کلی غریبه بایستم که
همه حواس شان به من است. اعتماد در دیگر زمینه‌ها هم داشت در من رشد
می‌کرد. کم کم لازم بود در مورد به روزرسانی‌ها و پچ‌های لینوکس تصمیمات
سریع و قاطع بگیرم و این تصمیمات باعث شده بود به عنوان رهبر یک جامعه در
حال شکل‌گیری، احساس اعتماد به نفس بیشتری داشته باشم. تصمیمات فنی هیچ‌وقت
مساله ساز نبودند؛ مشکل اصلی وقتی بود که باید به یک نفر - آنهم به
شیوه‌ای سیاستمدارانه - می‌گفتم که راه حل کس دیگری‌ را به راه حل او ترجیح
داده‌ام. گاهی کار به سادگی گفتن \dbquote{فلان اصلاح دارد به خوبی کار
  می‌کند. پس چرا همان را استفاده نکنیم؟} بود.

روش من همیشه این بود که اصلاحی که از نظر فنی بهتر بود را انتخاب
می‌کردم. با اینکار هیچ وقت لازم نبود بین دو برنامه‌نویس که کدهایشان در
رقابت با یکدیگر بود، داوری کنم. همچنین با اینکه آن زمان نمی‌دانستم، این
روشی بود که باعث شد مردم به من اعتماد کنند و وقتی مردم به شما اعتماد
می‌کنند، نصیحت شما هم کاراتر است و نصحیت خوب، باعث اعتماد دوباره می‌شود.

شکی نیست که باید قبل از به وجود آمدن اعتماد اولیه، برای آن یک زیربنا
فراهم کنید. به نظرم این زیربنا نه به هنگام نوشتن کرنل لینوکس که به
هنگام قرار دادن آزاد و رایگان آن در اینترنت ایجاد شد. مردم وقتی اعتماد
کردند که من لینوکس را روی اینترنت گذاشتم و به همگان اجازه دادم تا آن
را اصلاح کنند یا ارتقاء دهند. هرکسی که می‌خواست به پروژه کمک کند
می‌توانست کدهایش را ارسال کند و من در این مورد که آیا این کد به کل
پروژه کمک خواهد کرد یا نه تصمیم می‌گرفتم.

همان طور که هیچ‌وقت به این فکر نکرده بودم که لینوکس در جایی خارج از
کامپیوتر من اجرا شود، به این هم نیاندیشیده بودم که روزی رهبر گروهی
شوم. این موضوع خود به خود اتفاق افتاد. در دوره‌ای، یک گروه مرکزی پنج
نفره بیشترین فعالیت‌های دنیای لینوکس در بخش‌های مختلف را بر عهده
گرفتند. آن‌ها مثل فیلتر عمل می‌کردند و مسوولیت حوزه‌های مورد نظر خود را
برداشتند.

خیلی زود یاد گرفتم که بهترین و مفیدترین روش رهبری این است که به افراد
اجازه دهیم کارها را به این دلیل که به آن علاقه‌دارند انجام دهند، نه به
این دلیل که من به انجام شدن آن کار علاقه دارم. بهترین رهبرها، همچنین
درک می‌کنند که کی اشتباه می‌کنند و می‌توانند خود را از پروسه بیرون
بکشند. همچنین بهتر رهبر باید بتواند به دیگران اجازه بدهد که به جای
آن‌ها تصمیم بگیرند.

بگذارید دوباره جمله بندی کنم. بیشترین موفقیت لینوکس به خاطر این ضعف‌ها
در شخصیت من بود: ۱. من تنبل هستم و ۲. دوست دارم به خاطر فعالیت‌های
دیگران اعتبار کسب کنم. در صورتی که من این دو خصیصه را نداشتم، الان مدل
توسعه لینوکس - اگر این آن چیزی است که مردم آن را می‌نامند - به جای
شبکه‌ای درهم تنیده از صدها هزار مشارکت کننده که از طریق گروه‌های خبری
لینوکس و مباحثات میان توسعه دهندگان در مراسمی که شرکت‌ها پشتیبان مالی
آن هستند و در هر لحظه حدود ۴۰۰۰ پروژه را پیش‌ می‌برند،‌ محدود شده بود به
نیم دوجین گیک که هر روز به هم ایمیل می‌زدند و بعد برنامه می‌نوشتند. این
روزها در بالای شبکه برنامه‌نویسان کرنل لینوکس، رهبری است که هیچ گاه
غریزه رهبری نداشته و هنوز هم ندارد.

و کارها هم بسیار خوب پیش رفته. من چیزهایی که بهشان علاقه چندانی
نداشته‌ام را کنار گذاشته‌ام. اولین آن‌ها، سطح کاربر - در تضاد با سطح عمیق
کرنل - یعنی سطحی بود که با استفاده کنندگان از سیستم ارتباط برقرار
می‌کند. اول یک نفر داوطلب برعهده گرفتن آن شد. بعد روند نگهداری کل
زیربخش‌های آن به شکلی ارگانیک تقسیم شد. مردم می‌دیدند که چه کسی فعال است
و به چه کسی می‌شود اعتماد کرد و به او کار بیشتری دادند و اعتماد بیشتری
کردند. رای‌گیری‌ای در کار نبود. دستوری هم در کار نبود و کسی هم کسی را
حسابرسی نکرد.

اگر دو نفر یک نوع درایور نرم‌افزاری خاص را توسعه دهند، گاهی من هر دو را
قبول می‌کنم و بعد در عمل می‌بینیم که کدام مورد بیشتر استفاده
می‌شود. کاربران معمولا گرایش دارند که یکی را به دیگری ترجیح دهند. اگر
هم هر دو شاخه مورد قبول قرار گیرد و کاربران بین آن‌ها پخش شوند، آن‌ها را
در دو شاخه متفاوت نگه می‌داریم و هر کدام کاربران خاص خود را حفظ می‌کنند.

چیزی که معمولا آدم‌ها را متعجب می‌کند این است که مدل نرم‌افزار بازمتن
واقعا کار می‌کند.

به نظرم این می‌تواند برای درک ذهنیت هکرهای جهان نرم‌افزارهای بازمتن موثر
باشد (البته من معمولا سعی می‌کنم از لفظ \dbquote{هکر} استفاده نکنم. به
هنگام صحبت در جمع‌های خصوصی و فنی من گاهی خودم را هکر می‌خوانم ولی اخیرا
این لغت معنی دیگری پیدا کرده: بچه‌های نوجوانی کاری مفیدتر از رخنه به
بانک‌های اطلاعاتی شرکت‌ها بلد نیستند و وقتی که باید صرف کارهای داوطلبانه
در کتابخانه‌ها یا حداقل دوست پیدا کردن را بکنند، صرف اینجور فعالیت‌ها
می‌کنند).

هکرهایی - برنامه نویسانی - که روی لینوکس کار می‌کنند معمولا ارتقای
شغلی، خواب، مسابقات ورزشی کودکانشان و حتی سکس را فدای برنامه‌نویسی برای
لینوکس می‌کنند. آن‌ها از این لذت می‌برند که بخشی از یک پروژه جهانی هستند
- لینوکس بزرگترین پروژه جمعی جهان است - و تلاش خود را صرف این کرده‌اند
که زیباترین تکنولوژی جهان را در اختیار هر کسی قرار دهند که خواهان آن
است. به همین سادگی. به همین مفرحی.

خب. انگار دارم شبیه تبلیغ رسانه‌ای بی‌شرمانه‌ای می‌شوم که هدفش مشهور کردن
خودم است. هکرهای بازمتن، نمونه‌های تکنولوژیک مادر ترزا نیستند. اسم آن‌ها
به همراه مشارکت‌های شان در \dbquote{فهرست اعتبارات} و \dbquote{فایل
  تاریخچه} که همراه هر پروژه است، می‌آید. کسانی که مشارکت‌های خوبی داشته
باشند به راحتی توسط کارفرمایانی که برای پیدا کردن بهترین برنامه‌نویسان
این فایل‌ها را زیر و رو می‌کنند، استخدام می‌شوند. همچنین خیلی از هکرها به
دنبال اعتبار و‌ شخصیتی در نزد دیگران هستند که یک مشارکت درست و حسابی
می‌تواند نصیب آن‌ها کند. این انگیزه‌ای قوی است. همه دوست دارند در چشم
بقیه معتبر باشند، مشهور شوند و شان اجتماعی خود را بالا ببرند. توسعه
بازمتن، این امکان را در اختیار برنامه‌نویسان می‌گذارد.

نیازی نیست بگویم که سال ۱۹۹۳ را هم مثل سال‌های ۱۹۹۲ و ۱۹۹۱ گذراندم:
چسبیده به کامپیوتر. اما این جریان در حال تغییر بود.

در ادامه راه دانشگاهی پدربزرگ، من هم در دانشگاه هلسینکی کمک استاد شدم
و در نیمسال پاییز، کلاس سوئدی \dbquote{مقدمه‌ای بر علوم کامپیوتر} را
تدریس می‌کردم. این گونه بود که با تاو آشنا شدم. او نقشی حتی بیشتر از
کتاب سیستم‌عامل‌ها: طراحی و اجرای آندرو تاننباوم در زندگی من داشت ولی
قرار نیست حوصله شما را با جزییات آن ماجرا سر ببرم.

تاو یکی از پانزده دانشجوی کلاس من بود. او قبلا مدرکی در آموزش
پیش‌دبستانی داشت و حالا می‌خواست کامپیوتر هم بخواند اما سرعت یادگیری‌اش
از بقیه کلاس کمتر بود. در نهایت هم این رشته را کنار گذاشت.

آن کلاس بسیار ابتدایی بود. در سال ۱۹۹۳، اینترنت هنوز همگانی نشده
بود. من به عنوان تکلیف از دانشجویان خواستم تا برایم یک ایمیل
بفرستند. این روزها مضحک است ولی به هرحال گفتم: \dbquote{برای تکلیف، یک
  ایمیل برایم بفرستید.}

ایمیل اکثر دانشجویان عبارت‌هایی مثل \dbquote{آزمایش} بود یا حداکثر
نکاتی غیرمهم در مورد کلاس.

تاو با من قرار گذاشته بود. 

من با اولین دختری که در دنیای دیجیتال به من نزدیک شد، ازدواج کردم.

اولین قرار ما هیچ‌وقت تمام نشد. تاو معلم پیش‌دبستانی و قهرمان شش دوره
مسابقات کاراته فنلاند بود. او از یک خانواده معقول می‌آمد. البته من هر
خانواده‌ای که به اندازه مال خودم قاطی پاتی نباشد را معقول می‌خوانم. او
کلی دوست داشت و از همان لحظه اول که دیدمش به نظرم زن مناسبی آمد
(جزییات را برای تان نمی‌گویم). در عرض چند ماه، من و رندی (گربه‌ام) به یک
آپارتمان نقلی نقل مکان کردیم.

در طول دو هفته اول حتی زحمت آوردن کامپیوتر را هم به خودم ندادم. بدون
احتساب دوران سربازی، آن دو هفته طولانی‌ترین زمانی بعد از یازده سالگی
(زمانی که روی پای پدربزرگ‌ با کامپیوتر کار می‌کردم) است که بدون کامپیوتر
سر کرده‌ام. نمی‌خواهم بحث را کش بدهم, ولی تکرار می‌کنم که آن دوران
بیشترین زمانی است که به عنوان یک شهروند، بدون پردازنده سر کرده‌ام. به
هرحال آن دوره را گذراندم (بازهم جزییات را برای تان نمی‌گویم). مادرم در
چندباری که او را دیدم تکرار کرد که \dbquote{این پیروزی مادر طبیعت
  است.} احتمالا خواهر و پدرم دچار شوک شده بودند.

چند وقت بعد، تاو یک گربه آورد تا رندی تنها نباشد. بعد هم عادت کردیم تا
عصرها را تنهایی یا با دوستان بگذرانیم و پنج صبح هم بلند شویم تا او به
کارش برسد و من هم زود به دانشگاه بروم تا حینی که دیگران نیامده‌اند،
بدون مزاحت ایمیل‌های لینوکس را بخوانم.
