\chapter{تولد یک نرد}
\section{بخش یکم}
من بچه زشتی بودم.

چاره‌ای به جز گفتن‌اش ندارم. امیدوارم روزی هالیوود فیلمی درباره لینوکس
بسازد و مطمئن هستم در آن فیلم از کسی شبیه به تام‌کروز برای نقش اول
استفاده خواهد شد – ولی در نسخه غیرهالیوودی، جریان جور دیگری است.

البته اشتباه نکنید. مساله این نیست که من شبیه گوژپشت نوتردام
باشم. برای تصور کردن من، دندان‌های جلوی بزرگی را در نظر بگیرید که هر کس
به عکسی از جوانی‌های من نگاه کند، با دیدن آن‌ها به یاد سگ‌آبی
بیفتد. بی‌سلیقگی کامل در لباس را هم به تصویر اضافه کنید که با یک دماغ
بزرگ توروالدزی، کم کم می‌تواند چهره کودکی من را در ذهن شما شکل دهد.

من دماغ بزرگی دارم - البته آدم‌های خانواده ما به من گفته‌اند که اندازه
دماغ یک مرد، نشان دهنده چیزهای بزرگ دیگری هم هست - اما گفتن این چیزها
به یک نوجوان، دردی از او دوا نمی‌کند. برای او، تنها فایده بزرگی دماغ،
سایه انداختن بر دندان‌های پیش‌آمده است. عکس‌های پرسنلی سه نسل از مردهای
خانواده توروالدز، یادآور این واقعیت دردناک است که در این تصاویر بیش از
آنکه آدم‌ها دیده شوند، دماغ‌ها دیده می‌شوند. یا لااقل در آن دوره که برای
من این طور به نظر می‌رسید.

حالا برای کامل شدن‌ تصویر، شروع کنید به اضافه کردن جزییات. موی قهوه‌ای
(که البته اینجا در آمریکا به آن بلوند می‌گویند ولی در اسکاندیناوی،
دقیقا قهوه‌ای است)، چشم‌های آبی و ضعیفی که بهتر است به خاطرشان عینک
بزنید. و از آنجایی که عینک زدن، ممکن است حواس مردم را از دماغ پرت کند،
من همیشه آن‌ها را بر چشم داشتم. تمام اوقات.

آه و قبلا هم که به سلیقه وحشتناک در لباس پوشیدن اشاره کرده‌ام. رنگ
انتخابی من همیشه آبی بود و منظورم از آبی، یک جین آبی با یک یقه‌اسکی آبی
است یا شاید هم فیروزه‌ای. فرقی نمی‌کند. خوشبختانه خانواده ما چندان اهل
عکس گرفتن نبود و به همین علت شواهد کمی از آن جریان‌ها وجود دارد.

البته چند تایی عکس هست. در یکی از آن‌ها من تقریبا سیزده ساله‌ام و با
خواهرم سارا که شانزده ماه از من کوچکتر است، جلوی دوربین
ایستاده‌ایم. وضع او بد نیست، ولی من وحشتناکم. یک بچه رنگ و رو رفته و
استخوانی که خودش را برای عکاس، که احتمالا باید مادرم باشد، کج و کوله
کرده. او احتمالا این شاهکار را قبل از رفتن به محل کارش به عنوان
ویرایشگر خبرگزاری فنلاند، خلق کرده است.

به دنیا آمدنم در آخرین روز سال (۲۸ دسامبر) به این معنا بود که من در
مدرسه جوان‌ترین دانش‌آموز کلاس بودم و این یعنی کوچک‌ترین بودن در کلاس. در
سال‌های بعدی اینکه نیم‌سال از بقیه بچه‌ها کوچک‌تر باشید چندان مهم نیست،
ولی مطمئنا در اولین سال‌های مدرسه، موضوع مهمی بود.

و می‌دانید؟ جالب است که هیچ‌کدام از این مسایل چندان هم مهم نبودند. یک
سگ‌آبیِ کوتوله عینکی بودن با موهای نامرتب در اکثر روزها (و موهای واقعا
نامرتب در بقیه روزها) و لباس بد پوشیدن، چندان مهم نبودند. چون من شخصیت
دوست داشتنی‌ای داشتم.

نه!

نه! بگذارید با این حقیقت روبرو شوم؛ من یک نِرد بودم. یک گیک. تقریبا از
همان اوایل. البته دسته‌های عینکم را با چسب نچسبانده بودم ولی ممکن بود
این کار را هم بکنم، چون بقیه ویژگی‌ها را داشتم. ریاضی‌ام خوب بود، فیزیکم
خوب بود و توانایی‌های اجتماعی‌ام افتضاح بود. و این قبل از دورانی بود که
نِرد بودن، باحال به حساب بیاید.

احتمالا همه در مدرسه یکی مثل من را می‌شناخته‌اند. کسی که به خوب بودن در
ریاضی مشهور بود - نه به این خاطر که خوب درس می‌خواند بلکه فقط به این
خاطر که در ریاضی خوب بود. من همین آدم در کلاس خودم بودم.

ولی اجازه بدهید قبل از اینکه زیاد برایم افسوس بخورید، برگردم به کامل
کردن آن تصویر. شاید یک نِرد بودم و شاید یک کوتوله بودم، ولی وضعم بد
نبود. ورزشکار نبودم ولی اسکول هم نبودم. در مدرسه بازی ساعت‌های تفریح
برانبول
 \LFootnote{Brannboll}
 بود - یک بازی سرعتی و قدرتی که در آن
بازیکنان سعی می‌کنند با پرتاب یک توپ، بازیکنان تیم حریف را از بازی خارج
کنند. من هیچ وقت اولین بازی کنی نبودم که کشیده می‌شد ولی معمولا همان
اول‌ها انتخاب می‌شدم.

پس در سلسله‌مراتب اجتماعی ممکن بود یک نرد باشم ولی در کل، مدرسه خوب
بود. بدون اینکه مجبور باشم کار زیادی بکنم، نمره‌های خوبی می‌گرفتم. البته
هیچ وقت نمره‌های عالی نداشتم چون هیچ وقت کار نمی‌کردم. در سلسله‌مراتب
اجتماعی هم جای خوبی داشتم. دیگر کسی به دماغم توجه نمی‌کرد و حالا که به
گذشته نگاه می‌کنم می‌بینم که دلیل‌اش این بوده که آدم‌ها بیشتر از دماغ‌من،
درگیر مشکلات خودشان بوده‌اند.

با نگاه کردن به گذشته می‌بینم که اکثر بچه‌های دیگر هم سلیقه بدی در لباس
داشته‌اند. ما بزرگ می‌شویم و ناگهان کس دیگری مسوول این تصمیم‌گیری
می‌شود. در مورد من، کارمندان تبلیغاتی شرکت‌های بزرگ فناوری هستند که
لباسم را انتخاب می‌کنند. همان‌هایی که تی‌شرت‌ها و ژاکت‌ها را برای پخش
رایگان در کنفرانس‌ها انتخاب می‌کنند. این روزها، من تقریبا همه لباس‌هایم
را از شرکت‌های فنّآوری می‌گیرم و در نتیجه عملا نیازی به انتخاب لباس
ندارم. همسرم هم بقیه لباس‌ها مثل صندل‌ها و جوراب‌ها را انتخاب می‌کند و من
دیگر لازم نیست نگران لباس باشم.

و نسبت به دماغم هم رشد خوبی کرده‌ام. حداقل حالا بیشتر از آنکه دماغ
باشم، آدم هستم.

\section{بخش دوم}
احتمالا این موضوع که برخی از اولین و شادترین خاطرات من مربوط به
بازی‌کردن با ماشین‌حساب الکترونیکی پدربزرگم است، کسی را متعجب نخواهد
کرد.

پدربزرگ مورد بحث، لئو والدمار تورنکویست\LFootnote{Leo Waldmar
  Tornqvist}، پدرِ مادرم بود و استاد آمار در دانشگاه هلسینکی. یادم است
که با محاسبه سینوس اعداد اتفاقی کلی کیف کرده‌ام. نه به این خاطر که
جواب‌ها برایم مهم بوده‌اند (عملا برای هیچ کس مهم نیستند)، بلکه به این
خاطر که این جریان مدت‌ها پیش اتفاق افتاده و آن روزها ماشین‌حساب‌ها جواب
را به سرعت تحویل نمی‌دادند. آن‌ها واقعا جواب را حساب می‌کردند. در طول
حساب کردن هم کلی چشمک می‌زدند تا به شما بگویند که \dbquote{بعله! هنوز
  زنده هستم و حدود ده ثانیه‌ای طول می‌کشد تا این محاسبات را انجام بدهم و
  در این مدت برای شما چشمک می‌زنم تا متوجه باشید که چقدر مشغولم.}

این جذاب بود. بسیار جذاب‌تر از ماشین‌حساب‌های امروزی که برای انجام کاری
به سادگی سینوس‌گرفتن از یک عدد، عرق هم نمی‌کنند. با ماشین‌حساب‌های آن
روزها، می‌فهمیدید که کاری که مشغول انجام‌اش هستید، سخت است. آن‌ها این را
به وضوح به شما نشان می‌دادند.

اولین باری که کامپیوتر دیدم را دقیقا به خاطر ندارم ولی باید چیزی حدود
یازده‌ سالگی‌ام بوده باشد. احتمالا ۱۹۸۱ که پدربزرگم در آن سال یک کمودور
\lr{ VIC-20} خرید. از آنجایی که کلی از وقتم را به بازی‌کردن با
ماشین‌حساب جادویی‌اش گذرانده‌بودم، باید در شروع بازی با این کامپیوتر
جدید، بسیار هیجان زده بوده باشم - اما حقیقت این است که چیز چندانی در
این مورد به خاطرم نمی‌رسد. در اصل حتی یادم نیست که چه زمانی جذب
کامپیوترها شدم. آرام شروع شد و در من رشد کرد.

کمودور \lr{VIC-20} یکی از اولین کامپیوترهای آماده‌ای بود که برای
استفاده خانگی طراحی شده بودند. برای استفاده از آن به هیچ تنظیم
سخت‌افزاری نیاز نبود. کافی بود آن را به تلویزیون وصل کنید و بعد روشنش
کنید و یک مکان‌نمای چشمک‌زن با گفتن \code{READY} در بالای یک صفحه آبی
آماده باشد تا شما به آن بگویید چه باید بکند.

مشکل اصلی این بود که کار چندانی نبود که بتوانید با آن کامپیوتر انجام
دهید. بخصوص اوایل کار که زیرساخت نرم‌افزارهای تجاری ایجاد نشده
بود. تنها کاری که می‌شد با آن ماشین کرد، برنامه‌نویسی بیسیک\RFootnote{یک
  زبانه برنامه نویسی که در سال ۱۹۶۴ طراحی شد و در دهه هفتاد برای
  استفاده از کامپیوترهای کوچک بسیار محبوب بود} بود. دقیقا همان‌کاری که
پدربزرگ من شروع‌اش کرد.

پدربزرگ من این اسباب‌بازی جدید را دقیقا به عنوان یک اسباب بازی نگاه
می‌کرد، و همچنین به عنوان یک ماشین‌حساب مجلل. این کامپیوتر نه تنها
می‌توانست سینوس یک عدد را بسیار سریع‌تر محاسبه کند، که این قابلیت را هم
داشت که به شکل خودکار این کار را بر روی فهرست بزرگی از اعداد تکرار
کند. علاوه بر این، حالا او می‌توانست بسیاری از کارهایی را که پیش از این
باید در دانشکده و با کامپیوتر بزرگ آن‌جا انجام ‌داد، در خانه هم انجام
دهد.

و او می‌خواست من را هم در این تجربه شریک کند. همچنین می‌خواست من را به
ریاضی علاقمند کند.

پس من را روی زانو‌هایش می‌نشاند و از من می‌خواست تا برنامه‌هایی که با دقت
روی کاغذ نوشته بود را برایش تایپ کنم. می‌گفت خودش با کامپیوترها راحت
نیست. نمی‌دانم آن محاسبات راجع به چه چیزی بودند و بعید می‌دانم که آن
موقع هیچ درکی از کاری که می‌کردم هم داشته باشم ولی به هرحال آنجا بودم و
به او کمک می‌کردم. احتمالا کار از حالتی که او خودش به تنهایی برنامه‌ها
را وارد می‌کرد، خیلی بیشتر طول می‌کشید. ولی کسی چه می‌داند؟ من از همان
کودکی به صفحه‌کلید عادت کرده بودم، چیزی که پدربزرگم هیچ وقت امکان‌اش را
نداشت. بعد از مدرسه یا هر موقع دیگری که مادرم من را پیش پدربزرگم
می‌گذاشت، مشغول همین کار می‌شدیم.

بعد شروع کردم به خواندن راهنماهای کامپیوتر و وارد کردن برنامه‌های آماده
شده. مثال‌ها شامل بازی‌های ساده‌ای بودند که خودتان می‌توانستید آن‌ها را
وارد کنید. اگر همه چیز را درست تایپ می‌کردید، یک آقایی با گرافیک بد،
روی صفحه راه می‌رفت. بعد می‌توانستید برنامه را عوض کنید تا آقای راه
رونده، رنگش عوض شود. شما خودتان می‌توانستید این کار را بکنید.

این بالاترین لذت بود.

شروع کردم به نوشتن برنامه‌های خودم. اولین برنامه‌ای که نوشتم، اولین
برنامه‌ای بود که هر کسی می‌نویسد:

\begin{latin}
\begin{lstlisting}
10 PRINT "HELLO"
20 GOTO 10
\end{lstlisting}
\end{latin}

این برنامه دقیقا همان‌کاری را می‌کند که انتظار دارید بکند. روی صفحه
می‌نویسد \dbquote{سلام} و تا ابد به این کار ادامه می‌دهد. یا حداقل تا وقتی که
شما از شدت سر رفتن حوصله‌تان، برنامه را قطع کنید.

اما این قدم اول است. بعضی‌ها همین‌جا متوقف می‌شوند.برای آن‌ها این برنامه
احمقانه‌ای است چون \dbquote{چرا باید کسی علاقمند باشد به میلیون‌ها کلمه
  \xquote{سلام} خیره شود؟} اما به هرحال این برنامه تقریبا همیشه اولین
برنامه در راهنماهایی بود که آن روزها همراه کامپیوترهای شخصی داده
می‌شدند.

نکته جادویی اینجا است که شما می‌توانید این برنامه را تغییر دهید. خواهرم
می‌گوید که من یک تغییر ریشه‌ای در برنامه دادم تا نسخه دومی بسازم که به
جای نوشتن \dbquote{سلام}، روی صفحه بارها و بارها می‌نوشت \dbquote{سارا
  بهترین است.} در کل من برادر بزرگ‌تر مهربانی نبودم ولی این ژست
برنامه‌نویسی، تاثیر زیادی روی خواهرم گذاشت.

من این جریان را یادم نیست. هر بار که یک برنامه می‌نوشتم، آن را فراموش
می‌کردم و سراغ برنامه بعدی می‌رفتم.

\section{بخش سوم}

اجازه بدهید درباره فنلاند صحبت کنم. بعضی وقت‌ها در
اکتبر\RFootnote{حوالی مهر} هوا سایه‌های خاکستری غمگینی می‌گیرد و جوری
می‌شود که انگار هر لحظه می‌خواهد باران یا برف بیاید. هر روز صبح که از
خواب بلند می‌شوید، انتظار یک روز افسرده را دارید. باران سرد است و هر
خاطره‌ای از تابستان را می‌شوید. وقتی برف ببارد، این قدرت جادویی را خواهد
داشت که همه چیز را براق کند و فضا را با یک لایه خوش بینی، جلا
بدهد. مشکل این است که این خوش بینی فقط سه روز طول می‌کشد، ولی برف در
طول سه ماه آینده با سرمای استخوان سوزش باقی خواهد ماند.

اگر در ژانویه\RFootnote{حوالی دی} تصمیم بگیرید که از خانه بیرون بروید،
در سرما سرگردان خواهید شد. این فصل، فصل رطوبت، لباس‌های ضخیم و سُرخوردن
روی زمین هاکی‌ای است که بچه‌ها از آب بستن روی مسیری که شما برای رفتن به
سر کلاس دستور زبان از آن میان‌بُر می‌زنید، ساخته‌اند. راه رفتن در
خیابان‌های هلسینکی در این فصل یعنی جا خالی دادن از مسیر تلوتلو خوردن
پیرزن‌های مستی که احتمالا در سپتامبر\RFootnote{حوالی شهریور}
مادربزرگ‌های متشخصی بوده‌اند ولی در ساعت ۱۱ صبح یک روز وسط هفته ماه
ژانویه به خاطر خوردن ودکا در صبحانه، مشغول تلوتلو خوردن در پیاده‌رو
هستند. چه کسی می‌تواند به آن‌ها ایراد بگیرد؟‌ چند ساعت بعد هوا دوباره
تاریک خواهد بود و کاری هم نیست که انجام دهی . البته یک ورزش در فضای
بسته وجود دارد که من در زمستان جذب‌اش شده‌ام: برنامه‌نویسی.

مورفار\LFootnote{Morfar} (کلمه سوئدی برای \dbquote{پدرِ مادر})
معمولا کنار من است ولی نه همیشه. برای‌اش هم مهم نیست که وقتی حضور
ندارد، شما در اتاق‌اش بنشینید. پول برای خریدن اولین کتاب کامپیوترتان را
با التماس می‌گیرید. همه چیز به انگلیسی است و لازم است اول زبان را
رمزگشایی کنید. درک ادبیات فنی به زبانی که آن را بلد نیستید، سخت
است. پول توجیبی‌تان را خرج خرید مجلات کامپیوتری می‌کنید. یکی از آن‌ها
برنامه‌ای برای کدهای مورس دارد. نکته خاص درباره این برنامه، این است که
به زبان بیسیک نوشته نشده بلکه مجموعه‌ای از اعداد است که می‌شود مستقیما
آن‌ها را با دست به زبان ماشین ترجمه کرد - صفرها و یک‌هایی که کامپیوتر
آن‌ها را می‌فهمد.

اینجوری است که کشف می‌کنید زبان کامپیوترها، بیسیک نیست بلکه آن‌ها با یک
زبان بسیار ساده‌تر کار می‌کنند. بچه‌های هلسینکی دارند با پدر و مادرشان
هاکی بازی می‌کنند یا در جنگل‌ها اسکی می‌کنند. شما دارید یاد می‌گیرید که
کامپیوترها واقعاً چگونه کار می‌کنند. بدون اینکه بدانید برنامه‌هایی هستند
که می‌توانند اعداد قابل خواندن انسان‌ها را به صفر و یک‌های مورد علاقه
کامپیوترها تبدیل کنند، شروع می‌کنید به نوشتن برنامه‌ها با اعداد، و تبدیل
های لازم را هم با دست انجام می‌دهید. این برنامه‌نویسی به زبان ماشین است
و از طریق آن قادر هستید کارهایی را بکنید که قبلا فکر می‌کردید غیر ممکن
هستند. می‌توانید کامپیوترها را به مرز کارهایی که برای‌اش ممکن است
برانید. کوچک‌ترین جزییات را خودتان کنترل می‌کنید. شروع می‌کنید به فکر
کردن در این باره که چطور می‌توانید کار مشابهی را کمی سریع‌تر و با حجمی‌
کمی‌ کم‌تر انجام دهید. از آنجایی که هیچ لایه‌ای بین شما و کامپیوتر نیست،
تا حد ممکن به جواب نزدیک می‌شوید. این همان چیزی است که می‌توانید به آن
صمیمی شدن با ماشین بگویید.

دوازده‌ سال‌تان است، شاید هم سیزده یا چهارده، فرقی نمی‌کند. بقیه بچه‌ها در
بیرون دارند فوتبال بازی می‌کنند. کامپیوتر پدربزرگ‌تان جذاب‌تر
است. کامپیوترش دنیایی است که منطق بر آن حکم می‌راند. فقط سه نفر در کلاس
هستند که کامپیوتر دارند و فقط یکی از آن‌ها به این دلایل از آن استفاده
می‌کند. به گردهم‌آیی‌های هفتگی می‌روید. این تنها فعالیت اجتماعی در برنامه
روزانه‌ شما است البته به جز مواردی که گاه‌گداری در خانه یکی از کسانی که
کامپیوتر دارد جمع می‌شوید و شب آن‌جا می‌خوابید.

برای شما مهم نیست. دارید لذت می‌برید.

این ماجراها بعد از طلاق است. پدر در قسمت دیگری از شهر هلسینکی زندگی
می‌کند. او معتقد است که فرزندش باید بیش از یک سرگرمی داشته باشد و به
همین خاطر اسم شما را در کلاس بسکتبال،‌ ورزش مورد علاقه خودش،
می‌نویسد. این فاجعه است،‌ شما کوتوله تیم هستید. بعد از یک فصل و نیم از
هر زبانی استفاده می‌کنید تا به او بگویید که می‌خواهید تیم را رها کنید
چون این ورزش او است نه ورزش شما. \emph{*} نیمه‌برادر تازه شما، لئو،
ورزشکارتر است ولی بعدا او هم در نهایت مثل ۹۰ درصد جمعیت فنلاند،
لوتری\RFootnote{عقاید لوتریانیسم یکی از شاخه‌های عمده مسیحیت غربی است
  که با کلام مارتین لوتر، اصلاح طلب آلمانی مشخص است. لوتر برای اصلاح
  کلام و عمل کلیسا اصلاحات پروتستانی را راه اندای کرد.} می‌شود. این
موقعی است که پدر که یک بی‌خدای پر و پا قرص است، متوجه می‌شود که به عنوان
یک پدر وظیفه‌اش را درست انجام نداده - البته سال‌ها پیش که سارا به کلیسای
کاتولیک پیوسته بود هم به این جریان مشکوک شده بود.

پدربزرگی که کامپیوتر دارد از آن آدم‌های سرزنده نیست. دارد کچل می‌شود و
کمی چاق شده. عملا یکی از آن پروفسورهای بی‌حافظه‌ای که به سختی می‌توان به
آن‌ها نزدیک شد. حداقل این است که برون‌گرا نیست. ریاضی‌دانی را در ذهن مجسم
کنید که به فضا خیره می‌شود و تا وقتی که مشغول حل یک مساله است، هیچ حرفی
نمی‌زند. هیچ وقت نمی‌توانید بگویید دارد به چه چیزی فکر می‌کند. نظریه
تحلیل پیچیدگی؟ خانم سامکورپی\LFootnote{Mrs. Sammalkorpi} که
در طبقه پایین است؟ من هم برای این مدهوش شدن‌ها مشهور هستم. وقتی جلوی
کامپیوتر نشسته‌ام، اگر کسی مزاحمم بشود بسیار ناراحت می‌شوم. تاو می‌تواند
در این باره اطلاعات بیشتری به شما بدهد.

روشن‌ترین خاطرات من از مورفار نه پشت کامپیوترش که از کلبه قرمز کوچک‌اش
است. در هلسینکی مرسوم که مردم یک خانه ییلاقی داشت باشند، حتی اگر شده
یک چهار دیواری ده متر در ده متر. مردم برای رسیدگی به باغچه‌های کوچک‌شان،
به این خانه‌ها می‌روند. آن‌ها معمولا یک آپارتمان کوچک در شهر دارند و
اوقات فراغت را برای کاشت یا برداشت سیب‌زمینی یا رسیدگی به یکی دو درخت
سیب و بوته‌های رز به ییلاق‌های شان می‌روند. البته معمولا مسن‌ترها؛ چون
جوان‌ها مشغول کارهای شان هستند. این آدم‌ها سر چیزهایی که می‌کارند،
رقابت‌های خنده‌داری دارند. این همان‌جایی است که مورفار درخت سیب من را
کاشته، یک نهال کوچک. شاید هنوز هم آن‌جا باشد، مگر اینکه آن قدر خوب رشد
کرده باشد که یکی از همسایه‌های حسود در تاریکی یک شب تابستانی یواشکی
داخل شده و آن را بریده باشد.

مورفار چهار سال بعد از اینکه من را به دنیای کامپیوتر معرفی کرد، نیمی
از بدن‌اش به خاطر یک لخته خونی در مغز، فلج شد. این برای همه یک شوک
بود. نزدیک‌ترین فرد خانواده به شما، برای یک‌سال در بیمارستان بود ولی
برای شما چندان اهمیتی نداشت. شاید یک جور مکانیسم دفاعی بود یا شاید هم
به این دلیل بود که جوانان حساسیت کمتری به اینجور چیزها دارند. او دیگر
همان آدم قبلی نبود و در نتیجه شما نمی‌خواستید برای دیدن‌اش بروید. شاید
دو هفته یک‌بار پیش‌اش می‌رفتید. مادرتان بیشتر سر می‌زد. خواهرتان هم که
وظیفه مددکار اجتماعی فامیل بودن را از همان روزها بر عهده گرفته‌بود، به
همچنین.

بعد از اینکه پدربزرگ مرد، کامپیوترش آمد تا با شما زندگی کند. در این
مورد بحث خاصی در نگرفت.

\section{بخش چهارم}
اجازه بدهید کمی به عقب برگردیم.

فنلاند شاید این روزها یکی از پیشرفته‌ترین کشورهای دنیا باشد. ولی قرن‌ها
قبل، این کشور به زحمت چیزی بیشتر از یک توقف‌گاه برای وایکینگ‌هایی بود که
در \dbquote{تجارت} با کنستانتین بودند. بعدها، وقتی که همسایه‌های سوئدی
خواستند فنلاندی‌ها را مردمانی صلح‌جو کنند، اسقف هنری را به آن‌جا
فرستادند. این اسقف متولد انگلیس، در ۱۱۵۵ برای ماموریتی از سوی کلیسای
کاتولیک، وارد فنلاند شد. سوئدی‌های نوآیین، استحکامات نظامی فنلاند را
تقویت کردند تا از خود در برابر امپراتوری شرقی یعنی روسیه حفاظت کنند و
در نهایت هم نبرد بر سر کنترل فنلاند را از روس‌ها بردند. در طول قرن‌های
بعدی، سوئدی‌ها با پاداش‌ و تنبه بر اساس زمین و مالیات، فنلاندی‌ها را به
کار کشیدند و تا سال ۱۷۱۴ نمایش را اداره کردند. در این سال، روسیه با
تسخیر فنلاند یک میان‌پرده هفت ساله را به اجرا گذاشت. بعد سوئد دوباره
کنترل این مستعمره را به دست گرفت و تا سال ۱۸۰۹ آن را اداره کرد که طی
آن، ناپلئون و روسیه با هم به فنلاند حمله کردند و تا سال ۱۹۱۷ که انقلاب
کمونیستی روسیه به وقوع پیوست، فنلاند بخشی از روسیه بود. در این دوره،
جمعیت نسل اول مهاجران سوئدی به فنلاند، به ۳۵۰۰۰۰ نفر می‌رسید. این افراد
همان‌ سوئدی زبان‌هایی هستند که این روزها حدود پنج درصد جمعیت فنلاند را
تشکیل می‌دهند.

از جمله خانواده پخش و پلای من. 

جد مادری من یک کشاورز نسبتا فقیر ازجاپو\LFootnote{Jappo}
بود؛ یک شهر کوچک در کنار شهر واسا\LFootnote{Vasa}. او شش
پسر داشت که حداقل دوتای آن‌ها مدرک دکترا گرفتند. این مساله چیزهای
بسیاری در مورد امکان پیشرفت در فنلاند را نشان می‌دهد. بله! اعصاب آدم از
تاریکی فصل زمستان و درآوردن کفش‌ها موقع ورود به خانه خرد می‌شود ولی در
عوض حق دارید به رایگان مدرک دانشگاهی بگیرید. این با آمریکا که در آن
بسیاری از بچه‌ها بدون هیچ امیدی بزرگ می‌شوند، خیلی فرق دارد. یکی از آن
دو پسر، پدر بزرگ من یعنی لئو والدمار تورنکویست بود، همان رفیقی که من
را به دنیای کامپیوتر معرفی کرد.

می‌رسیم به جد پدریم. این همان آدمی است که اسم توروالدز را برای اسم وسط
خودش اختراع کرد. اسم او اوله توروالد الیس ساکسبرگ\LFootnote{Ole
  Torvald Elis Saxberg} بود. پدربزرگ من بدون پدر متولد شده بود
(ساکسبرگ اسم دوران دوشیزگی مادرش بود) و بعد از آشنایی مادرش با آقای
متشخصی که جده‌ام در نهایت با او ازدواج کرده بود،
کارانکو\LFootnote{karanko} نامیده می‌شد. فارفار\LFootnote{Farfar}
(\dbquote{پدر پدرم}) این آقا را دوست نداشت و در نتیجه اسم‌اش را عوض
کرد. او اسم آخرش را حذف کرد و با این نظریه که یک \lr{s} تشخص بیشتری به اسم
وسطش می‌دهد، یک \lr{s} به انتهای آن اضافه کرد. توروالد به خودی خود یعنی
\dbquote{سرزمین تور}. بهتر بود پدربزرگم برای ساختن یک اسم جدید از صفر
شروع کند چون اضافه شدن یک \lr{s} معنای اصلی اسم را از بین می‌برد و هم
سوئدی‌ها و هم فنلاندی‌ها را در مورد شیوه تلفظ این اسم، گیج می‌کند. آن‌ها
فکر می‌کنند که اسم باید \lr{Thorwalds} نوشته شود. در دنیا بیست و یک
توروالدز هست و همه با من فامیل‌اند. همه ما در این سردرگمی شریک هستیم.

شاید به همین خاطر است که در اینترنت من همیشه \dbquote{لینوس}
بودم. \dbquote{توروالدز} گیج کننده است.

این پدربزرگ در دانشگاه تدریس نمی‌کرد. یک روزنامه‌نگار و شاعر بود. اولین
شغل او، سردبیری یک روزنامه‌ محلی کوچک در ۱۰۰ کیلومتری هلسینکی بود. او
به خاطر زیاد‌ه‌روی در نوشیدن به هنگام کار اخراج شد. ازدواج‌اش با
مادربزرگم هم به هم خورد. با وجود مشکل همیشگیش با مشروب، به شهر
تورکو\LFootnote{Turku} در جنوب فنلاند رفت و در آنجا سردبیر یک روزنامه
شد و چند کتاب شعر هم منتشر کرد. ما برای کریسمس و عید پاک پیش او می‌رویم
و سری هم به مادربزرگ می‌زنیم. فارمار مارتا\LFootnote{Marta} در هلسینکی
زندگی می‌کند و به خاطر پختن پن‌کیک‌های عالی، شهرت دارد.

فارفار پنج سال پیش درگذشت.

قبول! من هیچ وقت هیچ کدام از کتاب‌هایش را نخواندم. این واقعیتی است که
پدر همیشه به غریبه‌ها متذکر می‌شود.

روزنامه‌نگارها همه جای خانواده من پراکنده‌اند. بر اساس افسانه‌های
خانوادگی، یکی از اجداد من، ارنست فون وندت\LFootnote{Ernst von Wendt}
روزنامه‌نگاری بود که به خاطر طرفداری از سفیدها در جنگ‌های داخلی فنلاند
که منجر به استقلال ما از روسیه در ۱۹۱۷ شد، توسط سرخ‌ها دستگیر شد (باشه!
کتاب‌های این یکی را هم نخوانده‌ام ولی همه می‌گویند چیز زیادی هم از دست
نداده‌ام). پدرم نیلز\LFootnote{Nils} (که همه او را به اسم
نیک\LFootnote{Nicke} می‌شناسند) یک روزنامه‌نگار رادیو و تلویزیون است که
از دهه ۱۹۶۰ و دوره دبیرستانش عضو فعال حزب کمونیست بوده است. اولین
گرایش‌های سیاسی او موقعی به وجود آمد که خبردار شد در فنلاند، خشونت‌هایی
علیه طرفداران کمونیسم در جریان است. چند دهه بعد، پذیرفت که شیفتگی‌اش به
کمونیسم شاید محصول خام‌اندیشی‌اش بوده باشد. او مادر من
آنا\LFootnote{Anna} (که به نام میکی\LFootnote{Mikke} شهرت داشت) را
موقعی ملاقات می کند که هر دو دانشجوهای شورشی دانشگاه‌های دهه ۱۹۶۰
بودند. داستان این است که آن‌ها برای شرکت در یک گلگشت کلوپ دانشجویان
سوئدی زبان که پدرم مسوولش بود به بیرون از شهر رفته بودند. پدرم که برای
جلب توجه مادرم، یک رقیب پیدا کرده بود، در موقع برگشت، رقیب را مسئول
نظارت بر سوار شدن همه بر اتوبوس کرد و با استفاده از این فرصت، خودش
کنار مادرم نشست و او را متقاعد کرد تا با او به خانه بیاید (و مردم من
را نابغه فامیل می‌دانند!).

من کمابیش در بین تظاهرات درون دانشکده و احتمالا با موسیقی جانی
میشل\LFootnote{Joni Mitchell} در پس‌زمینه متولد شدم. آشیانه عشق خانواده
من، اتاقی در خانه پدربزرگ و مادربزرگم بود. سبد لباس‌های چرک ما اولین
ننوی من بود. خوشبختانه به خاطر آوردن آن دوره کار راحتی نیست. در حالی
که من سه ماه بیشتر نداشتم،‌ پدرم ترجیح داد به جای رفتن به زندان به
عنوان یک آدم باوجدان، تن به ثبت نام در خدمت سربازی یازده ماهه بدهد. او
آن قدر سرباز و تیرانداز خوبی از آب درآمد که می‌توانست دائما از مرخصی‌های
آخر هفته استفاده کند. خاطره‌های خانوادگی می‌گویند که خواهرم سارا در همین
دوره به وجود آمد. مادرم در مواقعی که مشغول رسیدگی به دو بچه کوچک‌اش
نبود، به عنوان ویراستار اخبار خارجی خبرگزاری فنلاند، کار می‌کرد. این
روزها او ویراستار تصاویر است.

این همان خانواده روزنامه‌نگاران است که من به شکل معجزه‌آسایی از آن جان
سالم به در بردم. سارا دفتر خودش را دارد که در آن گزارش‌های خبری را
ترجمه می‌کند و همچنین با خبرگزاری فنلاند نیز همکاری دارد. برادر نا تنی
من، لئو توروالدز، از آن آدم‌های علاقمند سینما است که می‌خواهد روزی فیلم
خودش را کارگردانی کند. از آنجایی که همه افراد خانواده من روزنامه‌نگار
هستند، احساس می‌کنم محق هستم در این باره که آن‌ها چه وازده‌هایی هستند،
شوخی کنم. می‌دانم که با گفتن این حرف آدم مزخرفی به نظر می‌رسم، ولی در
این سال‌ها، خانه ما در فنلاند به اندازه کافی سهمش را به خبرنگارانی که
برای ساختن خبر به آن هجوم آورده‌اند و کسانی که اصولا خودشان از هیچ خبر
ساخته‌اند و همه آن‌هایی که همیشه به نظر می‌رسد کمی زیادی نوشیده‌اند، ادا
کرده است. خبرنگاران زیاد می‌نوشند.

این آن موقعی است که باید در اتاق خواب مخفی شد. شاید هم مادر وضعیت
احساسی مناسبی ندارد. ما در یک آپارتمان دو اتاق خوابه در طبقه دوم یک
ساختمان رنگ‌پریده زرد در استورا روبرتسگاتان\LFootnote{Reborstagan} در
رودبرگن\LFootnote{Rodbergen} زندگی‌ می‌کنیم که ناحیه‌ای کوچک در همسایگی
مرکز هلسینکی است. سارا و برادر نفرت‌انگیزش که شانزده ماه از او بزرگ‌تر
است، در یکی از اتاق خواب‌ها زندگی می‌کنند. کنار خانه یک بوستان کوچک هست
که به نام خانواده سینبریچف\LFootnote{Sinebrychoff} که یک
آبجو‌سازی محلی دارند، نام‌گذاری شده است. این مساله همیشه به نظر من عجیب
بوده ولی واقعاً چه فرقی هست بین این اسم‌گذاری و اسم‌گذاری یک استادیوم
بسکتبال به نام یک تولید کننده لوازم دفتری؟ (چون یک بار یک گربه در این
پارک دیده‌ایم، پارک سینبریچف در خانواده ما \dbquote{پارک گربه} نامیده
می‌شود). یک خانه مخروبه هم هست که کبوترها در آن لانه می‌کنند. پارک روی
یک تپه ساخته شده و در زمستان محل سرسره بازی است. محل دیگر بازی، حیاط
سیمانی پشت ساختمان ما است. وقتی قایم‌باشک بازی می‌کنیم، بالا رفتن پنج
طبقه توسط نردبان و رسیدن به سقف بسیار مفرح است.

ولی هیچ تفریحی به پای کار با کامپیوتر نمی‌رسد.حالا که کامپیوتر در خانه
است، می‌شود همه شب را بیدار ماند. همه پسرها شب‌ را با خواندن
\dbquote{پلی‌بوی} در زیر پتو بیدار می‌ماندند. در عوض من خودم را به خواب
می‌زدم تا مادرم سراغ کارهای خودش برود و بعد از تخت بیرون می‌پریدم و پشت
کامپیوتر می‌نشستم. این قبل از دوره چت‌روم‌ها بود.

\dbquote{لینوس! وقت غذا است!} بعضی وقت‌ها حتی غذا را هم بی خیال
می‌شدید. بعد مادرتان شروع می‌کرد به تعریف این داستان برای همکارانش که
شما بچه بسیار کم دردسری هستید و تنها کاری که برای راضی نگه‌داشتن‌تان
کافی‌است، این است که شما را با یک کامپیوتر در یک کمد تاریک بیاندازند و
گاه‌گداری هم کمی ماکارونی خشک برای تان بگذارند. خیلی هم بی‌راه
نرفته. هیچ کس نگران این نبود که این بچه را بدزدند (اصلا کسی متوجه
می‌شد؟). بدون شک کامپیوترها در آن دوره‌ای که کمتر پیچیده بودند، برای
بچه‌ها مناسب‌تر بودند. آن روزها هر تازه‌کاری مثل من، می‌توانست کاپوت
کامپیوتر را بالا بزند و موتورش را بررسی کند. حالا که کامپیوترها
پیچیده‌تر شده‌اند، دیگر هر کسی نمی‌تواند به راحتی کاپوت را بالا بزند و
موتور را پیاده و سوار کند و در نتیجه دیگر نمی‌تواند یاد بگیرد که این
ماشین‌ها دقیقا چطور کار می‌کنند. آخرین باری که خود شما موتور ماشین‌تان را
باز کردید و کاری پیچیده‌تر از تعویض فیلتر روغن کردید، کی بود؟

این روزها بچه‌ها به جای ور رفتن با موتور استعاره‌ای کامپیوتر، آن قدر با
آن بازی می‌کنند تا عقل‌شان را از دست بدهند. البته مشکلی با بازی‌های
کامپیوتری ندارم. در اصل اولین برنامه‌های خودم هم، بازی‌ها بوده‌اند.

در یکی از آن‌ها، شما یک زیر دریایی کوچک را در طول یک غار زیر آبی کنترل
می‌کردید. یک مفهوم کاملا استاندارد برای بازی‌. کل جهان از راست به چپ
حرکت می‌کرد و بازیکن در نقش زیر دریایی باید با بالا و پایین رفتن، از
برخورد با دیواره‌های غار و ماهی‌های بزرگ جلوگیری می‌کرد. ماهی هم با کل
جهان حرکت می‌کرد و حرکت مستقلی نداشت. همان طور که بازی ادامه پیدا
می‌کرد،‌ حرکت سریع و سریع‌تر و عرض غار، کم و کم‌تر می‌شد. در این بازی نمی‌شد
برنده شد و اصولا هم برنده شدن، هدف بازی نبود. می‌شد یک هفته‌ای با بازی
کردن تفریح کرد و بعد باید به سراغ بازی دیگری می‌رفتید. برای من تمام
مساله سر این بود که بتوانم برنامه این بازی را بنویسم و بعد سراغ برنامه
بعدی بروم.

اسباب‌ بازی‌های دیگری هم هست،‌ مثلا هواپیماها، خودروها، کشتی‌ها و قطارهای
مدل. یک‌بار پدر یک قطار مدل گران‌قیمت آلمانی برایم خرید. دلیل این کارش
این بود که خودش هیچ وقت در دوره بچگی قطار مدل نداشت و معتقد بود که این
می‌تواند یک سرگرمی خوب مشترک بین پدر و پسر باشد. چیز جالبی بود ولی
نمی‌توانست با کامپیوتر رقابت کند. محروم شدن از کار با کامپیوتر هیچ وقت
به خاطر کار زیاد با آن نبود بلکه دلایل دیگری مثل دعوا کردن با سارا
داشت. در طول مدارس ابتدایی و دبیرستان، شما همیشه مشغول رقابت با همدیگر
هستید بخصوص در مورد دروس اصلی.

رقابت حاصل خوبی داشت. بدون متلک‌های من سارا هیچ وقت این قدر انگیزه پیدا
نمی‌کرد که برای جلو افتادن از من، به جای پنج مقاله لازم برای
فارغ‌التحصیل شدن از دبیرستان‌های فنلاند، شش مقاله بنویسد. در طرف مقابل
من باید به خاطر اینکه انگلیسی‌ام قابل فهمیدن است، از سارا متشکر
باشم. او همیشه انگلیسی من را که در اصل مخلوط فنلاندی / انگلیسی بود،
دست می‌انداخت. به همین دلیل پیشرفت کردم. حالا که بحث به اینجا رسیده این
را هم بگویم که مادرم هم معمولا من را دست می‌انداخت. البته نه به خاطر
انگلیسی، برای این موضوع که هیچ وقت علاقمند نبودم دخترهایی که می‌خواستند
\dbquote{نابغه ریاضی} به آن‌ها درس دهد را به خانه بیاورم.

در آن دوره ما با پدرم و دوست‌دخترش زندگی می‌کردیم. بعضی وقت‌ها هم سارا با
پدر زندگی می‌کرد و من با مادر. گاهی هم هر دو پیش مادر بودیم. به هرحال
زبان سوئدی کلمه‌ای برای \dbquote{خانواده بدکارکرد} ندارد. به خاطر طلاق،
پول زیادی نداشتیم. یکی از روشن‌ترین خاطراتم مربوط به زمانی است که مادر
مجبور شد تنها دارایی‌اش را به گرو بگذارد؛ یک سهم از شرکت مخابرات
هلسینکی که به خاطر داشتن یک خط تلفن، هر شهروند صاحب آن می‌شود. احتمالا
ارزش‌اش چیزی حدود ۵۰۰ دلار بود و هر بار که دچار مشکل مالی می‌شدیم،‌ باید
سند آن را به مرکز کارگشایی می‌بردیم. یادم هست که یک بار با مادرم رفتم و
کلی خجالت کشیدم (حالا من یکی از اعضای هیات مدیره آن شرکت هستم. در اصل
تنها شرکتی است که من عضو هیات مدیره‌اش هستم). یادم هست که یک‌بار دیگر هم
احساس خجالت کردم؛ وقتی که برای خرید اولین ساعت مچی‌ام پول جمع کرده‌ بودم
و مادرم از من خواست که از پدربزرگ بخواهم بقیه پول ساعت را تقبل کند.

دوره‌ای هم بود که طی آن مادر شب‌ها کار می‌کرد و من و سارا باید تهیه غذای
خودمان را بر عهده می‌گرفتیم. مادر می‌خواست که ما به مغازه کنار خانه
برویم و با حسابی که داشتیم، مواد غذایی بخریم. ما به جای غذا، تنقلات
می‌خریدیم چون تا دیروقت پای کامپیوتر نشستن و تنقلات خوردن فوق العاده
بود. در شرایط مشابه، بقیه پسرها بیدار می‌ماندند و روی لحاف پلی‌بوی
\dbquote{می‌خواندند}

کمی بعد از اینکه پدربزرگ سکته کرد، مورمور (مادر مادرم) هم دیگر مواظبت
از خودش را فراموش کرد. او به خاطر چیزی که خودش \dbquote{کسلی} می‌نامید،
برای ده سال در یک خانه سالمندان بستری شد. دو سالی که از تاریخ
بستری‌شدن‌اش گذشت، ما به آپارتمان‌اش اسباب‌کشی کردیم. خانه‌ای در طبقه اول
یک ساختمان قرص و محکم مربوط به دوران روسیه که کنار یک پارک زیبا در
نزدیکی اسکله هلسینکی واقع شده بود. ساختمان یک آشپزخانه کوچک و سه اتاق
خواب داشت. سارا اتاق بزرگ‌تر را برداشت. پسر خلافی که با یک کمد تاریک و
کمی پاستای خشک و یک کامپیوتر خوشحال می‌شد، به کوچک‌ترین اتاق
رفت. پنجره‌ها را با پارچه‌های تیره مشکی پوشاندم تا نور آفتاب به داخل
اتاق سرک نکشد. کامپیوتر هم روی یک میز کوچک در فاصله نیم‌متری تختخواب
قرار گرفت.

\begin{journal}
وقتی که سردبیر مجله‌ سن‌جوز مرکوری نیوز\LFootnote{San Jose Mercury News}
در بهار ۱۹۹۹ از من خواست تا گزارشی در مورد لینوس توروالدز بنویسم، به
سختی از وجود همچین آدمی اطلاع داشتم. از بهار سال قبل، کلمه لینوکس
بیشتر و بیشتر شنیده می‌شد. یعنی از موقعی که چند شرکت با پیش‌قدمی
نت‌اسکیپ\RFootnote{\lr{Netscape} - که اولین جنگ مرورگرها را شروع کرد و در
  نهایت با انتشار آزاد متن مرورگرش، باعث به دنیا آمدن فایرفاکس شد.}
شروع به پذیرش مفهوم بازمتن\LFootnote{Open Source} برای
نرم‌افزارها و حتی سیستم‌های عامل خود کرده بودند. البته قبل از این هم من
در مورد لینوکس شنیده بودم. در اوایل دهه ۱۹۹۰، من ویراستار نشریه‌ای
مرتبط با یونیکس و نرم‌افزارهای بازمتن بودم و در نتیجه جمله‌ای از یک منبع
درباره اسم لینوس، در ذهنم وجود داشت. این منبع می‌گفت که لینوس یک
دانشجوی فنلاندی است که یک نسخه قوی از یونیکس را در خوابگاه ش نوشته و
آن را به رایگان در اینترنت پخش کرده است. این منبع چندان هم دقیق
نبود. دلیل زنگ زدن سردبیر این بود که لینوس برای یک سخنرانی و شرکت در
جلسه آشنایی با لینوکس به شهر ما یعنی سن‌جوز آمده بود. سردبیر با گفتن
اینکه \dbquote{ما امروز اینجا یک فوق ستاره داریم} و فکس کردن چند بریده روزنامه
درباره لینوس، من را مامور نوشتن یک گزارش کرد.

لینوس دو سال قبل به سلیکون‌ولی\LFootnote{Silicon Valley}
آمده بود و برای شرکت ترنسمتا\LFootnote{Transmeta} که آن
روزها شرکتی با پروژه‌های مخفی بود، کار می‌کرد. این شرکت سال‌ها مشغول
توسعه یک ریزپردازنده بود که قرار بود صنعت کامپیوتر را متحول کند. شغل
او به شکلی بود که اجازه می‌داد کماکان به فعالیت بسیار وقت‌گیرِ
تصمیم‌گیرنده نهایی بودن در هر تغییر پیشنهاد شده در لینوکس، ادامه
دهد. لینوس همچنین وقت کافی داشت تا به عنوان یکی‌ از مشهورترین چهره‌های
جنبش تازه جوانه زده نرم‌افزارهای بازمتن، به سرتاسر دنیا سفر کند.

او مشغول تبدیل شدن به قهرمان یک فرقه جدید بود. در حالی که
بیل‌گیتس\RFootnote{بنیان گذار و مدیر عامل شرکت مایکروسافت} به عنوان
الهه انتقام همه، در حال زندگی در زانادو\LFootnote{Xanadu}ی مجلل خود
بود، لینوس با همسر و دخترهای تازه‌پایش در یک مجتمع فشرده در سانتاکلاوس
زندگی‌ می‌کرد. او به وضوح به ثروت عظیمی که بر سر برنامه‌نویسان
کم‌استعدادتر در حال باریدن بود، بی‌اعتنایی می‌کرد. نفس حضور لینوس، معمای
حل نشدنی‌ای بود برای دیگر ساکنان سیلیکون‌ولی که تنها انگیزه‌شان، میزان
سود سهام‌های بورس بود؛ \dbquote{چطور شخصی با اینهمه استعداد نسبت به
  ثروت‌مندشدن بی‌اعتنا است؟}

دسترسی به لینوس راحت نبود. به پیام‌های صوتی‌اش گوش نمی‌داد و به ندرت
پیش‌می‌آمد که ایمیلی را جواب دهد. هفته‌ها طول کشید تا او را پای تلفن بکشم
ولی وقتی این‌کار انجام شد، به راحتی‌ پذیرفت که در اولین وقت خالی‌اش،
مصاحبه کند؛ یک ماه بعد در می ۱۹۹۹. با این احساس حرفه‌ای که بهتر است
مصاحبه در محیطی نزدیک به روحیات مصاحبه‌شونده انجام شود، تصمیم گرفتم که
محیط پس زمینه مقاله‌ام، یک سونای فنلاندی باشد. در یک
موستانگ\LFootnote{Mustang} اجاره‌ای که عکاس من راننده‌اش بود به سمت
سانتاکروز و سونایی راه افتادیم که به عنوان بهترین سونای فنلاندی منطقه،
به ما پیشنهاد شده بود. این سونا کنار یک منطقه مختص لختی‌ها بود.

وقتی از ورودی دفتر ترنسمتا در یک ساختمان بدون نام بیرون آمد، یک قوطی
کوکای باز شده در دستش بود. لباس رسمی برنامه‌ نویس‌ها یعنی یک شلوار جین،
تی‌شرت‌های پخش شده در کنفرانس‌ها و ترکیب جدا ناشدنی جوراب و صندل‌هایی را
داشت که ادعا می‌کرد حتی پیش از اینکه یک برنامه‌نویس دیگر را با آن‌ها
ببیند، به آن علاقمند بوده است. وقتی در مورد ترکیب جوراب و صندل پرسیدم
جواب داد که \dbquote{باید یکی از قوانین طبیعی مربوط به برنامه‌نویس‌ها باشد.}

همین که سوار شد، اولین سوال یک جور فرا فکنی بود. در حالی که داشتم با
ضبط‌صوت ور می‌رفتم از لینوس پرسیدم \dbquote{اطرافیان‌ات هم همه اهل
  فنّآوری هستند؟}

جواب داد \dbquote{نه. اکثرا روزنامه‌نگار هستند} و اضافه کرد که \dbquote{به همین دلیل
  می‌دانم چه وازده‌هایی هستید.}

می‌دانست که با این جواب نمی‌تواند در برود.

جواب دادم : \dbquote{آه! پس تو از یک خانواده وازده هستی؟}

بهترین برنامه‌نویس جهان آن قدر شدید خنده‌اش گرفت که یک قلپ کوکا به پشت
گردن عکاس / راننده من پاشید. قرمز شد. این می‌توانست شروع یک بعد از ظهر
به یاد ماندنی باشد.

جریان پیچیده‌تر هم شد. فنلاندی‌ها تعصب خاصی نسبت به سوناهای‌شان دارند و
این اولین بازدید لینوس از یک سونای فنلاندی در طول سه سال اخیر
بود. فوق‌ستاره رنگ‌پریده و لخت با عینک‌هایی که بخار گرفته بودند روی
بالاترین پله سونا نشسته بود و در حالی که موی خیس‌اش روی پیشانی‌اش آمده
بود، عرق از فرق سرش به سمت چیزی می‌ریخت که من با کمی بدجنسی آن را
\dbquote{کلنگ} می‌نامیدم. اطراف او پر از آد‌م‌های خودپسند و آفتاب‌گرفته‌ای بود که
درباره چیزهای بی‌اهمیت بحث می‌کردند. او به نظر فراتر از اطرافیان‌اش
می‌رسید و با اشتیاق در حال توضیح دادن درباره خواص اثبات شده سونا
بود. می‌توانستید لبخند نشاط را روی صورت‌اش ببینید.

نظر من این است که در بیشتر موارد، مردم در سیلیکون‌ولی شادتر از هر جای
دیگری هستند چون آن‌ها در پشت میز فرمان انقلاب اقتصادی نشسته‌اند. از این
مهمتر اینکه آن‌ها همه پولدار هستند، چه نیو ولی و چه اولد
ولی\LFootnote{New Valley and Old Valley}. اما هیچ وقت نمی‌بینید کسی در
این‌جا بخندد، حداقل خارج از حصار دفترش.

اولین خواسته معتبرترین افراد در فنّآوری - و حتی آن‌هایی که اعتباری
ندارند - این است که شما متوجه شوید چقدر استثنایی هستند و این که بفهمید
آن‌ها یکی از مهمترین بازیگران ماموریتی هستند که حتی از ماموریت برقراری
صلح جهانی هم مهمتر است. این صحبت درباره لینوس صدق نمی‌کرد. در حقیقت عدم
خودپسندی لینوس باعث می‌شد سرآمد جمع گزافه‌گوی سیلیکون‌ولی باشد. بالاتر از
میلیاردهای شاغل در فنّآوری‌های بالا. او بیشتر از یک گوزن‌شمالی که در
روشنایی‌های شهر گیر کرده باشد، شبیه به یک آدم فضایی بود که به زمین آمده
تا غیر عقلانی بودن روش‌هایی که برگزیده‌ایم را به ما گوشزد کند.

و احساس من این بود که چندان هم موفق نشده است.

لینوس قبلا به من گفته بود که یکی از بخش‌های مهم مراسم سونای فنلاندی این
است که بعد از سونا بنشینیم و در حال نوشیدن آب‌جو درباره مسايل جهان گپ
بزنیم. برای کسب آمادگی، چند قوطی فوسترز\RFootnote{\lr{Fosters} یک مارک
  آبجو} را در بوته‌ها مخفی کرده بودیم. قوطی‌ها را پیدا کردیم و در یک
حوضچه آب گرم \dbquote{ساکت} نشستیم و در حالی که عکاس عکس می‌گرفت، مشغول
نوشیدن شدیم. کشف کردم که لینوس بر خلاف انتظار من، اطلاعات بسیار خوبی
درباره تاریخ اقتصاد آمریکا و سیاست‌های جهانی دارد. به نظر او اگر شرکت‌ها
و سیاست‌مداران آمریکایی، شیوه میانجی‌گرایانه سیاستمداران اروپایی‌ را پیش‌
می‌گرفتند، وضع بهتری در جهان داشتند. عینک‌اش را در آب گرم فرو کرد تا
تمیز شود و توضیح داد که در حقیقت نیازی به عینک ندارد ولی از دوره بلوغ
با این تصور که عینک باعث می‌شود دماغ‌اش کوچک‌تر به نظر برسد، از آن
استفاده کرده است. در همین موقع یک کارمند زن با لباس کامل به کنار حوضچه
آب گرم ما می‌آید و دستور می‌دهد که آب‌جوهایمان را تحویل دهیم چون نوشیدن
آن‌ها در این محیط ممنوع است.

تنها گزینه باقی مانده، دوش گرفتن،‌ لباس پوشیدن و پیدا کردن یک کافه برای
ادامه گفت‌و‌گو است. بیشترین کسانی که در سیلیکون‌ولی می‌بینید، بسیار شیفته
خودشان هستند. آن‌ها موقع حرف زدن آن قدر جدی روی شرکت خود یا محصول
فوق‌العاده‌ای که در حال تولید کردنش هستند یا صنعت مورد علاقه‌شان تمرکز
می‌کنند که انگار هیچ چیز دیگری در دنیا وجود ندارد. هیچ کس نمی‌تواند حلقه
بی‌پایان صحبت ‌آن‌ها درباره خودشان را بشکند. اما ما آنجا در یک آبجو فروشی
کوچک زیر آفتاب نشسته بودیم و حین مزه‌مزه کردن
گادآوفول\LFootnote{Godawful}، با لینوس که مثل یک قناری مشغول اقرار
کردن به اعتیادش به موسیقی راک کلاسیک و دین کونتز\LFootnote{Dean
  Koontz}، عشق‌اش به کمدی‌های احمقانه تلویزیونی و اسرار خانوادگی بود،
گپ‌ مي‌زدیم.

و او هیچ علاقه‌ای نداشت که به حلقه صاحبان پول و قدرت وارد شود. از او
پرسیدم که در یک ملاقات فرضی دوست دارد چه چیزی به بیل‌گیتس بگوید و جواب
داد که اصولا علاقه‌ای به این ملاقات ندارد. می‌گوید: \dbquote{نقطه اتصال
  چندانی با هم نداریم، من هیچ علاقه‌ای به چیزی که او در آن بهترین در
  دنیا است، ندارم و او هم هیچ علاقه‌ای به چیزی ندارد که ممکن است من یکی
  از بهترین‌های آن در دنیا باشم. من نمی‌توانم در مورد تجارت به او
  توصیه‌ای بکنم و او هم درباره فنّآوری، توصیه‌ای برای من ندارد.}

در مسیر جاده‌ای کوهستانی که از آن به سانتا کلارا برمی‌گشتیم، یک جیپ
چروکی سیاه خودش را به کنار ماشین ما رساند و مسافرانش فریاد زدند
\dbquote{هی لینوس!} و برای گرفتن یک عکس یادگاری از قهرمان‌شان که در
صندلی پشت یک موستانگ روباز، در باد لبخند می‌زند، دوربینی یک‌ بار مصرف
بیرون آوردند.

هفته بعد درست موقع حمام به خانه آن‌ها رسیدم. تازه دختر بلوند یک ساله‌اش
را از وان حمام صید کرده بود و در حین صید دختر دوم، دنبال جایی می‌گشت تا
اولی را به زمین بگذارد. بچه اول را به من داد و جیغ بچه بلند شد. تاو که
در طول این مدت در اتاق کناری بود، برای کمک به فرزندش به اتاق دوید. زن
دوست‌داشتنی‌ای بود و یک بوته خار هم روی بازویش خالکوبی کرده بود. چند
دقیقه دیگر همگی مشغول خواندن کتاب‌های کودکانه سوئدی و انگلیسی بودیم تا
بچه‌ها به خواب بروند. بعد از اینکه بچه‌ها خوابیدند همگی به پارکینگ رفتیم
و شروع به باز کردن بسته‌هایی کردیم که هنوز فرصت نشده بود کسی بازشان
کند. توروالدز بدون اینکه برخورنده باشد، دائما توضیح می‌داد که
\dbquote{غیرممکن است بشود در سیلیکون ولی از پس هزینه یک خانه واقعی با
  یک حیاط خلوت واقعی برآمد.}

در آخر کار هم جی لنو\LFootnote{Jay Leno} نگاه کردیم و مشغول خالی کردن
قوطی‌های گینس\RFootnote{\lr{Guinness} یک مارک عالی آبجوی تیره ایرلندی}
شدیم. این همان زمانی بود که احساس کردم باید از این ماجرا یک کتاب در
بیاوریم.
\end{journal}

\section{بخش پنجم}
و من برای چهار سال پشت کامپیوتر نشستم. 

بله!‌ مدرسه هم می‌رفتم: دبیرستان نورسن\LFootnote{Norssen} که بین پنج
دبیرستان سوئدی زبان هلسینکی، مرکزیت داشت و از همه به خانه من نزدیک‌تر
بود. ریاضی و فیزیک جالب و به همین دلیل راحت بودند. اما همین که درسی به
حفظ ‌کردن مرتبط می‌شد، کل اشتیاق من به آن مبحث از بین می‌رفت. به همین
دلیل تاریخ تا وقتی درباره زمان جنگ هستینگز\LFootnote{Battle of
  Hastings} بود جذابیتی نداشت، اما وقتی کار به عوامل اقتصادی موثر بر
کشورها می‌رسید، مساله جالب می‌شد. منظورم این است که واقعاً برای چه کسی
مهم است که در بنگلادش چند نفر زندگی می‌کنند؟ البته حالا که به مساله فکر
می‌کنم می‌بینم که برای خیلی‌ها ممکن است مهم باشد. نکته این است که برای من
خیال بافی نکردن درباره کامپیوترها سر کلاسی که بحث در مورد بادهای موسمی
یا دلایل بادهای موسمی بود، بسیار راحت‌تر بود از کلاسی که در آن درباره
آمار صحبت می‌شد.

ورزش کلا یک پرونده جدا داشت. اینکه فاش کنم من ورزشکارترین فرد در شبه
جزیره اسکاندیناوی نبوده‌ام، مطمئنا هیچ‌وقت خبرساز نخواهد شد. چه باور
بکنید چه باور نکنید، آن روزها لاغر بودم. شرکت در تمرینات ژیمناستیک
قابل قبول بود ولی وقتی کار به فوتبال یا هاکی روی یخ می‌رسید باید کلاس‌ها
را جیم‌ می‌شدم.

این مساله در کارنامه‌هایم هم خودش را نشان می‌داد. در فنلاند نمر‌ها از
چهار تا ده هستند. من معمولا چند ده و تعدادی هم نه از ریاضی، فیزیک،
زیست و بقیه درس‌ها داشتم اما در ورزش معمولا هفت می‌گرفتم. یک بار هم شش
گرفتم. در نجاری هم شش شدم. این درس هم جزو نقاط ضعف من بود. بقیه
دوستانم از آن کلاس‌ها یک کمد زیبا یا چند ابزار نجاری به یادگار نگه‌
داشته‌اند. تمام چیزی که من دارم، چند تراشه فرورفته در انگشت شستم است و
هنوز هم که هنوز است، آن‌جا هستند. لازم است همین‌جا بگویم که تاب‌های زیبای
موجود در حیات پشتی که دخترم بیشتر وقت‌اش را روی آن‌ها می‌گذراند، ساخته
پدر زنم هستند.

دبیرستان من یکی از آن مدارس ويژه بچه‌های باهوش یا عقب‌افتاده که در
شهرهای آمریکا وجود دارند، نبود. این جور مدرسه‌ها عملا خلاف شیوه‌ای هستند
که فنلاند بر اساس آن اداره می‌شود. در فنلاند مدارس بچه‌های باهوش یا ضعیف
را از یکدیگر جدا نمی‌کنند اما هر مدرسه یک موضوع منحصر به فرد دارد که
گذراندن آن الزامی نیست، ولی در مدارس دیگر هم نمی‌شود پیدایش کرد. در
مدرسه نورسن، این درس، لاتین بود. آموختن لاتین جالب بود. بسیار جالب‌تر
از فنلاندی یا انگلیسی.

متاسفم که یک زبان مرده است. دوست داشتم رفقایی می‌بودند که دور هم جمع
بشویم و به لاتین جک تعریف کنیم یا به لاتین درباره سیستم‌های عامل گپ
بزنیم.

وقت‌گذرانی در کافی‌شاپ نزدیک مدرسه هم مفرح بود. خیلی از بچه‌ها اینجا دور
هم جمع می‌شدند. بخصوص آن‌ تیپ‌ بچه‌هایی که اهل مخفی شدن پشت دیوار مدرسه و
سیگار دود کردن نبودند. اگر از کلاس ورزش جیم می‌شدید، می‌توانستید به این
کافی‌شاپ بروید. همینطور اگر یک ساعتی بین دو کلاس وقت خالی داشتید؛ چیزی
که گاهی پیش می‌آمد.

این کافی‌شاپ محل اجتماع گیک‌ها هم بود. تنها کافی‌شاپ ی هم بود که
دانش‌آموزان می‌توانستند از آن خرید کنند و پول حساب خود را هر وقت که
داشتند بپردازند. یعنی می‌توانستید چیزی که می‌خواهید را سفارش بدهید و
مسوولین یک فهرست از غذا و نوشیدنی‌های شما نگه می‌داشتند و بعد هر وقت که
پولی گیرتان می‌آمد، می‌توانستید حساب خود را صاف کنید. با اطلاعی که از
شیفتگی فنلاندی‌ها به تکنولوژی دارم، مطمئن هستم که اگر کافی‌شاپ هنوز سر
جایش باشد، بانک‌های اطلاعاتی‌اش کامپیوتری شده‌اند.

سفارش من همیشه یکسان بود: یک کولا و یک دونات. 

جوان و بی‌توجه به غذاهای سالم. 

در کل من در مدرسه بهتر از خواهرم سارا بودم. او اجتماعی‌تر، قابل
نگاه‌کردن‌تر و مهربان‌تر بود و باید اضافه کنم که تقبل کرده این کتاب را به
سوئدی ترجمه کند. اما در نهایت او با نوشتن مقالاتی بیشتر، در مدرسه از
من جلو زد. علاقمندی‌های من محدودتر بودند. همه من را به عنوان
\dbquote{مرد ریاضی} می‌شناختند.

در حقیقت تنها باری که دختری را به خانه آوردم موقعی بود که از من
درخواست می‌کردند به آن‌ها درس بدهم. البته زیاد هم پیش‌نیامد و هیچ وقت هم
من پیشنهاد دهنده نبودم، ولی پدرم همیشه می‌گفت که آن‌ها دنبال چیزی بیشتر
از درس ریاضی هستند (به نظر او آن‌ها طرفدار معادله دماغ باشکوه = مردانگی
باشکوه بودند). به هرحال اگر آن‌ها دنبال دوستی با مرد ریاضی بودند، مرد
ریاضی‌شان چندان علاقه‌ای به جریان نشان نمی‌داد. منظورم این است که من هیچ
وقت نفهمیدم منظور آن‌ها از \dbquote{دوستی صمیمی‌تر} چیست. من گاهی از گربه همسایه
نگهداری کرده بودم و \dbquote{دوستی صمیمی} به نظرم چیز چندان خاصی نبود.

بله! بدون شک من یک گیک بودم. برو برگرد هم ندارد. این قبل از دوره‌ای بود
که گیک‌بودن سکسی حساب شود. البته به نظر من گیک بودن سکسی نیست ولی به
نوعی جذاب است. به هرحال چیزی که من بودم، یک پسر گیک خجالتی بود؛ البته
اگر گفتن این حرف،زائد نباشد.

اینجا بودیم که من می‌توانستم جلوی کامپیوتر بنشینم و کاملا خوشحال باشم. 

برای فارغ‌التحصیلی از مدرسه در فنلاند، باید یک کلاه پشمالوی سفید با یک
نوار سیاه بپوشید. جشنی است که طی آن دیپلم‌تان را می‌دهند و بعد با کلی
شامپاین و گل و کیک به خانه می‌آیید. یک جشن هم برای کل کلاس در یک
رستوران محلی برگزار می‌شود. من هم همین برنامه‌ها را داشتم و احتمالا بهم
خوش گذشته است، ولی چیز چندانی از آن جریان یادم نیست. اما در مورد
مشخصات کامپیوتر مبتنی بر ۶۸۰۰۸ی که داشتم بپرسید و می‌توانم مثل بلبل همه
آن‌ را از حفظ بگویم.

\section{بخش ششم}
اولین سال من در دانشگاه، بسیار پر حاصل بود. توانستم تمام امتیازهایی که
باید در یک سال کسب شوند - که در سیستم فنلاندی \dbquote{هفته‌های کاری}
نامیده می‌شوند - را با موفقیت به دست بیاورم. همان سال تنها سالی بود که
این امر اتفاق افتاد. شاید به خاطر هیجان محیط جدید بود یا امکان عمیق
شدن در موضوعات مورد علاقه یا حتی این موضوع که درس‌ خواندن‌ برایم راحت‌تر
بود از تبدیل شدن به یک حیوان اجتماعی و بیرون رفتن با دوستان. نمی‌دانم
موفقیت سال اول دانشگاه را باید تقصیر چه کسی بدانم، ولی به شما اطمینان
می‌دهم که دیگر تکرار نشد. موفقیت دانشگاهی من از همان سال به بعد به سرعت
رو به قهقرا رفت.

در آن مرحله هنوز رشته اصلی‌ام را انتخاب نکرده بودم. در نهایت کامپیوتر
را به عنوان رشته اصلی و فیزیک و ریاضی را به عنوان رشته‌های فرعی انتخاب
کردم. یکی از مشکلات این بود که در کل دانشگاه هلسینکی فقط یک دانشجوی
سوئدی زبان دیگر بود که کامپیوتر را به عنوان رشته اصلی برگزیده بود؛
لارس ویرزنیوس\LFootnote{Lars Wirzenius}. ما دو نفر
اسپکتروم را تاسیس کردیم که عبارت بود از سازمان اجتماعی دانشجویان سوئدی
زبان رشته‌ علوم که در نهایت به یک سازمان مفرح تبدیل شد. این سازمان
تشکیل شده بود از دانشجویان علوم پایه مثل فیزیک و شیمی و این یک معنا
بیشتر نداشت: همه اعضا پسر بودند.

اما ما اتاق باشگاه‌مان را برای استفاده در اختیار سازمان همتای خود یعنی
سازمان اجتماعی دانشجویان سوئدی زبان علوم نرم (مثل زیست و روان‌شناسی) هم
قرار می‌دادیم. با این روش این امکان برای ما فراهم می‌شد با دخترها رابطه
داشته باشیم که البته برای بعضی از ما این رابطه بسیار ناشیانه
بود. قبول!‌ برای همه ما!

اسپکتروم دردسرهای همه سازمان‌های اخوت به سبک آمریکایی را داشت ولی نکته
این بود که در آن مجبور نبودید با کسانی که به علوم بی‌علاقه بودند، زندگی
و تفریح کنید. ما چهارشنبه‌ شب‌ها همدیگر را می‌دیدیم و در همین جلسات بود
که من فرق آب‌جوی انگلیسی و پیلسن را یاد گرفتم. گاهی هم مسابقه ودکا خوری
می‌دادیم. البته این اتفاق معمولا در سال‌های آخر دانشگاه می‌افتاد که برای
من خیلی هم طولانی بودند: من هشت سال در دانشگاه درس خواندم و در نهایت
هم به چیزی بیشتر از لیسانس نرسیدم (البته دکترای افتخاری‌ای که در ژوئن
۲۰۰۰ به من داده شده را حساب نمی‌کنم).

به هرحال سال اول دانشگاه برای من خاطره‌ای مبهم است از سفرهای درون‌شهری
با مترو از اتاق خواب که پر بود از کتاب و قطعات کامپیوتر به کلاس‌های درس
و برعکس. من روی تخت دراز می‌کشیدم و سه گانه علمی تخیلی دوگلاس
آدامز\RFootnote{\lr{Douglas Adams} - منظور لینوس سه گانه راهنمای
  مسافران مجانی کهکشان است که شدیدا خواندنش به هر گیک توصیه می شود. یک
  رمان طنز علمی تخیلی که جلد اول آن هم به فارسی ترجمه شده} را
می‌خواندم. بعد آن را زمین می‌گذاشتم و به سراغ کتاب درسی فیزیک
می‌رفتم. بعد از تخت بیرون می‌آمدم و سراغ کامپیوتر می‌رفتم و برنامه یک
بازی جدید را می‌نوشتم. آشپزخانه درست بیرون اتاق من بود و گاهی برای کمی
قهوه و چیپس ذرت، سری به آن‌جا می‌زدم.

شاید خواهرتان جایی همان دور و بر بود. شاید هم با دوستانش بیرون رفته
بود. این امکان هم وجود داشت که این روزها برای زندگی پیش پدر رفته
باشد. شاید مادر در خانه بود و شاید هم با دوستان روزنامه‌نگارش بیرون
رفته بود. شاید هم دوستی به خانه‌تان آمده بود و با هم در آشپزخانه نشسته
بودید و چای پشت چای می‌نوشیدید و در ام.تی.وی برنامه بویس و
باتهد\LFootnote{Bevis and Butthead} را به انگلیسی نگاه می‌کردید و به
این فکر می‌کردید که برای بازی اسنوکر بیرون بروید، ولی هوا بیش از اندازه
سرد بود.

و خوشبختانه در این دوره از زندگی دیگر ورزشی در کار نبود.

ورزش مال سال بعد است. زمانی که ارتش فنلاند همه مردهای کشور را به
سربازی فرا می‌خواند. خیلی‌ها درست بعد از دبیرستان به سربازی می‌روند اما
من احساس می‌کردم که بهتر است قبل از رفتن به سربازی، سال اول دانشگاه را
تمام کنم.

در فنلاند می‌توانید انتخاب کنید: یا هشت ماه سربازی اجباری یا یک‌سال
خدمات اجتماعی. البته اگر دلیل دینی یا دلیل موجه دیگری ارائه دهید که
نباید این کارها را انجام دهید، می‌توانید از هر دو معاف شوید. من از این
دلایل نداشتم و خدمات اجتماعی هم برایم انتخاب مناسبی نبود.

دلیل اینکه خدمات اجتماعی را انتخاب نکردم این نبود که علاقه‌ای به خدمت
به نوع بشر نداشتم. علت اصلی احتمالا این بود که می‌ترسیدم وظایف مربوط به
خدمات اجتماعی بیشتر از خدمت نظام، حوصله سربر باشند. از هر کسی که به
جای سربازی،‌ خدمات اجتماعی را انتخاب کرده است بپرسید و او به شما خواهد
گفت که اگر برنامه خاصی برای خدمت اجتماعی نداشته باشید، جایی که به شکل
اتفاقی به شما خواهد افتاد، جای جذابی نخواهد بود. من نمی‌توانستم دلیل
اعتقادی بیاورم که نباید به سربازی بروم. در حقیقت من از نظر اعتقادی به
این باور دارم که وقتی کار به زور کشیده می‌شود و چاره‌ای هم نیست، کاربرد
اسلحه یا کشتن آدم‌ها می‌تواند لازم باشد.

تازه اگر سربازی را انتخاب می‌کردید، بازهم باید بین دو حالت انتخاب
می‌کردید. می‌توانستید هشت ماه به عنوان سرباز ساده خدمت کنید یا به مدت
یازده ماه درجه‌دار باشید. نظر من این بود که با وجود ۱۲۹۶۰۰ دقیقه اضافی،
درجه‌داری انتخاب جالب تری است. تازه این امکان هم وجود داشت که چیزی یاد
بگیرم.

این گونه شد که قهرمان (آن زمان‌ها) ۵۵ کیلویی شما، به ستوان دوم ذخیره
ارتش فنلاند تبدیل شد. کار من کنترل آتش توپخانه بود. البته ربطی به
مهندسی موشک ندارد. مختصات توپ‌ها به شما داده می‌شود. شما نقشه را
می‌خوانید و کشف می‌کنید که کجا هستید و بعد یک مثلث بین خودتان و توپ و
هدف ترسیم می‌کنید. کمی عملیات ریاضی انجام می‌دهد و بعد با یک خط تلفن که
خودتان در سیم‌کشی‌اش شریک بوده‌اید، به توپ‌ها می‌گویید که با چه زاویه‌ای به
کدام طرف شلیک کنند.

یادم هست که قبل از رفتن به ارتش در این باره که با چه چیزی روبرو خواهم
شد، خیلی مضطرب بودم. بعضی‌ها برادر بزرگ‌تر یا دوستی را دارند که به
سربازی رفته باشد و حین گفت و گو با او پیشاپیش از اینکه چه چیزی در
انتظار آن‌ها است، مطلع می‌شوند اما در مورد من هیچ کس نبود که بتواند
بگوید در ارتش چه اتفاقی خواهد افتاد. همه می‌دانند که ارتش در کل جای
جالبی نیست. این مساله را همه کسانی که درباره ارتش صحبت می‌کنند،
می‌گویند. ولی از آنجایی که من اصلا نمی‌دانستم آن‌جا چه می‌گذرد، استرس
داشتم. این همان احساسی است که وقتی به این فکر می‌کنم که مردم قرار است
این کتاب را بخوانند، در من به وجود می‌آید.

سخت‌ترین دوره ارتش، موقعی بود که باید با کابل‌هایی که به نظر چند تن وزن
داشتند، در جنگل لاپلند\LFootnote{Lapland} پیاده روی
می‌کردیم. من حقیقتاً فکر می‌کنم آن کابل‌ها چند تنی وزن داشتند. قبل از
رسیدن به آموزشگاه، دستور می‌دادند که با یک حلقه کابل در گردن و دو حلقه
کابل بر پشت، بدویم. باید تقریبا پانزده کیلومتر لعنتی را می‌دویدیم. در
مواقع دیگر هم باید منتظر می‌ماندیم تا شاید چیزی پیش بیاید.

گاهی هم باید مسیر طولانی تا محلی که قرار بود اردو بزنیم را اسکی
می‌کردیم. این همان موقعی بود که فهمیدم اگر خدا می‌خواست آدم‌ها اسکی کنند،
به جای پا به آن‌ها صفحات پهنی از جنس فایبرگلاس می‌داد. البته یک لحظه صبر
کنید، باید بگویم که الزاما به خدا باور ندارم.

قبل از غذا خوردن، باید چادر می‌زدیم و آتش را به راه می‌کردیم. گرسنه و
سرما زده و خسته بودیم چون دو شبانه روز بود که نخوابیده بودیم. بعضی‌ها
هستند که برای شرکت در برنامه‌های منجر به این \dbquote{تجربیات شخصیت
  ساز} کلی پول خرج می‌کنند. آن‌ها فقط کافی است تا در ارتش فنلاند ثبت نام
کنند.

این ماراتن‌های اسکی، زیاد نبودند ولی به هرحال بودند. طبق محاسبات من،‌ در
طول یازده ماه حضورم در ارتش، بیش از ۱۰۰ روز را در جنگل‌ها
خوابیده‌ام. فنلاند جنگل‌های زیادی دارد و حدود ۷۰ درصد کشور از جنگل
پوشیده شده. احساس می‌کنم در طول ارتش کل این جنگل‌ها را دیده‌ام.

به عنوان یک درجه‌دار، کار من این بود که در یک گروه پنج نفره، فرمانده
کنترل توپخانه باشم. این وظیفه به این معنا بود که باید درک می‌کردم چیزها
چطور کار می‌کنند و بعد تظاهر می‌کردم که کارکرد آن‌ها پیچیده‌تر از آنی است
که واقعاً هست. این کار جالبی نبود و من هم فرمانده خوبی نبودم. بدون شک
در دستور دادن ضعیف بودم. خوب دستور می‌گیرم - رمز کار این است که دستورات
را شخصی برداشت نکنید - ولی این احساس را ندارم که ماموریت ما در این
جهان، انجام به نحو احسن هر چیزی است.

لااقل آن روزها نبود.

گفتم که لاپلند چقدر سرد بود؟

حالا که به جریان فکر می‌کنم می‌بینم که آن روزها از آن تجربیات متنفر بودم
ولی حضور در ارتش یکی از چیزهایی بود که همین که تمام می‌شوند، احساس
می‌کنید تجربه خوبی داشته‌اید.

در عین حال حالا این امکان را دارم که تا آخر عمر، تقریبا با تمام مردان
فنلاندی، در مورد موضوع مشترکی صحبت کنم. بعضی می‌گویند اصولا دلیل خدمت
اجباری این است که مردان فنلاندی تا آخر عمرشان موضوعی داشته باشند که
حین آبجو خوردن درباره‌اش گپ بزنند. همه مردان فنلاند یک بدبختی مشترک
داشته‌اند. آن‌ها از ارتش متنفرند ولی خوشحالند که می‌توانند درباره‌اش صحبت
کنند.

\section{بخش هفتم}
حالا که بحث در مورد فنلاند است، اجازه بدهید کمی بیشتر توضیح بدهم. ما
احتمالا بیشترین گوزن‌شمالی جهان را داریم. همچنین فنلاند پر است از
طرفداران مشروبات الکلی و رقص تانگو. یک زمستان که در فنلاند بمانید،
ریشه تمام شادنوشی‌ها را کشف می‌کنید اما طرفداران تانگو هیچ عذر موجهی
ندارند. نکته خوب این است که آن‌ها در شهرهای کوچک متمرکز شده‌اند و احتمالا
کمی دارد به آن‌ها بر بخورید.

یک تحقیق جدید نشان داده که مردان فنلاندی، در اروپا بیشترین نیروی
مردانگی را دارند. یا به خاطر گوشت گوزن‌شمالی است یا به خاطر آن همه ساعت
در سونا ماندن. اینجا کشوری است که تعداد سوناهایش بیشتر از تعداد
خودروها است. کسی نمی‌داند این مذهب چطور به وجود آمده ولی حداقل در بعضی
مناطق کشور، اول سونا را می‌سازند و بعد خانه را. خیلی از مجتمع‌های مسکونی
در طبقه اول یا آخر یک سونا دارند که در هر ساعت از هفته، مثلا پنجشنبه‌ها
ساعت ۷ تا ۸ شب، به یک خانواده اختصاص دارد. پنجشنبه و جمعه روزهای خاص
سونا است. با اینکار احتمال لخت دیدن همسایه‌ها کمتر می‌شود. چند وقت قبل
یک کتاب راهنمای توریست‌های انگلیسی زبان در فنلاند را می‌خواندم و دیدم که
به شکل مفصلی توضیح داده که فنلاندی‌ها در سونا سکس نمی‌کنند و اگر بفهمند
چنین اتفاقی افتاده یا این یکی از فانتزی‌های مرسوم توریست‌ها در فنلاند
است، ناراحت می‌شوند. وقتی کتاب را خواندم نمی‌توانستم جلوی خنده‌ام را
بگیرم چون سونا در فنلاند بخشی از خانه است و درست مثل این بود که کتاب
در مورد سکس در کف آشپزخانه تذکر بدهد. به نظر من که مساله این قدرها هم
اهمیت ندارد. در مناطق دور افتاده‌تر بچه‌ها در سونا به دنیا می‌آیند چون
تنها جایی است که آب گرم دارد. بر اساس بعضی سنن، سونا محل مرگ هم
هست. البته این قواعد در خانواده من که نگرشی آسان‌گیر به همه چیز داشتند،
جایی نداشت.

خصیصه‌های دیگری هم هست که فنلاندی‌ها را از بقیه نژاد بشر متمایز
می‌کند. مثلا سنت سکوت. هیچ کس زیاد حرف نمی‌زند. ممکن است مردم فقط کنار
هم بایستند و حرف نزنند. البته این قانون هم درباره خانواده من که گاهی
آن‌ها را \dbquote{وصله ناجور} می‌خوانم صدق نمی‌کند.

فنلاندی‌ها در مقابل مشکلات برخوردی رواقی‌ دارند. تحمل بی‌صدای مصائب و
قدرگرایی چیزی بوده که به ما کمک کرده طی سلطه روسیه، نتیجه سلسله جنگ‌های
خونبار و همچنین در مقابل آب‌هوای مزخرف دوام بیاوریم. نویسنده آلمانی
برتولت برشت در طول جنگ جهانی دوم مدت کوتاهی را در فنلاند زندگی کرده و
نقل قول معروفش در مورد مشتریان کافه کنار ایستگاه راه آهن گفته است که
\dbquote{به دو زبان ساکت می‌مانند.} او در اولین فرصت ممکن است از طریق
ولادیوستک\LFootnote{Vlodivostock} کشور را به مقصد آمریکا ترک کرد.

حتی امروز هم اگر وارد کافه‌ای در یک شهر فنلاندی شوید - بخصوص شهرهای
کوچک‌تر - به احتمال زیاد با چهره‌های آهنینی روبرو خواهید شد که تنها
نشسته‌اند و خیره به فضا نگاه می‌کنند. مردم فنلاند به خلوت یکدیگر بسیار
احترام می‌گذارند - این یک خاصیت دیگر است - و در نتیجه هیچ‌ وقت به ذهن
کسی خطور نمی‌کند که سر میز یک غریبه برود و صحبتی را با او شروع
کند. پیچیدگی مساله اینجا است که فنلاندی‌ها آدم‌های خوش مشربی هستند، ولی
افراد کمی فرصت می‌کنند این خوش مشربی را کشف کنند.

درک می‌کنم که وضع در جشن‌هایی که در بارهای همجنس‌گرایان زن برگزار می‌شود،
کاملا فرق می‌کند.

از آنجایی که فنلاندی‌ها از صحبت‌های رو در رو بیزارند، این کشور بازار فوق
العاده‌ای است برای گوشی‌های تلفن همراه. ما با اشتیاقی که در هیچ کشور
دیگری دیده نمی‌شود، سراغ ابزارهای جدید می‌رویم. حالا که درست فکر می‌کنم
می‌بینم نمی‌توان در این باره که کدام کشور سرانه گوزن‌شمالی بالاتری دارد
به راحتی نظر داد - ممکن است این عنوان به نروژ برسد - ولی در این باره
که کدام کشور بیشترین نسبت گوشی همراه نسبت به مردان، زنان و کودکان را
دارد، هیچ شکی وجود ندارد. حتی این صحبت در فنلاند وجود دارد که سیم‌کارت
را موقع تولد نوزاد به بدن‌اش پیوند بزنند!

موارد کاربرد گوشی‌های تلفن همراه هم فراوان است. فنلاندی‌ها دائما به
یکدیگر پیام کوتاه می‌فرستند یا برای تقلب در امتحانات به گوشی‌های شان
وابسته هستند (سوال را برای دوستی بفرستید و منتظر جواب بمانید). ما در
گوشی‌ها از ماشین حسابی استفاده می‌کنیم که بسیاری از آمریکایی‌ها اصولا
نمی‌دانند که وجود دارد. برای قدم بعدی فقط همین مانده که شماره فردی که
در آن طرف کافه تنها نشسته‌ است را بگیری و با او کمی گپ بزنید. به نظرم
می‌توان ادعا کرد که پس از اختراع سونا، هیچ چیز مثل موفقیت نوکیا، چهره
فنلاند را دگرگون نکرده بود.

عمومیت یافتن تلفن‌های همراه در فنلاند جای تعجب ندارد. این کشور تجربه
پذیرش سریع و گرم فنّآوری‌های جدید را دارد. مثلا بر خلاف هر جای دیگری در
کره زمین، فنلاند کشوری است که مردمش همه عملیات بانکی و پرداخت قبوض خود
را از طریق بانکداری الکترونیکی انجام می‌دهند و تازه نه این بانک‌داری شبه
الکترونیکی که در آمریکا هست. فنلاند همچنین دارای بالاترین سرانه
استفاده از اینترنت است. بعضی‌ها این تکنولوژی دوستی را به سیستم آموزشی
فنلاند ربط می‌دهند. فنلاند بالاترین نرخ باسوادی جهان را دارد و
دانشگاه‌ها در آن رایگانند و به همین سبب است که بسیاری از دانشجویان پنج،
شش یا حتی هفت سال در دانشگاه می‌مانند. البته در مورد من هشت سال! به
هرحال با گذراندن این همه از عمر در دانشگاه، چاره‌ای نیست جز اینکه چیزی
یاد بگیرید. بعضی‌های دیگر هم معتقد هستند که این سطح بالای تکنولوژی،
مربوط می‌شود به زیرساخت‌های قدرتمندی که در زمان جنگ با روس‌ها برای توسعه
صنایع کشتی‌ سازی ساخته شد. در نهایت عده‌ای هم می‌گویند این مساله مربوط
می‌شود به جمعیتی که گاهی به شکل غیرقابل تحملی، یکنواخت و یکسان است.

\begin{journal}
لینوس و من پشت میز غذاخوری نشسته‌ایم. به تازگی از یک مسابقه
اتوموبیل‌رانی برگشته‌ایم. تاو مشغول تمام کردن سالاد است و پاتریشا و
دانیلا سر کتابی که من برای یکی از آن‌ها خریده‌ام، دعوا می‌کنند. عروسک
پنگوئن کنار میز را کمی نوازش می‌کنم و با کنار زدن یک ظرف بزرگ کره
بادام‌زمینی روی میز برای ضبط صوتم که تازه روشن‌اش کرده‌ام جا باز
می‌کنم. از لینوس می‌خواهم درباره کودکی‌اش صحبت کند.

با لحنی یکنواخت می‌گوید: \dbquote{در واقع چیز زیادی از بچگی‌ام یادم نمی‌آید.}

\dbquote{چطور ممکنه؟ فقط چند سال قبل بود!}

\dbquote{از تاو بپرس. من در به یاد آوردن اسم‌ها یا قیافه‌ها یا کارهایی
  که کرده‌ام خیلی ضعیفم. حتی شماره تلفن خانه را از او می‌پرسم. قواعد و
  شیوه تنظیم امور یادم می‌ماند ولی جزییات چیزها نه. جزییات بچگی‌هایم
  یادم نیست. یادم نیست وقایع چطور اتفاق افتاده‌اند یا وقتی بچه بودم چه
  فکرهایی می‌کردم.}

\dbquote{خب مثلا آیا دوستانی داشتی؟}

\dbquote{کم. هیچ وقت خیلی اجتماعی نبودم. الان خیلی خیلی بیشتر از دوران کودکی‌ام
  اجتماعی هستم.}

\dbquote{مثلا یادت می‌آید یک روز تعطیل از خواب بیدار شده باشی و با پدر و مادر و
  خواهرت جایی رفته باشی؟}

\dbquote{آن وقت‌ها پدر و مادرم از هم جدا شده بودند}

\dbquote{چند سال‌ات بود که از هم جدا شدند؟}

\dbquote{نمی‌دانم. شاید شش. شاید ده. یادم نیست.}

\dbquote{کریسمس چی؟ کریسمس را یادت هست؟}

\dbquote{آه بله. خاطرات مبهمی دارم از اینکه لباس می‌پوشیدیم و به خانه پدربزرگ
  پدریم در تورکو می‌رفتیم. در جشن شکرگزاری هم همینطور. به جز این چیزی
  یادم نیست.}

\dbquote{در مورد اولین کامپیوترت چی؟}

\dbquote{یک \lr{VIC-20} مشهور بود که پدربزرگ مادری‌ام خریده بود. کل‌اش در یک جعبه
  بود.}

\dbquote{جعبه‌اش بزرگ بود؟ مثلا اندازه جعبه یک جفت پوتین زمستانی؟}

\dbquote{تقریبا همان اندازه.}

\dbquote{و پدربزرگ‌ات چی؟ چیزی از او به یادت می‌آید؟}

\dbquote{او احتمالا نزدیک‌ترین فامیل به من بود ولی چیز زیادی یادم نیست... او
  کمی اضافه وزن داشت ولی چاق نبود. داشت طاس می‌شد و گوشه‌گیر بود. شبیه
  پروفسورهای کم‌حافظه. واقعا هم همینطور بود. من روی زانوهایش می‌نشستم و
  برنامه‌هایش را برایش تایپ می‌کردم.}

\dbquote{یادت هست که چه بویی داشت؟}

\dbquote{نه. این دیگر چه سوالی است؟}

\dbquote{پدربزرگ‌ها معمولا بوی مخصوصی دارند. عطرهای ارزان. شراب. سیگار. او چه بویی داشت؟}

\dbquote{نمی‌دانم. آن قدر با کامپیوتر مشغول بودم که متوجه بویی نمی‌شدم.}
\end{journal}
