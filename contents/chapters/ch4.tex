\chapter{فرش قرمز}\label{ch4}
\section{بخش یکم}
تولد نسخه \code{1.0} برای لینوکس فتح باب جدیدی بود به نام روابط
عمومی. اگر به من بود با همان شیوه قدیمی معرفی نسخه‌های جدید راضی
بودم. من یک ایمیل در گروه می‌زدم و می‌نوشتم که \dbquote{لینوکس نسخه
  \code{1.0} بیرون آمد. با آن ور بروید.} (البته نه دقیقا با این کلمات)

از نظر اکثر آدم‌ها تولد نسخه \code{1.0} موضوع مهمی بود و فکر می‌کردند که
باید آن را به دیگران معرفی کرد. در عین حال تعداد زیادی شرکت تجاری به
وجود آمده بودند که لینوکس را به دیگران می‌فروختند. برای آن‌ها نسخه
\code{1.0} نه از نظر فنی که از نظر روانی قدم بزرگی بود. من هم مخالف
این نظر نبودم. حقیقت این است که استفاده از نسخه \code{0.96} یک
سیستم‌عامل چندان جذاب نیست.

من هم طرفدار ارائه نسخه جدید بودم چون به معنای یک گام بزرگ رو به جلو
بود. همچنین ارائه این نسخه به من اجازه می‌داد که مدتی باگ‌زدایی را متوقف
کنم و برگردم بر سر توسعه سیستم. شرکت‌ها و جامعه لینوکس هم می‌خواستند که
ارائه این نسخه را با سر و صدا جشن بگیرند و توجه دیگران را به این
سیستم‌عامل جدید جلب کنند.

ما نیاز به یک استراتژی در روابط عمومی داشتیم. قرار نبود در این نمایش‌ها
من نقش محوری داشته باشم. من علاقه‌ای به نوشتن برای مطبوعات یا سخنرانی و
بازاریابی نداشتم و انجام این کارها، نظر جمع بود. افراد برای ایفای این
نقش‌ها داوطلب شدند. این همان روشی بود که خود لینوکس را هم به وجود آورده
بود و به نظر می‌رسید که به خوبی کار می‌کند.

لارس یکی از کسانی بود که ارائه نسخه یک لینوکس را به یک رویداد پر سر و
صدا تبدیل کرد. از نظر او و دیگران، دانشگاه بهترین جا برای ارائه این
نسخه بود. منطقی هم بود. اتاق خواب من برای اینکار خیلی کوچک بود و
برگزار شدن مراسم در آن‌جا باعث می‌شد رسانه‌ها و تبلیغات‌چی‌ها درباره هدف
لینوکس گمراه شوند. پس لارس داوطلب شد تا موضوع را با دانشگاه هماهنگ
کند. دانشکده علوم کامپیوتر دانشگاه هلسینکی به اندازه کافی کوچک بود که
او بتواند مستقیما با رییس دانشکده صحبت کند.

دانشکده علوم کامپیوتر دانشگاه هلسینکی از اینکه سالن اصلی دانشکده را در
اختیار ما قرار دهد تا مراسم معرفی نسخه یک لینوکس را در آن برگزار کنیم
بسیار هم خوشحال شد. چرا؟ واضح است. مگر یک دانشکده چندبار در سال فرصت
می‌کند خبری بسازد که در تلویزیون هم پوشش داده خواهد شد؟

من قبول کردم که سخنرانی بکنم. در مقایسه با سخنرانی در اد، کار سختی
نبود. البته حالا که فکر می‌کنم باید اعتراف کنم که این سخنرانی هم واقعا
سخت بود.

اولا چون که پدرم هم در جمع نشسته بود و ثانیا به این دلیل که سخنرانی در
تلویزیون فنلاند پخش می‌شد. این اولین باری بود که در تلویزیون نشان داده
می‌شدم. پدر و مادرم در بین حضار بودند. ولی مطمئن هستم که کنار هم ننشسته
بودند. تاو هم بود. این اولین باری بود که پدرم تاو را می‌دید پس برای من
آن جلسه چیزی بیشتر از معرفی نسخه \code{1.0} لینوکس بود. از آنجایی که
من تا آخرین لحظه مشغول آماده کردن سخنرانی و بررسی اسلایدهایم بودم،
متوجه نشدم که تاو و پدرم کی ملاقات کردند. احتمالا این دیدار باید حین
ورود به سالن بوده باشد. شاید هم از گوشه چشمم دیدم. نمی‌دانم.

در آن سخنرانی هم مثل اکثر سخنرانی‌هایی که در سال‌های بعد داشتم بیشتر از
آنکه در مورد فنّآوری حرف بزنم، در مورد جنبش بازمتن صحبت کردم. جلسه
خوبی بود. نظر بعضی از افراد دانشکده کامپیوتر در مورد لینوکس را هم
تغییر داد. قبل آن جلسه لینوکس چیزی بود که دانشکده کامپیوتر به آن
افتخار و تا حدی هم از آن حمایت می‌کرد. بعد از جلسه، افراد در مورد
لینوکس بسیار جدی‌تر حرف می‌زدند. هرچه باشد آن را در اخبار هم دیده بودند.

در طول سال‌های بعدی بعضی‌ها گفتند که دانشکده به دنبال کسب اعتبار از طریق
لینوکس است. به نظرم این طور نیست. دانشکده همیشه حامی خوبی بوده و حتی
شغلی به من داد که بتوانم طی آن، روی لینوکس کار کنم. این جریان متعلق به
اولین روزها است و در نتیجه هیچ‌کس نمی‌تواند بگوید که آ‌ن‌ها اینکار را
می‌کردند چون می‌دانستند که لینوکس روزی در جهان مشهور خواهد شد. اما به
هرحال آن‌ها هم از اینکه بخشی از مراسم باشکوه معرفی لینوکس باشند، خوشحال
بودند. روابط عمومی دانشگاه با این پروژه کلی پیشرفت کرد. می‌دانم که این
روزها دانشجویان سوئدی بیشتری در دانشکده‌ای هستند که همیشه زیر سایه
دانشگاه پلی‌تکنیک بوده.

غبطه پیروزی دیگران را خوردن بخشی از خصوصیات فنلاندی‌ها به حساب می‌آید و
همان طور که لینوکس بیشتر و بیشتر در سطح جهانی به موفقیت می‌رسید،‌ دائما
از من سوال می‌شد که آیا با حسادت هم‌دانشکده‌ای‌ها مشکلی دارم یا نه. عملا
خلاف این موضوع صادق بود؛ آن‌ها شدیدا حمایت می‌کردند. از همان مراحل اول
آن‌ها شروع کردند به مرخص کردن ترمینال‌های قدیمی و به خدمت گرفتن
کامپیوترهای شخصی دارای لینوکس.

مراسم افتتاحیه نسخه یک، باعث شد لینوکس به حوزه رادار نشریات فنلاند
وارد شود و در بقیه دنیا هم اخبار آن به گوش برسد. اکثر اخبار ناشی از
برخورد اتفاقی یک روزنامه‌نگار با لینوکس بود و هیجاناتی که در پی
داشت. از نظر تجاری، نسخه \code{1.0} رقیب هیچ یک از بازیگران بازار به حساب
نیامد. لینوکس داشت بازار مینیکس و کوهیرنت را تصاحب می‌کرد. ولی خارج از
این حوزه، کسی توجه چندانی به آن نداشت که خوب هم بود. همین توجه هم خیلی
بیشر از آن‌چیزی بود که من انتظار داشتم.

این را هم بگویم که خبرنگاران نشریات تجاری شروع کردند به کوبیدن در خانه
من - به معنی واقعی کلمه. تاو اصلا از این موضوع راضی نبود که صبح روز
تعطیل در بزنند و بعد از بازکردن در ببیند که یک خبرنگار ژاپنی با چند
هدیه - معمولا ساعت - پشت در ایستاده و درخواست مصاحبه با من را دارد چون
جایی شنیده که من حرف‌های جالبی برای گفتن دارم. وقتی من این خبرنگارها را
به داخل راه می‌دادم، ناراضی‌تر هم می‌شد. (این کاری بود که سال‌ها
کردم. خانه جدیدمان را که خریدیم، آن را یک منطقه بدون خبرنگار اعلام
کردیم. در موارد حادتر حتی پیش می‌آمد که فراموش کنم به تاو بگویم که
خبرنگاری که درخواست مصاحبه داشته را به خانه دعوت کرده‌ام. البته خودم هم
فراموش می‌کردم! در این حالت تاو مجبور بود خبرنگار را به خانه راه بدهد و
سرگرمش کند تا من به خانه برسم). بعد هم وب‌سایت‌های هواداران شروع به رشد
کردند؛ مثل آن وب‌سایت فرانسوی که به شکلی عجیب با عکس‌های مایه خجالت من،
همیشه به روز بود. مثلا آن عکسی که من را در نشست اسپکتروم نشان می‌داد:
بدون بلوز، با یک آبجو در دست و قرص و محکم!

نه.

فقط خبرنگاران و هکرهای لینوکس نبودند که به من علاقه نشان می‌دادند. به
ناگهان، آدم‌هایی با حساب‌های بانکی عظیم هم به سراغ من آمدند تا با من در
مورد فنّآوری‌های شان صحبت کنند. سال‌ها بود که یونیکس به خاطر قدرت بالا و
توانایی‌هایش در اجرای همزمان چند برنامه، به عنوان سیستمی با پتانسیل‌های
بسیار شناخته می‌شد. به همین دلیل شرکت‌هایی که همیشه نگاهی به یونیکس
داشتند، لینوکس را هم زیر نظر گرفتند. یکی از آن‌ها کمپانی شبکه‌ای
ناول\LFootnote{Novell} بود که در همان دوران یک پروژه مبتنی بر لینوکس
را تعریف و شروع کرد. این پروژه یک میزکار یونیکس به نام شیشه
بینا\LFootnote{Looking Glass} بود. این پروژه زیبا بود ولی چاره‌ای جز
شکست نداشت چون یکی از استانداردهای دوره خودش را نادیده گرفته بود: محیط
عمومی میزکار\RFootnote{\lr{Common Desktop Enviroment} - یکی از میزکارهای
  ساده ای که ابتدا برای یونیکس ها و اوپن وی ام اس توسعه پیدا کرد و تا
  مدت ها به شکل پیش فرض در سیستم‌های سولاریس استفاده می شد. این میزکار
  هنوز هم در حال توسعه بوده و در لینوکس قابل نصب و استفاده است.}.



در آگوست ۱۹۹۴ ‌آن‌ها به اطلاع من رساندند که می‌خواهند من را در اورمن
ایالت یوتا\LFootnote{Ormen, Utah} ببینند تا با هم در مورد میزکارشان
صحبت کنیم. ناول این فرصت را برای من فراهم کرده بود تا آمریکا را ببینم
و من هم جواب دادم که اگر پول کافی برای دیدن یک شهر دیگر آمریکا را هم
تقبل کنند، به دیدنشان خواهم رفت. حتی به عنوان یک فنلاندی جهان‌ندیده هم
درک می‌کردم که اورمن نمی‌تواند نمونه خوبی از یک شهر آمریکایی باشد. آن‌ها
واشنگتن را پیشنهاد کردند ولی من علاقه‌ای به آن نداشتم چون به نظرم همه
پایتخت‌ها مثل یکدیگرند. پیشنهاد بعدی آن‌ها نیویورک بود ولی من ترجیح
می‌دادم کالیفرنیا را ببینم.

در دفتر مرکزی ناول بود که متوجه شدم این پروژه تا چه حد از نظر آن‌ها جدی
است (البته بعدها این جدیت را کنار گذاشتند و حتی پروژه را هم متوقف
کردند و نه نفری که مشغول کار روی آن بودند، به سراغ پروزه
کلدرا\RFootnote{\lr{Caldera} - لینوکسی بود که ناول سعی می‌کرد با آن ایده‌های
  جدیدش در مورد دسکتاپ را اجرایی کند. پروژه در ۱۹۹۵ متوقف شد.}
رفتند). به هرحال این فرصتی بود تا من آمریکا را ببینم. جایی که به نظرم
می‌رسید به دلیل مرکزیت تکنولوژیکش، برای زندگی‌ آینده‌ام جای مناسبی باشد.

دیدن آمریکا یک تلنگر حسابی بود. اولین چیزی که توجه من را جلب کرد تازه
بودن همه چیز بود در مقایسه با اروپا. کلیسای مورمون تنها چند سال قبل از
اینکه من به آمریکا بروم جشن ۱۵۰ سالگی‌اش را گرفته بود و به همین مناسبت،
ساختمان اصلی را تمیز کرده بودند و گنبد سفید آن در نور آفتاب
می‌درخشید. در مقایسه با اروپایی که همه کلیساهایش آن قدر قدیمی هستند که
آثار زمان از آن‌ها پاک شدنی نیست، دیدن یک گنبد سفید فقط من را به یاد یک
چیز می‌انداخت: دیزنی‌لند. آن ساختمان بیشتر شبیه یک قلعه اسباب‌بازی بود تا
یک کلیسا. البته در هتل اورم این اشتباه را هم کردم که به سونا بروم. یک
سونای جمع و جور ساخته شده از پلاستیک که داخل آن فقط کمی گرم‌تر از بیرون
آن بود. از آن که خارج می‌شدم به این فکر می‌کردم که آمریکایی‌ها اصلا
نمی‌دانند سونا چیست و دلم هم برای خانه تنگ شده بود.

محدودیت‌ها را هم سریعا آموختم. دقیقا همان طور که گردشگرانی که به فنلاند
می‌آیند، سریعا یاد می‌گیرند که نباید در بارها با غریبه‌ها شروع به صحبت‌
کنند، من هم یاد گرفتم که در یوتا - و بعدا فهمیدم که در تمام آمریکا -
نباید درباره موضوعاتی مثل سقط جنین یا اسلحه با کسی بحث منطقی
کرد. پنجاه درصد احتمال دارد به کسی بر بخورید که درباره این موضوعات
عقاید بسیار احساسی دارد و سریعا درگیر دعوا بر سر موضوعی می‌شوید که اصولا
نباید بر سر آن دعوا کرد. در اروپا مردم نیازی نمی‌بینند که درباره اینجور
موضوعات با یکدیگر دعوا کنند. دلیلی که در آمریکا مردم اینقدر سرسختانه
سر موضع خود می‌ایستند، این است که بیش از حد، موضع دیگران را
شنیده‌اند. به احتمال زیاد فنلاندی‌ها بیشترین نسبت اسلحه به جمعیت را
دارند ولی معمولا از آن‌ها برای شکار استفاده می‌کنند و اصولا مساله برای
شان آن قدرها مهم نیست.

یکی دیگر از چیزهایی که در طول اولین روزهای اقامت در آمریکا یاد گرفتم
این بود که مارین هدلندز\LFootnote{Marin Headlands} با خودم فکر می‌کردم
که کمپینگ در آن کوه‌ها باید واقعا لذت بخش باشد. با رسیدن به آن طرف پل،
آن قدر خسته بودم که دیگر نمی‌توانستم به راه رفتن فکر کنم. آن روز اصلا
با خودم فکر نمی‌کردم که شش سال بعد، در همان کوه‌های بادخیز خواهم نشست و
حین نگاه کردن به اقیانوس‌ آرام، خلیج سانفرانسیسکو، پل، مه و خود شهر
سانفرانسیسکو، اینها را برای ضبط صوت دیوید تعریف خواهم کرد.

فقط یک سال طول کشید تا دوباره به آمریکا برگردم. این بار آمده بودم تا
در جان هال\RFootnote{\lr{Maddog a.k.a. Jon Hall} - یکی از شخصیت‌های بزرگ
  دنیای گنو/لینوکس.} را دیدم. او بازاریاب فنی دیجیتال یونیکس و یک
کاربر قدیمی یونیکس بود. او کسی بود که من را برای سخنرانی دعوت کرده
بود. مدداگ که به خاطر ریشش که تا روی سینه می‌رسید و قدرت طنز
فوق‌العاده‌اش مشهور بود (و البته خرخر بلندش موقع خواب) مدیر عامل موسسه
\dbquote{لینوکس اینترنشنال} بود که وظیفه پشتیبانی از لینوکس و کاربرانش
را بر عهده داشت. او همچنین پدرخوانده دختر من پاتریشیا است.

یک نکته مثبت دیگر دیدار نیواورلئان: مدداگ جور کرد که به من یک آلفا قرض
داده شود و این مبنایی شد برای پورت شدن لینوکس به سخت‌افزاری غیر از
کامپیوترهای شخصی. البته قبل از این هم مردم لینوکس را به معماری‌های
دیگری پورت کرده بودند. پورتی برای سری ۶۸۰۰۰ وجود داشت که کامپیوترهای
آتاری و آمیگا از آن استفاده می‌کردند ولی در آن دوره لینوکس به شکل هم
زمان روی هر دو معماری قابل اجرا نبود. آلفا اولین پورت واقعی لینوکس
بود. حالا دیگر یک سورس روی هر دو سخت‌افزار قابل کمپایل بود. کافی بود یک
لایه تجرید اضافه کنید تا کدی مشابه به دو شیوه و بر اساس اینکه بر روی
چه معماری‌ای قرار است استفاده شود، کمپایل شود. کد هنوز هم یکی است ولی
روی معماری‌های مختلف قابل استفاده است.

وقتی در مارس ۱۹۹۵ نسخه \code{1.2} را ارائه کردیم، کد کرنل به ۲۵۰ هزار خط
برنامه رسیده بود، مجله تازه کار \dbquote{ژورنال لینوکس} ادعا می‌کرد که
۱۰۰۰۰ خواننده دارد و این سیستم‌عامل روی پردازنده‌های اینتل، دیجیتال و
سان اسپارک اجرا می‌شد. یک قدم بزرگ پیش رفته بودیم.

\section{بخش دوم}
سال ۱۹۹۵ است و نسخه‌های لینوکس تجاری زیادی به وجود آمده‌اند و هر شرکت
ارائه دهنده لینوکس، پیروان خاص خود را دارد. دانشگاه، موقعیت من را از
کمک‌استادی به دستیار تحقق ارتقاء داده و این یعنی پول بیشتر و تدریس
کمتر. به آرامی - و واقعا به آرامی - مشغول انجام پروژه لیسانسم هستم که
عبارت است از پورت کردن لینوکس به معماری‌های مختلف. تاو به من اسکواش یاد
داده و به شکل هفتگی با هم مسابقه می‌دهیم و انصافا هم معمولا مساوی
می‌شویم.

این زندگی سعادت‌مندانه مشکلاتی هم دارد. معلوم می‌شود که یک آدم فرصت‌طلب
در بوستن لینوکس را به عنوان یک علامت تجاری برای خود ثبت کرده. مساله هم
فقط این نیست. او نامه‌هایی برای \dbquote{ژورنال لینوکس} و چند موسسه
دیگر فرستاده و از آن‌ها خواسته تا \dbquote{به عنوان قدردانی} نسبت به اینکه
علامت تجاری او را استفاده می‌کنند، پنج درصد سود خود را به او بدهند.

وقتی این را شنیدم، برایم یک دجاوو\RFootnote{\lr{Dejavu} - همان حالتی
  که مغز تصور می کند چیزی که در حال اتفاق افتادن است را قبلا در خواب
  دیده.}  بود. اسم این آدم برایم آشنا بود. ایمیل‌های قدیمی را گشتم و
کشف کردم که تقریبا یکسال و خورده‌ای قبل برای من ایمیلی زده و پرسیده که
آیا به خدا اعتقاد دارم یا نه و اینکه یک فرصت تجاری خوب برایم فراهم
است. این قبل از دورانی بود که اسپم‌ها یک دردسر عمومی شده باشند و هنوز
کسی برای شما ایمیل نمی‌زد تا روشی را معرفی کند که شما را یک شبه پولدار
می‌کند. نه. من هیچ وقت به آن ایمیل جواب ندادم ولی چون این ایمیل با
معیار آن‌ روزها خیلی عجیب بود، جایی ذخیره‌اش کردم.

این آقا یک سوءاستفاده‌چی حرفه‌ای از علامت‌های تجاری نبود و فقط یک بار سعی
کرد این کار را بکند. علامت‌های تجاری باید در شاخه‌های متفاوت ثبت شوند و
او لینوکس را در شاخه کامپیوتر ثبت کرده بود. برای ثبت شدن علامت تجاری،
باید شواهدی دال بر مالکیت آن ارائه کنید و این آقا یک دیسک حاوی
برنامه‌ای به نام لینوکس را به مسوول مربوطه نشان داده بود.

بعضی‌ها وحشت زده شدند. تمام جامعه لینوکس اعتقاد داشت که باید علامت
تجاری را برای خودمان ثبت کنیم اما مشکل اینجا بود که سازمانی نداشتیم که
دعوا را از طرف آنجا مطرح کنیم. حتی پول کافی برای وکیل را هم
نداشتیم. هیچ شرکت مستقلی نمی‌توانست پول لازم یعنی تقریبا ۱۵۰۰۰ دلار را
بپردازد. در نهایت \dbquote{ژورنال لینوکس} و چند شرکت دیگر تصمیم گرفتند
تا به موسسه لینوکس اینترنشنال کمک مالی کنند تا این موسسه نسبت به
مالکیت علامت تجاری لینوکس توسط آن فرد شکایت کند. لینوکس اینترنشنال در
استرالیا و توسط شخصی به نام پاتریک دکروز\LFootnote{Patrick D'Cruze}
تاسیس شده بود که در سال ۱۹۹۴ برای گسترش لینوکس در سطح جهانی به آمریکا
مهاجرت کرد. سالی که مساله علامت تجاری لینوکس پیش آمد، مدداگ مدیرعامل
موسسه بود. همه به او اعتماد داشتند و هنوز هم دارند.

من در فنلاند بود و سعی می‌کردم تاو را در اسکواش یا آوتون را در اسنوکر
شکست دهم و هیچ علاقه‌ای هم به داخل شدن در این ماجرا نداشتم. تنها چیزی
که من می‌خواستم این بود که این کابوس تمام شود. آرزوی من این بود که کلا
جریان علامت تجاری ملغی شود؛ مثلا به این دلیل که قبل از ادعای آن فرد در
صنعت استفاده شده است. ما به اندازه کافی سند و مدرک داشتیم که لینوکس از
سال‌ها قبل در این صنعت یک کلمه رایج بوده است. اما مشکل در اینجا بود که
به اعتقاد وکیل ما، اینکار وقت تلف کردن بود و ما به هیچ وجه نباید تلاش
می‌کردیم تا لینوکس را یک کلمه معمول در صنعت معرفی کنیم. او ما را متقاعد
کرد که برای عمومی کردن کلمه لینوکس لازم است نشان دهیم که این کلمه در
صنعت کاربرد عمومی دارد و این نه فقط در آن زمان که حتی این روزها هم
ممکن نیست. وکیل گفت که با این استراتژی، پرونده را خواهیم باخت. همچنین
برای مان توضیح داد که حتی اگر در این پرونده پیروز شویم، دوباره شخص
دیگری ادعای مالکیت این نام را مطرح خواهد کرد.

راه حل پیشنهادی او این بود که ادعای مالکیت علامت تجاری لینوکس را بکنیم
و آن را به یک شخص منتقل کنیم. پیشنهاد اولیه من موسسه لینوکس اینترنشنال
بود ولی عده‌ای با این نظر مخالف بودند. لینوکس اینترنشنال یک موسسه جوان
بود که هنوز موقعیت خود را تثبیت نکرده بود. مردم نگران این بودند که
لینوکس اینترنشنال به دست موسسات تجاری بیافتد (که لازم است اضافه کنم
هیچ گاه این اتفاق نیُفتاد). البته این نگرانی هم بود که در صورت انصراف
مدداگ، چه کسی کنترل موسسه را در دست خواهد گرفت.

حالا همه چشم‌ها به من دوخته شده بود. وکیل هم می‌گفت که دعوا بسیار راحت‌تر
به نتیجه خواهد رسید اگر من به عنوان اولین کاربر لینوکس در جهان، ادعای
مالکیت علامت تجاری آن را بکنم. همین استراتژی را پذیرفتیم و در نهایت هم
پیش از دادگاه با طرف مقابل به توافق رسیدیم، چون راحت‌تر و کم‌خرج‌تر
بود. مثل اکثر توافقات بیرون از دادگاه، نمی‌توانم چندان درباره جزییات
توضیح بدهم. البته من هم اطلاع چندانی از ماجرا ندارم و فقط خوشحالم که
تمام شده.

بعد که به سراغ نامه‌ای رفتم که آن مرد برایم فرستاده بودم، به این نتیجه
رسیدم که حداقل در آن دوره اصلا در فکر پول درآوردن از نام تجاری لینوکس
نبوده و فقط می‌خواسته کمی با من گپ بزند یا کمی پول از من بگیرد. شاید هم
اگر کشف می‌کرد که من یک هم‌دین خوب هستم، علامت تجاری را به خودم پس
می‌داد.

این واقعیت را می‌پذیرم که همه آدم‌ها از نظر اخلاقی در سطح یکسانی نیستند
ولی مشکل اخلاقی من با کل این جریان علامت تجاری این بود که به خاطر کار
یک نفر دیگر،‌ منی که هیچ کار بد و نامناسبی نکرده بودم مجبور شدم به
میدان بروم و با آن آدم بجنگم.

نتیجه آن جنگ و گریز کثیف این بود که علامت تجاری لینوکس به نام من ثبت
شد. معنی دیگر این جمله این است که وقتی شرکتی مثل
وی.ای. لینوکس\RFootnote{شرکتی در ویرجیانا که در ۱۹۹۳ تاسیس شد و پشت
  سایت‌هایی مانند \lr{ThinkGeek} و سورس فورج است. این شرکت از ۲۰۰۹ به بعد
  گیک‌نت نام گرفته است.} تصمیم گرفت تا سهامی عام شود، موظف بود به کسانی
که سهام را خریداری می‌کنند اطلاع دهد که دارایی معنوی نیمی از نام شرکت
متعلق به شخص دیگری است (البته در آن مورد خاص، شرکت به شکل قانونی از من
اجازه گرفته بود که از اسم لینوکس استفاده کند). گاهی من را به خاطر
مواردی مثل این شماتت می‌کنند.

داستان علامت تجاری لینوکس یک دردسر غیرمترقبه در دنیای لینوکس بود و یک
انحراف از مسیر. مدت زیادی از حل شدن این مساله نگذشته بود که داستان
جدیدی رو شد: مهندسی از آزمایشگاه تحقیقاتی اینتل در پورتلند اورگون به
من اطلاع داد که شرکتش در حال استفاده از لینوکس برای کشف معماری‌های جدید
است. او از من پرسید که آیا علاقه‌دارم شش ماه به عنوان کارآموز به اینتل
بروم یا نه.

من و تاو به شکلی مبهم درباره امکان مهاجرت به آمریکا و زندگی در آن‌جا با
یکدیگر صحبت کرده بودیم. او می‌دانست که از آبجوی ریشه گذشته، من از بودن
در آمریکا بسیار لذت برد‌ه‌ام. ما به این نتیجه مشترک رسیده بود که فرصت‌ها
در آمریکا بهتر هستند - چه برسد به آب و هوا - (تازه من به این نتیجه هم
رسیده بودم که شیوه ایجاد انگیزه در کارمندان در آمریکا در مقایسه با
اروپا واقع‌بینانه‌تر است و منجر به محصولات بهتری هم می‌شود. در فنلاند اگر
یک نفر از همکاران خود بسیار بهتر باشد، بدون سر و صدا کمی بیشتر پول
می‌گیرد. در آمریکا این پول بیشتر است و خیلی هم بهتر کار می‌کند). دوره
کارآموزی در اینتل فرصت خوبی بود برای محک زدن شرایط و ما توافق کردیم که
من از این موقعیت استفاده کنم. من دل‌نگرانی‌هایی هم داشتم. مثلا راحت
نبودم که دانشکده را بدون گرفتن مدرک ترک کنم. شاید به خاطر تجربه
پدربزرگ پروفسورم، احساس خوبی نسبت به این نداشتم که دانشگاه راه رها
کنم. در نهایت، احساسات من اصولا در تصمیم‌گیری دخیل نشد چون همان مهندس
با من تماس گرفت و گفت که برای رفتن به اینتل و کار در آن‌جا، نیازمند
ویزای کار شش ماهه هستم که احتمال موافقت اداره مهاجرت با آن کم است.

ما در هلسینکی ماندیم. در روزهای شادمانی آخر سال ۱۹۹۶، من تلاش می‌کردم
تا سانتیمتر به سانتیمتر به مدرک لیسانسم نزدیک‌تر شوم. برای کسب امتیازات
لازم، فقط باید یک دوره کوچک دیگر می‌گذراندم و پایان نامه‌ام را هم
می‌نوشتم. عجیب است که این اولین باری می‌شد که لینوکسی که اکثر ساعت‌های
عمرم را به آن گذرانده بودم، برایم کاربردی تحصیلی پیدا می‌کرد.

سال ۱۹۹۶ سال بیداری من بود. در فنلاند برابری طلب، حقوق شما بعد از سه
سال کار باید افزایش پیدا کند. افزایش حقوق برای من مانند تلنگری بود که
می‌گفت سه سال است دارم در دانشکده کار می‌کنم. آیا می‌خواستم همه زندگی‌ام
را همین جا بگذارنم؟ آیا برنامه‌ام این بود که جای پدربزرگم را بگیرم؟
یادتان هست که او را در ابتدای همین کتاب چطور توصیف کردم؟ کچل، دارای
اضافه وزن و بدون هیچ بویی. حالا دیگر خودم را بیشتر از قبل در آینه نگاه
و بررسی می‌کردم. خط جلوی موهایم چند میلیمتری عقب رفته بود و چند کیلو
چربی اضافی در شکم سابقا لاغرم دیده می‌شد. بیست و شش ساله بودم و برای
اولین بار در زندگی احساس پیری می‌کردم. هفت سال بود که در دانشکده بودم و
می‌دانستم که همین‌که اراده کنم، می‌توانستم کارها را جمع و جور کنم و
فارغ‌التحصیل شوم.

\begin{journal}
دختر ده ساله من کلی، می‌گوید این اوج خوش‌بختی است که کسی یک پنگوئن داشته
باشد. ما دور آتش کمپ نشسته‌ایم و لینوس دارد از این می‌گوید که گروه
کاربران لینوکس بریستول در انگلستان برایش یک پنگوئن خریده‌اند. کلی باور
نمی‌کند که لینوس تا به حال به دیدن آن حیوان نرفته و لینوس توضیح می‌دهد
که آنها در اصل یک پنگوئن نخریده‌اند بلکه از طرف لینوس، حمایت مالی از یک
پنگوئن را پذیرفته‌اند. مطمئن نیست ولی می‌گوید که مدت زمان این حمایت
مالی، یک سال است.

از لینوس می‌پرسم که جریان نماد پنگوئن چیست. تاو جواب می‌دهد که
\dbquote{پنگوئن ایده من بود.} لینوس دنبال پیدا کردن یک نماد بود چون
مردم دائما می‌گفتند که دوست دارند یک نماد داشته باشند. او به چیزهایی که
دیده بود فکر می‌کرد. شرکت‌های مبتنی بر لینوکس، نمادهای خود را
داشتند. مثلا یکی از آن‌ها از یک مثلث صورتی به عنوان نماد استفاده می‌کرد
که تا جایی که من می‌دان علامت بین‌المللی همجنس‌گرایان است و قبلا رزرو
شده. لینوس می‌گفت نمادی می‌خواهد که قشنگ باشد و علاقه ایجاد کند.

پنگوئن به فکر من رسید. یک بار یک پنگوئن، لینوس را در باغ وحش استرالیا
گاز گرفته بود. او دوست دارد که به حیوان‌ها غذا بدهد و همیشه هم چیزی
برای دادن دارد. آن پنگوئن‌ها تقریبا سی سانتیمتر قد داشتند و لینوس عملا
رفت داخل قفس تا به آن‌ها غذا بدهد. انگشتانش را طوری تکان داد که انگار
ماهی هستند و یکی از پنگوئن‌ها جلو آمد و انگشتش را گاز گرفت تا ببیند که
ماهی است یا نه. هرچند یک پنگوئن او را گاز گرفته بود ولی از پنگوئن‌ها
خوشش آمده بود و سعی می‌کرد هر جا که بشود،‌ به دیدن پنگوئن‌ها برود.

پس وقتی من پیشنهاد کردم که \dbquote{حالا که این قدر عشق پنگوئن هستی
  چرا به عنوان نماد انتخابش نمی‌کنی؟ گفت که در این مورد فکر خواهد کرد.}

لینوس که دو سه نفر آنطرف‌تر نشسته، سرش را به علامت مخالفت تکان می‌دهد و
می‌گوید: \dbquote{نه. اشتباه می‌کند. پنگوئن نظر تاو نبود.}

اولین بار است که می‌بینم لینوس و تاو در موضوعی اختلاف نظر دارند. آن‌ها
معمولا کارها را به خوبی بین خود تقسیم می‌کنند. تاو مسوولیت کودکان و یک
شوهر مشهور را بر عهده دارد و معمولا خبرنگاران را با قابلیت‌های بالای
کاراته‌اش متحیر می‌کند و لینوس گاه‌گداری مسوول شستن و تا کردن لباس‌ها و
درست کردن کاپوچینوی صبحگاهی است. حالا هم حتی بعد از استرس یک رانندگی
ده ساعته و توقف‌های مکرر به خاطر نیازهای دو بچه کوچک، به خوبی با هم
کنار می‌آیند.

بنا به روایت لینوس، هرچند که ممکن است تاو در مراحل اولیه پنگوئن را
پیشنهاد کرده باشد، اما انتخاب این موجود یخ نشین به عنوان نماد شانس این
سیستم عامل جدید، حاصل گفتگو با دو تن از افراد رده بالای دنیای لینوکس
بوده است.

تاو در این نسخه هم یک بل می‌گیرد \dbquote{چون من گفته بودم به نظرش خوب
  نیامد. هنوز داشت دنبال نماد می‌گشت تا بالاخره در بوستون با هنری
  هال\LFootnote{Henry Hall} در این مورد صحبت کرد. من آنجا هم پنگوئن را
  پیشنهاد کردم و آن‌ها موافقت کردند. به نظرم موافقت آن‌ها بود که باعث شد
  لینوس هم پنگوئن را قبول کند.}

\dbquote{هنری هال گفت هنرمندی را می‌شناسد که می‌تواند پنگوئن را برایمان
  طراحی کند. ولی این اتفاق هیچ وقت نیفتاد. یادم هست که در قدم بعدی
  لینوس در اینترنت از مردم خواست تا برایش عکس پنگوئن بفرستند و در
  نهایت نسخه لری اوینگ\LFootnote{Larry Ewing}، هرمند طراح شاغل در
  موسسه علوم کامپیوتری در دانشگاه \lr{A\&M} تگزاس را انتخاب کرد.}

البته قرار نبود هر پنگوئنی پذیرفته شود. لینوس دنبال یک پنگوئن خوشحال
بود. پنگوئنی که تازه یک خمره آبجو نوشیده و از یک سکس خوب برگشته
باشد. از این گذشته، لینوس می‌خواست که نماد شانس لینوکس منحصر به فرد
باشد و به همین دلیل به جای منقار و پاهای سیاه، به دنبال منقار و پاهای
نارنجی بود تا به نظر برسد که پدر پنگوئن لینوکس، اردک بوده. چیزی شبیه
به اینکه دافی داک در راه قطب جنوب هوس خوشگذرانی به سرش زده باشد و یک
شب را پیش یک خانم پنگوئن گذرانده باشد.
\end{journal}

\section{بخش سوم}
خبر تصمیم من مبنی بر کار کردن در ترنسمتا، از طرف جامعه لینوکس با همان
برخوردی روبرو شد که وقتی در سال ۱۹۹۶ اعلام کرده بودم که می‌خواهم بچه‌دار
شوم دیده بودم.

وقتی در بهار این خبر در اینترنت پیچید که تاو حامله است، افراد فعال‌تر
گروه‌های لینوکس از من می‌پرسیدند که چه برنامه‌ای برای ایجاد تعادل بین
نیازهای مربوط به توسعه و نگهداری لینوکس و درخواست‌های خانواده‌ام
دارم. چند ماه بعد که مشخص شد بالاخره دانشگاه هلسینکی را ترک خواهم کرد
و برای کار در موسسه رازآلود ترنسمتا به سیلیکون‌ولی خواهم رفت، دوباره
این بحث‌های طولانی محوریت پیدا کردند که آیا من خواهم توانست در این محیط
تجاری جدید نیز مانند محیط دانشگاهی به فلسفه آزادی نرم‌افزار پایبند بمانم
یا نه. دوستان مخالف می‌گفتند که ترنسمتا توسط یکی از بنیانگذاران
مایکروسافت به نام پاول آلن\LFootnote{Paul Allen} ایجاد شده است و حتی
بعضی‌ها گفتند که کل این جریان نقشه‌ای حساب شده است برای کنترل لینوکس
توسط شرکت‌های تجاری.

نمی‌گویم که این دغدغه‌ها از طرف هوداران صادق لینوکس منطقی نبود ولی
فقط... یک لحظه به من فرصت بدهید! واقعیت این است که نه تولید پاتریشیا
در دسامبر ۱۹۹۶ (و دانیلا شانزده ماه بعد و سلسته چهل و هشت ماه بعد) و
نه کارم در ترنسمتا، که در فوریه ۱۹۹۷ شروع شد تاثیر منفی در لینوکس
نداشته‌اند. اگر کوچکترین احساسی داشتم که چیزی ممکن است روی تمرکزم بر
لینوکس تاثیر منفی بگذارد، صادقانه قدم‌های مناسب برای انتقال کل جریان به
یک فرد مورد اعتماد دیگر را برمی‌داشتم.

البته دارم از خودم جلو می‌زنم.

در بهار ۱۹۹۶ و در زمانی که سوز سرما داشت می‌شکست، من آخرین واحدهای لازم
برای لیسانس را پاس کردم. همین موقع‌ها بود که پیتر آروین را ملاقات کردم،
عضوی از جامعه لینوکس که سه سال قبل جمع‌آوری آنلاین پول برای پرداخت
اقساط کامپیوتر من را هماهنگ کرده بود. او هم مثل هر کس دیگری که خواننده
گروه لینوکس بوده باشد، می‌دانست که من در حال فارغ‌التحصیل شدن هستم. او
حدودا یکسال بود که در ترنسمتا کار می‌کرد و حالا پیش رییسش رفته بود و
گفته بود که یک نفر را در فنلاند می‌شناسد که می‌تواند برای شرکت مفید
باشد. او وقتی برای دیدن مادربزرگش به سوئد آمده بود، سر کوتاهی هم به من
زد. درباره ترنسمتا توضیح داد که البته به دلیل پروژه مخفی آن، کار بسیار
مشکلی بود. بین برنامه‌نویس‌ها این شایعه دهن به دهن می‌شد که ترنسمتا در
حال ساختن \dbquote{چیپ‌های قابل برنامه‌ریزی} است. به هرحال دیدن پیتر از
نزدیک فوق‌العاده بود.

یک هفته بعد از برگشت به کالیفرنیا، پیتر ایمیل زد و پرسید که کی به آنجا
خواهم رفت. این موقعیت کاملا با چیزی که سال قبل از طرف اینتل پیشنهاد
شده بود و قرار بود من به عنوان کارآموز برای شش ماه به آمریکا بروم و به
دلایل اداری انجام نشد، فرق داشت.

به نظر من حتی امکان سفر به کالیفرنیا هم عالی بود. این اولین مصاحبه
کاری من بود. رزومه هم نداشتم و نمی‌دانستم ترنسمتا مشغول چه کاری
است. انگار در یک سرزمین بیگانه بودم.

من بیشتر نگران برنامه‌ریزی برای مهاجرت به آمریکا بودم تا به دست آوردن
کار. آن دیدار برای من حالت مصاحبه نداشت و به نظر می‌رسید که آن‌ها هم از
استخدام من مطمئن هستند. به عنوان یک مصاحبه کاری، موقعیت عجیبی بود.

بعد از روز اول، به هتلم در آن طرف خیابانی رفتم که دفاتر مرکزی ترنسمتا
در آن واقع شده بودند. در حالتی که به خاطر پرواز چندین ساعته ذهنم به هم
ریخته بود، حس می‌کردم که کار جالبی است ولی ترنسمتایی‌ها دیوانه‌اند. در آن
مرحله هیچ سیلیکونی در شرکت وجود نداشت. هیچ سخت‌افزاری آنجا نبود. همه
چیز روی شبیه‌سازها اجرا می‌شد و نمایش اینکه آن‌ها می‌توانستند ویندوز
\code{3.1} را بوت کنند و بعد سولایتر را اجرا کنند من را متقاعد نمی‌کرد
که کاری انجام شده. موقع خواب به این فکر می‌کردم که آیا مشغول وقت تلف
کردن نیستیم. دقیقا یادم هست که فکر می‌کردم: شاید هیچکدام به جایی نرسد؛
نه یک نوآوری تکنولوژیک در ترنسمتا و نه به یک شغل درست و حسابی برای من.

آن شب را واقعا با این خیالات خوابیدم؛ البته خواب چندانی هم نکردم. در
تخت دراز کشیده بودم و به برنامه‌های ترنسمتا فکر می‌کردم. بعد درباره
اینکه می‌توانم در حیاط پشتی خانه‌ام یک درخت نخل داشته باشم خیال‌پردازی
کردم. بعد هم رفتم به سراغ نشخوار ذهنی چیزهایی که در شبیه‌ساز دیده
بودم. شب پرخاطره‌ای است از خیالات گذران که البته قابل مقایسه با
پیچیدگی‌های ذهنی \dbquote{حوا} در برخورد با میوه ممنوعه نیست.

صبح فردا، هیجان زده بودم و تا شب هیجانم خیلی بیشتر شده بود. این درست
همان موقعی بود که استرس هم شروع شد.

قبل از پذیرفتن شغل در ترنسمتا، با خیلی‌ها درباره آن مشورت کرده
بودم. همین که خبر پذیرفتن شغل از طرف من مطرح شد، ایمیل‌های حاوی
پیشنهادهای شغلی دیگر هم به سویم سرازیر شدند. در رد
هت\LFootnote{RedHat} هم صحبت کردم. آن‌ها به من شغلی در رد هت پیشنهاد
کردند و گفتند که حقوق بالاتری از ترنسمتا می‌دهند. هرچند که اطلاعی از
مفاد قرارداد ما نداشتند و من هم هنوز درباره حقوق با ترنسمتا صحبت نکرده
بودم. حتی گفتند که سهام بیشتری از ترنسمتا هم به من می‌دهند، هرچقدر هم
که پیشنهاد ترنسمتا بالا باشد. اما من هیچ علاقه‌ای به کار در یک شرکت خاص
توزیع کننده لینوکسی نداشتم - حتی در شرکتی که از سر خوش‌شانسی دقیقا در
وسط کالیفرنیای شمالی باشد.

در نهایت من بدون اینکه رسما دنبال کار بگردم، پنج پیشنهاد کار خوب
داشتم. ترنسمتا، هیجان‌انگیزترین آن‌ها بود.

من با اینکه احساس غریبی داشتم، گفتم بله. در مرحله بعد به دانشکده اعلام
کردم که می‌خواهم آن‌جا را ترک کنم و همین جا بود که استرس شروع شد. برای
من این قدم بزرگی بود که راه بازگشت نداشت. ما یک فرزنده جدید داشتیم،
مشغول مهاجرت به یک کشور جدید بودیم و من داشتم آشیانه امن دانشگاه
هلسینکی را ترک می‌کردم - البته اول باید پایان‌نامه‌ام را می‌نوشتم. الان که
به گذشته نگاه می‌کنم، به نظرم می‌رسد که انجام هم زمان کلیه این تغییرات
ایده خوبی بود. البته دیوانگی هم بود.

هیچ اطلاعیه رسمی‌ای در کار نبود (چرا باید می‌بود؟). فقط جریان در اینترنت
پخش شد و بحث‌هایی که قبلا به آن‌ها اشاره کرده‌ام درگرفت. بحث اینکه آیا من
واقعا در مواجه با شرکت‌های تجاری، و حین عوض کردن پوشک بچه‌، به لینوکس و
آزادی نرم‌افزار پایبند خواهم ماند؟ آن روزها تصور مردم این بود که چیزی
مثل لینوکس توسط دانشجویان نوشته می‌شد نه توسط آدم‌های جا افتاده. درک
می‌کنم که نگرانی آن‌ها به جا بود.

پایان‌نامه‌ را در طول یک آخر هفته طولانی نوشتم و درست چند دقیقه قبل از
اینکه تاو را برای به دنیا آوردن پاتریشیا به بیمارستان برسانم، آن را
تمام کردم. پاتریشا چهل ساعت بعد به دنیا آمد. پنجم دسامبر ۱۹۹۶. در یک
لحظه پدر بودن به نظرم طبیعی‌ترین کار دنیا رسید.

در طول هفته‌های بعد دائما مشغول کارهای پاتریشیا و کامل کردن مدارک مورد
نیاز برای مهاجرت به آمریکا بودیم. کاری که گویی تا ابد طول می‌کشید. کشف
کردیم که برای مهاجرت ساده‌تر است که ازدواج کرده‌باشیم پس در یکی از
روزهای ژانویه - که همیشه باید تاریخ دقیقش را از تاو بپرسم - به دفتر
دولتی رفتیم و رسما ازدواج کردیم. ازدواج ما سه مهمان داشت: پدر و مادر
تاو و مادر من (پدرم در مسکو بود). دوران عجیبی بود. اکثر وسایل مان را
به ایالات متحده فرستاده بودیم بدون اینکه بدانیم چه زمانی خودمان
می‌توانیم به آنجا پرواز کنیم. برای خداحافظی از همه دوستان، یک مهمانی
خداحافظی گرفتیم. بیست نفر آدم در آپارتمان کوچک یک خوابه و نیمه خالی ما
جمع شدند. بنا به یک سنت خوب فنلاندی، همه حسابی مست کردند.

ویزاهای ما بالاخره آماده شد و در ۱۷ فوریه ۱۹۹۷، سوار هواپیمای روز به
مقصد سانفرانسیسکو شدیم. درجه حرارت هلسینکی یادم مانده که منفی هفده
درجه بود و خانواده تاو را که وقتی خداحافظ کرد، داشتند گریه
می‌کردند. آن‌ها بسیار به هم نزدیک بودند. یادم نیست که خانواده خودم به
فرودگاه آمده بودند یا نه. باید آمده باشند. شاید هم نه.

در آمریکا فرود آمدیم و با یک بچه و دو گربه از گمرک رد شدیم. پیتر آروین
آنجا بود و به ما خوشامد گفت. یک ماشین کرایه کردیم و به سمت سانتاکلارا
و آپارتمانی که در سفر چند ماه قبل دیده و اجاره کرده بودیم، راه
افتادیم. کل جریان برای من یک تجربه سورئال بود، بخصوص اختلاف دمای ۲۰
درجه‌ گرمتر از فنلاند.

تا دو ماه دیگر، بقیه وسایل نمی‌رسید.شب اول را روی یک تشک بادی که
همراهمان آورده بودیم گذراندیم و روز بعد برای خرید یک تخت واقعی به
فروشگاه رفتیم. تا رسیدن وسایل مان، پاتریشیا در سبد حمل و نقلش
می‌خوابید. تاو از این جریان ناراضی بود. ولی دیوید با اشاره به من که سه
ماه اول زندگی‌ام را در سبد لباس‌های چرک گذرانده بودم، می‌گفت که این
جریان، تکرار تاریخ است. ما چندان آشپزی نمی‌کردیم (هنوز هم نمی‌کنیم) و
نمی‌دانستیم هم که برای شام باید کجا برویم. ما اکثر وعده‌های غذایی را در
رستوران مرکز خرید محله یا در یک فست‌فود می‌خوردیم. یادم هست که به تاو
می‌گفتم که باید جاهای جدیدی برای غذا خوردن پیدا کنیم.

با توجه به سفر و شغل جدیدم در ترنسمتا و زمانی که تطبیق با محیط جدید
می‌برد، در یکی دو ماه اول فرصت چندانی برای کار روی لینوکس پیدا
نکردم. سر کار تمام وقتم اشغال بود و بعد از کار را هم با تاو و پاتریشیا
به کشف محیط جدید می‌گذراندیم. دوران شلوغی بود. تقریبا هیچ پولی
نداشتیم. حقوقم عالی بود ولی همه را صرف خرید مبلمان کرده بودیم. خرید
ماشین یک دردسر بزرگ بود چون سابقه مالی برای کارت اعتباری نداشتیم. حتی
اثبات اینکه از پس پرداخت قبض تلفن برخواهیم آمد هم برای خودش دردسری شده
بود.

کامپیوتر من در یک کشتی بود که با سرعت لاک‌پشتی در حال دور زدن شاخ
آفریقا بود. آن دوره اولین غیبت من در وب بود و این خیلی‌ها را نگران
کرد. داستان چیزی شبیه به این بود که: خب بعله... حالا او دارد برای یک
شرکت تجاری کار می‌کند و...

خیلی‌ها رک و راست می‌پرسیدند که: آیا این به معنای پایان عمر لینوکس به
عنوان یک سیستم‌عامل آزاد است؟ من توضیح می‌دادم که در قراردادم با ترنسمتا
ذکر شده که حق دارم کار روی لینوکس را ادامه بدهم. به ذهنم نمی‌رسید که
چطور باید به مردم بگویم که می‌خواهم چند روزی نفسی بکشم.



\subsection*{زندگی در سرزمین ترنسمتا}
یکی از مشکلات توضیح این امر که نقل مکان به آمریکا و ورود به دنیای
تجاری قرار نیست من را تغییر دهد، این واقعیت بود که ترنسمتا یکی از
رمزآلودترین شرکت‌های تجاری بود. در مورد اینکه افراد حق داشتند در مورد
ترنسمتا چه چیزهایی را به دیگران بگویند فقط یک قانون وجود داشت که همه
کارمندان باید آن را رعایت می‌کردند. آن قانون این بود: \dbquote{حق
  ندارید هیچ چیزی بگویید.} این باعث شده بود مردم به فکر فرو روند که من
به چه فرقه عجیبی پیوسته‌ام و آیا شانس بیرون آمدن از آن را دارم یا
نه. من حتی به مادرم هم نمی‌توانستم درباره کاری که می‌کنم توضیح دهم -
البته علاقه‌ای هم به این جریان نداشت.

کاری که من در ترنسمتا می‌کردم کار عجیبی نبود. عملا اولین کاری که کردم
حل کردن چند مشکل و باگ در لینوکس بود که ترنسمتا به آن برخورد کرده
بود. شرکت از چندین سیستم لینوکسی چند پروسسوره استفاده می‌کرد. من تا آن
روز با یک سیستم چند پروسسوره واقعی کار نکرده بودم و معلوم شد که بخش
\lr{SMP}\LFootnote{Symmetric MultiProcessing} اشکالاتی دارد و آنطور که
باید، کار نمی‌کند. این جریان برای من یک مساله شخصی بود و بدون درنگ
مشغول حل کردنش شدم.

کار اصلی من در ترنسمتا، عضویت در تیم سافتبال بود. اوه ببخشید. منظورم
تیم نرم‌افزار است.\RFootnote{لینوس از شباهت ورزش \lr{Softball} به
  \lr{Software} استفاده کرده تا بگوید عضو تیم سافت‌تبال ترنسمتا هم بوده
  است} خیلی زیاد سافتبال بازی نمی‌کردیم. هیچکدام از تیم‌های سیلیکون‌ولی
تا وقتی که نمی‌گفتیم داریم روی چه پروژه‌ای کار می‌کنیم، حاضر نبودند با ما
بازی کنند.

نمی‌دانم که مردم چقدر با ترنسمتا آشنایی دارند. حالا که دارم اینها را
تایپ می‌کنم در دوره سکون قبل از عمومی کردن سهام هستیم (آه خدایا! کاری
کن سهام ما را بخرند) و دیگر هم یک شرکت مخفی نیستیم، هرچند که بنا به
قواعد سازمان تجارتی آمریکا، باید پیش از عمومی کردن سهام، بعضی از
پروژه‌ها را بی و صدا پیش ببریم. بگذارید همین جا دعا کنیم که وقتی مشغول
خواندن این کتاب هستید ترنسمتا کلی مشهور شده باشد و همه پردازنده‌های آن
را خریده باشند. این چیزی است که ترنسمتا مشغول آن است: پردازنده،
سخت‌افزار.

البته کار ترنسمتا چیزی بیشتر از سخت‌افزار است و این شانسی است که من
آورده‌ام چون حتی تفاوت بین ترانزیستور و دیود را هم نمی‌فهمم. کاری که
ترنسمتا می‌کند این است که سخت‌افزاری ساده بسازد و سپس با استفاده از یک
نرم‌افزار هوشمند کاری کند که این سخت‌افزار ساده، مانند یک سخت‌افزار
پیچیده مثلا یک \code{x86}، عمل کند. این سخت‌افزار ساده باعث خواهد شد تا
پردازنده‌ها تعداد ترانزیستورهای کمتری داشته باشند و در نتیجه توان
بسیاری کمتری مصرف کنند و این چیزی است که در دنیای متحرک امروزی، همه به
دنبالش هستند. این نرم‌افزار هوشمند، همان چیزی است که باعث شده ترنسمتا
یک تیم نرم‌افزاری بزرگ داشته باشد و من هم جزو آن باشم.

شرایط برای من کاملا مناسب بود. یک شرکت غیر لینوکسی که از نظر فنی مشغول
کار جالبی بود (اعتراف می‌کنم که تا حالا هم ندیده‌ام شرکت دیگری حتی به
سراغ آزمایش ایده ترنسمتا رفته باشد) و از من کاری را می‌خواست که در آن
تخصص داشتم: برنامه‌نویسی سطح پایین روی پردازنده‌های خانواده
\code{80x86}. مطمئنا هستم یادتان نرفته که ماجرای نوشتن لینوکس اصولا از
داستان علاقه من به تجربه برنامه‌نویسی سطح پایین روی کامپیوتر جدیدم که
یک \code{x86} بود، شروع شد.

اینکه ترنسمتا یک شرکت لینوکسی نبود هم برای من مهم بود. البته اشتباه
نکنید: من عاشق حل مشکلات لینوکس در ترنسمتا و انجام پروژه‌های داخلی
مرتبط با لینوکس بودم (و واقعیت این است که این روزها عملا غیرممکن است
که شرکتی را پیدا کنید که مشغول کار روی فنّآوری‌های جدید باشد و اینگونه
پروژه‌ها را نداشته باشد). در ترنسمتا لینوکس در جایگاه دوم قرار داشت؛
دقیقا همان چیزی که من می‌خواستم. من فرصت داشتم روی لینوکس کار کنم بدون
اینکه احساس کنم اجباری نسبت به رعایت ترجیحات شرکتم در مورد شیوه توسعه
لینوکس و اهداف بلندمدت آن دارم. من کماکان این حس را داشتم که لینوکس یک
سرگرمی شخصی است که هیچ چیزی جز مباحث تکنولوژیک، در تصمیم‌گیری‌های مربوط
به آن دخیل نیست.

نتیجه این بود که من در طول روز برای ترنسمتا کار می‌کردم. من
\textbf{مفسر \code{x86}} را می‌نوشتم و پشتیبانی می‌کردم (هنوز هم از آن
استفاده می‌کنیم ولی در حال حاضر افراد دیگری مسوول پشتیبانی آن هستند.)
مفسر مورد نظر، بخشی از نرم‌افزار ترنسمتا بود که یکی یکی دستورات اینتل
را برمی‌داشت و آن‌ها را اجرا می‌کرد (یعنی دستورات \code{80x86} را به زبان
مورد نظر ما \dbquote{تفسیر} می‌کرد.) بعدها وظایف دیگری به من محول شود
ولی دروازه ورودم به شبیه‌سازهای سخت‌افزاری، همان پروژه بود.

شب‌ها هم می‌خوابیدم.

در قرارداد من با ترنسمتا به روشنی ذکر شده بود که حق دارم در طول ساعات
کاری هم روی لینوکس کار کنم و شک نکنید که از این بند استفاده کافی را
کردم.

بعضی از آدم‌ها اعتقاد دارند که باید طولانی کار کرد. آن‌ها گاهی دوبرابر،
سه برابر یا حتی چهاربرابر یک شیفت معمولی کار می‌کنند. من یکی از آن‌ها
نیستم. نه ترنسمتا و نه لینوکس هیچ‌گاه باعث نشدند که یک خواب خوب را از
دست بدهم. اگر بخواهید حقیقت را افشا کنم، باید بگویم که من به خواب
اعتقاد مبرم دارم. بعضی‌ها می‌گویند این یعنی تنبل بودن، ولی در جواب فقط
حاضرم بالشتم را به سمت شان پرتاب کنم. من برای خوابیدن همیشه یک استدلال
خوب دارم و حاضرم از آن دفاع کنم: شاید خوابیدن باعث شود چند ساعتی را از
دست بدهید، مثلا ده ساعت را، ولی در عوض باعث می‌شود همان ساعت‌های محدودی
که مشغول کار هستید، کاملا سرحال باشید و مغزتان شش سیلندر کار کند. شاید
هم چهار سیلندر یا هرچند تا که دوست دارید.

\section{بخش چهارم}
به سیلیکون‌ولی خوش آمدید. اولین کاری که من بعد از ورود به این کهکشان
غریب کردم، سر زدن به ستاره‌ها بود.

یک ایمیل از منشی استیو جابز دریافت کردم که می‌گفت استیو جابز بسیار
خوشحال خواهد شد اگر همدیگر را ببینیم و یکی دو ساعتی را با هم
بگذرانیم. بدون اینکه بدانم داستان چیست، جواب مثبت دادم.

ملاقات در دفتر مرکزی اپل که در آوی توانیان\LFootnote{Avie Tevanian} هم
حضور داشت. در آن روزها، اپل داشت روی \lr{OS X} کار می‌کرد که
سیستم‌عامل مبتنی بر یونیکسی بود که در نهایت در سپتامبر ۲۰۰۰ به بازار
آمد. ملاقات چندانی رسمی نبود و جابز سعی کرد به من توضیح دهد که در
دنیای سیستم‌عامل‌های رومیزی، دو بازیگر بیشتر حضور ندارند: اپل و
مایکروسافت و بهترین کار برای من این است که با اپل روی هم بریزم و
طرفداران بازمتن را به حمایت از اپل و \lr{OS X} ترغیب کنم.

من صحبت را ادامه دادم چون دوست داشتم درباره سیستم‌عامل جدید بیشتر
بدانم. این سیستم‌عامل مبتنی بر میکروکرنلی به نام
تالیجنت\LFootnote{Taligent} استفاده کردند که هیچ گاه به جایی نرسید.

جابز سعی کرد با گفتن اینکه کرنل سطح پایین ماخ بازمتن است، من را تحت
تاثیر قرار دهد ولی این را نگفت که وقتی لایه مک روی کرنل قرار می‌گیرد،
یک برنامه بسته و انحصاری است، بازمتن بودن لایه پایینی برای کسی ارزش
چندانی ندارد.

او راهی نداشت تا بداند که به نظر شخصی من ماخ چیز چندان دلچسبی هم
نیست. صادقانه بگویم که ماخ به نظر من چیز چرتی است. ماخ حاوی همه مشکلات
ممکنی است که یک طراح احتمال دارد بکند و البته چند اشتباه طراحی جدید هم
دارد. یکی از انتقادات همیشگی به میکروکرنل، سرعت کند آن بوده است و به
همین خاطر افراد زیادی روی این مساله تحقیق کرده‌اند که چگونه می‌توان
راندمان این سیستم‌ها را ارتقا داد. طراحان ماخ سعی کردند همه این
پیشنهدات را پیاده کنند و در نتیجه ماخ سیستمی بسیار پیچیده شد که هنوز
هم راندمانش بهبود چندانی پیدا نکرده.

آوی توانیان یکی از کسانی بود که در دوره دانشجویی روی ماخ کار کرده
بود. صحبت کردن درباره چیزهایی که به نظر او و جابز مشکل‌زا می‌رسیدند جالب
بود. بخصوص که ما در مورد مسایل تکنیکی نظرات کاملا متفاوتی داشتیم. به
نظر من هیچ دلیلی وجود نداشت که علاقمندان لینوکس در آن پروژه مشارکت
کنند. البته درک می‌کنم که چرا آن‌ها به دنبال کسب همکاری برنامه‌نویسان
بازمتن بودند؛ آن‌ها می‌دیدند که نیروی جنبشی پشت لینوکس در حال افزایش است
ولی به نظرم متوجه نشده بودند که این انرژی تا چه حد زیاد است. به نظرم
جابز احساس نمی‌کرد که لینوکس این توان بالقوه را دارد که کاربرانی بیشتر
از مک داشته باشد.

توضیح دادم که چرا ماخ را دوست نداشتم. دلایل من دلایلی قابل فهم بودند
که تشریح دقیق شان بسیار مشکل بود. آن‌ها هم مطمئنا قبلا این استدلال‌های
مخالف را شنیده بودند. مشخص بود که من شدیدا طرفدار لینوکس بودم و آن‌ها
شدیدا طرفدار مچ. صحبت درباره شیوه برخورد آن‌ها با بعضی از مشکلات فنی
جذاب بود. یکی از مشکلاتی که از همان موقع مشخص بود، برنامه حمایتی آنان
از نرم‌افزارهای قدیمی مک در سیستم‌عامل جدید بود. آن‌ها می‌خواستند با یک
لایه سازگار کننده، کل برنامه‌های سابق را قابل اجرا نگه دارند. قرار بر
این بود که با استفاده از این لایه کل برنامه‌های قدیمی اجرا شوند. ولی
یکی از نقاط ضعف جدی سیستم‌های قدیمی این بود که حفاظت حافظه در آن‌‌ها وجود
نداشت و راه حل فعلی هیچ فکری برای این موضوع نکرده بود. قرار بود فقط
نرم‌افزارهای جدید، حفاظت حافظه داشته باشند و من این را درک نمی‌کردم.

ما در نگاه مان به جهان تفاوت‌های بنیادی داشتیم. استیو دقیقا همان استیوی
بود رسانه‌ها از او ساخته‌اند. بسیار علاقمند به اهدافش و بخصوص مسایل
مربوط به سهم اپل از بازار. من علاقمند به مسایل فنی بودم و توجهی به
اهداف او یا بحث‌هایش نداشتم. استدلال اصلی او این بود که اگر من به دنبال
بازار کامپیوترهای رومیزی هستم، باید با او متحد شوم. جواب من این بود:
چرا باید علاقمند به این بازار باشم؟ و چرا باید علاقمند به ماجرای اپل
باشم؟ به نظرم من در دنیای اپل هیچ چیز جذابی وجود ندارد و هدف من هم
تسخیر دنیای کامپیوترهای رومیزی نیست (البته شکی نیست که این امر اتفاق
خواهد افتاد ولی هدف من نبوده است).

او استدلال‌های چندانی نداشت. پیش‌فرض او این بود که من از این اتحاد
خوشحال خواهم شد. به هیچ وجه نمی‌پذیرفت و برایش غیرقابل درک بود که ممکن
است بخشی از بشریت وجود داشته باشد که علاقمند به بالابردن سهم اپل از
کامپیوترهای رومیزی نباشد. فکر کنم با دیدن اینکه من هیچ علاقه‌ای به
دانستن میزان سهم اپل از بازار - یا مایکروسافت از بازار - ندارم، شگفت
زده شد. به هیچ وجه هم حق ندارم او را برای اینکه نمی‌دانست من از ماخ بدم
می‌آید، سرزنش کنم.

البته با وجود اینکه تقریبا با هر چیزی که گفت مخالف بودم، یک جورهایی از
او خوشم آمد.

بعد برای اولین بار بیل جوی\LFootnote{Bill Joy} را دیدم و به عنوان
اعتراض ترکش کردم.

خب. بگذارید صادق باشیم. اولین باری که او را دیدم متوجه نبودم که چه کسی
را دیده‌ام. ما در یک پیش‌نمایش جینی\LFootnote{Jini} همدیگر را
دیدیم. جینی زبان تعاملی سان میکروسیستمز و یک افزونه جاوا بود. کار اصلی
این زبان ایجاد ارتباط بین سیستم‌های مختلف بود. می‌توانستید یک چاپگر آگاه
از جینی را تقریبا با هر دستگاه دیگری که بتواند به جینی صحبت کند، به
شکل خودکار کنترل کنید.

سان، از من و حدود ده دوازده نفر فعال بازمتن و آدم فنی دیگر دعوت کرده
بود تا طی \textbf{جهان جاوا}\LFootnote{Java World} در اتاقی از یک هتل
در سن جونز پیش‌نمایشی از این سیستم را ببینیم. دلیل دعوت من این بود که
سان می‌خواست جینی را تحت چیزی عرضه کند که خودش به آن \dbquote{بازمتن}
می‌گفت.

وقتی به آنجا رفتم تا حدی اطلاع داشتم که بیل جوی هم آنجا خواهد بود. او
فرد اصلی پشت یونیکس‌های بی.اس.دی. بود که بعدها به عنوان دانشمند ارشد به
سان پیوسته بود. قبلا هیچ وقت او را ندیده بودم. او جلو آمد و گفت که بیل
جوی است و من عکس‌العمل خاصی نشان ندادم. من برای دیدن کسی به آنجا نرفته
بودم بلکه در آنجا بودم تا ببینم نظر سان درباره بازمتن چیست و چگونه
می‌خواهد به این دنیا وارد شود. چند دقیقه بعد، خود بیل مشغول توضیح در
این مورد که چرا تصمیم‌گرفته‌اند پروژه را به شکل بازمتن عرضه کنند و بعد
هم شیوه کار آن را نمایش داد.

بعد شروع کردند به توضیح مفاد مجوز کاربری برنامه. وحشتناک بود. شاید هم
ابلهانه. خلاصه جریان این می‌شد که اگر کسی بخواهد از سیستم حتی به شیوه‌ای
نیمه‌تجاری هم استفاده کند، دیگر برنامه بازمتن نخواهد بود. به نظرم این
یک ایده کاملا احمقانه بود. از این ناراحت بودم که موقع دعوت روی بازمتن
بودن برنامه پافشاری کرده بودند. بازمتنی از نظر آن‌ها این بود که شما
می‌توانستید متن برنامه را بخوانید ولی همین که می‌خواستید در آن تغییرات
ایجاد کنید یا از آن در زیرساخت‌های خود استفاده کنید، باید از سان مجوز
می‌گرفتید. اگر یک نفر در ردهت تصمیم می‌گرفت که نسخه بعدی سی دی ردهت را
با جینی سازگار کند، لازم بود که برای این کار از شرکت سان درخواست مجوز
بکند.

چند سوال کردم تا مطمئن شوم که موضوع را درست درک کرده‌ام. 

بعد به شکلی اعتراض آمیز از جلسه خارج شدم. 

شدیدا از این ناراحت بودم که آن‌ها آدم‌ها را با ادعای بازمتنی به آنجا
کشیده‌اند و وقتی فهمیدم که برنامه واقعا بازمتن نیست، گفتم \dbquote{بی
  خیالش بشوید. من علاقمند نیستم.} و بیرون رفتم.

به نظرم این طور آمد که من آنجا هستم تا هم سیستم را ببینم و هم اگر از
آن خوشم آمد و چیزی گفتم، درباره جینی نقل قولی از من در رسانه‌ها
باشد. به آن‌ها پاتک زدم و امیدوارم از آن درس گرفته باشند. بعدها کسان
دیگری آن‌ها را متقاعد کردند که استارآفیس\RFootnote{\lr{Star Office} - بسته
  نرم افزارهای اداری شرکت سان میکروسیستمز.} را آزاد کنند. به نظرم زمان
معلم خوبی است.

به من گفتند که جلسه ادامه پیدا کرد و شام هم خوردند و به جز من همه همان
جا ماندند.

ملاقات بعدی با بیل جوی تجربه بسیار بهتری بود. تقریبا یکسال و نیم بعد،
او مرا دعوت به خوردن سوشی کرد.

منشی‌اش زنگ زد تا قرار را تنظیم کند. بیل در کلورادو کار و زندگی می‌کند و
ماهانه یک هفته را در سیلیکون‌ولی می‌گذراند. ما به سانیویل\LFootnote{
  Sunnyvale} که بهترین سوشی تُن ادویه زده را دارد قابل مقایسه نیست؛ ولی
خوب است.

خب، ما در فوکی سوشی نشسته بودیم و بیل داشت سعی‌ می‌کرد برای ما
واسابی\RFootnote{سس تند سوشی که در بهترین حالت از گیاهی به همین نام
  گرفته می‌شود.} واقعی مهیا کند. من آن موقع این را نمی‌دانستم ولی بعد
فهمیدم که ادویه‌ای که به عنوان واسابی در اکثر سوشی فروشی‌های آمریکا سر
میز گذاشته می‌شد،‌ در اصل ادویه‌ای است نزدیک به واسابی و نه خود
واسابی. بیل همانجا کشف کرد که گیاه واسابی فقط در ژاپن رشد می‌کند و
میزان رشد آن‌هم پاسخگوی نیازهای تجاری نیست. بیل داشت سعی می‌کرد این
موضوع را پیشخدمت توضیح دهد ولی او نکته را نمی‌گرفت. پیشخدمت دختری ژاپنی
بود که دائما اصرار می‌کرد واسابی، واسابی است. بیل از او خواست تا در این
مورد از آشپزها توضیح بخواهد.

این آمد و رفت یک جورهایی بامزه بود. این یک ناهار اجتماعی بود. بیل به
من اطمینان داد که اگر بخواهم برای سان کار کنم، کافی است لب تر کنم و او
کارها را درست خواهد کرد. ولی موضوع اصلی این نبود. او برای من تعریف کرد
که چطور پنج سال مسوول نگهداری کد یونیکس بی‌.اس.دی. بوده است و برایم گفت
که این امر باعث شد فرصت کار تجاری خوبی در سان برایش فراهم شود. گفت که
از نظر او، این امکان که شرکت بزرگی مثل سان بتواند آینده مالی آدم را
تامین کند، امر مهمی است. وقتی از روزهای اول یونیکس حرف می‌زند من واقعا
جذب شدم. اینکه در آخر کار هم نتوانستیم واسابی اصل را بچشیم به نظرم اصلا
مهم نیامد. کاملا یادم هست که به نظرم او احتمالا بهترین و جذاب‌ترین آدم
مشهوری آمد که در سیلیکون‌ولی دیده بودم.

سه سال جلو برویم. به محض برداشتن مجله وایرد\RFootnote{\lr{Wired
    magazine} - مجله تکنولوژیک که این روزها سایتی هم به همنن نام را
  اداره می‌کند.}، با مقاله منفی او در مورد تکنولوژی با عنوان
\textbf{آینده نیازی به ما نخواهد داشت} مواجه شدم. یک جورهایی ناراحت
  شدم. مشخص است که آینده نیازی به ما نخواهد داشت اما لازم نیست اینقدر
  منفی در این باره صحبت کنیم.

حسم این نبود که مقاله را جر و واجر کنم ولی به نظر من یکی از
ناراحت‌کننده‌ترین چیزهایی که ممکن است برای انسانیت رخ دهد، این خواهد بود
که همین طور حرکت کند و حرکت کند. این خلاف تکامل است. به نظر می‌رسید که
در نگاه بیل، پیشرفت‌هایی مثل مهندسی ژنتیک باعث از دست رفتن انسانیت
خواهند بود. افراد همیشه تصور کرده‌اند که هر چیزی غیر از آن چیزی که ما
این روزها هستیم، چیزی به دور از انسانیت است. مشخص است که ما در طول
۱۰۰۰۰ سال تکامل خواهیم یافت و با معیارهای امروز، دیگر انسان نخواهیم
بود. از نظر من، آن‌ روز ما تعریف جدیدی از انسان خواهیم داشت.

از مقاله بیل این طور برمی‌آمد که نگران این موضوع است. به نظر من تلاش
برای جلوگیری از تکامل، غیرطبیعی و بی‌نتیجه است. به جای گشتن به دنبال دو
سگ مناسب برای جفتگیری با هم، ما می‌توانیم به سراغ مهندسی ژنتیک برویم و
شکی نیست که این مساله بالاخره برای انسان‌ها هم اتفاق خواهد افتاد. به
نظر من استفاده از ژنتیک برای بهبود نژاد، بهتر از زیستن در وضع موجود
است. به نظر در مقایسی کلی‌تر، شدیدا ارزشمند است اگر بتوانیم تکامل نه
فقط انسان‌ها، که جوامع را هم تضمین کنیم. جوامع هم باید بتوانند در
مسیرهایی که حس می‌کنند پیش بروند. شما نمی‌توانید تکنولوژی را متوقف کنید
یا سعی‌کنید جلوی پیشرفت دانش بشری و ادراک اینکه جهان و ما چگونه ساخته
شده‌ایم را بگیرید. مساله این است که این پیشرفت آن قدر سریع شده که
افرادی مثل بیل جوی آن را وحشتناک می‌یابند. اما به نظر من این پیشرفت
بخشی طبیعی از تکامل ما است.

من با بیل اختلاف نظر دارم؛ چه در مورد شیوه تعامل ما با آینده و چه در
مورد مفهوم بازمتن بودن نرم‌افزار. با استیو جابز در مورد تکنولوژی اختلاف
نظر دارم. شاید به نظر برسد که در طول سال اول اقامتم در سیلیکون‌ولی همه
وقتم به مخالفت گذشته است اما این حقیقت ندارد. من در طول سال اول کلی کد
نوشتم و پاتریشیا را به باغ وحش بردم تا به حیوان‌ها غذا بدهد و در کل،
مشغول وسعت دادن دیدگاهم بودم - مثلا همان حقیقتی که درباره واسابی
آموختم.

\section{بخش پنجم}
\subsection*{موفقیت یک شبه ما}
آیا گروه‌های خبری تبلیغاتی را می‌خوانید؟ تمام فعالیت آن‌ها این است که
چیزی را مشهور کنند و این یعنی کل تلاش آن‌ها این است که چیز دیگری را
پایین بکشند. اگر به آن‌ها داخل شده باشید می‌دانید که چیزی جز این مهملات
که \dbquote{سیستم من از سیستم تو بهتر است} در آن پیدا نمی‌شود. این یک
نوع استمناء آنلاین است.

دلیلی که بحث گروه‌های خبری تبلیغاتی را پیش کشیدم این است که با وجود
پوچی مطالب، آن‌ها می‌توانند سرنخی باشند برای اینکه ببینید چه چیزی در حال
وقوع است. وقتی شرکت‌ها برای اولین بار به این نتیجه رسیدند که لینوکس
سیستم‌عامل دوست‌داشتنیی است، ایده حمایت تجاری، برای اولین بار در مطبوعات
مطرح نشد بلکه برای اولین بار این گروه‌های خبری تبلیغی بودند که به این
فکر افتادند.

بگذارید مرتب پیش بروم. در بهار ۱۹۹۸، سومین بلوند به دنیای من گام
گذاشت: دانیلا یولاندا توروالدز\LFootnote{Daniela Yolanda Torvalds} در
۱۶ آوریل متولد و به عنوان اولین توروالدزی که شهروند ایالات متحده است،
ثبت شد. او شانزده ماه با پاتریشیا اختلاف سن داشت، یعنی دقیقا برابر با
اختلاف سن و من سارا. البته به خاطر برخورد معتدل کننده تاو، مطمئن هستم
که جنگ‌های من و خواهرم بین آن‌ها تکرار نخواهد شد.  شاید هم به خاطر
توانایی‌های کاراته‌اش.

دو هفته قبل از تولد دانیلا، جامعه بازمتن - که تا همین چند وقت جامعه
نرم‌افزار آزاد خوانده می‌شد - با بزرگترین جهش خود مواجه شد. این اتفاق
زمانی افتاد که موزیلا\LFootnote{Mozilla} بازمتن اعلام کرد. از یک طرف
این خبر باعث شادی تمام گروه‌های خبری شد. چون می‌دانستند که بازمتن، بیشتر
و بیشتر در رسانه‌ها خبرساز خواهد شد. در طرف مقابل، این خبر خیلی‌ها از
جمله من را دچار استرس کرد. نت‌اسکیپ آن روزها دچار مشکل بود و با تشکر
بسیار از مایکروسافت، بازمتن اعلام شدن مرورگر موزیلا، حرکتی از سر
ناامیدی ارزیابی شد (جالب است بدانید که این مرورگر ریشه در مفاهیم بازمتن
داشت و به عنوان یک پروژه در دانشگاه ایلینویز شروع شده بود).

افراد حاضر در گروه‌های خبری می‌گفتند که این حرکت باعث هدر رفتن زحمت‌ها
خواهد شد و نام بازمتن را خراب خواهد کرد. حالا دو پروژه بازمتن بزرگ
وجود داشت: موزیلا و لینوکس. و اگر نت‌اسکیپ که بیشتر شناخته شده بود شکست
می‌خورد، بدنامی آن دامن لینوکس را هم می‌گرفت.

و نت‌اسکیپ تا حد زیادی هم شکست خورد. شرکت برای مدتی طولانی در جذب
برنامه‌نویسان بازمتن برای مرورگرش ناموفق بود. این برنامه یک کد عظیم بود
که فقط و فقط کارمندان خود نت‌اسکیپ از آن سردرمی‌آوردند.

پروژه نه فقط از این نظر نفرین شده بود که کد آن بسیار بزرگ بود، بلکه
این مشکل هم وجود داشت که نت‌اسکیپ فقط می‌توانست بخشی از آن یعنی نسخه تحت
توسعه را بازمتن کند که آن هم در زمان انتشار، شدیدا مشکل داشت. شرکت
نمی‌توانست تمام کد را تحت جی.پی.ال. منتشر کند، چون صاحب بخش‌هایی از آن
نبود. برای مثال کدهای جاوا تحت مجوز سان بودند. در گروه‌های خبری هم
کسانی بودند که با مجوز نت‌اسکیپ مشکل داشتند. در کل، مجوز ارائه شده خیلی
مهربانانه بود ولی افرادی مثل ریچارد استالمن حاضر نیستند با یک مجوز نرم
و نازک کنار بیایند.

من از اینکه نت‌اسکیپ این گام را برداشته بود بسیار خوشحال بودم ولی آن را
یک موفقیت شخصی نمی‌دانستم. یادم هست که اریک ریموند\RFootnote{\lr{Eric
    Raimond} - از شخصیت‌های بسیار تاثیرگذار دنیای گنو/لینوکس} برداشتی
شخصی از این جریان داشت. او از این جریان خیلی خوشحال بود. مقاله او به
نام \dbquote{کلیسای جامع و بازار} که فوق‌العاده خوب فلسفه بازمتنی و
تاریخچه آن را توضیح می‌داد، یک سال قبل چاپ شده و به عنوان یکی از دلایل
بازمتن شدن موزیلا، از آن نام برده شده بود. او به شکلی فعال جنبش بازمتن
را به پیش می‌راند. او چندین بار در مناسبت‌های مختلف به نت‌اسکیپ رفته و از
آن‌ها خواسته بود تا مرورگر خود را بازمتن اعلام کنند. من فقط یک بار این
کار را کردم. در واقع اریک با پیام بازمتنی، به شرکت‌های مختلفی رفته
است. من بیشتر اهل تکنولوژی بودم تا پیامبری.

بیست و چهار ساعت از بازمتن شدن موزیلا نگذشته بود که تیمی استرالیایی به
نام گروه موزیلا کریپتو\LFootnote{Mozilla Crypto Group} ماجولی برای
رمزگذاری ارتباطات ارائه کرد. آن روزها شهروندان غیرآمریکایی اجازه
نداشتند از سایت‌های رمزگذاری شده آمریکایی استفاده کنند. حالا در یک شب
گروهی استرالیایی کاری کرده بودند که تمام مردم جهان می‌توانستند با
استفاده از آن به هر سایتی که می‌خواهند متصل شوند. البته یک مشکل هم وجود
داشت. با توجه به قوانین صادرات تجاری آمریکا، موزیلا نمی‌توانست از یک کد
استرالیایی در برنامه خود استفاده کند. این شد که نت‌اسکیپ نتوانست از یکی
از بهترین موفقیت‌های بازمتن کردن برنامه‌اش استفاده کند.

ما بسیار نگران این بودیم که نت‌اسکیپ موضوع روز رسانه‌ها شده بود. در طی
سال اول، ما بسیار محتاط حرکت می‌کردیم، چون از این می‌ترسیدیم که کسی نقد
کوچکی از موزیلا بکند و مطرح شدن این موضوع در رسانه‌ها باعث شود که دیگر
شرکت‌ها نسبت به بازمتن کردن نرم‌افزارهای شان بدبین شوند.

ولی فقط دو ماه بعد از تصمیم بزرگ نت‌اسکیپ، سان میکروسیستمز هم به بازی
بازمتن پیوست و اعلام کرد که اولین شرکت بزرگ سخت‌افزاری خواهد بود که به
لینوکس اینترنشنال می‌پیوندد. سان می‌خواست روی سرورهایش از لینوکس پشتبانی
کند. شرکتی که قبلا آن مجوز دوست‌نداشتنی را روی جینی گذاشته بود، حالا به
این نتیجه رسیده بود که لینوکس ارزش جدی گرفته شدن دارد. گروه‌های خبری از
تبریک افراد به خودشان پر شد. حالا که سان با ما بود، اخبار توسعه لینوکس
از گروه‌های خبری به رسانه‌های تجاری منتقل شده بود. افرادی خارج از
گروه‌های ما هم به ناگهان به لینوکس علاقمند شدند، البته افراد فنی.

و حالا نوبت آی‌.بی.ام. بود. 

آی.بی.ام. همیشه به سخت‌گیر بودن مشهور بود و در نتیجه وقتی این شرکت در
ماه ژوئن اعلام کرد که آپاچی یعنی مشهورترین نرم‌افزار وب‌سرور لینوکس را
روی کامپیوترهایش خواهد فروخت و از آن پشتیبانی خواهد کرد، همه به وجد
آمدند. قبل از این می‌توانستید آپاچی را روی \lr{AIX} یا یونیکس
آی.بی.ام. اجرا کنید و این دقیقا همان کاری بود که اکثر خریداران این
سرورها، به دنبالش بودند. سر همین جریان، آپاچی مورد توجه آی.بی.ام. قرار
گرفت. احتمالا کسی در آی.بی.ام. فهمیده بود که اکثر افراد روی سرورهای
خریداری شده آپاچی نصب می‌کنند و نتیجه گرفته بود که اگر آی.بی.ام. کسانی
را داشته باشد که این سرورها را از پیش روی سیستم‌عامل نصب کنند و
پشتیبانی آن را نیز ارائه دهند، فروش بالا خواهد رفت. شاید هم این تصمیم
نتیجه بازخورد افرادی بود که به آی.بی.ام. می‌گفتند که قصدشان از خرید
سرور، اجرای آپاچی است.

نصب لینوکس روی یک کامپیوتر کار نسبتا ساده‌ای است. در عوض چیزی که در
اکثر شرکت‌ها دردسرساز می‌شود پاسخ به این سوال است که در صورت بروز مشکل
چه کسی باید شماتت شود؟ مطمئنا شرکت‌هایی مثل ردهت بودند که این پشتیبانی
را تقبل می‌کردند، ولی حضور غولی همچون آی.بی.ام. برای پشتیبانی، تاثیر
زیادی روی احساس رضایت مشتریان داشت. خیلی‌ها در ابتدا این برداشت را
داشتند که این سرویس آی.بی.ام. یک سرویس محدود و حاشیه‌ای خواهد بود، اما
معلوم شد که این طور نیست. آی.بی.ام. با پشتیبانی از لینوکس و آپاچی در
سرورهایش فقط مشغول آزمایش آب دریا با نوک پا بود و بعد از احساس رضایت،
شروع به پشتیبانی از لینوکس روی سرورها کرد و بعد با نصب لینوکس روی
سرورها، با سر به دنیای بازمتن شیرجه زد. بعد هم نوبت سرورهای کوچکی بود
که روی کامپیوترهای شخصی اجرا می‌شدند و بعد هم کامپیوترهای شخصی معمولی و
بعد هم لپ‌تاپ‌ها. امسال آی‌بی‌ام اعلام کرد که در بودجه سالانه‌اش یک میلیارد
دلار صرف لینوکس خواهد کرد.

آی.بی.ام. بخش عمده‌ای از فعالیت‌های لینوکسی‌اش را به تنهایی انجام داده
است. فکر کنم اصولا یکی از دلایل علاقه آن‌ها به لینوکس این باشد که
می‌توانند هرکار که به ذهن شان می‌رسد بکنند، بدون اینکه نگران مسایل مربوط
به مجوزهای مختلف باشند. این کمپانی چشمش از مسایل مربوط به مجوز ترسیده
است. مایکروسافت یک بار با پروژه دوساله \lr{OS/2} که بعدا معلوم شد چیزی
بیشتر از یک ویندوز دوپینگ کرده نیست، ترتیب آی.بی.ام. را داده
بود. مایکروسافت از \lr{OS/2} پشتیبانی نکرد، چون حاضر نبود بازارش را با
آی.بی.ام. تقسیم کند. مایکروسافت ویندوز ان.تی. را از طرف خودش به بازار
عرضه کرده بود و دوست نداشت این سیستم‌عامل رقیبی از طرف آی.بی.ام. داشته
باشد. آی.بی.ام. هیچ وقت پول‌های میلیاردی که خرج \lr{OS/2} کرده بود را
زنده نکرد. یک بار هم همین بلا با جاوا سر آی.بی.ام. آمد. فکر کنم از
اینکه قرار نبود درباره مجوز لینوکس از کسی اجازه بگیرند، بسیار خوشحال
بودند.

هیچ شکی نیست که آی.بی.ام. بزرگترین خبر برای لینوکس بود و هیجان زیادی
در گروه‌های خبری ایجاد کرد؛ البته نه از آن نوع هیجان استرس‌زای مربوط به
اخبار نت‌اسکیپ یا بحث‌های احساسی که گاه گداری (باشه؛ معمولا!) در مورد
مبارزه با کالایی شدن لینوکس از طرف هواداران دو آتشه‌اش در می‌گیرند.

جولای آن سال به پایان نرسیده بود که اینفورمیکس\LFootnote{Informix}
اعلام کرد که بانک‌اطلاعاتی‌اش را به لینوکس پورت خواهد کرد. به عبارت دیگر
اعلام کرده بود که اگر شما از لینوکس استفاده کنید، خواهید توانست از
بانک‌های اطلاعاتی اینفورمیکس بهره بگیرید. این مساله آن‌روزها خیلی مهم
تلقی نشد. شرکت مشکلات مالی زیادی داشت هرچند که هنوز هم یکی از سه شرکت
اصلی بانک‌اطلاعاتی به حساب می‌آمد. آدم‌های لینوکسی از این ماجرا هم راضی
بودند و باز گروه‌های خبری آکنده شده از پیام‌های تبریک.

چند هفته نگذشته بود که به ناگهان اوراکل\LFootnote{Oracle} هم پیوستن
خود به طرح را اعلام کرد. اوراکل سلطان بانک‌های اطلاعاتی بود. مدت‌ها قبل
از اعلام خبر در گروه‌های خبری این موضوع درز کرده بود که اوراکل مشغول
پورت کردن بخش‌هایی از نرم‌افزارهایش به لینوکس است. از آنجایی که اوراکل
عملا یک شرکت یونیکسی بود، پورت کردن نباید کار سختی بوده باشد. اما اگر
گروه‌های خبری را دنبال می‌کردید، می‌دیدید که پورت شدن اوراکل به لینوکس
حتی اگر از نظر فنی کار مهمی نبوده باشد، از نظر روانی تاثیر عظیمی روی
جامعه گذاشته بود.

تصمیم اوراکل، درست مانند اطلاعیه قبلی از طرف آی.بی.ام.، باعث جلب توجه
گروهی جدید به جز طرفداران لینوکس هم شد. این گروه افرادی بودند که معمولا
خودشان را تصمیم‌گیران مدیریتی می‌خوانند، هرچند که بین ما بیشتر به نام
\dbquote{کت‌شلوارپوش‌ها} مشهورند. آن‌ها دیگر نمی‌توانستند به این بهانه که
سازمان‌شان وابسته به بانک‌های اطلاعاتی است، لینوکس را کنار بگذارند.

این اخبار با وجود خوشایند بودن، تغییری در زندگی من ایجاد نکردند. من و
تاو به دو بچه دوست‌داشتنی‌مان سرگرم بودیم. اکثر اوقات غیرخانوادگی من صرف
نگهداری لینوکس می‌شد؛ چه در خانه و چه در دفتر. برای جلوگیری از ترجیح
یکی از نسخه‌های لینوکس توسط من، در دفتر کار از ردهت و در محیط کار از
زوزه که نسخه‌ای اروپایی است استفاده می‌کردم. یک روز فکر کردم به اندازه
کافی ورزش نمی‌کنم و شروع کردم به رکاب زدن مسیر ده کیلومتری خانه تا دفتر
ترنسمتا. دوشنبه بود. راه سربالایی نداشت ولی باد شدید مخالفی می‌وزید که
کار را از آنچه من می‌خواستم سخت‌تر می‌کرد. ده ساعت بعد که کارم در دفتر
تمام شد، باد معکوس شده بود و بازهم باید در خلاف جهت باد رکاب می‌زدم. به
تاو زنگ زدم و با ماشین به دنبالم آمد و سوارم کرد. نیازی به گفتن نیست
که دیگر هیچ وقت سعی نکردم این مسیر را رکاب بزنم.

این خاطره بی‌ضرر را گفتم تا تاکید که کنم که توسعه لینوکس، باعث تغییر
زندگی من نشده بود. اکثر کارها در شرکت‌ها انجام می‌شد. مدیران شرکت که
دائما مقاله‌های درمورد لینوکس در مطبوعات را می‌دیدند به سراغ آدمی‌های فنی
شرکت‌شان می‌رفتند که به فعالیت‌های لینوکسی شهره بودند و از آن‌ها جویا
می‌شدند که این سر و صداها برای چیست. دانستن تفاوت‌ها و آگاه شدن از
مزایای لینوکس، آن‌ها را وا می‌داشت که از لینوکس روی سرورهای شان استفاده
کنند.

این داستان، ماجرایی بود که در همه دفاتر تکنولوژی اطلاعات سراسر جهان در
حال گسترش بود ولی نمی‌شود انکار کرد که اصلی‌ترین مرکز آن، ایالات متحده
بود. در این تصمیم، مساله رایگان بودن لینوکس کمترین نقش را داشت چرا که
در سرمایه‌گذاری‌هایی به این وسعت، نرم‌افزار هزینه چندانی محسوب
نمی‌شود. معمولا سرویس و پشتیبانی بیشترین هزینه را به خود اختصاص
می‌دهند. مساله‌ای که باعث می‌شود کت‌شلوارپوش‌ها لینوکس را انتخاب کنند یک
واقعیت فنی ساده بود: لینوکس از تمام رقبا قوی‌ترین بود، از جمله ویندوز
ان.تی. و انواع یونیکس. در عین حال این هم بسیار مهم بود که شرکت‌ها از
این متنفر بودند که بخشی از تصمیم‌گیری‌های آن‌ها در دست شرکت‌هایی همچون
مایکروسافت یا شرکت‌های صاحب یونیکس باشد. شما می‌توانید کارهایی با لینوکس
بکنید که با هیچ سیستم‌عامل دیگری نمی‌توانید بکنید. اولین کاربران لینوکس
به این دلیل لینوکس را انتخاب کرده بودند که می‌توانستند بر خلاف تمام
نرم‌افزارهای انحصاری دیگر، به متن آن دسترسی داشته باشند.

از این منظر که نگاه کنیم، از زمانی که من نسخه \code{0.01} را از اتاق
خوابم منتشر کردم،‌ چیز زیادی تغییر نکرده است. لینوکس از همان زمان از
دیگر سیستم‌ها قابلیت تطابق بیشتری داشته. شما می‌توانستید رییس خودتان
باشید. و حداقل در مورد سرورهای وب، این سیستم‌عامل از پف و باد بیخودی -
قابلیت‌های غیرمرتبطی - که دیگر سیستم‌عامل‌های رقیب را سنگین کرده بود،
عاری بود.

نکته دیگری هم بود که باعث موفقیت لینوکس شد: با وجود شهرت فزاینده‌اش به
عنوان یک سرور وب، لینوکس می‌توانست پایش را در هر کفشی بکند. درک این
موضوع برای فهم موفقیت لینوکس الزامی است.

کامپیوترهای بزرگ و قدیمی، تک منظوره بودند. حتی یونیکس هم چند جای پای
خاص داشت؛ مثلا سوپرکامپیوترهای وزارت دفاع آمریکا یا بانک‌ها. آدم‌هایی هم
بودند که کارشان فروختن سیستم‌عامل به این مراکز خاص بود. آن‌ها پول زیادی
در می‌آوردند چون برای هر سیستم‌عامل، کلی صورت حساب می‌دادند. بعد
مایکروسافت آمد که برای هر سیستم‌عامل نود دلار پول می‌گرفت. مایکروسافت به
دنبال بازار بانک‌ها و وزارت‌خانه‌ها نبود اما چشم مان را که باز کردیم
دیدیم همه جا را تسخیر کرده. درست مثل حمله ملخ‌ها. از این جور حمله‌ها جان
سالم به در بردن واقعا سخت است (البته ملخ‌ها بد نیستند. من همه جانوران
را دوست دارم).

خیلی بهتر است که آدم همه جا باشد و بتواند پایش را در هر کفشی بکند و
مایکروسافت هم همین نقشه را پیش گرفته. یک ارگانیسم مایع را در نظر
بگیرید که هر جایی را که بتواند اشغال می‌کند. اگر هم جایی در امان بماند
هیچ مشکلی نیست چون کم کم همه دنیا از این مایع زنده پر خواهد شد.

لینوکس این روزها به همین شکل پیش می‌رود و مشغول پر کردن هر سوراخ جذابی
است. لینوکس تک منظوره نیست. کوچک و قابل ارتجاع است. می‌تواند به هر جایی
سرک بکشد. در سوپرکامپیوترها و جاهای مهمی مثل دستگاه‌های
درون‌ساز\RFootnote{\lr{Embeded Devices} - دستگاه‌های کامپیوتری که درون
  دیگر دستگاه‌ها جاسازی می‌شوند؛ مانند کامپیوتر خودروها.} داخل ابزارها
بیابید؛ از سیستم‌های ضد قفل ترمز گرفته تا ساعت‌های مدرن.

نگاه کنید: لینوکس واقعا دارد جاری می‌شود.

بهترین قابلیت لینوکس در عمق ساختار آن نهفته است. بهترین‌ها و
باهوش‌ترین‌های نسل آینده از محصولات شما استفاده خواهند کرد. چون چیزی که
شما نوشته‌اید، آن‌ها را به هیجان می‌آورد. در نسل جوان ما، این مایکروسافت
و داس نبود که بچه‌های باهوش را به هیجان می‌آورد بلکه کامپیوترهای شخصی
بودند که قلب آن‌ها را به ضربان وا می‌داشت. اگر شما عاشق کامپیوترهای شخصی
می‌شدید، عاشق داس هم بودید چون انتخاب دیگری نبود.

این نکته نفهته در رشد سریع مایکروسافت بود. 

اگر به باهوش‌ترین بچه‌های دور و برتان نگاه کنید، می‌بینید که خیلی از آن‌ها
به لینوکس گرایش پیدا کرده‌اند. مطمئنا یکی از دلایل اینکه چرا فلسفه
بازمتن و لینوکس در دانشگاه‌ها پیروان بیشتری دارند، \textbf{احساسات ضد
  امر مستقر}\LFootnote{Antiestablishmentsentiment} است (همان احساسی که
بخش عمده‌ای از زندگی پدر من را شکل داد.) ذهنیت این دانشجویان، نبرد بین
شرکت بزرگ و شیطانی مایکروسافت به همراه بیل گیتس-لعنتی-پولدار-حریص علیه
عاشقانه‌های-لینوکس-و-نرم‌افزار آزاد-برای-همه به همراه
لینوس-توروالدز-خودشکن-قهرمان است. این بچه‌ها بعد از فارغ‌التحصیلی در
شرکت‌های تجاری استخدام می‌شوند و عشق به لینوکس را با خود به آنجا می‌برند.

رفقایی که جرات کرده‌اند به عمق شرکت مایکروسافت نفوذ کنند، اطلاع داده‌اند
که عکس من روی صفحات دارت آنجا نصب شده. تنها حرفی که در این مورد دارم
بزنم این است: واقعا مگر ممکن است کسی نتواند دماغ من را هدف قرار دهد؟

البته بازهم دارم از خودم جلو می‌زنم. اطلاعیه آی.بی.ام. در بهار ۱۹۹۸،
اطلاعیه‌های پیاپی شرکت‌های سخت‌افزاری بزرگ دیگر را در پی داشت. در ماه
آگوست، مجله فوربس\RFootnote{\lr{Forbes} - مجله اقتصادی تجاری آمریکا}
دنیای کوچک ما را \dbquote{کشف کرد} و روی جلدش، عکسی از من چاپ کرد که
زیرش نوشته بود \dbquote{صلح، عشق، نرم‌افزار آزاد} حالا که دیگر شرکت پشت
شرکت سرسپردگی‌ غیرقابل اجتنابش به لینوکس را اعلام می‌کرد، نیازی نبود
برای خواندن آینده، گروه‌های خبری تبلیغاتی را دنبال کنم.

\section{بخش ششم}
لینوکس درست مثل یک قهرمان ناشناس جهان سوم که ناگهان برنده مدال طلای المپیک شود، قلب مردم را تسخیر کرده بود. 

من مرکز توجه بودم. اریک ریموند در یک مصاحبه خبری گفته بود که دلیل کشش
(یا هرچیز دیگری که اسمش را می‌گذارید) بیشتر رسانه‌ها به من به دلیل
\dbquote{ظاهری کمتر غیرعادی در مقایسه با دیگر هکرها} است. این نظر یک
هکر است. همه از این ماجرا راضی نبودند. ریچارد استالمن کمپینی به راه
انداخت تا با این منطق که من به کمپایلر جی.سی.سی. پروژه گنو و بسیاری
ابزار و نرم‌افزارهای آزاد دیگر وابسته بوده‌ام، نام لینوکس را به
گنو/لینوکس تغییر دهد. عده‌ای هم از این ناراضی بودند که لینوکس داشت در
دنیای تجاری جای خودش را باز می‌کرد.

رسانه‌ها نیز به این انشعاب بین ایده‌آلیست‌ها و عملگرایان (این ها لغات من
نیستند!) بین لینوکسی‌هایی که حالا تعدادشان به بیش از صدها هزار نفر
رسیده بود، دامن می‌زدند. در این تقسیم بندی گروهی که توسط رسانه‌ها
ایده‌آلیست نام گرفته بودند، اعتقاد داشتند که لینوکس با اهداف جامعه
سرمایه‌داری ناسازگار است و من هم شده بودم رهبر گروه عملگرایان. از نظر
من این تقسیم‌بندی یکی دیگر از حرف‌های بی‌معنی روزنامه‌نگاران است که فقط
برای سیاه و سفید نشان دادن جهان، کاربرد دارد (همین مشکل را با دوستانی
هم دارم که لینوکس را کلا در معنای جنگ لینوکس و مایکروسافت معنا می‌کنند
در حالی که لینوکس چیزی کاملا متفاوت و با هدفی عام‌تر است. لینوکس شیوه
رشد ارگانیک تکنولوژی، دانش، ثروت و تفریح است. شیوه‌ای متفاوت با هر آن
چیزی که در پیش از آن در دنیای تجاری وجود داشته).

برای من این مسایل اصلا موضوعیت نداشتند. بدون مصالح تجاری، لینوکس چگونه
می‌توانست به بازارهای جدید دست یابد؟ چه کسی ممکن بود به سراغ نوآوری
برود؟ چه راه دیگری وجود داشت که لینوکس به دست کاربرانی برسد که به
دنبال گزینه‌هایی به جز گزینه‌های تجاری موجود در بازار بودند؟ آیا واقعا
گزینه دیگری به جای کسب حمایت تجاری شرکت‌ها وجود داشت؟ راستی چگونه می‌شد
بدون حضور شرکت‌های تجاری، به کارهای نه چندان جذاب و حتی حوصله‌سربری
همچون پشتیبانی و حل مشکل پرداخت؟

بازمتن یعنی اینکه همه حق داشته باشند در بازی شرکت کنند. چه دلیلی دارد
در این مفهوم شرکت‌هایی که اتفاقا می‌توانند بازیگران خوبی باشند را از
بازی بیرون کنیم - البته تا وقتی که به قوانین بازی احترام می‌گذارند؟
بهترین کاری که بازمتن می‌کند این است که تکنولوژی‌های شرکت‌ها را بهبود
می‌بخشد و باعث می‌شود شرکت‌ها کم کم خست کمتری به خرج دهند.

تازه اگر هم تصمیم می‌گرفتیم که بازیگران تجاری را بیرون گود نگه داریم چه
کاری از ما ساخته بود؟ من دوست ندارم مخفی شوم، به زیرزمین بروم یا از
صحبت با شرکت‌ها طفره بروم.

احساسات ضد تجاری همیشه بخشی از جامعه بازمتن بوده است ولی تا وقتی که
لینوکس به یک کلمه مرسوم بین افراد نه چندان فنی تبدیل نشد، بروز چندانی
نداشت. بعد از شهرت لینوکس، گروه‌های خبری مرتبط پر شد از مشاجرات و گاهی
نوشته‌های بیمارگونه آدم‌های پر سر و صدا. هیچ کدام از توسعه دهندگانی که
من با آن‌ها کار کرده‌ام از این نگرانی‌ها نداشته‌اند، ولی افراد دیگری را
دیده‌ام که سر این موضوع که ردهت چگونه مشغول از بین بردن مفهوم بازمتن
است یا مردم چگونه دارند ایده‌آلیسم پشت لینوکس را فراموش می‌کنند، دائما
در حال جنگ هستند.

این مساله تا حدی صحت دارد که بعضی از فعالان بازمتن، ایده‌آلیسم سابق خود
را کنار گذاشته‌اند و به شرکت‌های تجاری پیوسته‌اند. شاید بعضی‌ها این امر را
از دست دادن نیروها بدانند ولی به نظر من این امر فقط به معنای مطرح شدن
یک گزینه جدید است. آدم‌های فنی‌ای که نگران تهیه زندگی برای کودکان شان
بوده‌اند، حالا این فرصت را دارند که شغلی درست و حسابی داشته باشند. حالا
اگر بخواهید می‌توانید به میزان سابق ایده‌آلیست باشید یا به یک شرکت تجاری
بپوندید و به خاطر کارهایی که سابقا هم می‌کردید، حقوق بگیرید. شاید سابقا
پیوستن به یک شرکت تجاری به معنی کنار گذاشتن ایده‌آل‌ها بود، ولی حالا شما
می‌توانید کارهای توسعه لینوکس را در قالب یک شرکت تجاری انجام دهید.

به هرحال من هیچ وقت این تصور را نداشتم که در گروه ایده‌آلیست‌ها
هستم. مطمئنا همیشه به این اعتقاد داشته‌ام که جنبش بازمتن در حال بهتر
کردن جهان است ولی برای من بازمتن به معنی تفریح و لذت هم هست. این مساله
شاید خیلی ایده‌آلیستی نباشد.

به نظر من، آدم‌های ایده‌آلیست دوست داشتنی هستند اما گاهی حوصله سربر و
ترسناک هم می‌شوند.

برای داشتن یک نظر خیلی قرص و محکم، باید دیگر نظرها را کنار گذاشت. این
مساله به نوبه خود به معنای غیرمنطقی شدن است و این همان چیزی است که
باعث می‌شود در مقایسه با سیاست‌های اروپایی، من سیاست‌های آمریکایی را
مشکل‌دار ببینم. در سیاست آمریکایی، حریف به عنوان دشمن تلقی می‌شود و نکات
مثبتش در سایه قرار می‌گیرد. در اروپا سیاستمداران از طریق نمایش قدرت
همکاری شان، پیروز می‌شوند.

من همیشه سعی می‌کنم تعادل را برقرار کنم. تنها دوره‌ای که واقعا نگران
تجاری شدن جریان بودم، همان روزهای اول بود که لینوکس صاحب اسم و رسم
نبود. در آن زمان اگر شرکت‌های تجاری تصمیم می‌گیرفتند که لینوکس را صاحب
شوند، شاید کاری از دست من بر نمی‌آمد. امروز دیگر وضعیت تغییر کرده. یکی
از بحث‌های دنباله‌دار گروه‌های خبری در سال ۱۹۹۸ این بود که شرکت‌های تجاری
هیچ چیزی به لینوکس برنمی‌گردانند. به نظرم من باید همان قدر که توسعه
دهندگان لینوکس به من اعتماد داشتند، به شرکت‌های تجاری اعتماد
می‌کردم. این شرکت‌ها ثابت کردند که قابل اعتمادند. هرچند که گاهی هیچ
منفعت مستقیمی به لینوکس برنگشت اما رابطه مثبت بود.

من به عنوان شخصیت مشهور دنیای لینوکس، صاحب نام تجاری آن و نگهدارنده
اصلی کرنل، احساس فزاینده‌ای داشتم. وقتی می‌دیدم که میلیون‌ها نفر در جهان
در حال استفاده از این برنامه هستند، احساس می‌کردم که باید نهایت تلاشم
را بکنم تا مطمئن شوم که لینوکس به اندازه کافی و تا حداکثر ممکن، قابل
اتکا است. برایم مهم بود که شرکت‌ها معنای حقیقی بازمتنی را درک کنند و تا
جایی که من می‌دیدیم، هیچ جنگی بین شرکت‌های حریص و هکرهای انسان‌دوست در
جریان نبود.

نه، وقتی اینتل از من خواست تا برای حل باگ \code{F0 0F} پنتیوم که باعث
قفل شدن این پروسسورها می‌شد به آن‌ها کمک کنم ایده‌آل‌هایم را کنار نگذاشته
بودم. (\dbquote{باگ \code{F0 0F} پنتیوم}؟ بله دوباره ما مهندس‌های عجیب
و غریب هستیم که اسم‌های عجیب و غریب ابداع می‌کنیم. \code{F0 0F} دو بایت
اول دستورات غیرمجازی هستند که باعث هنگ کردن پروسسسورهای پنتیوم
می‌شدند. این اسم هم از همین دستورات گرفته شده). و همچنین وقتی که در
شرکتی کار گرفتم که آن قدر بسته بود که حتی درباره اینکه مشغول چه کاری
است به کسی توضیح نمی‌داد، بازمتن فکر کردن را کنار نذاشتم. واقعیت این
است که هنوز هم چیپی که ترنسمتا در حال ساخت آن بود را یکی از جذاب‌ترین
تکنولوژی‌های حال حاضر می‌دانم و معتقدم که قابلیت‌های بسیاری در زندگی
روزمره ما خواهد داشت. برای ثبت در تاریخ باید بگویم که من یکی از کسانی
بودم که در انتشار بخشی از کد آن پروژه، نقش داشت.

من احساس می‌کردم که برای حفظ موقعیت خودم به عنوان کسی که هم از نظر
تکنولوژیکی و هم از نظر اخلاقی قابل اعتماد است، از طرف جامعه بازمتن تحت
فشارم. برای من مهم بود که بین شرکت‌های رقیبی که لینوکس‌های متنوع را عرضه
می‌کردند، از هیچکدام طرفداری نکنم. نه، من با پذیرفتن سهام پیشنهادی رد
هت که به خاطر تشکر به من داده شده بود، خودم را نفروختم. من آن را قبول
کردم چون ایرادی نداشت. اما پیشنهاد ۱۰ میلیون دلاری یک تاجر انگلیسی که
می‌خواست با پرداخت این مبلغ من را به هیات مدیره شرکت لینوکسی‌اش اضافه
کند، رد کردم. او می‌گفت که درک نمی‌کند چرا من حاضرم به خاطر چیزی کوچک
مثل اعتبارم در جامعه بازمتن، این پیشنهاد عظیم را رد کنم. او می‌گفت:
\dbquote{من درک نمی‌کنم تو کدام بخش از این ده میلیون دلار را نمی‌فهمی!}

در ابتدا فکر نمی‌کردم درگیر چنین موضوعاتی شویم اما شهرت جدید لینوکس
باعث شد لحظات حساسی نه فقط برای من، که برای کل جامعه بازمتن به وجود
بیایند. در حقیقت از سال ۱۹۹۸ که بازمتن بودن توجه جهانیان را جلب کرد،‌
خود اسم این ماجرا به یک موضوع بحث عمده تبدیل شد. تا آن موقع ما به
مفهوم اشتراک نرم‌افزار به آن شیوه‌ای که مثلا در لیسانس \lr{GP} توضیح
داده شده \dbquote{نرم‌افزار آزاد} می‌گفتیم و کلیت جریان را هم
\dbquote{جنبش نرم‌افزار آزاد} می‌نامیدیم. این دو اصطلاح ریشه در بنیاد
نرم‌افزار آزاد داشت که ریچارد استالمن در ۱۹۸۵ برای توسعه نرم‌افزارهای
آزاد و نوشتن \lr{GNU} که شکلی از یک یونیکس آزاد بود، پایه گذاری کرده
بود. حالا ناگهان مبلغانی مثل اریک ریموند کشف کرده بودند که مطبوعاتی‌ها
سردرگم شده‌اند. آیا کلمه \dbquote{آزاد} به معنی مجانی بود؟ آیا
\dbquote{آزاد} را می‌شد \dbquote{نداشتن محدودیت} تعریف کرد؟ برایان
بهلندورف\LFootnote{Brian Behlendorf} که از طرف آپاچی با روزنامه‌نگاران
صحبت می‌کرد،‌ مشکل مشابهی دارد. بعد از چندین هفته ایمیل‌بازی که البته من
در آن مشارکتی نداشتم و فقط سی.سی. می‌شدم (به مسایل سیاسی‌اش علاقمند
نبودم) به این تفاهم رسیدیم: به جای \dbquote{آزاد} به آن \dbquote{باز}
خواهیم گفت. برای آن‌هایی هم که کل جریان را به شکل یک جنبش می‌دیدند، جنبش
نرم‌افزار آزاد می‌شد جنبش بازمتن و نظر من هم همین بود. البته بنیاد
نرم‌افزار آزاد کماکان همان بنیاد نرم‌افزار آزاد باقی می‌ماند و ریچارد
استالمن هم کماکان مغز متفکر پشت آن می‌بود.

به عنون یکی از بازیگران اصلی جنبش بازمتن، همه بیشتر و بیشتر به سراغ من
می‌آمدند. هر بار که تلفنم در ترنسمتا زنگ می‌زد - و آن روزها همیشه زنگ
می‌زد - یکی از این دو نفر پشت خط بود: یا خبرنگاری که می‌خواهد با من
مصاحبه کند یا کسی که می‌خواهد از طرف سازمانش من را برای سخنرانی به یک
جلسه دعوت کند. من به عنوان یک شخص مشهور احساس می‌کردم که برای گسترش
اندیشه بازمتن و خود لینوکس، باید هر دوی اینها را قبول کنم. اینکه اریک
ریموند گفته بود من مورد علاقه روزنامه‌نگاران هستم چون از بقیه هکرها
ظاهر معقول‌تری دارم را فراموش کنید. به نظر خودم بخشی از کشش (یا هر چیزی
که اسمش را می‌گذارید) من برای روزنامه‌نگاران این است که من بیل گیتس
نیستم.

به نظر می‌رسد که روزنامه‌نگاران از این خوش شان می‌آید که من برخلاف بیل
گیتس که در قصر تکنولوژی مدرنش در کنار ساحل زندگی می‌کند، در یک خانه سه
خوابه در سانتاکلارای حوصله سر بر زندگی می‌کنم و پایم روی اسباب‌بازی‌های
دخترهایم می‌لغزد. این را هم دوست دارند که سوار پونتیاک قدیمی می‌شوم و
شخصا به تلفن‌هایم جواب می‌دهم. چه کسی ممکن است من را دوست نداشته باشد؟

از لحظه‌ای که به نظر رسید لینوکس می‌تواند به عنوان رقیبی برای مایکروسافت
مطرح شود - و بخصوص از وقتی که مشکلات قانونی ضد انحصار گریبان
مایکروسافت را گرفت و این شرکت نیازمند یک رقیب شد - ، رسانه‌ها طوری هر
پیشرفت را گزارش می‌کردند که گویی مشغول گزارش جنگ جهانی سوم هستند. یک
نفر ناشناس باعث درز \dbquote{سند هالووین} شد؛ یادداشتی درون سازمانی که
مدعی می‌شد مایکروسافت، نگران لینوکس است. استیو بالمر هم مدتی بعد اعلام
کرد \dbquote{مطمئنا من هم نگرانم} واقعیت این است که حتی اگر مایکروسافت
به خاطر منافعش روی رقابت بین ویندوز ان.تی. و لینوکس تبلیغ کرده باشد،
این روزها رقابت در حال جدی‌تر شدن است.

من نمی‌خواهم روی چهارپایه بروم و علیه مایکروسافت سخنرانی کنم. فایده‌اش
چیست؟‌ وقایع، واقعیت را روشن خواهند کرد و تا امروز هم وقایع به نفع
لینوکس پیش رفته‌اند. روزنامه‌نگاران این مساله را دوست دارند. مساله برای
آن‌ها مانند مبارزه بین حضرت داوود نرم‌زبان و جالوت
تمامیت‌خواه\RFootnote{بنا به افسانه‌های یهود، جالوت حکمران فلسطین بود که
  در نبرد با داوود جوان، مغلوب شد.} است. و از آنجایی که به راحتی در
این مورد با آن‌ها صحبت می‌کنم، بیشتر به سراغ من می‌آیند. درست است که من
به خبرنگارها گفته‌ام وازده ولی اکثر کسانی که به آن‌ها مصاحبه می‌کنم، جذاب
هستند. داستانی که من تعریف می‌کنم برای روزنامه‌نگاران هم جذاب است. کدام
روزنامه‌نگاری است که دوست نداشته باشد از طرف ضعیف‌تر، خبر تهیه کند؟

بعد از تهیه گزارش از آمیبی که مایکروسافت را نابوده کرده، آن‌ها دوست
دارند به سراغ مفهوم بازمتن بروند. رساندن این پیام هم دارد ساده‌تر و
ساده‌تر می‌شود چون نمونه‌های بیشتر و بیشتری در دسترس هستند. مطلب بعدی که
آن‌ها را شگفت زده می‌کند، شیوه مدیریت لینوکس است. آن‌ها این را متوجه
نمی‌شوند که چطور ممکن است بزرگترین پروژه جمعی طول تاریخ بشر، به این
بهینگی مدیریت شود در حالی که یک شرکت معمولی با سی کارمند، نیاز به کلی
دفتر و دستک مدیریتی دارد.

 یک بار وقتی کسی می‌خواست به من و شیوه مدیریتم اشاره کند، از عبارت
 \dbquote{دیکتاتور خیرخواه} استفاده کرد. اولین بار که این عبارت را
 شنیدم، یاد یک ژنرال در یک کشور فقیر استوایی افتادم که مشغول توزیع موز
 بین مردم گرسنه‌اش است. نمی‌دانم که با تصویر \dbquote{دیکتاتور خیرخواه} راحت
 هستم یا نه. من کرنل لینوکس و کل ساختار آن را کنترل می‌کنم چون تا به
 امروز تمام کسانی که با لینوکس در ارتباط بوده‌اند، به من بیش از هر کس
 دیگری، اعتماد داشته‌اند. روش من برای مدیریت پروژه در این روزها که صدها
 هزار نفر مشغول توسعه آن‌ هستند، هیچ تفاوتی با زمانی که در اتاق خوابم
 کد می‌نوشتم نکرده: تا وقتی که کسی خودش جلو نیامده و داوطلب انجام کاری
 نشده، کاری به کسی محول نمی‌کنم. این جریان اولین بار وقتی پیش آمد که
 احساس کردم برای نوشتن بخش‌هایی از سیستم‌عامل انگیزه کافی ندارم. مثلا از
 کدهای سطح کاربر. آدم‌ها جلو می‌آمدند و داوطلب نوشتن زیربخش‌های مختلف
 می‌شدند. همه چیز توسط مسوولین زیربخش‌ها کنترل و به من ارجاع می‌شد.

من کار آن‌ها را قبول یا رد می‌کردم، البته اکثرا انتخاب‌ها به شکل طبیعی
واقع می‌شدند. اگر بخشی توسط دو نفر نوشته می‌شد، من هر دو را نگاه می‌داشتم
تا ببینم کدام مورد قبول جامعه واقع می‌شود. گاهی هر دو کد مورد استفاده
واقع می‌شوند و حتی مسیرهای مختلفی را در پیش می‌گیرند. گاهی هم پیش می‌آید
که دو برنامه نویس دائما پچ‌هایی را می‌فرستند که رقیب یکدیگرند و به
اینکار ادامه می‌دهند. در این حالت من آن قدر پچ هیچکدام را قبول نمی‌کنم
تا یکی از آن دو نفر حوصله‌اش سر برود. این دقیقا همان روشی است که اگر
سلیمان بود، ممکن بود یک مهدکودک را اداره کند.

دیکتاتور خیرخواه؟ نه! من فقط تنبلم. من سعی می‌کنم از طریق کاری نکردن و
اجازه دادن به امور برای سیر روند طبیعی‌شان، وضعیت را کنترل کنم. در این
حالت بهترین نتایج حاصل می‌شوند.

شیوه من خبرساز هم شد. 

اما نکته عجیب اینجاست که با اینکه روش مدیریت لینوکس توجه خیلی از
رسانه‌ها را جلب کرده، مدیریت کوتاه و محدود من در یکی از بخش‌های ترنسمتا،
به یک شکست فاحش تبدیل شد. در یک مرحله، تصمیمی گرفته شد مبنی بر اینکه
من مدیر تیمی از توسعه‌دهندگان باشم. گند زدم. هرکسی که سری به اتاق و میز
پر از آت و آشغال من زده باشد می‌داند که من آدم بسیار نامرتبی هستم. من
نتوانستم جلسات هفتگی، بررسی پیشرفت کار و مراحل کار را هماهنگ کنم. بعد
از سه ماه مشخص شد روش کاری من که این همه از طرف رسانه‌ها مورد توجه قرار
گرفته، قادر نیست کوچکترین پیشرفتی در وضعیت ترنسمتا ایجاد کند.

در همین حین، رسانه‌ها توجه خود را از این جریان به یک موضوع جدید معطوف
کردند: شاخه‌شاخه‌ شدن\LFootnote{Fragmentation}. کسانی که تاریخ پرماجرا و
نه چندان شاد یونیکس را بررسی کرده‌اند از کشمکش‌های مرتبط با این جریان
بین توسعه‌دهندگان آگاهند. این سوال در سال ۱۹۹۸ مطرح شد: آیا تاریخ
درباره لینوکس هم تکرار خواهد شد؟ جواب من همیشه این بوده است که با وجود
کشمکش بین توسعه‌دهندگان بر سر این جریان، بلایی که سر یونیکس آمد هیچگاه
بر سر لینوکس نخواهد آمد. مشکل یونیکس این بود که به خاطر بسته بودن و
روابط تجاری، توسعه‌دهندگان بسیاری سال‌ها صرف نوشتن بخش‌های تکراری و مشابه
کردند چون به کد منبع یکدیگر دسترسی نداشتند. پیاده‌سازی موازی یک قابلیت
توسط شرکت‌های متفاوت باعث شاخه‌شاخه‌ شدن بسیار و همچنین جنگ‌های لعنتی‌ای شد
که سال‌های گرآن قدری را از ما گرفت. مطمئنا به خبرنگارها نمی‌گویم که
توسعه‌دهندگان لینوکس برای یکدیگر نامه عاشقانه می‌فرستند ولی به دلیل
اینکه حتی توسعه دهندگانی که با یکدیگر مخالف هستند هم می‌توانند کد
یکدیگر را ببینند و حتی از آن در برنامه‌های خود استفاده کنند، لینوکس مثل
یونیکس شاخه‌شاخه نخواهد شد. کد منبع، انباری است که هرکس اجازه برداشت و
استفاده از آن را دارد.

هر چه قدر که روزنامه‌نگاران بیشتری این نکات را درک کنند، من بیشتر و
بیشتر علاقمند به ملاقات با آن‌ها می‌شوم (بر خلاف روزنامه‌نگارانی که در
بچگی در هلسینکی می‌دیدم، روزنامه‌نگاران آمریکایی اکثرا میانه‌رو و منطقی
هستند). در مواردی هم اصولا از صحبت و بحث با آن‌ها لذت می‌برم.

اما به هرحال سخنرانی داستان دیگری است. من مجلس گرم‌کن نیستم. فراموش
نکنید که من بچه‌ای بودم که به ندرت اتاق خواب تاریکم را ترک کرده‌ام. حتی
در نوشتن سخنرانی هم مهارتی ندارم و به همین دلیل معمولا نوشتن سخنرانی
تا شب آخر به تاخیر می‌افتد.

البته یک جورهایی به نظر می‌رسد که این مساله اهمیت چندانی هم
ندارد. معمولا همین که وارد محوطه سخنرانی می‌شوم همه روی پاهای شان بلند
می‌شوند و حتی قبل از اینکه دهن بازکنم، دست می‌زنند و هورا
می‌کشند. نمی‌خواهم ناشکر باشم، ولی به نظرم این وضعیت خیلی ناجور است. هر
چیزی که بگویم به نظر بی‌ربط می‌آید حتی جمله استاندارد
\dbquote{ممنون. حالا لطفا بنشینید.} در این مورد پیشنهادهای شما را با
استقبال می‌پذیرم.

البته همه تلفن‌ها هم از طرف خبرنگاران یا سازمان‌دهندگان کنفرانس‌ها
نیستند. یک شب با تاو در خانه نشسته بودم. داشتیم برای دخترهای مان قصه
می‌خواندیم که تلفن زنگ زد.

من جواب دادم: \dbquote{توروالدز هستم. بفرمایید.}

\dbquote{اوه. شما همان آقای لینوکس هستید؟}

\dbquote{بله}

دو ثانیه سکوت شد و بعد طرف تلفن را قطع کرد.

یک بار دیگر هم دوستی از لاس‌وگاس به خانه‌ام زنگ زده بود و سعی می‌کرد مرا
متقاعد کند که قراردادی مربوط به فروش تی‌شرت‌های لینوکس را امضا کنم.

راه حل مشخص بود؛ باید شماره تلفنی می‌گرفتم که در فهرست تلفن‌ها ثبت نشده
باشد. اولین باری که به کالیفرنیا آمدیم زحمت این کار را به خودم ندادم
بخصوص که برای شماره‌های فهرست نشده، باید مبلغ بیشتری می‌پرداختیم. کم کم
هزینه‌ای که به خاطر این صرفه‌جویی متحمل شدم را فهمیدم و حالا یک تلفن ثبت
نشده دارم. یک بار قبل از اینکه تلفنم را از فهرست خارج کنم، دیوید برایم
تعریف کرد که تلفن من را همراهش نداشته و برای پیدا کردنش به شرکت تلفن
زنگ زده. بعد از اینکه اسم من را گفته بود، اپراتور تلفن با تعجب اضافه
کرده بود: \dbquote{عجب! آدمی که این همه میلیون به جیب زده، تلفن فهرست
  شده دارد؟}

اما نه. میلیونی در کار نبود. البته شکی نیست که میلیون‌ها نفر کاربر
لینوکس بودند ولی لینوس یک میلیون هم پول نداشت.

و خیلی هم خوب بود.

\section{بخش هفتم}
خیلی روزها با این خیال از تخت بیرون می‌آمدم که خوش‌شانس‌ترین آدم روی
زمینم. یادم نیست که چهارشنبه ۱۱ آگوست ۱۹۹۹ هم یکی از این روزها بود یا
نه ولی منطقا باید بوده باشد.

در دومین روز از سن جوز مرکوری نیوز\LFootnote{San Jose Mercury News} را
خواندم، البته به جز بخش ورزشی و تبلیغات را و بعد سوار تویوتایم شدم تا
مسیر ده کیلومتری به سمت سن جوز را طی کنم.

یادم هست که با کلی آدم دست دادم. 

این روزی بود که قرار بود رد هت سهام خود را عمومی کند. شرکت سال‌ها قبل
به من پیشنهاد سهام کرده و اخیرا هم کاغذهایی برایم فرستاده بودم که حتی
فرصت نکرده بودم به آن‌ها نگاه کنم. پاکت سهام، یک جایی دور و بر کاغذهای
انباشته شده در اطراف کامپیوترم جا خوش کرده بود. یادم هست که واقعا دوست
داشتم کار ردهت به خوبی پیش برود. البته منظورم در مورد جزییات بورس نیست
چون از آن سر در نمی‌آورم. من به دلیل دیگری به این جریان علاقه داشتم و
آن این بود که موفقیت عمومی شدن سهام ردهت، به معنای موفقیت تجاری لینوکس
خواهد بود. به همین دلیل از صبح کمی عصبی بودم. البته مشخصا تنها کسی
نبودم که عصبی بود. چند هفته‌ای بود که بازار بورس وضع خوبی نداشت و افراد
اصولا مشکوک به این بودند که شاید ردهت نتواند کل سهام خود را بفروشد.

در واقع وضعیت \textbf{\lr{Liquidity Event}}\RFootnote{Liquidity Event
  - برنامه ای که طی آن یک شرکت سهام خود را می خرد یا می فروشد. ممنون
  می شوم کسی که اقتصاد می‌داند توضیح و ترجمه صحیح‌تری به \code{jadijadi}
  روی جیمیل ایمیل کند.} واقع شد. در سالن کنفرانس به ما گفتند که سهام
اولیه ردهت به مبلغ ۱۵ دلار فروخته شده. شاید هم ۱۸ دلار. یادم
نیست. نکته مهم این است که در آخر معاملات آن روز، این رقم به ۳۵ دلار
رسیده بود. رکورد نشکسته بودیم ولی اوضاع خوب بود.

یادم هست که حین رانندگی به سمت خانه به همراه تاو و دیرک، احساس آسودگی
می‌کردم. بعد که در مورد پول فکر کردم، هیجان زده شدم. پشت ترافیک شاهراه
۱۰۱ بود که کشف کردم در عرض چند ساعت،‌ از حساب بانکی در حد صفر به وضعیتی
نزدیک به نیم میلیون دلار ارتقاء یافته‌ام. قلبم شروع کرد به تند زدن. این
ارتقاء مالی را به سختی باور می‌کردم.

هیچ ایده‌ای در مورد بورس نداشتم و در نتیجه تصمیم گرفتم که بیشتر یاد
بگیرم. پس به وی.ای. لینوکس\RFootnote{\lr{VA Linux} - شرکتی که پشت سایت‌هایی
  مانند سورس فورج و \lr{ThinkGeek} بود و امروزه نام خود را به گیک‌نت تغییر
  داده.} بود. به او گفتم که در آشنایان من تنها کسی است که از بورس سر
در می‌آورد. دقیقا این را گفتم: \dbquote{تو یک کارگزار بورس یا کسی شبیه
  به این را سراغ داری؟ چون نمی‌خواهم برای فروش سراغ \lr{eBay} بروم.}

ردهت به جای چند سهم سرراست، یکسری گزینه جلوی من گذاشته بود. نمی‌دانستم
که برای استفاده از این سهام باید چکار کنم. می‌دانستم که سهام را نمی‌شود
همان لحظه فروخت ولی نمی‌دانستم که این امر شامل من هم می‌شود و هیچ نظری
هم در مورد مالیات مترتب بر سهام نداشتم. لری که از این کارها سر در
می‌آورد و خیلی‌ها را هم می‌شناخت من را به لمن برادرز\LFootnote{Lehman
  Borthers} معرفی کرد. به نظرم اگر لری من را معرفی نکرده بود، لمن اصلا
من را تحویل نمی‌گرفت چون مشتری‌های بسیار بزرگ‌تری داشت. او قول داد که
بهترین گزینه را پیدا و به من اعلام می‌کند. در همین حین و دو روز بعد از
اینکه سهام عام شده بود، کسی از اداره نیروی انسانی ردهت یا شاید هم وکیل
آن‌ها با من تماس گرفت و گفت که پیش از عام کردن سهام، آن را قسمت
کرده‌اند. از این جمله هم سر در نمی‌آوردم پس به سراغ پاکت سهام رفتم که
هنوز هم بازش نکرده بودم. در پاکت به زبان ساده توضیح داده بود که سهام
من دوبرابر شده است.

نیم‌میلیون دلار من، حالا شده بود یک میلیون دلار!

با وجود تصویری که به عنوان یک گیک توده‌ای و پرهیزگار که در فقر زندگی
می‌کند از من در رسانه‌ها بازتاب پیدا کرده بود، این جریان عملا باعث شده
بود که به هذیان‌گفتم بیفتم.

ماجرا همین بود. 

من نشستم و کل کاغذهای قانونی ردهت را خواندم. بله من برای فروش سهام
باید ۱۸۰ روز صبر می‌کردم.

درک می‌کنید ۱۸۰ روز چقدر طولانی‌است وقتی که شما برای اولین بار روی کاغذ
میلیونر شده‌اید؟

حالا یک ورزش جدید داشتم: بررسی روزانه ارزش سهام ردهت که در طول شش ماه
بعد از عام شدن، افزایش می‌یافت. سهام به شکل پیوسته زیاد می‌شد و چند باری
هم جهش کرد و باز هم به رشد ادامه داد. یکبار هم سهام ردهت دوباره تقسیم
شد و در بهترین حالت، من ۵ میلیون دلار پول داشتم.

ردهت با مبلغ پایینی شروع کرد و در وال‌استریت قدم به قدم بالا رفت و هر
واقعه‌ای که حتی ربط اندکی هم به اینترنت داشت، باعث رشد آن می‌شد چون به
نوبه خود باعث \dbquote{کشف} مجدد لینوکس می‌شد. ما در طول زمستان ۱۹۹۹،
سهام منتخب بورس بودیم. متخصصان بورس به تلویزیون می‌آمدند و در مورد این
سیستم‌عامل عجیب و کوچک که در حال به زانو درآوردن مایکروسافت است صحبت
می‌کردند. تلفن من هم دائما زنگ می‌زد. اوج لذت، روزی بود که وی.ای. لینوکس
هم سهام خود را در نهم دسامبر به بورس عرضه کرد. این موفقیت ماورای تصور
همگان بود.

من و لری آگوستین برای حضور در اولین جلسه خرید و فروش عمومی سهام به
سانفرانسیسکو رفته بودیم.  من لباس همیشگی‌ام را پوشیده بودم یعنی یک
تی‌شرت رایگان و صندل. همسر و بچه‌ها را هم با خودمان برده بودیم و صحنه
وول خوردن بچه‌های نوپا در مسیر رفت و آمد بانکداران بزرگ دنیا باید صحنه
بامزه‌ای بوده باشد.

همه چیز خیلی سریع اتفاق افتاد. نمودارها نشان می‌دادند که لینوکس
وی.ای. در روز اول مبادله سهام به مبلغ ۳۰۰ دلار برای هر سهم مبادله شده
است. این سابقه نداشت. حتی اگر نمودارها را نمی‌دیدیم، به راحتی می‌شد
موفقیت را از رفتارهای عجیب بانکدارانی که گویا کانال‌های سی.ان.ان. و
بلومبرگ\RFootnote{\lr{Bloomberg} - شرکتی با محوریت نرم افزارهای
  اطلاعاتی، رسانه و اطلاعات اقتصادی.}  جادوی شان کرده بود، حدس زد. لری
خونسردی همیشگی‌اش را حتی در این مرحله هم از دست نداد. البته فکر می‌کنم
در کل جریان فروش، یک مژه هم نزد. البته دقیق نمی‌توانم بگویم چون مشغول
تعقیب و مهار بچه‌ها بودم.

حتما حتی قبایل جنگل‌های بارانی ماداگاسکار هم می‌دانند که این داستان چقدر
لری را پولدار کرد. او که تقریبا بدون هیچ پشتوانه مالی خودش را به
سانفرانسیسکو رسانده بود، در بازگشت به سیلیکون‌ولی چیزی در حدود ۱.۶
میلیارد دلار پول داشت و همان طور که رسانه‌ها هنوز هم علاقه دارند تذکر
دهند؛ تازه بیست و خورده‌ای سال داشت.

اما قضیه من این طور بود که لینوکس وی.ای. هم چند گزینه برای دریافت سهام
به من پیشنهاد کرده بود. این بار هم مثل مورد ردهت، تا شش ماه حق نداشتم
سهامم را بفروشم، ولی بر خلاف ردهت که سهامش دائما افزایش یافته بود،
سهام لینوکس وی.ای. فقط و فقط پایین رفت. بعد از آن شروع طوفانی، برای
یکسال سهام فقط پایین رفت و به ۶.۶۲ هم رسید. بخشی از این سقوط به خاطر
اصلاح بازاری بود که در ماه آوریل، کل سهام‌های تکنولوژیک را با کاهش ارزش
مواجه کرد. اما علاوه بر این دوره سهام ماه بودن لینوکس هم با آب شدن
یخ‌ها در بهار، گذشته بود. به خاطر دوره انتظار، من نمی‌توانستم سهامم را
حینی که قیمت آن هنوز بالا بود بفروشم. این دفعه بر خلاف دفعه قبل، دنبال
کردن وضعیت بازار، از نظر روانی برایم مشکل بود، چون هربار که به تختواب
می‌رفتم، می‌دانستم که فردا صبح با پشتوانه مالی کمتری از خواب برخواهم
خواست.

البته هنوز احساس می‌کردم که خوش‌شانش‌ترین آدم روی زمینم.

\begin{journal}

لینوس یک روز ژانویه با ماشین به دفتر کارم در
ساسالیتو\LFootnote{Sausalito} آمد. بعد از اینکه به خاطر استفاده از
مکینتاش و سیستم‌عامل غیرلینوکسی کمی با من شوخی کرد، پشت دستگاه نشست تا
پیش‌نویس اولیه مقدمه مفصلی که از زبان اول شخص، یعنی خودش، نوشته بودم را
بخواند. من شاید فقط پنج سانتیمتر، آن طرف‌تر نشستم. تنها صدایی که از
لینوس درآمد، وقتی بود که به پاراگرافی رسید که در آن می‌گفتم هیچ وقت فکر
نمی‌کرد به جز جین سیبلیویس\LFootnote{Jean Sibelius} و نیکی
ریندیر\LFootnote{Nikki the Reindeer}، تنها فوق‌ستاره‌ای باشد که فنلاند
تحویل جهان داده است. بعد از شاید حداکثر ده دقیقه، خواندن را تمام کرد و
تنها نظرش این بود که: \dbquote{پسر عجب جمله‌های طولانی‌ای می‌نویسی.} دو
ساعت بعدی را صرف کوتاه‌تر کردن جمله‌ها، استفاده از بعضی کلمات که او ممکن
بود برای گفتن همان حرف‌ها استفاده کند و تمرین کار دو نفره کردیم. در
نهایت فصل اول را بستیم.

بعد لینوکس سعی کرد وضوح تصویر نمایشگر مسطح من را بهتر کند. موفق هم
نشد. این نمایشگر سال گذشته جدیدترین مدل بازار بود و من با داشتن آن
احسا س تشخص می‌کردم. لینوس گفت: \dbquote{از روی یک همچین چیزی چطور
  می‌توانی چیزی بخوانی؟» بالاخره هم نتوانست وضوح تصویر را به چیزی که از
نظر خودش قابل قبول باشد ارتقاء دهد. بعد یک کاغذ پیدا کرد و شروع کرد به
کشیدن نمودارهایی برای توضیح اینکه نمایشگر چطور کار می‌کند. یک جایی
بالاخره متوقفش کردم و گفتم: «برویم کمی سوشی بخوریم.}

لینوس گفت: \dbquote{این جریان پول دارد من را دیوانه می‌کند. مجبورم صبر
  کنم تا دوره انتظار سهام تمام شود. مثل این است که کلی پول دارم ولی
  اصلا پول ندارم. نمی‌توانم از فکرم بیرونش کنم.}

من ساکی سفارش دادم. او آب میوه سفارش داد چون می‌خواست رانندگی کند.

\dbquote{تا امروز ما هیچ وقت بیشتر از ۵۰۰۰ دلار در حسابمان نداشته‌ایم،
  البته به جز کمی سهام که به عنوان پس‌انداز خریده بودیم و قرار نبود به
  آن دست بزنیم. این همه پولی بود که می‌توانستیم خرج کنیم. حالا یکهو روی
  کاغذ اینهمه پول داریم و...}

\dbquote{مثلا چقدر؟ یکی دو میلیون؟}

\dbquote{تقریبا ۲۰ میلیون دلار. این ارزش سهام لینوکس وی.ای. است به
  شرطی که بیشتر سقوط نکند. اما تا شش ماه آینده که دوره انتظار تمام
  شود، نمی‌توانم به این پول دسترسی داشته باشم. نه! حالا شده پنج ماه.}

\dbquote{راستش من متوجه مشکل نمی‌شوم. مشکل این است که باید قبل از خریدن
  یک خانه بزرگ، پنج ماه صبر کنی؟ نمی‌خواهم از همدلی دریغ کنم ولی این
  ...}

\dbquote{هی صبر کن! اول به نظر می‌رسد با این پول می‌شود هر خانه‌ای را
  خرید ولی توجه کن که ما یک خانه پنج اتاق‌خوابه لازم داریم که دورش زمین
  کافی باشد تا بتوانیم صدای حیوانات را بشنویم و تازه من هر روز سر کار
  بیلیارد بازی می‌کردم پس یکی از اتاق‌ها باید آن قدر بزرگ باشد که یک میز
  بیلیارد در آن جا شود. یک واحد مجزا هم می‌خواهیم که وقتی پدر و مادر
  تاو می‌آیند یا وقتی که دوستان خواهر من می‌خواهد به من سر بزند و برای
  نگهداری از بچه‌ها چند ماهی اینجا بمانند، جایی برای خوابیدن داشته
  باشند. بامزه است. وقتی از فنلاند به آمریکا آمدیم، پاتریشیا آمد. وقتی
  دانیلا آمد داشتیم از آپارتمان سابق به خانه دوبلکسمان می‌رفتیم و
  حالا...}

\dbquote{پس شما دو نفر دنبال برنامه یک بچه جدید را دارید؟}

\dbquote{خب ما به امور اجازه می‌دهیم به شکل طبیعی پیش بروند}

\dbquote{از جایی که من می‌آیم به جای جمله تو می‌گویند: داریم سعی می‌کنیم
  یک بچه دیگر داشته باشیم رفیق}

\dbquote{به هرحال ما به جای زیادی نیاز داریم. به
  وودساید\LFootnote{Woodside} دیدیم که هیچ زمینی اطرف آن‌ها نبود و عملا
  هم به مخروبه تبدیل شده بودند. مناسبترین خانه‌ای که دیدیم ۵ میلیون
  دلار قیمت داشت. این را باید بدانی که وقتی ۲۰ میلیون پول داری، نصفش
  صرف مالیات خواهد شد پس از این ۲۰ میلیون فقط می‌شود روی ۱۰ میلیونش
  حساب کرد و نکته وحشتناک این است که خرج یک خانه ۵ میلیونی، سالیانه ۶۰
  هزار دلار است پس باید پولی هم برای این کار کنار گذاشت. نمی‌دانم. این
  اولین و احتمالا آخرین باری است که این قدر پول نصیب من شده و نمی‌خواهم
  زندگی‌ام را طوری گسترش دهم که بعدا از پس ادامه زندگی برنیایم. هیچ وقت
  هم دوست ندارم وام بگیرم.}

\dbquote{وضعت بد هم نیست. برایت متاسف نیستم. احتمالا اگر سهام ترنسمتا
  خوب فروش برود، زندگی‌ات تضمین خواهد بود.}

\dbquote{بعله ولی من آنجا فقط یک مهندس معمولی هستم پس سهام چندانی به
  من نخواهد رسید. حقوقم هم که آن‌ قدرها زیاد نیست.}

\dbquote{لینوس، در موقع لزوم می‌توانی پیش هر سرمایه‌دار بزرگ این شهر
  بروی و هر چقدر که بخواهی پول بگیری...}

\dbquote{فکر کنم حق با تو باشد.}
\end{journal}

\section{بخش هشتم}
حالا رسیده‌ایم به جایی که باید قانون‌های طلاییم را افشا کنم. قانون اول
این است: \dbquote{با دیگران چنان رفتار کن که می‌خواهی آن‌ها با تو رفتار
  کنند} اگر پیرو این قانون باشید، در هر موقعیت به راحتی خواهید دانست
که چه رفتاری بهتر است. قانون دوم این است که: \dbquote{به خودتان افتخار
  کنید} و قانون سوم هم اینکه \dbquote{و از کارها لذت ببرید.}

مطمئنا اینکه آدم به خودش افتخار کند و از کارش لذت ببرد همیشه هم آسان
نیست. یک ماه قبل از سهامی عام شدن لینوکس وی.ای. من در اجرای هر دوی این
قوانین ناکام بودم؛ یعنی درست وقتی که سخنرانی افتتاحیه نمایشگاه
کامدکس\RFootnote{\lr{COMDEX} - نمایشگاه تکنولوژی که تا سال ۲۰۰۳ در لاس‌وگاس
  برگزار می‌شد.} لاس وگاس به من سپرده شد. همان طور که همه می‌دانند
نمایشگاه کامدکس بزرگترین و بدترین نمایشگاه تجاری است که بشر تا به حال
به راه انداخته. شهر خواب‌آلود لاس وگاس نوادا برای یک هفته تبدیل می‌شود
به آهنربایی که جذب کننده هر تکنولوژی جدیدی است که ممکن است خریداری
داشته باشد و هر آدمی که ممکن است محصول جدیدی را بخرد یا بفروشد. چند
روز مانده به شروع نمایشگاه، کافی است در خیابان پنجره تاکسی را پایین
بکشید و از هر زن خیابانی بپرسید که سخنرانی افتتاحیه کامدکس سر چه ساعتی
شروع می‌شود و جواب صحیح را تحویل بگیرید.

این که برگزار کنندگان کامدکس از دیکتاتور خیرخواه سیاره لینوکس خواسته
بودند که صحبت افتتاحیه را بر عهده بگیرد، به خودی خود ماجرای عظیمی
بود. دادن این سخنرانی به من چیزی بود که در صنعت به معنای به رسمیت
شناختن ارزش لینوکس، تعبیر می‌شد.

بیل گیتس سخنرانی اولین شب نمایشگاه یعنی یکشنبه را داشت. اتاق سخنرانی
او، سالن رقص هتل ونتیان\LFootnote{Venetian Hotel} بود که گنجایشی برابر
هفت فروشگاه متوسط ایکیا\RFootnote{\lr{IKIA} - فروشگاه زنجیره لوازم
  خانگی سوئدی} داشت. از ساعت‌ها قبل جمعیت عظیمی برای شنیدن سخنان او در
آن‌جا جمع شده بودند. بعضی‌ها می‌خواستند صحبت‌های احتمالی او در مورد
دادگاه‌های ضد انحصار را بشنوند - که در همان زمان علیه مایکروسافت در
جریان بود - و عده‌ای هم آن جا بودند تا بعدا برای نوه‌های شان تعریف کنند
که پولدارترین مرد کره زمین را از نزدیک دیده‌اند. صحبت‌های گیتس با جوکی
در مورد وکلا شروع شد و بعد هم با نمایش تکنولوژی‌ جدید وب مایکروسافت و
بخش‌های گرافیکی آن ادامه یافت که در آن گیتس مانند آوستین
پاور\RFootnote{ \lr{Autosin Powers} - بازیگر کمدی} لباس پوشیده بود و
ادای او را در می‌آورد. این جریان باعث خنده طولانی حضار شد.

من در این سالن نبودم چون داشتم تاو را در خرید لباس شب همراهی می‌کردم. 

شب بعد من در همان سالن سخنرانی کردم. 

البته فکر می‌کنم ترجیح می‌دادم بازهم برای خرید بیرون می‌رفتم. نه... واقعا
نه...

مساله این نبود که آمادگی نداشتم. اتفاقا اوضاع از همیشه بهتر بود. من
معمولا شب قبل از سخنرانی متنم را آماده می‌کنم. ولی این بار برای سخنرانی
دوشنبه، از روز شنبه آماده شده و کامپیوتر را هم برای پخش اسلایدها تنظیم
کرده بودم. همه چیز به نظر خوب می‌رسید. حتی سخنرانی را روی چند فلاپی
مختلف کپی کرده بودم تا اگر یکی از آن‌ها خراب شد، مشکلی پیش نیاید. یکی
از معدود چیزهایی که به نظرم از سخنرانی‌ هم بدتر است، سخنرانی در شرایطی
است که چیزی به مشکل برخورده باشد. حتی به خاطر آماده بودن در برابر
موقعیتی که همه فلاپی‌ها خراب شده باشند، سخنرانی‌ام را در اینتنرت هم
آپلود کرده بودم.

به خاطر ترافیک ناشی از نمایشگاه، دیرم شد و فقط نیم ساعت مانده به
سخنرانی‌ام، به هتل ونتیان رسیدم. با تاو، دخترهایم و چند دوست دیگر
بودیم. وقتی به هتل رسیدیم، به خاطر اشتباه یکی از کارمندان در صدور
مجوزها، برای رساندن خودمان به پشت صحنه به مشکل برخوردیم. می‌خواهم بگویم
که هم چیز به مشکل برخورده بود.

در نهایت داخل شدیم. من برای صحبت جلوی چهل نفر آدم هم مضطرب می‌شوم چه
برسد به بزرگترین جمعیتی که به عمرم دیده‌ام. بعد آن اتفاق افتاد.

من کشف کردم که کامپیوتری که از دو روز قبل آن همه با آن ور رفته بودم که
از همه نظر آماده باشد، کنارم نیست. احمقانه بود. یکی جلو آمد و به من
اطلاع داد که جمعیت از حدود چهار ساعت قبل در سالن انتظار تجمع کرده‌اند و
منتظر سخنرانی من هستند و جای خالی حتی برای یک نفر هم باقی نمانده
است. در همین حال، من و بقیه داشتیم مثل مرغ سر کنده در پشت صحنه این طرف
و آن طرف می‌رفتیم تا شاید کامپیوتر را پیدا کنیم.

این کامپیوتر یک کامپیوتر رومیزی معمولی بود که روی آن
استارآفیس\RFootnote{مجموعه آفیس شرکت سان میکروسیستمز} نصب کرده بودم که
یکی از بسته‌های نرم‌افزارهای اداری لینوکس است. برنامه این بود که فلاپی
را داخل دستگاه بگذارم و همه چیز به خوبی کار کند اما حالا کامپیوتر کلا
غیب شده بود. در نهایت حدس زدیم که آن کامپیوتر احتمالا برچسب نداشته یا
برچسب اشتباه خورده بوده و به همین دلیل توسط کارمندان از پشت صحنه خارج
شده. خوشبختانه لپ‌تاپم همراهم بود و اسلایدها و استار‌آفیس را هم روی آن
داشتم.

چون این لپ‌تاپم بود، همه فونت‌ها را روی آن نصب نکرده بودم. نتیجه این بود
که آخرین خط اسلایدها دیده نمی‌شد ولی وقتی متوجه این نکته شدم با خودم
گفتم: چه اهمیتی دارد؟ به هرحال من از جلسه زنده بیرون خواهم آمد. حالا
باید کابل‌ها را وصل می‌کردم. قبل از اینکه من بتوانم کارم را تمام کنم،
مردم وارد شده بودند. من مشغول ور رفتن با کابل‌ها بودم که آدم‌ها به داخل
سالن سرازیر شدند و تک تک صندلی‌ها که سهل است، همه جاهای ایستادن بین
ردیف‌ها و گوشه‌های سالن را پر کردند. خوشبختانه قبل از اینکه دهانم را
برای حرف زدن باز کنم، همه بلند شدند و شروع کردند به تشویق کردنم.

صحبتم را با اشاره به لطیفه‌ای که دیروز بیل گیتس در مورد وکلا گفته بودم
شروع کردم و یک جمله درباره اینکه ترنسمتا مشغول چه پروژه‌ای است به آن
افزودم. در مطبوعات شدیدا شایعه شد بود که من از فرصت سخنرانی در کامدکس
استفاده خواهم کرد تا (بالاخره) توضیح دهم که ترنسمتا مشغول چه پروژه‌ای
است. اما ما هنوز آماده اعلام عمومی نبودیم. بخش عمده سخنرانی من مربوط
به مزیت‌های بازمتن بودن نرم‌افزار بود. در اواسط صحبت دانیلا که همراه تاو
و پاتریشیا در ردیف اول نشسته بود، شروع به گریه کرد و شک ندارم که صدایش
در همه کازینوها و استریپ‌کلاب‌های لاس‌وگاس شنیده شد.

کسی این سخنرانی را به عنوان یک خطابه خوب در تاریخ ثبت نخواهد
کرد. بعدها یک نفر سعی کرد با گفتن اینکه بیل گیتس هم به هنگام شروع
سخنرانی در شب قبل به وضوح مضطرب بوده، من را دلداری بدهد. به هرحال
سخنرانی بیل گیتس با مشکلاتی شبیه من مواجه نشده بود اما او در حالی
سخنرانی می‌کرد که بخش عدالت تجاری دولت آمریکا، گلویش را چسبیده بود و
فشار می‌داد. به نظرم وضع من بهتر بود.

\begin{journal}
اگر بگویم استراتژی من این بود که کسی را پیدا کنم که از همه بیشتر منتظر
و مشتاق سخنرانی افتتاحیه لینوس است و همراه او وارد سالن شوم، به نظر
خواهد آمد که دارم بخشی از راهنمای ساده برای خبرنگار شدن را
می‌نویسم. واقعا هم چه راه بهتری هست برای درک شیفتگانی که لینوس را مثل
یک خدای ملبس به جین و تی‌شرت‌های تبلیغاتی می‌بینند.

ساعت ۵ عصر است و من روی یک پله برقی به سمت ووداستوک\RFootnote{
  \lr{Woodstock} - یکی از بزرگترین و پرهیجان ترین فستیوال‌های موسیقی که
  در اینجا لینوس با اشاره به آن، هیجان ماجرا را متذکر می‌شود.} گیک‌ها
پایین می‌روم. در ابتدای صف طولانی‌ای که مثل مار در کل راهروها پیچیده، یک
دانشجوی علوم کامپیوتر خوره از کالج والاوالا\LFootnote{Walla Walla
  College} ایستاده که با خوشحالی می‌پذیرد در حین سخنرانی با او باشم. او
تا این لحظه دو ساعت و نیم است که در انتظار دیدن لینوس در صف ایستاده و
می‌داند که بعد از اینکه دو ساعت و نیم دیگر هم منتظر بماند، اولین نفری
خواهد بود که به سالن وارد می‌شود. هم‌کلاسی‌هایش چند نفری عقب‌تر
هستند. آن‌ها نیم‌ساعت دیرتر رسیده‌اند و دلیل تاخیر این بوده که دیشب به
همراه یکی از اساتید دانشگاه شان از ایالت واشنگتن تا اینجا رانندگی
کرده‌اند و شب را هم در سالن ورزشی یک دبیرستان گذرانده‌اند. آن‌ها یکی دو
ماه قبل به همراه یکدیگر یک شرکت طراحی وب راه‌انداخته‌اند. به نظر می‌رسد
که در دید آن‌ها تمام افراد بالغ دنیا به دو بخش هکرها و کت‌شلوارپوش‌ها
تقسیم شده‌اند و دائما با دیدن یک کت‌شلوارپوش‌ به هم اشاره می‌کنند و
می‌گویند \dbquote{هی پسر! ببین چقدر کت‌شلوارپوش‌ اینجاست.} درست همان طور
که یک همکلاسی غیرکامپیوتری آن‌ها ممکن است در ساحلی قدم بزند و دائما
بگوید که \dbquote{واو... چه تیکه‌هایی!} البته این بچه‌ها هم مثل
همکلاسی‌های غیرکامپیوتری در حال کشمکش با همدیگر و متلک گفتن هستند اما
متلک این‌ها هم اکثرا به مادربردها یا گیگابایت‌ها مرتبط می‌شود.

و بعد حرف به لینوس می‌رسد. اسم لینوس با ابهت برده می‌شود. مثلا می‌گویند
که \dbquote{لینوس نباید در شرکتی کار می‌کرد که محصولش بازمتن نیست. نه
  نباید آنجا کار می‌کرد.} برای هم نقل قول‌ها و ارجاعات دقیقی از سایت‌هایی
مثل اسلش‌دات\RFootnote{منظور لینوس \code{slashdot.org} است که یکی از
  منابع خبری گیک‌ها به شمار می‌رود.} می‌آورند و جوری در مورد افشاگری‌هایی
که این سایت و امثالش در مورد ترنسمتا کرده‌اند صحبت می‌کنند که گویی در
مورد رسوایی اخیر زندگی عشقی یک بازیگر هالیوود حرف می‌زنند. این شیفتگی،
هیجان و صحبت در مورد شایعات، منحصر به این گروه جوان نیست.

به دستشویی مردانه می‌روم و در حینی که در حال استفاده از تنها جای خالی
هستم، به صحبت‌ دو نفر کناری‌ام گوش می‌دهم.

نفر سمت چپ می‌گوید: \dbquote{این سخنرانی در مقایسه با سخنرانی افتتاحیه
  گیتس، حوصله‌سربر خواهد بود.}

نفر دوم پاسخ می‌دهد: \dbquote{چه انتظاری داری؟ لینوس یک هکر است نه یک
  کت‌شلوارپوش‌. نباید سخت‌گرفت.}

بالاخره وقتی در سالن باز می‌شود، ما در ردیف جلویی جایی پیدا نمی‌کنیم و
کمی عقب‌تر از وسط سالن، می‌نشینیم. هم ردیفی‌های والاوالایی من برای چند
لحظه فراموش می‌کنند که قرار است قهرمان شان را رو در رو ببینند و مشغول
جر و بحث در این مورد می‌شوند که حق آن‌ها بوده‌ است که در ردیف جلو
بنشینند. چند لحظه بعد هم شروع می‌کنند به کشف کت‌شلوارپوش‌های حاضر در
سالن. با اینکه شصت، هفتاد متری تا صحنه فاصله داریم و چراغ‌های صحنه هم
خاموش است، می‌توانم لینوس را تشخیص بدهم که روی صحنه، مشغول کار با
لپ‌تاپش است. او در حالی که چند مسوول نمایشگاه احاطه‌اش کرده‌اند، تند و
تند مشغول تایپ روی کامپیوتر است. چه خبر است؟ یک جور نمایش نرم‌افزاری که
همین چند دقیقه قبل آماده شده؟‌

در نهایت لینوس و بقیه صحنه را ترک می‌کنند. یک نفر مدیر بین‌المللی لینوس
یعنی مدداگ (جان هال) را معرفی می‌کند. همراه والاوالایی من به وضوح به
هیجان آمده. می‌گوید \dbquote{ریشش را نگاه کن} مدداگ می‌گوید از اینکه
قرار است فردی را معرفی کند که مثل پسرش می‌ماند، خوشحال است. لینوس به
روی صحنه می‌آید و یک ماچ و روبوسی پر پشم و پیل از مدداگ تحویل
می‌گیرد. حتی از این صندلی ارزان دور از صحنه هم می‌توانم بگویم که عصبی
است.

لینوس می‌گوید: \dbquote{من می‌خواستم صحبت‌هایم را با لطیفه‌ای در مورد وکلا
  شروع کنم ولی یک نفر قبلا آن را استفاده کرده}. منظور او طنز دیشب بیل
گیتس در مورد وکلا است که با تشویق خوبی هم روبرو شد.

بعد در یک جمله از ترنسمتا و عملیات سری آن می‌گوید و باقی سخنرانی به
تکرار جملاتی می‌گذرد که در بالای سر لینوس و در اسلایدهای بزرگ نمایش
داده می‌شوند. جملات درباره موفقیت و اهمیت روزافزون جنبش بازمتن
هستند. نه حرف شگفت‌آوری هست. نه چیز جدیدی.

سخنرانی با صدایی بشاش ولی یکنواخت ادا می‌شود و یک جا هم یکی از دخترهایش
گریه می‌کند. لینوس وسط حرفش می‌گوید \dbquote{این بچه من است.} اگر به
مونیتورها نگاه کنید، به راحتی انعکاس نور صحنه از عرق پیشانی لینوس را
می‌بینید.

بعد از اتمام سخنرانی، افراد برای پرسش و پاسخ صف می‌کشند. از گفتن اینکه
کدام بسته اداری لینوکس را ترجیح می‌دهد، طفره می‌رود و در جواب کسی که
می‌پرسد در خانه چند پنگوئن عروسکی دارد می‌گوید \dbquote{یک چندتایی} نفر
بعدی می‌پرسد که زندگی در کالیفرنیا را چقدر دوست دارد. لینوکس خوشحال
می‌شود و توضیح مبسوطی می‌دهد که \dbquote{الان ماه نوامبر است و من هنوز
  شلوار کوتاه می‌پوشم. اگر در هلسینکی اینکار را می‌کردم، تا حالا جواهرات
  سلطنتی‌ام یخ زده بودند.} یک نفر از حضار هم به پشت میکروفون سوال می‌رود
و به سادگی می‌گوید که \dbquote{لینوس، تو قهرمان منی.} لینوس به شکلی که
انگار میلیون‌ها بار این عبارت را شنیده و میلیون‌ها بار آن را جواب داده
است می‌گوید: \dbquote{ممنون}

بعد از پایان پرسش و پاسخ، صدها نفر به سمت پایین صحنه می‌آیند. یعنی جایی
که حالا لینوس آمده و دارد سعی می‌کند تا جایی که امکان دارد، با همه دست
بدهد.
\end{journal}

\section{بخش نهم}

\textbf{\large آیا انقلاب لینوکس به پایان رسیده است؟} \\
\emph{\large نوشته اسکات بریناتو، \lr{PC Week}}

\vspace*{10pt}
\textbf{\Large \dbquote{ از تماس شما ممنونم. انقلاب به پایان رسیده است. اگر اطلاعات بیشتری در مورد لینوکس می‌خواهید، لطفا یک را فشار دهید...}}

\begin{journalinside}
به نظر می‌رسد لینوس توروالدز پیام گیرش را به این پیغام تغییر داده و لابد معنایش این است که لینوکس در دنیای سرورها به سیستم‌عامل اصلی تبدیل شده  و بهتر است انقلاب را فراموش کنیم و به کار روی میزکارهای ویندوزی خود برگردیم. 

روزگاری بود که طی آن خبرنگاران می‌توانستند به نویسنده لینوکس زنگ بزنند
و در دفتر رازآلود ترنسمتا با او کمی صحبت کنند. آن روزها وقتی زنگ
می‌زدید، یک نفر با لهجه شیرینش گوشی را برمی‌داشت و از آن طرف خط می‌گفت
\dbquote{توروالدز} با حوصله بود و به سوالات شما جواب می‌داد. اگر هم وقت
نداشت به شما می‌گفت. حتی ممکن بود به شما بگوید که مشغول پرسیدن سوالاتی
هستید که هر برنامه‌نویس نوآموزی هم جواب‌هایش را می‌داند. اما به هرحال به
تلفن شما جواب می‌داد.

امروز، اگر شماره ترنسمتا را بگیرید و کد داخلی لینوس را وارد کنید، صدای
زنانه مهربانی به شما می‌گوید \dbquote{متشکرم که به لینوس توروالدز زنگ
  زده‌اید. این صندوق پست صوتی، پیام جدید قبول نمی‌کند. برای تماس با
  لینوس،‌ فکسی بفرستید به شماره...}

احساس من این است که او با من تماس نخواهد گرفت. به اندازه کافی با
خبرنگاران حرف زده. حالا او یک آدم مشهور است و حرف زدن با او هم همان
دردسرهایی را دارد که برای حرف زدن با آن یکی آدم مشهور دنیای کامپیوتر
باید متحمل شوید. صدای زنانه دارد شماره فکس را می‌خواند و من به این فکر
هستم که همان ترکیب قدیمی \code{0\#} را بزنم برای ارتباط با منشی...

\dbquote{متاسفانه منشی‌های شرکت نمی‌توانند پیامی برای توروالدز
  بپذیرند. امکان برنامه‌ریزی برای ملاقات هم ندارند.} این خانم هم مهربان
است ولی اوضاع بد می‌شود وقتی که تکرار می‌کند \dbquote{با خوشحالی می‌توانم
  فکس شما را قبول کنم و آن را به ایشان بدهم.} شاید مشکلات بیل گیتس کمتر
شده باشد.

قبول، انقلاب لینوکس تمام نشده اما مثل هر انقلاب دیگر، جوش و خروش اولیه
آن در غریو همراهانش گم شده است. موسیقی جدید در حال خاموش کردن جوش و
خروش پانک راک است. مالیات‌دهندگان فقیر حالا دارند به ثروتمندانی تبدیل
می‌شوند که کارخانه‌ها را در دست گرفته‌اند.

در واقع لینوس تا اینجا هم خوب دوام آورده است. کنار کشیدن او از دنیای
خبرنگاران غیرقابل اجتناب بود. فقط کافی است به تعداد تلفن‌ها از طرف
رسانه‌ها و حجم دیوانه کننده موضوعاتی که از او پرسیده می‌شود، فکر کنید.

نگاهی به جلسه پرسش و پاسخ نمایشگاه بین‌المللی لینوکس که همین ماه گذشته
در سن جوز برگزار شد بیاندازید. لینوس قبول کرد در این جلسه پرسش و پاسخ
شرکت کنند چون برایش امکان نداشت به بی‌نهایت درخواست مصاحبه فردی پاسخ
بدهد. پاسخ‌های او پاسخ‌هایی تکراری به سوالات همیشگی بودند. آیا بازمتن در
دنیای تجاری هم کارایی دارد؟ آیا تو تلاش می‌کنی همان طور که بیل گیتس
دنیای نرم‌افزار را کنترل می‌کند، دنیای نرم‌افزار را کنترل کنی؟ نظرت
درباره مایکروسافت چیست؟ بازمتن چیست؟‌ لینوکس چیست؟ چرا پنگوئن؟

در اینجا است که احساس می‌کنید توروالدز به یک قهرمان ورزشی تبدیل شده که
جواب همه سوال‌های مرسوم را می‌داند و ماشین‌وار آن‌ها را تکرار می‌کند. به
تیم رابینز فکر کنید که در فیلم می‌گوید \dbquote{من باید هر طور شده به
  زمین می‌رفتم و ۱۱۰ درصد تلاشم را می‌کردم تا تیم نتیجه بگیرد.}

بدون نیاز به طولانی کردن بحث، این را هم اضافه کنم که بعضی پرسش‌های
خبرنگاران هم واقعا بی‌ربط است. مثلا در همان کنفرانس مطبوعاتی، خبرنگاری
فنلاندی می‌پرسد که هدف لینوس برای تسخیر بازارهای تجاری کوچک و متوسط
چیست (جواب معمول توروالدز این است که در تلاش برای تسخیر هیچ جا
نیست). کمی بعد، یکی از آن‌ خبرنگارانی که می‌خواهد ثابت کند درک و دیدگاه
جدیدی درباره بازمتن دارد از توروالدز نظرش را درباره تلاش بعضی شرکت‌ها
برای ثبت تجاری ژنوم‌های محصولات کشاورزی می‌پرسد (جواب معمول توروالدز این
است که \dbquote{وقتی صحبت از ثبت تجاری می‌شود من معتقدم که هم نوع خوب
  آن وجود دارد و هم نوع بد آن}).

راهنما برای برنامه‌نویسان در حال پیشرفت: هر وقت کسی از شما در مورد ژنوم
محصولات کشاورزی پرسید، یعنی وقت آن شده که روی تلفن‌های تان منشی خودکار
نصب کنید.

پس شاید تصمیم توروالدز برای جواب ندادن تلفن‌ها، چیز بدی هم نباشد. البته
ما صفا و سادگی توروالدز را از دست می‌دهیم که برای خبرنگارانی که اکثر
مواقع با فشرده شدن گلوی شان توسط تاجران از خود راضی مواجه بوده‌اند،
جذاب بوده است؛ اما امیدواریم که اگر فکس‌ها به او برسند و او واقعا به
آن‌ها پاسخ بدهد، آن جنبه دوست داشتنی توروالدزی‌اش حفظ شود. چرا که اگر
صدای منشی تلفنی، بر صدای لینوس غلبه کند، لینوکس دیگر به مفرحی سابق
نخواهد بود.
\end{journalinside}

خب، فکر می‌کنم به آقای بریناتو یک توضیح، و نه عذرخواهی مدیون هستم. کسی
که این ستون از مجله \lr{PC Week} را بخواند، فکر خواهد که من نردی هستم که
تازگی‌ها از خودم یک آدم مزخرف ساخته‌ام. ولی این طور نیست. من همیشه یک
آدم مزخرف بوده‌ام.

بگذارید از اول شروع کنم. به نظر من پست صوتی یک چیز شیطانی است. نمونه
خوبی از یک تکنولوژی بد. به نظرم این ایده بدترین تکنولوژی موجود است و
با تمام احساس، از آن متنفرم. در ترنسمتا، هر یک از ما یک صندوق پستی
داشتیم که بیست دقیقه گنجایش داشت و بعد از پر شدن، تلفن زننده را به سمت
منشی هدایت می‌کرد. مال من همیشه پر بود.

به نظرم دردسر از جایی شروع شد که خبرنگارها دائما به منشی‌های شرکت زنگ
می‌زنند و از پر بودن صندوق پست صوتی من شکایت می‌کردند. احتمالا بعد از صد
شکایت اول،‌ منشی‌ها هم خسته شده‌اند و با بداخلاقی صحبت کرده‌اند. آن‌ها
می‌دانستند که من نمی‌خواهم با خبرنگاران صحبت کنم و در عین حال دوست
نداشتند که وظیفه گفتن این حرف به افراد، بر عهده آن‌ها باشد.

برای خلاص کردن منشی‌های شرکت، من شروع کردم به پاک کردن پیام‌های صوتی
بدون گوش کردن به آن‌ها. مشکلی هم نبود چون خیلی وقت‌ها حتی به پیام‌های
خودم هم گوش نمی‌دهم. نمی‌دانم چرا ولی به نظر می‌رسد مردم وقتی که با سیستم
منشی خودکار طرف هستند، صدایشان را مبهم می‌کنند تا من مجبور باشم هر پیام
را برای فهمیدن منظور و شماره تماس، پانزده بار گوش دهم. از طرف دیگر من
وقتی دلیلی برای اینکار وجود ندارد، به کسی که برایم تلفنش را گذاشته زنگ
نمی‌زنم. فکر کنم آدم‌ها بعد از پیام گذاشتن خوشحال هستند که من به آن‌ها
زنگ خواهم زد و بعد که می‌بینند این طور نشد، ناراحت می‌شوند.

در این وقت است که مراجعان به سراغ منشی‌ها می‌روند. آن‌ها نمی‌دانستند در
این شرایط باید چه بگویند و در نتیجه از آن‌ها خواستم که شماره فکسم را به
مراجعان بدهند. خوبی فکس این است که به همان راحتی پست صوتی می‌توان از
دستش خلاص شد و تازه پیدا کردن شماره تماس هم در آن راحت‌تر است. البته
اگر بخواهید؛ که من هیچ وقت نخواستم.

در ابتدا،‌ اوضاع خوب بود و منشی‌ها محترمانه شماره فکس من را به افراد
می‌دادند. در نهایت افراد کشف کردند که من فکس‌ها را نمی‌خوانم و بعد از یک
هفته تلفن‌های جدیدی شروع شد که در جواب به \dbquote{فکس بفرستید} می‌گفتند
که فکس فرستاده‌اند و ترتیب اثر داده نشده. منشی‌ها دوباره گیر کرده بودند
و من می‌دانستم که پاسخگویی تلفن‌های من، وظیفه آن‌ها نیست.

درست است. تعبیر آقای بریناتو از روزهای قدیم لینوکس، این روزها دیگر
وجود ندارد. ولی این را بدانید که من همیشه آدم مزخرفی بوده‌ام. این جریان
یک جریان جدید نیست.

راه‌حل مبتنی بر فکس، زیاد دوام نیاورد. در نهایت ترنسمتا یک دسترسی ویژه
برای منشی تلفنی در اختیار من گذاشت که اصولا صندوق پستی نداشت. در همان
زمان ترنسمتا یک منشی جدید استخدام کرد که حاضر بود داوطلبانه تلفن‌های من
را جواب دهد. به من گفته بودند که او برای اینکار آموزش حرفه‌ای دیده
است. این روزها به من توصیه شده که حتی اگر نخواهم با آن‌ها مصاحبه کنم،
حتما پاسخ تلفن خبرنگاران را بدهم یا بعدا به آن‌ها زنگ بزنم چون احساس
مثبتی نسبت به من و لینوکس در آن‌ها ایجاد می‌کند. جواب من این است که:
احساس مثبت آن‌ها برای من مهم نیست.

امروز، اگر پشت میزم نشسته باشم، جواب تلفن‌های خودم را خودم می‌دهم ولی
نباید از این موضوع برداشت کنید که به راحتی در دسترسم. این تصمیم سیاسی
هم نیست. مفهوم بازمتن هیچ وقت به این معنا نبوده که من از بقیه قابل
دسترس‌تر باشم. من هم هیچ وقت از بقیه در دسترس‌تر نبوده‌ام. مساله اصلا این
نیست. مساله اساسی این است که حتی اگر من شیطان مجسم باشم و مستقیما از
جهنم نزول کرده باشم، مردم حق دارند بیخیال من شوند و خودشان تغییرات
مورد نظرشان را در سیستم‌عامل اعمال کنند. مساله درباره در دسترس بودن
نیست، مساله این است که آن‌ها حق دارند من را کنار بگذارند و خودشان به متن
دسترسی پیدا کنند. این است که مهم است.

هیچ نسخه \dbquote{رسمی}ای از لینوکس وجود ندارد. چیزی که هست نسخه شخصی
من است و نسخه شخصی یک نفر دیگر. نکته این است که مردم اعتماد بیشتری به
نسخه شخصی من دارند و از آن مثل یک نسخه رسمی استفاده می‌کنند چون دیده‌اند
که نه سال تمام است که من متعهدانه روی آن کار کرده‌ام. من اولین نفری
بودم که روی لینوکس فعالیت کردم و ایده اکثر مردم هم این است که کارم خوب
بوده است. فرض کنید امروز سرم را بتراشم و ببینید که روی آن نوشته
۶۶۶\RFootnote{در فرهنگ غربی این عدد به شیطان نسبت داده می‌شود.} و بعد
فریاد بکشم که \dbquote{جلوی من زانو بزنید وگرنه همگی نفرین خواهید شد.}
مردم خواهند خندید و عده‌ای خواهند گفت: \dbquote{ما کد کرنل را برداشتیم
  و رفتیم سراغ کار خودمان. تو هم هرکاری دوست داری بکن.}

مردم به من اعتماد دارند و تنها دلیل این اعتماد، این واقعیت است که من
در طول نه سال گذشته، قابل اعتماد بوده‌ام.

این حرف اصلا به معنی نیست که من علاقمندم به پیام‌های صوتی گوش کنم تا
جواب هر کسی که به من زنگ زده را بدهم. اصلا هم دوست ندارم به این عنوان
یک آدم خوب و دوست داشتنی که با هر کسی حرف می‌زند و جواب هر تلفن یا
ایمیلی را می‌دهد مشهور شوم. حالا که مرتبط است، این را هم اضافه کنم که
داستان‌هایی که من را یک راهب از خودگذشته معرفی می‌کنند که علاقه‌ای به پول
و زندگی مجلل ندارد، به نظرم عجیب و بی‌ربط می‌آیند. بارها سعی کرده‌ام با
این تصویر از خودم مبارزه کنم ولی این نوشته‌ها هیچ وقت به چاپ
نرسیده‌اند. من دوست ندارم چیزی باشم که رسانه‌ها دوست دارند باشم.

واقعیت این است که با تصویر آن راهب از خودگذشته مشکل دارم چون بامزه
نیست، حوصله‌سربر است و غیرحقیقی است.

\section{بخش دهم}
از خواب که بیدار می‌شوم، خواب آلود به سمت پنجره می‌روم و با خودم فکر
می‌کنم که مجبورم خیلی چیزها درباره زندگی یاد بگیرم. چیزهایی که بقیه در
مهدکودک یاد گرفته‌اند را من تازه تجربه می‌کنم. مثلا هیچ وقت نفهمیدم که
مردم چرا اینقدر من یا کارهایم را جدی می‌گیرند. دو نمونه ذکر می‌کنم که
شباهت‌هایی هم با هم دارند.

وقتی در دانشکده بودم، روی کامپیوتر خودم شناسه ریشه\RFootnote{\lr{root}
  - بالاترین سطح دسترسی در سیستم‌های یونیکس} داشتم و هر شناسه ریشه یک
اسم هم دارد. این اسم فقط کاربرد اطلاعاتی دارد و استفاده دیگری
نمی‌شود. من اسم کاربر ریشه خودم را \textbf{لینوس توروالدز
  خدا}\LFootnote{Linus God Torvalds} گذاشته بودم. من خدای ماشینی بودم
که در دفتر کارم قرار داشت.

این روزها، \lr{finger} کردن یک کاربر روی یک ماشین دیگر به منظور اینکه
چک کنیم که آیا به سیستم لاگین کرده یا نه، تقریبا منسوخ شده. دلیل این
امر استفاده روزافزون از فایروال‌ها است. اما سال‌ها قبل، مردم دائما
کامپیوترهای یکدیگر را \lr{finger} می‌کردند تا بررسی کنند که آیا کاربر
مورد نظرشان پشت کامپیوتر هست یا نه و اگر هست، آیا ایمیل‌هایش را
خوانده. این دستور علاوه بر وضعیت کاربر،
\textbf{برنامه}\LFootnote{Plan} او و کمی از اطلاعات شخصی مربوط به او
را هم برمی‌گرداند؛ چیزی شبیه به جد وب امروزی. من همیشه آخرین نسخه کرنل
را در \dbquote{برنامه}ام می‌گذاشتم و در نتیجه یکی از راه‌های فهمیدن
جدیدترین نسخه لینوکس، این بود که افراد کامپیوتر مرا \lr{finger}
کنند. بعضی‌ها حتی این کار را اتوماتیک کرده بودند. آن‌ها هر ساعت یکبار
کامپیوتر مرا \lr{finger} می‌کردند تا سریعا از به روز شدن نسخه کرنل،
مطلع شوند. مستقل از اینکه افراد به چه منظوری کامپیوتر مرا \lr{finger}
می‌کردند، نام کاربر ریشه که \dbquote{لینوس توروالدز خدا} بود هم به آن‌ها
نمایش داده می‌شد. اوایل این امر مشکلی نداشت. اما کم کم شروع کردم به
دریافت نامه‌هایی مبنی بر اینکه این اسم نوعی کفرگویی است. در نهایت
تغییرش دادم. اینها آدم‌هایی هستند که خودشان را بیش از حد جدی می‌گیرند و
این مرا دیوانه می‌کند.

بعد هم که معلوم است باید از چه حرف بزنم؛ از جریان کارولینای
شمالی\LFootnote{North Carolina}. وای! خیلی بد بود. یک کتاب که اخیرا در
مورد رد هت چاپ شده، مساله را جوری جلوه داده که انگار ممکن بوده یک
فاجعه بین‌المللی اتفاق بیافتد. این قدرها هم بد نبود.

من دعوت شده بودم تا در گردهمایی کاربران لینوکس رد هت که در دورهایم
برگزار می‌شد صحبت کنم. سالن سخنرانی کیپ تا کیپ پر بود. لحظه‌ای که وارد
شدم، همه روی پای شان ایستادند و شروع به دست زدن کردند. اولین کلماتی که
به زبان آوردم، اولین کلماتی بودند که به ذهنم رسیدند:

\dbquote{من خدای شما هستم.} 

شک ندارم که قرار بود این یک شوخی باشد.

ماجرا این نبود که \dbquote{من کاملا متقاعد شده‌ام که خدای شما هستم و
  شما هرگز نباید این را فراموش کنید.} بلکه قرار بود این باشد که
\dbquote{خب، خب، خب. می‌دانم که خدای شما هستم. حالا با وجود اینکه
  اشتیاق شما را درک می‌کنم ولی لطفا احساسات خود را کنترل کنید و بنشینید
  تا من بتوانم حرف بزنم و شما بشنوید.}

باور نمی‌کنم که دارم شخصا این ماجرا را دوباره زنده می‌کنم. 

بعد از آن چهار کلمه اول، همه برای یک لحظه ساکت شدند. چند ساعت بعد، آن
چهارکلمه شده بود اصلی‌ترین موضوع بحث گروه‌های خبری. می‌پذیرم که کار بی
مزه‌ای بود ولی من می‌خواستم بامزه باشد. در واقع آن حرف روشی بود برای شرم
ساری از اینکه مردم ایستاده‌اند و تشویقم می‌کنند فقط به این خاطر که در
حال رفتن به سمت تریبون سخنرانی هستم.

مردم مرا زیادی جدی می‌گیرند. البته مردم خیلی‌ چیزها را زیادی جدی
می‌گیرند. درسی که از چندین سال مکانیک اصلی لینوکس بودن گرفته‌ام از این
هم تلخ‌تر است: بعضی دوستان، به این هم راضی نمی‌شوند که شخصا مسایل را جدی
بگیرند. آن‌ها خوشحال نخواهند بود تا لحظه‌ای که به بقیه هم بقبولانند که
باید فلان مساله را جدی بگیرند. این یکی از مسایلی است که من برایش غصه
می‌خورم.

هیچ وقت شده به این فکر کنید که چرا سگ‌ها اینقدر عاشق ما انسان‌ها هستند؟
نه، دلیلش این نیست که ما شش هفته یکبار آن‌ها را به سلمانی می‌بریم یا گاه
گداری جامانده‌های آن‌ها را از کنار خیابان برمی‌داریم. دلیلش این است که
سگ‌ها دوست دارند یک نفر به آن‌ها بگوید که چکار باید بکنند. این موضوع به
آن‌ها دلیلی برای زندگی می‌دهد (مساله بخصوص وقتی خیلی برای شان مهم می‌شود
که بدانیم اکثر سگ‌های ما عقیم شده‌اند و دیگر قادر به ادای تنها وظیفه
طبیعی‌شان که ادامه بقای این نسل پشمالو است هم نیستند. در عین حال به جز
چند استثنا، هیچ سگی به دنبال کارهایی که قابلیت‌هایش را دارد نیست و تنها
کاری که ممکن است بکند ، بوکردن گاه گداری یک سوسک است).شما در نقش یک
انسان، فرمانده سگ‌ها هستید و به آن‌ها می‌گویید که چگونه باید رفتار
کنند. پیروی از دستورات شما، دلیل وجودی بعضی از سگ‌ها است و خودشان هم از
این موضوع لذت می‌برند.

متاسفانه آدم‌ها هم به همین روش ساخته شده‌اند. مردم دوست دارند از بقیه
بشنوند که چکار باید بکنند. این بخشی از برنامه کرنل ما است. هر حیوان
اجتماعی‌ای باید به همین شیوه رفتار کند.

گفته بالا به این معنی نیست که ما موجودات پستی هستیم. تنها معنی
پاراگراف بالا این است که ما اگر کسی به ما بگوید چکار کنیم، به احتمال
زیاد در همراهی با دیگران به حرفش گوش خواهیم کرد.

آدم‌هایی هم هستند که نظرات و ایده‌های فردی دارند. این آدم‌ها این قدرت را
دارند که در بعضی مواقع و در برابر بعضی درخواست‌ها بگویند که
\dbquote{نه، من این کار را نمی‌کنم.} و این آدم‌ها هستند که رهبر دیگران
می‌شوند. رهبر شدن ساده است (باید هم ساده باشد، چون من هم یک رهبر
شده‌ام. این طور نیست؟). حالا آدم‌هایی که در همان حوزه علایقی دارند، با
خوشحالی از این فرد پیروی خواهند کرد و اجازه خواهند داد که رهبر برای
آن‌ها تصمیم بگیرد یا حتی به آن‌ها بگوید که چه کنند.

این یکی از حقوق پایه‌ای انسان‌ها است. کاری را بکنند که کسی به عنوان رهبر
برگزیده‌اند، از آن‌ها می‌خواهد. من با این موضوع مخالف نیستم؛ هرچند که آن
را ناراحت کننده می‌یابم. مخالفت من وقتی است که یکی از رهبرها یا یکی از
افراد جامعه بخواهد دیدگاه‌های خودش را به دیگران تحمیل کند. این موضوع
فقط ناراحت کننده نیست بلکه ترسناک است. ناراحت کننده است که آدم‌ها از هر
کسی - از جمله من - ممکن است پیروی کنند ولی وحشتناک‌ است اگر مردم
بخواهند این پیروی را به دیگران - از جمله من - نیز تحمیل کنند.

آن آدم آهنی مبلغ که درست وقتی پشت کامپیوتر مشغول تمرکز روی یک مساله
هستید یا درست در لحظه‌ای که بچه دارد خوابش می‌برد، می‌آید و در می‌زند و
می‌خواهد شما را به راه راست هدایت کند را فراموش کنید. مثال بسیار با
ربط‌تری در همین جامعه بازمتن خودمان هست: آدم‌های متعصبی که فکر می‌کنند هر
ابداعی باید مجوز جی.پی.ال. داشته باشد (به قول هکرها،
جی.پی.ال. شود). ریچارد استالمن می‌خواهد همه چیز را بازمتن کند. بازمتن
برای او یک مبارزه سیاسی است و از جی.پی.ال. به عنوان موتور پیشبرنده این
مبارزه استفاده می‌کند. برای او هیچ جایگزین دیگری وجود ندارد. واقعیت این
است که من لینوکس را به خاطر این دلایل والا، بازمتن نکردم. من فیدبک
می‌خواستم. آن روزها همه چیز همین طور بود. اکثر پروژه‌ها در دانشگاه‌ها
انجام می‌شدند و برای فهمیدن نظر دیگران، باید بسیار باز برخورد
می‌کردید. وقتی دانشگاه دیگری در مورد برنامه می‌پرسید، متن برنامه را به
آن‌ها می‌دادید. کاری که استالمن کرد این بود که بعد از جدا شدن از
پروژه‌های مورد علاقه‌اش، عامدانه به انتشار بازمتن برنامه‌ها ادامه داد.

بله! بازکردن پروژه‌ها و قابل استفاده کردن آن برای همه به شکلی که لینوکس
برای همگان قابل استفاده است، راهی است به سوی کسب مزایای بسیار زیاد از
جمله امکان دادن به دیگران برای سهیم شدن در خلاقیت. برای درک نتیجه این
تصمیم، کافی است به استانداردهای پایین نرم‌افزارهای بسته در مقابل
نرم‌افزارهای بازمتن نگاه کنید. بازمتنی و جی.پی.ال، فرصتی است برای خلق
بهترین تکنولوژی ممکن. موضوع بسیار ساده است. بازمتنی، از احتکار
تکنولوژی جلوگیری می‌کند و به هر کسی که علاقه‌ای به پیشبرد آن دارد اجازه
می‌دهد که در این کار مشارکت کند. بازمتن بودن یک پروژه باعث می‌شود هیچ
علاقمندی از دایره آفرینش و خلاقیت، بیرون گذاشته نشود.

این نکته کوچکی نیست. استالمن که لایق یک بنای یادبود برای بنیان نهادن
جی.پی.ال. است، زمانی شروع پروژه و ایجاد مفهوم نرم‌افزار آزاد را کلید زد
که همکارانش پروژه‌های آزاد و بازمتن آکادمیک در موسسه تکنولوژی
ماساچوست\RFootnote{همان دانشگاه \lr{MIT} معروف} را که برایش جذاب
بودند، به مقصد محیط‌های بسته تجاری ترک کردند. مشهورترین این پروژه‌ها
لیسپ\RFootnote{\lr{LISP} - زبانی که در حوزه‌های هوش مصنوعی شهرت بسیاری
  داشت و هنوز هم در این حوزه و حوزه‌های دیگر از آن استفاده می‌شود.}
بود. لیسپ به عنوان بخشی از یک پروژه هوش مصنوعی شروع شد و تا آن‌جا پیش
رفت که یک نفر احساس کرد این زبان آن قدر پیشرفت کرده که می‌شود آن را با
موفقیت تجاری کرد و از آن به پول رسید. در دانشگاه‌ها این زیاد اتفاق
می‌افتد. ریچارد آدمی تجاری‌ای نبود و به همین دلیل وقتی پروژه لیسپ در سال
۱۹۸۱ زیر نظر شرکت سیمبولیکس\LFootnote{Symbolics} رفت تا به پول برسد،
او از پروژه کنار گذاشته شد. برای مضاعف شدن دردناکی ماجرا، سیمبولیکس،
بسیاری از همکاران خوب او را هم استخدام کرد و نتیجه این شد که آن‌ها
آزمایشگاه هوش مصنوعی را ترک کردند.

همین اتفاق،‌ چند بار دیگر هم تکرار شد. برداشت من این است که انگیزه
فعالیت‌های بازمتن او، بیشتر از اینکه ضدتجاری باشد، در مخالفت با حذف
افراد از پروژه‌ها بوده است. برای او بازمتن به معنای بیرون نماندن از
پروژه‌ها است؛ توانایی باقی ماندن در هر پروژه‌ای مستقل از اینکه چه سازمان
تجاری‌ای حمایت آن را برعهده می‌گیرد.

جنبه فوق‌العاده جی.پی.ال. در این است که به هر کسی اجازه ورود به بازی را
می‌دهد. به این فکر کنید که این چه قدم بزرگی در پیشرفت تمدن بشری است!
ولی آیا این پیشرفت به این معنا است که هر چیزی باید جی.پی.ال. شود؟

به هیچ وجه! این همان بحث سقط جنین در تکنولوژی است. انتخاب جی.پی.ال. یا
استفاده از کپی‌رایت‌های سنتی، باید به فرد فرد مبتکران و برنامه‌نویسان
واگذار شود. هر کسی حق دارد در این مورد برای خودش تصمیم بگیرد. چیزی که
درباره ریچارد من را دیوانه می‌کند، گرایش او به سیاه و سفید دیدن همه
چیزها است. این دید باعث به وجود آمدن گرایش‌های سیاسی مختلف می‌شود. او
هیچ‌وقت دیدگاه دیگران را درک نمی‌کند. اگر او همین بحث‌ها را در مورد دین
می‌کرد، همه او را یک بنیادگرا می‌دانستند.

در واقع دومین چیز آزار دهنده دنیا - بعد از مبلغین مذهبی‌ای که در خانه‌ام
را می‌زنند و توضیح می‌دهند که من باید به چه چیزی باور داشته باشم - کسانی
هستند که در خانه‌ام را می‌زنند (یا صندوق پستی الکترونیکی‌ام را بمباران
می‌کنند) و به من می‌گویند که برنامه‌هایی که نوشته‌ام را باید تحت چه مجوزی
منتشر کنم. این یک مساله سیاسی نیست. مردم باید حق داشته باشند در مورد
خودشان تصمیم بگیرند. اینکه به کسی پیشنهاد بدهید که به فلان دلایل بهتر
است از مجوز جی.پی.ال. استفاده کند یک چیز است و اینکه روی این امر اصرار
کنید یک چیز دیگر. خیلی بد است وقتی مردم به من اعتراض می‌کنند که چرا
برای شرکتی کار می‌کنم که تمام محصولاتش را جی.پی.ال. نکرده. جواب من فقط
این است که این موضوع به آن‌ها مربوط نیست.

چیزی که باعث می‌شود من از ریچارد برنجم این اعتقاد او نیست که لینوکس به
دلیل استفاده از ابزارهای پروژه گنو\RFootnote{\lr{GNU} - پروژه‌ای که
  توسط استالمن و به منظور تولید یک سیستم عامل آزاد شروع شده است} باید
گنو/لینوکس نامیده شود. مشکل من این هم نیست که او آشکارا از شهرت من به
عنوان چهره محبوب بازمتن ابراز ناراحتی می‌کند و می‌گوید که وقتی من در سبد
لباس‌ها خوابیده بودم، او متن برنامه‌هایش را به رایگان در اختیار دیگران
می‌گذاشته. چیزی که باعث آزار من است، اصرار او است به اینکه همه مردم
باید از جی.پی.ال. استفاده کنند.

من ریچارد را به دلایل بسیاری تحسین می‌کنم. کلا هم حس می‌کنم که گرایش
دارم به افرادی ریچارد که اصول اخلاقی مشخص و محکمی دارند، احترام
بگذارم. اما چرا این آدم‌ها نمی‌توانند این اصول اخلاقی را برای خودشان نگه
دارند؟ از چیزی که بدم می‌آید این است که مردم به من بگوید باید چکار بکنم
و چکار نکنم. نفرت دارم از کسانی که فکر می‌کنند حق دارند در تصمیمات شخصی
من مداخله کنند (البته احتمالا به جز همسرم).

در طول دوران توسعه لینوکس، متخصصینی مثل اریک ریموند\RFootnote{\lr{Eric
    Raymond} - از شخصیت‌های بسیار مهم دنیای آزاد که حمایت‌های او از
  جنبش‌های اجتماعی و سیاسی نیز شهرت دارد} گفته‌اند که شاید موفقیت لینوکس
و عمر دراز جنبش بازمتن مدیون توانایی من در دوری از جناح‌بندی‌ها و برخورد
پراگماتیکم با مسایل باشد. هرچند که شاید اریک یکی از بهترین مفسران
مفهوم بازمتن باشد (هرچند که با دیدگاه‌های طرفدار اسلحه او به شدت
مخالفم)، اما به نظرم در این تعبیر از من نظرش چندان صحیح نیست. مساله
این نیست که من از جناح‌بندی‌ها دوری می‌کنم. مساله این است که من شدیدا از
هر کسی که بخواهد اصول اخلاقی خودش را به دیگران تحمیل کند متنفرم. در
این جمله می‌توانید \dbquote{اصول اخلاقی} را با \dbquote{دین}،
\dbquote{ترجیحات کامپیوتری} یا هر چیزی جایگزین کنید.

همان طور که تحمیل اصول اخلاقی اشتباه است، سازماندهی کردن آن نیز اشتباه
مضاعف است. من یکی از معتقدین جدی انتخاب فردی‌ام و این به آن معناست که
به نظرم وقتی صحبت از اصول اخلاقی است، افراد باید شخصا تصمیم گیری کنند.

من دوست دارم انتخاب خودم را داشته باشم. من شدیدا مخالف قوانین بی‌موردی
هستم که اجتماع تحمیل می‌کند. من عمیقا اعتقاد دارم که افراد تا وقتی به
دیگران صدمه نمی‌زنند، حق دارند در خلوت خانه‌های خود هر کاری که دوست
دارند بکنند. هر قانونی که این حق را نقض کند، قانونی بسیار بسیار شکننده
است. و قانون‌هایی هستند که این حق را نقض می‌کنند. من قانون‌هایی دیدم که
بسیار ترسناک بوده‌اند، بخصوص در مورد مدارس و کودکان. فقط به این فکر
کنید که قانونی برای تدریس تکامل تصویب کنند و خوب کار نکند. به نظرم
ترسناک است. این وجدان اجتماعی بی‌ریختی است که در جاهایی که اصلا به آن
مربوط نیست سرک می‌کشد.

در عین حال من معتقدم که چیزی که از من و اصول اخلاقی فردی‌ام و حتی از
نژاد بشری هم مهم تر است، تکامل است. به نظرم تا جایی که به تکامل صدمه
نمی‌زنم، حق دارم بنا بر اصول اخلاقی‌ام، در موضوعات جمعی مداخله
کنم. البته این احتمالا یک مفهوم داخلی\LFootnote{Build in} در انسان
است. به نظرم بخشی از مبانی زیست‌شناختی انسان است که ما خود را به جمع
پیوند می‌زنیم. اگر این طور نبود، هزاران سال پیش منقرض شده بودیم.

تنها چیزی که حالا باید درباره‌اش حرف بزنم: آدم‌هایی که زیادی نصیحت
می‌کنند. کلی آدم دیده‌ام که همیشه مشغول نصحیت دیگران هستند و از این کار
احساس نیکوکاری به آن‌ها دست می‌دهد.

و حالا خودم شبیه یکی از همان‌ها شده‌ام.

این یک تله معمول است، همین که مردم شما را زیادی جدی گرفتند، گرفتار آن
می‌شوید.

\section{بخش یازدهم}
آمریکایی‌ها در ۱۷ مارس (روز سنت‌پاتریک)، ۵ می (سینکو د مایو) و ۱۲ اکتبر
(روز کلمبوس) کلی سر و صدا راه می‌اندازد. ولی تقریبا کسی از روز ۶ دسامبر
مطلع نیست. از هر فنلاندی که بپرسید، به شما خواهد گفت که ۶ دسامبر، روز
استقلال فنلاند است.

اکثر فنلاندی‌ها، روز ششم دسامبر را به همان روشی جشن می‌گیرند که بقیه
جشن‌ها برگزار می‌شوند؛ نوشیدن بیش از حد.  آن‌ها شب جشن را با افراط
می‌گذارنند - حتی بنا به استانداردهای فنلاند - و تقریبا تمام روز استقلال
را جلوی تلویزیون لم می‌دهند تا حال شان جا بیاید. به هر حال تنها گزینه
دیگر این است که روز جشن ملی از خانه بیرون بروند و در برف‌ها برای راه
رفتن تقلا کنند.

تنها چیزی که همه مردم را در آن روز خاص به تلویزیون‌ها می‌چسباند، جشن
رییس جمهور است. در فنلاند زیاد از این جور جشن‌ها نداریم و در نتیجه جشن
سالانه رییس جمهوری،‌ عملا تنها مراسم عظیم سالانه است. این مراسم به شکل
مستقیم در سطح کل کشور از تلویزیون پخش می‌شود تا مردم نیمه مست را در
خانه نگه دارد و جلوی تصادفات رانندگی گرفته شود. علاوه بر این، جشن رییس
جمهور تلاش می‌کند به مردم یادآوری کند که ما خودمان هم اسکار داریم. شاید
هم \dbquote{مسابقه نهایی فوتبال} بین جامعه ممتاز فنلاند، استعاره بهتری
باشد.

در طول روز جشن، از اوتسجوکی\LFootnote{Utsjoki} شمالی گرفته تا
هنکو\LFootnote{Hanko}ی جنوبی، گراولکس\RFootnote{\lr{Gravlax} - نوعی
  ماهی آزاد نمک‌سود که در کشورهای اسکاندیناوی از آن به عنوان مزه یا
  تنقلات استفاده می‌شود.} و آسپیرین می‌خورند و به دعوت شدگانی که یکی یکی
جلو می‌آیند و با رییس جمهور دست می‌دهند نگاه می‌کنند. مردها کت‌های دامن‌گرد
می‌پوشند و زن‌ها آرایش عصر می‌کنند (البته باز هم بنا به استاندارد کشورهای
اسکاندیناوی).

هزار و نهصد و نود و نه، سالی بود که من هم به مراسم دعوت شدم. 

اگر سفیر کشوری در فنلاند باشید یا عضو مجلس باشید، خود به خود دعوت
می‌شوید. شاید صد یا دویست نفر هم به شکل اتفاقی از سطوح مختلف دعوت
شوند. بعضی از آن‌ها ممکن است مدال المپیک برده باشند و بعضی‌ها ممکن است
به رییس جمهور در برنامه‌هایش یاری رسانده باشند. اگر کاپیتان تیم هاکی
باشید که اخیرا قهرمان جهان شده هم دعوت خواهید شد. راه دیگر این است که
سیستم‌عاملی که نوشته‌اید، توجه جهانیان را جلب کرده باشد. همسر یا همراه
شما هم دعوت است.

در واقع شانس آوردیم که من و تاو هر دو توانستیم برویم. در آگوست از
اداره مهاجرت درخواست کرده بودیم که بتوانیم به فلاند برویم و
برگردیم. تا اواخر نوامبر مجوز ما صادر نشده بود. دو هفته بعد، دعوت‌مان
به جشن رییس جمهور، به ما ابلاغ شد.

حالا صحنه را تصور کنید. دو هزار فنلاندی - و دو هزار فنلاندی مهم - که
در قصر رییس جمهور جمع شده‌اند. این قصر خانه‌ای است که قدیم‌ها برای سکونت
یک بازرگان روس ساخته شده بود. این خانه یک عمارت بزرگ است اما به هرحال
برای یک خانواده ساخته شده؛ حالا گیریم خانواده‌ای با کلی آشپز و مستخدم و
این جور افراد. خانه زیاد هم بزرگ نیست.

وقتی رسیدید، یک نفر کت شما را تحویل می‌گیرد و بعد وارد می‌شوید و دیگر
بخشی از جمعیت عظیم هستید. نمی‌دانید کجا باید بروید.
پانچ\RFootnote{مشروبی الکلی حاوی آب‌میوه که در ظرف‌های بزرگ سرو می‌شود.}
در مجلس گردانده می‌شود و بدون شک پر از ودکا است. اگر ودکا نباشد، یعنی
شما در فنلاند نیستید. مدتی طول می‌کشد تا کسی را برای صحبت کردن پیدا
کنید. در نهایت با خبرنگاران مشغول صحبت می‌شوید چون صادقانه کشف می‌کنید
که جذاب‌‌ترین آدم‌های این جمع هستند (شاید هم پانچ باعث شده این آدم‌ها
جذاب‌تر از مثلا نمایندگان مجلس به نظر برسند).

انتظار نداشتم که مراسم مفرحی باشد چون به هرحال آدم‌های خیلی کمی را
می‌شناختم. من تنها کسی از گروه بازمتنی‌ها بودم که دعوت شده بود. حس
اولیه‌ام این بود که باید جایی شبیه ارتش باشد،‌ جایی که مفرح نیست ولی
بعدا می‌شود در موردش با خنده صحبت کرد. اما واقعیت این است که جای جالبی
بود.

تاو یک لباس سبز پوشیده بود که حتی اگر در اسکار بودیم هم مایه توجه
خبرنگاران می‌شد چه برسد به مراسم جشن رییس جمهور. به دلیل جذابیت تاو و
با توجه به اینکه آن سال فنلاند قهرمان هاکی جهان نشده بود، رسانه‌ها لقب
شاه و ملکه جشن رییس جمهور را به من و تاو اعطا کردند.

به هرحال. 

\begin{journal}
\dbquote{تو به عنوان یک دوست وارد این خانه می‌شوی نه به عنوان یک
  خبرنگار. هیچ خبرنگاری اجازه ورود به این خانه را ندارد.}

هیچ وقت تاو را اینقدر پر جوش و خروش ندیده بودم. درست در ورودی در خانه
جدیدی ایستاده بودیم که روز قبل لینوس و تاو کلیدش را تحویل گرفته
بودند. یکی از آن خانه‌های غول با اتاق صوتی تصویری‌ای که حالا جای میز
بیلیارد لینوس شده بود. این خانه به راحتی توان تبدیل شدن به یک مهدکودک
را هم داشت. یک راهروی وسیع که از هال می‌گذشت، در ورودی را به اتاق نشیمن
که در سمت دیگر ساختمان قرار داشت متصل می‌کرد. کافی بود کاشی‌های
ایتالیایی شیک را حذف کنند تا یک مسیر عالی برای تمرین اسکیت‌بورد برای
دخترها فراهم شود. اتاق کار لینوس در طبقه اول واقع شده و با یک در
شیشه‌ای از سر و صدای خانه ایزوله شده است. پنج تا هم حمام دارند و شاید
تا الان چند تای دیگر هم در گوشه‌ و کنار خانه پیدا کرده باشند. کل این
مجموعه با یک دروازه مستقل، از سیلیکون‌ولی جدا شده است.

نیک توروالدز هم اینجا است تا فامیلش را ببیند. پدر و پسر تازه از یک
گردش کوتاه با بی.ام.و. زد ۳ اجاره‌ای لینوس، برگشته‌اند. این همان مدل
ماشینی است که لینوس به زودی خواهد خرید. قرار است امروز عصر، نیک با این
ماشین به کتابخانه دانشگاه استنفورد برود اما پیش از این کار می‌خواهد در
آب داغ کمی لم بدهد و در حال رفتن به سمت حیاط پشتی که وان در آن واقع
شده، اعلام می‌کند که این بزرگترین خانه‌ای است که یک توروالدز، صاحب آن
بوده. برمی‌گردد و روی یک ورق کاغذ اسم همه بیست توروالدزی که در جهان هست
را می‌نویسد. خبر ندارد که بیست و یکمی هم در راه است.

لینوس هم از این خانه بزرگ ولی خالی به هیجان آمده. نیک دارد از اطراف
خانه فیلم می‌گیرد و من از لینوس خواهش می‌کنم تا چرخی با تاو بزند و من از
آن‌ها عکس بگیرم. این غیر فنلاندی‌ترین روش برای بروز هیجان است.

تاو می‌گوید: \dbquote{هیچ وقت فکر می‌کردی خانه‌مان به این بزرگی باشد؟}
\end{journal}

\begin{journal}
تاو می‌خواهد در لحظه باز شدن فروشگاه ایکیا آنجا باشد تا برای خانه جدید،
تجهیزات بخرد و من پیشنهاد می‌کنم که لینوس بچه‌ها را به آپارتمانی که من
به تازگی در استینسون بیچ\LFootnote{Stinson Beach} اجاره کرده‌ام بیاورد
تا تاو بدون دردسر به کارهای مورد علاقه‌اش برسد. همین که می‌رسند، به
توروالدز اصرار می‌کنم که کایاک سواری را امتحان کند. چند دوری می‌زند و
بعد بچه‌ها را هم یکی یکی سوار می‌کند. وقتی به خانه بر می‌گردد، شلوارش خیس
است.

از لینوس می‌خواهم تا فصل \dbquote{آیا موفقیت مرا به فساد خواهد کشاند؟}
را بخواند و نظرش را بگوید و برای اینکه راحت باشد، بچه‌ها را به ساحل
می‌برم. پاتریشیا و دانیلا نیم ساعتی دنبال ستاره‌دریایی می‌گردند و کمی هم
پاهایشان را در دریا خیس می‌کنند ولی چیزی نمی‌گذرد که یکی از آن‌ها می‌گوید
\dbquote{\lr{Kisin kommer}}، یعنی \dbquote{می‌خواهم به دستشویی بروم.}

به خانه که برمی‌گردیم، لینوس که فقط شورت پوشیده، با یک بسته اسنک پشت
کامپیوتر نشسته و تند و تند تایپ می‌کند. شاید پانزده ثانیه‌ای طول می‌کشد
تا متوجه حضور ما شود. سرش را از کامپیوتر بیرون می‌آورد و از بالای
مونیتور به من نگاه می‌کند و اولین کلماتی که می‌گوید این ها هستند:
\dbquote{هی پسر! این مکینتاشت واقعا چیز مزخرفی است.}

و بعد: \dbquote{آه، شلوارم را انداختم در خشک کن.}

عنوان فصل را به \dbquote{شهرت و ثروت} تغییر داده و استدلالش این است که
\dbquote{آیا موفقیت مرا به فساد خواهد کشاند؟} زیادی خودبینانه
است. می‌گوید که به وقت بیشتری نیاز دارد و در نتیجه دوباره بچه‌ها را برای
گردش بیرون می‌برم.
\end{journal}

\section{بخش دوازدهم}
اگر ندانید که جنگ با آسیاب‌های بادی مشکل است، جنگیدن با آن‌ها ساده خواهد
بود. پنج سال پیش که مردم از من می‌پرسیدند که آیا لینوکس موفق خواهد شد
در جنگ کامپیوترهای رومیزی کوچکترین ضربه‌ای به مایکروسافت بزند، همیشه
تردید را در صدای شان تشخیص می‌دادم. آن‌ها به این موضوع شک داشتند. واقعیت
این است که احتمالا آن‌ها بیش از من در مورد حقیقت موضوع اطلاع داشتند.

من واقعا درکی از همه قدم‌های مورد نیاز برای رسیدن به آن هدف نداشتم. نه
تنها در مورد مشکلات تکنولوژیک پیش روی ایجاد یک سیستم‌عامل کامل و قابل
اتکا اطلاع چندانی نداشتم، که حتی این را هم نمی‌دانستم که برای عرضه چنین
سیستم‌عاملی به جز یک تکنولوژی موفق، به چه چیزهایی نیاز است. احتمالا اگر
از اول می‌دانستم که برای موفقیت لینوکس به چه میزان زیرساخت نیاز است، از
همان ابتدا روحیه‌ام را از دست می‌دادم. مساله این نیست که کار ما باید خوب
باشد، مساله این است که علاوه بر خوب بودن ما، کلی چیز دیگر هم باید خوب
پیش برود.

هر آدم عاقلی که به کوه ناهموار پیش رو نظر می‌انداخت، از ایده صعود به آن
به وحشت می‌افتاد. برای نمونه به مشکلات ایده پشتیبانی از سخت‌افزارهای
کامپیوترهای شخصی نگاه کنید که بدون شک یکی از متنوع‌ترین گونه‌های
کامپیوتری روی زمین‌اند. برای داشتن یک سیستم‌عامل موفق، باید بتوان باگ‌های
نرم‌افزارهایی را اصلاح کرد که نه امکان تکرار و بررسی عملی آن‌ها را دارید
و نه علاقه‌ای به اصلاح آن‌ها. اما شما به لینوکس علاقه دارید پس باید به
تلاش برای حل مشکلات مرتبط با آن هم علاقمند باشید.

حتی برای فکر کردن به داشتن سهمی در بازار تجاری، باید میزان قابل توجهی
کاربر داشته باشید. از همان روزهای اول لینوکس، می‌شد با یک شرکت مساله
پشتیبانی را تا حدی پاسخ گفت ولی برای مقایس بزرگتر، نیاز به تعداد افراد
زیاد و تجهیزات فراوان می‌بود. نمی‌شود برای سی روز اول نصب، یک شماره تلفن
رایگان به طرف داد تا اگر مشکلی داشت زنگ بزند. البته پشتیبانی هیچ وقت
در لینوکس به موضوعی مشکل زا تبدیل نشد چون شرکت‌های زیادی بودند که در
این مورد به مشتریان کمک می‌کردند، از لینوکس کر\LFootnote{Linuxcare} و
ردهت گرفته تا آی.بی.ام. و سیلیکون‌ گرافیکس و کامپک و دل. من برای مدت‌های
مدید اصولا متوجه این موضوع نبود در حالی که مساله پشتیبانی همیشه یک
نقطه حساس و ارزشمند بوده است.

من بر خلاف افرادی که پیش زمینه فنی کافی در صنعت داشتند و بر خلاف
روزنامه‌نگارانی که متخصص صنعت بودند، فقط یک برنامه‌نویس با دید محدود
بودم که چیز چندانی در مورد نیازهای آینده نمی‌داند. حتی از نظر فنی هم
شاید اطلاعات کافی نداشتم چرا که اگر از اول می‌دانستم که چه راه دشواری
قرار است پیموده شود و اگر می‌دانستم که ده سال بعد از شروع هم باید
کماکان به همان کار ادامه دهم - و در تمام این ده سال،‌ لینوکس عملا شغل
تمام وقتم باشد - احتمالا هیچ وقت کار را شروع نمی‌کردم.

و از مزاحمت‌ها هم بگویم! این روزها مزاحمت زیادی ندارم ولی گاه گداری
افرادی که از بازمتن خوش شان نمی‌یاید یا کسانی که با یک باگ روبرو
می‌شوند، به من ایمیل‌های تند می‌زنند و از سر ناامیدی به من فحش می‌دهند. در
مقایسه با ایمیل‌های مثبتی که دریافت می‌کنم، این ایمیل‌ها به حساب نمی‌آیند،
ولی به هرحال آن‌ها را دریافت می‌کنم.

بعله! اگر می‌دانستم که چقدر کار باید انجام شود و انجام آن چقدر دردسر
دارد و پیش بردن بعضی چیزها چقدر انرژی می‌طلبد، شاید اصلا شروع
نمی‌کردم. اگر سواد کافی برای درک مشکلات پیش رو داشتم، احتمالا هیچ وقت
لینوکس را از همان دو سه نسخه اولیه، جلوتر نمی‌بردم. اگر می‌دانستم که
چقدر نکات جزیی را باید مراعات کنید و اگر می‌دانستم که مردم چه انتظاراتی
از یک سیستم‌عامل دارند، سناریوهای ترسناکی در ذهنم شکل می‌گرفت که هرگز
نمی‌توانستم بر آن‌ها غلبه کنم.

البته جنبه‌های مثبت را هم پیش‌بینی نکرده بودم. مثل اینکه چقدر حمایت
خواهم شد و چند نفر برای انجام این پروژه دست به دست هم خواهند داد. پس
اجازه دهید تا نظرم را عوض کنم. به نظرم اگر جنبه‌های مثبت را از پیش
می‌دانستم، به احتمال زیاد این کار را شروع می‌کردم.
