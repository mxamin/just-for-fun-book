\chapter{مقدمه}
\textbf{\Large معنای زندگی یک} \vspace*{10pt} \\
{\Large (سکس، جنگ، لینوکس)} \vspace*{10pt} \\

\noindent
\textbf{صحنه:} اولین جرقه‌های نوشته شدن این کتاب به داخل یک فورد سیاه
قدیمی که در جاده بین ایالتی حوالی سنترال ولی کالیفرنیا در حرکت بود،
باز می‌گردد. لینوس و تاو توروالدز و دخترهای کوچک آن‌ها، یعنی پاتریشیا و
دانیلا، به همراه یک آدم فضول مشغول ۵۶۴ کیلومتر رانندگی برای رسیدن به
لوس‌آنجلس هستند تا در آن‌جا به باغ‌وحش و یک فروشگاه
آیکیا\RFootnote{فروشگاه مشهور لوازم خانگی سوئدی} بروند.

\begin{dialogue}
\david
 حالا یک سوال اساسی دارم که باید به آن فکر کنی، یک سوال
مهم. از این کتاب می‌خواهی به چه چیزی برسی؟

\linus
خب، می‌خواهم معنای زندگی را توضیح دهم.

\tove
لینوس! یادت بوده که باید باک ماشین را پر کنی؟

\linus
من درباره معنای زندگی، یک نظریه دارم. ما می‌توانیم در فصل
اول برای مردم توضیح بدهیم که معنای زندگی چیست. با این کار جذب کتاب
خواهند شد. وقتی با خواندن این فصل جذب شدند و کتاب را خریدند، می‌توانیم
بقیه کتاب را با خزعبلات پر کنیم.

\david
اوه بله. به نظر نقشه جالبی می‌آید. یک بار شخصی به من گفت که
از بدو پیدایش بشر دو سوال همیشه ذهن او را مشغول کرده است . اول:
\dbquote{معنای زندگی چیست؟} و دوم: \dbquote{با این همه پول خرد که آخر
  روز در ته جیبم جمع می‌شود چکار کنم؟}

\linus
من جواب سوال اول را دارم. 

\david
و جواب سوال اول چی است؟

\linus
یک جواب ساده و دوست داشتنی. این جواب هیچ معنایی به زندگی
شما نمی‌دهد، ولی نشان‌تان می‌دهد که پشت پرده چه چیزی در جریان است. در
زندگی سه چیز معنادار هست. این‌ها سه انگیزه اصلی در زندگی شما
هستند. عواملی که باعث می‌شوند شما کارهایی را انجام دهید که یک موجود
زنده می‌کند: اولی بقا است، دومی نظم اجتماعی و سومی تفریح. هر چیزی در
زندگی، به همین ترتیب است و بعد از تفریح هم دیگر چیزی نیست. این به نوبه
خود مستلزم این است که در زندگی هر کاری معطوف به رسیدن به مرحله سوم
باشد و وقتی به منطقه سوم برسید، کارتان تمام شده است. البته پیش از
رسیدن به مرحله آخر، باید از مراحل قبل بگذرید.

\david
این بحث که گفتی‌، توضیح بیشتری لازم دارد.

\patricia
بابا! می‌شود جایی بایستیم و بستنی شکلاتی بخوریم؟ همین حالا
می‌خواهم بستنی شکلاتی بخورم.

\tove
نه عزیزم. باید کمی صبر کنی. وقتی ایستادیم که بروی جیش کنی،
بستنی هم می‌خریم.

\linus
با یک مثال مساله روشن می‌شود و نکته را می‌گیری. بهترین مثال
هم سکس است. سکس به عنوان راه بقا شروع شد و بعد به یک مساله اجتماعی
تبدیل شد. به همین دلیل است که ازدواج می‌کنیم. اما حالا سکس دارد به یک
جور تفریح تبدیل می‌شود.

\patricia
من باید برم جیش کنم.

\david
چطور به تفریح تبدیل شده؟

\linus
انگار برای تو مثال مناسبی نبود. خب بگذار برویم سراغ یک چیز
دیگر...

\david
نه! برگردیم به سکس.

\linus
در یک سطح دیگر اگر به مفهوم سکس به معنای بیولوژیکی‌اش نگاه
کنیم باید بپرسیم که سکس چگونه شروع شده است؟ برای بقا. سکس در اول کار
یک تفریح نبود. فقط رسیدن به هم بود. شاید بهتر باشد صحبت در مورد سکس را
کنار بگذاریم.

\david
نه نه. فکر کنم می‌تواند خودش یک فصل باشد.

\linus
می‌توانیم به جایش سراغ جنگ برویم. مشخص است که اولین جنگ‌ها
برای بقا بوده‌اند چون یک یارویی بین شما و چاله آب ایستاده بود. بعد باید
سر زن با یک مرد می‌جنگیدید و بعد جنگ تبدیل می‌شد به یک موضوع
اجتماعی. این جریان سال‌ها قبل از قرون وسطی واقع شد.

\david
یعنی جنگ به عنوان شیوه‌ای برای برقراری نظم اجتماعی.

\linus
درست است ولی شاید بهتر باشد بگوییم روشی برای پیدا کردن جایی برای افراد در نظم اجتماعی؛ چون هیچ کس برای خودِ نظم اجتماعی اهمیت چندانی قایل نیست. در این ماجرا، فرقی هم نمی‌کند شما یک مرغ در مرغدانی باشید یا یک انسان در جامعه. 

\david
و می‌خواهی بگویی این روزها جنگ به یک سرگرمی تبدیل شده؟

\linus
دقیقا.

\david
شاید برای کسانی که آن را روی تلویزیون می‌بیند این حرف درست
باشد. برای آن‌ها جنگ یک سرگرمی شده.

\linus
همچنین در بازی‌های کامپیوتری. بازی‌های جنگی. سی‌ان‌ان. تازه
دلیل جنگ هم ممکن است به تفریح مربوط باشد. علاوه بر این، برداشت از جنگ
هم به یک سرگرمی تبدیل شده. و دلیل سکس هم معمولا تفریح است. البته
مطمئنا ادامه بقا هنوز مساله مهمی است، بخصوص اگر کاتولیک باشید. ولی حتی
اگر کاتولیک باشید هم، گاهی به سکس به عنوان یک تفریح نگاه می‌کنید. پس
قرار نیست صد در صد تفریح باشد. در هر چیزی بخشی از انگیزه، به بقا مربوط
می‌شود، بخشی به نظم اجتماعی و بقیه به تفریح. حالا به فناوری نگاه
کنید. فناوری برای بقا به وجود آمد و فقط هم نه برای بقای صرف، بلکه برای
بقای راحت‌تر. شما آسیاب‌بادی می‌سازید که آب را از چاه بیرون بیاورد.

\david
یا آتش.

\linus
بله. هنوز صحبت بر سر بقا است و به نظم اجتماعی و تفریح
نرسیده.

\david
و چطور فناوری به نظم اجتماعی شکل می‌دهد؟

\linus
خب در حقیقت بیشتر جریان صنعتی‌شدن مربوط به بقا و بقای بهتر
است. در صنعت خودروسازی، فناوری به سراغ ساخت خودروهای سریع‌تر و بهتر
رفته است. ولی در جاهایی هم بحث شکل دادن به نظم اجتماعی مطرح بوده. مثلا
اختراع تلفن و تا حدی تلویزیون. بخش زیادی از برنامه‌های اولیه تلویزیون،
در مورد آموزش‌ عقاید بوده است. رادیو هم همین‌طور. به دلیل همین کارکرد
نظم اجتماعی است که بسیاری از دولت‌ها در رادیو و تلویزیون این همه
سرمایه‌گذاری کرده‌اند.

\david
برای ایجاد و دوام نظم اجتماعی مورد نظرشان...

\linus
درست است، ولی بعد جریان پیشرفت کرد. این روزها مشخصا
تلویزیون به منظور تفریح استفاده می‌شود. این روزها هر جا را که نگاه کنید
پر است از تلفن همراه. این وسیله اول برای یک نظم اجتماعی تولید شده بود،
ولی کم کم دارد یک وسیله تفریحی می‌شود.

\david
پس آینده فناوری چیست ؟ به نظر می‌رسد از مرحله بقا فراتر
رفته‌ایم و حالا در مرحله نظم اجتماعی هستیم.

\linus
باز هم درست است. فناوری همیشه برای ساده‌تر کردن زندگی بوده
است. سعی می‌کرده ما را سریع‌تر به مقصد برساند، قیمت کالاها را کم‌تر کند،
خانه‌ها را بهتر کند و این جور چیزها. ولی فرق فناوری اطلاعات با
فناوری‌های قدیمی چیست؟ این واقعیت که همه به هم متصل خواهند بود، به چه
چیزی منجر می‌شود؟ مشخص است که ارتباط بهتر شده ولی این یک تفاوت پایه‌ای
نیست. به نظر من قدم بزرگ بعدی، تفریح است.

\david
هر چیزی در نهایت به تفریح تغییر شکل می‌دهد...

\linus
و این توضیح می‌دهد که چرا لینوکس در حد خودش موفق است. به سه
انگیزه اصلی نگاه کنید. اولی انگیزه بقا است که کسانی که کامپیوتر دارند
منطقا قبلا آن را تامین کرده‌اند. واضح است که اگر کامپیوتر داشته باشید،
غذا برای خوردن و اینجور چیزها هم دارید. دومین انگیزه، نظم اجتماعی
است. لینوکس به گیک‌هایی که در گوشه اتاق‌های شان نشسته‌اند، در نظم اجتماعی
جایی می‌دهد.

\david
در کامدکس\RFootnote{\lr{COMDEX} بعد از سبیت بزرگترین
  نمایشگاه کامپیوتری بود که تا سال ۲۰۰۳ هر ساله در لاس‌وگاس برگزار
  می‌شد.} حرف بسیار جالبی زدی. انگار گفتی که توسعه لینوکس یک جور ورزش
است که یک تیم جهانی مشغول آن است. تو این را ممکن کردی رفیق!

\linus
لینوکس مثال خوبی است از اینکه چرا مردم عاشق تیم‌های ورزشی و
بخصوص عضویت در این تیم‌ها هستند.

\david
آره!‌ در حالی که تمام روز پشت کامپیوتر نشسته‌ای، احتمالا
ترجیح می‌دهی عضو یک چیزی باشی. هر چیزی.

\linus
یک چیز اجتماعی است، مثل یک تیم ورزشی. به یک تیم فوتبال فکر
کن یا از آن بهتر به یک تیم فوتبال دبیرستانی. جنبه اجتماعی لینوکس خیلی
خیلی مهم است ولی لینوکس تفریح هم است؛ از آن نوع تفریح‌هایی که با پول به
این راحتی‌ها قابل خریدن نیست. وقتی در مرحله بقا باشید، پول انگیزه بسیار
خوبی است چون به راحتی می‌تواند مواد مورد نیاز برای بقا را بخرد.  به
راحتی می‌شود پول را با این جور چیزها عوض کرد، ولی همین که به مرحله
تفریح برسید...

\david
پول به درد نخور می‌شود؟

\linus
نه. به درد نخور نیست چون می‌توانید با پول فیلم، ماشین‌های
سریع و تعطیلات بخرید. با پول می‌توانید کلی چیز بخرید که وضعیت زندگی شما
را بهتر می‌کنند.

\tove
لینوس، باید پوشک دانیلا را عوض کنیم و پاتریشیا باید به
دستشویی برود. من هم یک کاپوچینو می‌خواهم. فکر کنی اینجا
استارباکس\RFootnote{یک فروشگاه زنجیره‌ای قهوه} پیدا بشود؟ اصلا کجا
هستیم؟

\speakbf{دیوید (به بالا نگاه می‌کند)}
با توجه به بویی که می‌آید، باید نزدیک کینگ
سیتی باشیم.

\linus
حالا مقیاس را بزرگ‌تر کنید. مساله فقط درباره افراد نیست،‌
درباره کل زندگی است. مثل قانون آنتروپی. در این قانون آنتروپی زندگی،
همه چیز در حال حرکت از سوی بقا به تفریح است، ولی نه به این معنا که در
مقیاس کوچک، این روند هیچ وقت نمی‌تواند در جهت معکوس حرکت کند. اتفاقا
گاهی این حرکت مشخصا معکوس می‌شود. بعضی مواقع به سمت تلاشی پیش می‌رویم.

\david
ولی به عنوان یک سیستم، همه چیز در یک جهت پیش می‌رود...

\linus
همه چیز در یک جهت پیش می‌رود ولی نه در یک زمان خاص. پس می‌شود
دید که سکس به مرحله تفریح رسیده است، جنگ به آن نزدیک شده و فناوری عملا
به آن رسیده. چیزهای جدید فقط به بقا توجه می‌کنند. مثلا، امیدوارم که،
سفر فضایی اوایل مربوط به بقا باشد، بعد به یک امر اجتماعی تبدیل شود و
در نهایت نوعی تفریح باشد. اگر به تمدن هم به عنوان یک کیش نگاه کنید،
الگوی مشابهی را خواهید دید. تمدن به خاطر بقا ایجاد شده. وقتی با هم
هستید، راحت‌تر زنده می‌مانید و بعد شروع به ایجاد ساختارهای اجتماعی
می‌کنید. اما در نهایت تمدن منحصرا به خاطر تفریح حفظ می‌شود. قبول! منحصرا
را پس می‌گیرم و تفریح هم تفریح بدی نیست. رومی های قدیم به خاطر نظم
اجتماعی بسیار زیاد و همچنین تفریح‌های حسابی‌شان معروفند. آن‌ها بهترین
فیلسوفان دوره خودشان را داشتند.

\david
و چطور همه این‌ها به معنای زندگی مرتبط می‌شوند؟

\linus
راستش را بگویم نمی‌شوند. مساله این است که... این است که مشکل
همین جا است.

\david
این همان رابطه‌ای است که باید به آن فکر کنی. 

\patricia
مامان! گاوها را نگاه کن.

\linus
خب اگر بدانید که کل زندگی مربوط به این پیشرفت است، بعد هدف
شما این خواهد بود که خودتان هم در این مسیر حرکت کنید. این مسیر هم فقط
مسیر یک چیز نیست. هر کاری که می‌کنید، بخشی از حرکت در مسیرهای مختلف
است. می‌توانید از خودتان بپرسید \dbquote{چکار کنم که جامعه بهتر شود؟} چون شما
بخشی از جامعه هستید و می‌‌دانید که جامعه در این مسیر در حال حرکت
است. شما هم می‌توانید به این حرکت کمک کنید.

\speakbf{تاو (دماغ‌اش را گرفته)}
چه بوی گندی می‌یاد.

\linus
شاید بتوانیم این طور نتیجه گیری کنیم که در نهایت امر، همه
ما اینجا هستیم تا تفریح کنیم. تازه می‌توانیم بنشینیم و استراحت کنیم و
از این مسیر لذت ببریم.

\david
فقط برای تفریح؟
\end{dialogue}
